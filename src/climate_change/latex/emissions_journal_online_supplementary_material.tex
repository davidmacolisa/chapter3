%======================================================================================================================%
\documentclass[authoryear, preprint, twocolumn, 1p]{elsarticle}
%======================================================================================================================%
%% Preamble
\usepackage{natbib}
%\biboptions{longnamesfirst, angle, semicolon, sort} % Citation options
\usepackage{graphicx}
\usepackage{parskip}
\usepackage{float}
\usepackage{booktabs}
\usepackage[toc, page]{appendix}
\usepackage{amsmath}
\usepackage{amssymb}
\usepackage{amsfonts}
\usepackage{mathrsfs}
\usepackage[doublespacing]{setspace}
\usepackage{lineno}
\linenumbers % Uncomment this line if you need line numbers for review
\usepackage{geometry}
\geometry{paperwidth=8.5in, paperheight=11in, left=0.5in, right=0.5in, top=0.5in, bottom=0.5in}
\usepackage{hyperref}
\hypersetup{colorlinks = true, citecolor = blue, linkcolor = blue, urlcolor = blue, hypertexnames = true}
\newcommand{\shortlink}[1]{\href{https://www.#1}{\texttt{#1}}}
\journal{Research Policy} % Specify the journal name
%======================================================================================================================%
\begin{document}
%======================================================================================================================%
    \begin{frontmatter}
        % Title Page
        \title{Online Supplementary Material: Unforeseen Minimum Wage Consequences}
        \author[1]{Davidmac Olisa Ekeocha\corref{cor1}} % Include corresponding author footnote
        \ead{davidmac.ekeocha@liverpool.ac.uk} % Add your email address
        \cortext[cor1]{Corresponding author} % Corresponding author footnote

        \affiliation[1]{organization={University of Liverpool}, % Add affiliation details
            addressline={Chatham Street},
            postcode={L69 7ZH},
            city={Liverpool},
            country={United Kingdom}}
%----------------------------------------------------------------------------------------------------------------------%
        \begin{abstract}
            \noindent This online supplementary material presents results on additional heterogeneous characterizations, the offsite and publicly owned treatment works (POTWs), and the state-level baseline results for the article. Although, I lost some statistical power in the state analyses due to few state clusters, the results here show consistency in the direction of the effects reported in the original manuscript. Additionally, there are significant positive differences in the intensities of total offsite releases including land releases, but with one to two-year anticipations. The total release intensity transferred to POTWs increased one-year later, but waste treatment at POTWs declined two years later, suggesting greater adverse environmental impacts.
        \end{abstract}

        \begin{keyword}
            Minimum Wage \sep Toxic Releases \sep Emissions \sep Staggered Difference-in-Differences \sep United States
        \end{keyword}
    \end{frontmatter}
%======================================================================================================================%


    \section{Additional Heterogeneous Effects}\label{sec:additional-heterogeneous-effects}

    \subsection{Highest Emitting Industries}\label{subsec:highest-emitting-industries}
    There are manufacturing industries that release and emit more toxic chemicals than others. This subsection investigates if the documented effect is heterogeneous and entirely driven by such highest emitting manufacturing industries. They include the chemical, food, leather, and allied products and wood manufacturing industries (see their distributions in Figures~\ref{fig:releases-distribution-naics},~\ref{fig:air-emissions-distribution-naics},~\ref{fig:water-discharge-distribution-naics} and~\ref{fig:land-releases-distribution-naics} of Appendix~\ref{sec:appendix-distribution-of-industries-and-pollution-emissions-intensities}). To investigate the differential effect of higher MW on onsite total releases intensity of highest emitting manufacturing industries (HEIs), I estimate the following model:
    \begin{align}
        P_{f,cp,c,t}^{heis} &= \beta (Treated^{e} \cdot D)_{f,s,t} + \psi (Treated^{e})_{s,t} + \vartheta (Treated \cdot D)_{f,s,t} + \mu (Post \cdot D)_{f,s,t} \nonumber \\
        &\quad + \tau Treated_{s,t} + \rho D_{f,s,t} + \alpha Post_{t} + \delta X_{v,c,t-1} + \omega F_{f,t} + \lambda_{t} + \gamma_{f} + \phi_{cp} \nonumber \\
        &\quad + \zeta_{c} + \eta_{c,t} + \theta_{cp,t} + \varepsilon_{f,cp,c,t},\label{eq:heterogeneous-onsite-releases-intensity-heis}
    \end{align}
    where $P_{f,cp,c,t}^{heis}$ is the vector of total onsite releases intensity (air, water and land) of toxic chemicals at a manufacturing industry facility, $f$ in a cross-border county pair, $cp$ through a toxic chemical, $c$ in the year, $t$. $Treated_{s,t}$ is a dummy that is equal to $1$ for the treated states, and $0$ the control states. $Post_{t}$ is a dummy that is equal to $1$ if the year $t$ is a post-treatment year, and $0$ otherwise. And $D_{f,s,t}$ is a dummy that is unity for the set of HEIs, $f$ in state, $s$ in the year, $t$ and $0$ low and lowest emitting manufacturing industries (LEIs).
    % Please add the following required packages to your document preamble:
% \usepackage{booktabs}
% \usepackage{graphicx}
\begin{table}[H]
    \centering
    \caption{Onsite Releases Intensity given Highest Emitting Manufacturing Industries}
    \label{tab:heterogeneous-onsite-releases-int-heis}
    \resizebox{\columnwidth}{!}{%
        \begin{tabular}{@{}llllllll@{}}
            \toprule\toprule
            Onsite releases intensity (log) & total     & air emissions & point air & fugitive air & water discharge & land releases & surface impoundment \\ \midrule
            $Treated^{e} \cdot D$           & 0.360***  & 0.224***      & 0.032     & 0.235***     & 0.094           & 0.002         & -0.011**            \\
            & (0.116)   & (0.077)       & (0.066)   & (0.067)      & (0.087)         & (0.019)       & (0.005)             \\
            $Treated^{e}$                   & 0.020     & 0.023         & 0.048     & -0.011       & 0.007           & 0.007         & 0.021**             \\
            & (0.039)   & (0.039)       & (0.036)   & (0.030)      & (0.011)         & (0.013)       & (0.010)             \\
            cohort 2014 $\cdot D$           & 0.350***  & 0.305***      & 0.259**   & 0.077        & -0.079          & 0.041         & -0.011**            \\
            & (0.113)   & (0.114)       & (0.107)   & (0.077)      & (0.067)         & (0.044)       & (0.005)             \\
            cohort 2015 $\cdot D$           & 0.369***  & 0.149**       & -0.178*   & 0.381***     & 0.254**         & -0.034        & -0.012              \\
            & (0.141)   & (0.069)       & (0.091)   & (0.090)      & (0.122)         & (0.021)       & (0.007)             \\
            cohort 2017 $\cdot D$           & 0.052     & 0.032         & 0.166     & 0.145        & 0.033           & 0.025         & -0.002              \\
            & (0.174)   & (0.179)       & (0.141)   & (0.142)      & (0.091)         & (0.025)       & (0.009)             \\
            controls                        & Yes       & Yes           & Yes       & Yes          & Yes             & Yes           & Yes                 \\
            year FE                         & Yes       & Yes           & Yes       & Yes          & Yes             & Yes           & Yes                 \\
            facility FE                     & Yes       & Yes           & Yes       & Yes          & Yes             & Yes           & Yes                 \\
            border-county FE                & Yes       & Yes           & Yes       & Yes          & Yes             & Yes           & Yes                 \\
            toxic chemical FE               & Yes       & Yes           & Yes       & Yes          & Yes             & Yes           & Yes                 \\
            toxic chemical LTs              & Yes       & Yes           & Yes       & Yes          & Yes             & Yes           & Yes                 \\\midrule
            Observations                    & 1,893,689 & 1,893,689     & 1,893,689 & 1,893,689    & 1,893,689       & 1,893,689     & 1,893,689           \\
            $R^2$                           & 0.720     & 0.739         & 0.712     & 0.661        & 0.585           & 0.501         & 0.127               \\ \bottomrule \bottomrule
        \end{tabular}%
    }
    \begin{minipage}{\columnwidth}
        \vspace{0.05in}
        \tiny NOTES: These results are obtained from estimating this equation: $P_{f,cp,c,t}^{heis} = \beta (Treated^{e} \cdot D)_{f,s,t} + \psi (Treated^{e})_{s,t} + \vartheta (Treated \cdot D)_{f,s,t} + \mu (Post \cdot D)_{f,s,t} + \tau Treated_{s,t} + \rho D_{f,s,t} + \alpha Post_{t} + \delta X_{v,c,t-1} + \omega F_{f,t} + \lambda_{t} + \gamma_{f} + \phi_{cp} + \zeta_{c} + \eta_{c,t} + \varepsilon_{f,cp,c,t}$. Three-way clustered robust standard errors are reported in parentheses, and clustered at the toxic chemical, industry and state levels. ***, **, and * denote significance levels at the less than $1\%$, $5\%$ and $10\%$, respectively.
    \end{minipage}
\end{table}

    The parameter of interest here is the triple-differences parameter $\beta$ which measures the differential impact on onsite total releases intensities due to a higher MW policy for HEIs. $\psi$ measures the relative change in onsite total releases intensity for non-carcinogenic chemicals. And $\beta + \psi$ measures the total relative change in onsite total releases intensities for HEIs.

    The results are presented in Table~\ref{tab:heterogeneous-onsite-releases-int-heis}. I find that the increasing toxic release intensity is predominantly observed in higher-emitting manufacturing industries, particularly within the $2014$ and $2015$ cohorts. This increase is primarily driven by rises in total air emissions (both point and fugitive) and surface water discharge intensities, especially notable in these cohorts. However, I find a decline effect in point air emissions intensity in $2015$ cohort. No significant effect is recorded on surface impoundment intensity. Conversely, in lower-emitting manufacturing industries, except for a notable positive impact on surface impoundment intensities, the overall impact on total toxic releases, including all air emissions and surface water discharge intensities, remains relatively muted. Additionally, there is limited evidence of significant changes in land release intensity across the industries.
    \begin{figure}[H]
    \centering
    \includegraphics[width=1\textwidth, height=0.5\textheight,keepaspectratio]{fig_sdid_total_onsite_releases_int_EMITT}
    \caption{Triple-Differences: Onsite Total Releases Intensity given Highest Emitting Manufacturing Industries}
    \label{fig:heterogeneous-onsite-releases-intensity-emitt}
    \begin{minipage}{18cm}
        \vspace{0.05in}
        NOTES: The event study model of equation~\ref{eq:heterogeneous-onsite-releases-intensity-heis} is $G_{f,cp,c,t}^{gdp} = \sum_{{e = -3},{e \neq -1}}^{3} \beta (Treated^{e} \cdot D)_{f,s,t} + \psi (Treated^{e})_{s,t} + \vartheta (Treated \cdot D)_{f,s,t} + \mu (Post \cdot D)_{f,s,t} + \tau Treated_{s,t} + \rho D_{f,s,t} + \alpha Post_{t} + \delta X_{v,c,t-1} + \omega F_{f,t} + \lambda_{t} + \gamma_{f} + \phi_{cp} + \zeta_{c} + \eta_{c,t} + \varepsilon_{f,cp,c,t}$. Three-way clustered robust standard errors are reported in parentheses, and clustered at the toxic chemical, industry and state levels. HEIs mean highest emitting industries.
    \end{minipage}
\end{figure}

    Figure~\ref{fig:heterogeneous-onsite-releases-intensity-emitt} illustrates the dynamic effects of higher MW policies. The results indicate that the positive effects on toxic release intensities are present in both the highest and lowest emitting manufacturing industries. These effects are especially evident in air emissions (both point and fugitive), surface water discharge, and land release intensities. The impact emerges immediately and persists for up to three years later. For HEIs, the effect on surface impoundment intensity declines, whereas, for LEIs, both land releases and surface impoundment intensities increase instantaneously and persist for up to three years later.

    \subsection{Industry Concentration}\label{subsec:industry-concentration}
    In theory, companies operating in highly concentrated industries (HCIs), which face less competition, benefit from economies of scale and maintain dominant market positions~\citep{baumol1982contestable}. These firms can absorb the costs associated with minimum wage increases due to their large production scales. Moreover, they can transfer the increased labour costs to consumers through higher sales prices or to suppliers by negotiating lower purchase prices. Consequently,~\citet{zhang2023unintended} argued that HCI industries will exhibit a smaller increase in total releases intensities in response to MW hikes. I test whether high concentration (less competitiveness) implies smaller increases in total releases intensities in response to MW increases. To gauge market concentration, I employ the Herfindahl-Hirschman Index (HHI)~\citep{zhang2023unintended, weinstock1982using}. This index is derived each year by summing the squared market shares of all firms within a six-digit NAICS industry codes. Utilizing revenue data to compute the HHI, I then average these values over the entire sample period and classify industries based on the median HHI value. A lower Herfindahl-Hirschman Index (HHI) indicates that the industry is less concentrated and exhibits greater competitiveness. These are industries below the median HHI value. Those above the median HHI value are classified as high-concentrated industries and exhibit less competition. To investigate the differential effect of higher MW on onsite total releases intensity given industry concentration, I estimate the following model:
    \begin{align}
        P_{f,cp,c,t}^{ind-conc} &= \beta (Treated^{e} \cdot D)_{f,s,t} + \psi (Treated^{e})_{s,t} + \vartheta (Treated \cdot D)_{f,s,t} + \mu (Post \cdot D)_{f,s,t} \nonumber \\
        &\quad + \tau Treated_{s,t} + \rho D_{f,s,t} + \alpha Post_{t} + \delta X_{v,c,t-1} + \omega F_{f,t} + \lambda_{t} + \gamma_{f} + \phi_{cp} \nonumber \\
        &\quad + \zeta_{c} + \eta_{c,t} + \theta_{cp,t} + \varepsilon_{f,cp,c,t},\label{eq:heterogeneous-onsite-releases-intensity-lcis}
    \end{align}
    where $P_{f,cp,c,t}^{ind-conc}$ is the vector of total onsite releases intensity (air, water and land) of toxic chemicals at a low-concentrated manufacturing industry facility, $f$ in a cross-border county pair, $cp$ through a toxic chemical, $c$ in the year, $t$. $Treated_{s,t}$ is a dummy that is equal to $1$ for the treated states, and $0$ the control states. $Post_{t}$ is a dummy that is equal to $1$ if the year $t$ is a post-treatment year, and $0$ otherwise. And $D_{f,s,t}$ is a dummy that is unity for a low-concentrated manufacturing industry facility, $f$ in state, $s$ in the year, $t$ and $0$ for facilities in high-concentrated industries.

    The parameter of interest here is the triple-differences parameter $\beta$ which measures the differential impact on onsite total releases intensity due to a higher MW policy for manufacturing facilities in low-concentrated industries. $\psi$ measures the relative change in onsite total releases intensity for manufacturing facilities in high-concentrated industries. And $\beta + \psi$ measures the total relative change in onsite total releases intensity for manufacturing facilities in low-concentrated industries. The results are presented in Table~\ref{tab:heterogeneous-onsite-releases-int-lcis}. Except for the significant differential increases in fugitive air emissions and surface water discharge intensities in the $2015$ cohort for low-concentrated or high competitive manufacturing facilities, the effect on the intensities of total toxic releases, air emissions from point sources, land releases, and surface impoundment are muted. I also find evidence of a declining surface water discharge intensity in the $2014$ cohort. Conversely, except for the significant increases in surface impoundment intensity for the high-concentrated or less-competitive manufacturing industries, there is limited evidence of a significant effect on the intensities of total toxic releases, air emissions from point and fugitive sources, surface water discharge, and land releases.
    % Please add the following required packages to your document preamble:
% \usepackage{booktabs}
% \usepackage{graphicx}
\begin{table}[H]
    \centering
    \caption{Onsite Releases Intensity given Industry Concentration}
    \label{tab:heterogeneous-onsite-releases-int-lcis}
    \resizebox{\columnwidth}{!}{%
        \begin{tabular}{@{}llllllll@{}}
            \toprule \toprule
            Onsite total releases intensity (log) & total     & air emissions & point air & fugitive air & water discharge & land releases & surface impoundment \\ \midrule
            $Treated^{e} \cdot D$                 & -0.040    & -0.025        & -0.082    & 0.054        & 0.017           & 0.028         & 0.000               \\
            & (0.068)   & (0.061)       & (0.062)   & (0.054)      & (0.027)         & (0.028)       & (0.004)             \\
            $Treated^{e}$                         & 0.085     & 0.054         & 0.045     & 0.031        & -0.001          & 0.016         & 0.009***            \\
            & (0.064)   & (0.060)       & (0.057)   & (0.050)      & (0.034)         & (0.011)       & (0.004)             \\
            cohort 2014 $\cdot D$                 & -0.103    & -0.044        & -0.092    & 0.015        & -0.073**        & -0.020        & 0.003               \\
            & (0.083)   & (0.082)       & (0.082)   & (0.067)      & (0.034)         & (0.037)       & (0.007)             \\
            cohort 2015 $\cdot D$                 & 0.030     & -0.003        & -0.072    & 0.097*       & 0.120**         & -0.037        & -0.003              \\
            & (0.088)   & (0.076)       & (0.073)   & (0.056)      & (0.050)         & (0.035)       & (0.003)             \\
            cohort 2017 $\cdot D$                 & 0.062     & 0.018         & -0.042    & 0.098        & 0.050           & 0.036         & -0.004              \\
            & (0.137)   & (0.108)       & (0.099)   & (0.104)      & (0.065)         & (0.034)       & (0.005)             \\
            controls                              & Yes       & Yes           & Yes       & Yes          & Yes             & Yes           & Yes                 \\
            year FE                               & Yes       & Yes           & Yes       & Yes          & Yes             & Yes           & Yes                 \\
            facility FE                           & Yes       & Yes           & Yes       & Yes          & Yes             & Yes           & Yes                 \\
            border-county FE                      & Yes       & Yes           & Yes       & Yes          & Yes             & Yes           & Yes                 \\
            toxic chemical FE                     & Yes       & Yes           & Yes       & Yes          & Yes             & Yes           & Yes                 \\
            toxic chemical LTs                    & Yes       & Yes           & Yes       & Yes          & Yes             & Yes           & Yes                 \\
            border-county LTs                     & Yes       & Yes           & Yes       & Yes          & Yes             & Yes           & Yes                 \\\midrule
            Observations                          & 1,893,689 & 1,893,689     & 1,893,689 & 1,893,689    & 1,893,689       & 1,893,689     & 1,893,689           \\
            $R^2$                                 & 0.728     & 0.746         & 0.719     & 0.670        & 0.595           & 0.507         & 0.159               \\ \bottomrule\bottomrule
        \end{tabular}%
    }
    \begin{minipage}{\columnwidth}
        \vspace{0.05in}
        \tiny NOTES: These results are obtained from estimating model~\ref{eq:heterogeneous-onsite-releases-intensity-lcis}. Three-way clustered robust standard errors are reported in parentheses, and clustered at the toxic chemical, industry and state levels. ***, **, and * denote significance levels at the less than $1\%$, $5\%$ and $10\%$, respectively.
    \end{minipage}
\end{table}
    \begin{figure}[H]
    \centering
    \includegraphics[width=1\textwidth, height=0.5\textheight,keepaspectratio]{fig_sdid_total_onsite_releases_int_lowindconc}
    \caption{Triple-Differences: Onsite Total Releases Intensity Industry Concentration}
    \label{fig:heterogeneous-onsite-releases-intensity-lcis}
    \begin{minipage}{18cm}
        \vspace{0.05in}
        \tiny NOTES: The event study model of equation~\ref{eq:heterogeneous-onsite-releases-intensity-lcis} is $G_{f,c,i,cp,s,t}^{ind-conc} = \sum_{{e = -3},{e \neq -1}}^{3} \beta (Treated^{e} \cdot D)_{f,s,t} + \psi (Treated^{e})_{s,t} + \vartheta (Treated \cdot D)_{f,s,t} + \mu (Post \cdot D)_{f,s,t} + \tau Treated_{s,t} + \rho D_{f,s,t} + \alpha Post_{t} + \delta X_{v,c,t-1} + \omega F_{f,t} + \lambda_{t} + \gamma_{f} + \phi_{cp} + \zeta_{c} + \eta_{c,t} + \theta_{cp,t} + \varepsilon_{f,cp,c,t}$. Three-way clustered robust standard errors are reported in parentheses, and clustered at the toxic chemical, industry and state levels.
    \end{minipage}
\end{figure}

    Similar patterns are observed in the dynamic effects presented in Figure~\ref{fig:heterogeneous-onsite-releases-intensity-lcis}. In high competitive industries, while point air emissions intensity is declining, fugitive air emission intensity and surface water discharge intensity is rising, usually between the first and second year of post-treatment. However, the effects on the intensities of total toxic releases, total air emissions, land releases, and surface impoundment are muted. In contrast, in less-competitive industries, the effects on the intensities of total toxic releases, air emissions from point sources, surface water discharge, and land releases are also muted, whereas, I document increases in fugitive air emission intensity and surface impoundment intensity. This effect is observed immediately and persists throughout the post-treatment years. There is no evidence of significant pre-trends.

    \subsection{Economic Growth Patterns}\label{subsec:economic-growth-patterns}
    In this subsection, I check whether the increasing onsite releases intensity due to a higher MW floor is peculiar to treated counties with high economic growth patterns. Theory suggests that economic growth patterns are correlated with pollutant emissions with a turning point at higher economic growth~\citep{grossman1995economic, shapiro2018pollution}. To investigate the differential effect of higher MW on onsite total releases intensity, I estimate the following model:
    \begin{align}
        P_{f,cp,c,t}^{gdp} &= \beta (Treated^{e} \cdot D)_{h,s,t} + \psi (Treated^{e})_{s,t} + \vartheta (Treated \cdot D)_{h,s,t} + \mu (Post \cdot D)_{h,s,t} \nonumber \\
        &\quad + \tau Treated_{s,t} + \rho D_{h,s,t} + \alpha Post_{t} + \delta X_{v,c,t-1} + \omega F_{f,t} + \lambda_{t} + \gamma_{f} + \phi_{cp} \nonumber \\
        &\quad+ \zeta_{c} + \eta_{c,t} + \theta_{cp,t} + \varepsilon_{f,cp,c,t},\label{eq:heterogeneous-onsite-releases-intensity-gdp}
    \end{align}
    where $P_{f,cp,c,t}^{gdp}$ is the vector of total onsite releases intensity (air e.g., point and fugitive, water and land) in a high GDP county at a manufacturing industry facility, $f$  in a cross-border county pair, $cp$ through a toxic chemical, $c$ in the year, $t$. $Treated_{s,t}$ is a dummy that is equal to $1$ for the treated states, and $0$ the control states. $Post_{t}$ is a dummy that is equal to $1$ if the year $t$ is a post-treatment year, and $0$ otherwise. And $D_{h,s,t}$ is a dummy that is unity for a high gross domestic product (GDP) of county, $h$ in state, $s$ in the year, $t$ and $0$ otherwise (i.e., low GDP). High GDP is defined as those counties with GDP above the median quantile of the GDP distribution of all counties in $2013$.

    The parameter of interest here is the triple-differences parameter $\beta$ which measures the differential impact on onsite total releases intensity due to a higher MW policy in high GDP counties. $\psi$ measures the differential change in onsite total releases intensity in low GDP counties. And $\beta + \psi$ measures the overall relative change in onsite total releases intensity in high GDP counties. The results are presented in Table~\ref{tab:heterogeneous-onsite-releases-int-gdp}. Notice that the $2017$ cohort is contained in the $\psi$ parameter, as their $2013$ GDP is less than the median. The triple-differences coefficient shows a negligible and insignificant effect on onsite total toxic releases intensity for high GDP counties. Similarly, I find a differential increase in the intensities of fugitive air emissions and surface water discharge in the $2017$ cohort. The effect on total air emissions intensity from point sources, and land releases intensity including surface impoundment are muted. In contrast, there is significant differential increase in total toxic release and surface impoundment intensities for low GDP counties, and limited evidence on air emissions intensity from point and fugitive sources, surface water discharge, and land releases intensities.
    % Please add the following required packages to your document preamble:
% \usepackage{booktabs}
% \usepackage{graphicx}
\begin{table}[H]
    \centering
    \caption{Onsite Releases Intensity given GDP Patterns}
    \label{tab:heterogeneous-onsite-releases-int-gdp}
    \resizebox{\columnwidth}{!}{%
        \begin{tabular}{@{}llllllll@{}}
            \toprule\toprule
            Onsite releases intensity (log) & total     & air emissions & point air & fugitive air & water discharge & land releases & surface impoundment \\ \midrule
            $Treated^{e} \cdot D$           & 0.037     & 0.094         & 0.087     & 0.036        & -0.036          & -0.009        & 0.005               \\
            & (0.082)   & (0.061)       & (0.059)   & (0.047)      & (0.042)         & (0.030)       & (0.004)             \\
            $Treated^{e}$                   & 0.050     & 0.024         & -0.027    & 0.034        & 0.012           & 0.003         & 0.012*              \\
            & (0.063)   & (0.057)       & (0.042)   & (0.056)      & (0.036)         & (0.010)       & (0.007)             \\
            cohort 2014 $\cdot D$           & -0.056    & 0.094         & 0.089     & 0.011        & -0.152          & 0.016         & 0.012**             \\
            & (0.144)   & (0.078)       & (0.064)   & (0.074)      & (0.099)         & (0.016)       & (0.006)             \\
            cohort 2015 $\cdot D$           & 0.187*    & 0.094         & 0.083     & 0.111*       & 0.148**         & -0.050        & -0.008              \\
            & (0.110)   & (0.067)       & (0.092)   & (0.062)      & (0.072)         & (0.061)       & (0.006)             \\
            controls                        & Yes       & Yes           & Yes       & Yes          & Yes             & Yes           & Yes                 \\
            year FE                         & Yes       & Yes           & Yes       & Yes          & Yes             & Yes           & Yes                 \\
            facility FE                     & Yes       & Yes           & Yes       & Yes          & Yes             & Yes           & Yes                 \\
            border-county FE                & Yes       & Yes           & Yes       & Yes          & Yes             & Yes           & Yes                 \\
            toxic chemical FE               & Yes       & Yes           & Yes       & Yes          & Yes             & Yes           & Yes                 \\
            toxic chemical LTs              & Yes       & Yes           & Yes       & Yes          & Yes             & Yes           & Yes                 \\\midrule
            Observations                    & 1,893,689 & 1,893,689     & 1,893,689 & 1,893,689    & 1,893,689       & 1,893,689     & 1,893,689           \\
            $R^2$                           & 0.720     & 0.739         & 0.712     & 0.660        & 0.586           & 0.500         & 0.127               \\ \bottomrule \bottomrule
        \end{tabular}%
    }
    \begin{minipage}{\columnwidth}
        \vspace{0.05in}
        \tiny NOTES: These results are obtained from estimating model~\ref{eq:heterogeneous-onsite-releases-intensity-gdp}. Three-way clustered robust standard errors are reported in parentheses, and clustered at the toxic chemical, industry and state levels. ***, **, and * denote significance levels at the less than $1\%$, $5\%$ and $10\%$, respectively.
    \end{minipage}
\end{table}
    \begin{figure}[H]
    \centering
    \includegraphics[width=1\textwidth, height=0.5\textheight,keepaspectratio]{fig_sdid_total_onsite_releases_int_GDP}
    \caption{Triple-Differences: Onsite Total Releases Intensity given Growth Patterns}
    \label{fig:heterogeneous-onsite-releases-intensity-gdp}
    \begin{minipage}{\columnwidth}
        \vspace{0.05in}
        \tiny NOTES: The event study model of equation~\ref{eq:heterogeneous-onsite-releases-intensity-gdp} is $G_{f,cp,c,t}^{gdp} = \sum_{{e = -3},{e \neq -1}}^{3} \beta (Treated^{e} \cdot D)_{f,s,t} + \psi (Treated^{e})_{s,t} + \vartheta (Treated \cdot D)_{f,s,t} + \mu (Post \cdot D)_{f,s,t} + \tau Treated_{s,t} + \rho D_{f,s,t} + \alpha Post_{t} + \delta X_{v,c,t-1} + \omega F_{f,t} + \lambda_{t} + \gamma_{f} + \phi_{cp} + \zeta_{c} + \eta_{c,t} + \theta_{cp,t} + \varepsilon_{f,cp,c,t}$. Three-way clustered robust standard errors are reported in parentheses, and clustered at the toxic chemical, industry and state levels.
    \end{minipage}
\end{figure}

    Figure~\ref{fig:heterogeneous-onsite-releases-intensity-gdp} illustrates dynamic trends across counties, showing limited evidence of a differential increase in total onsite releases intensity including both point and fugitive air emissions, surface water discharge and land releases and surface impoundment intensities, for high GDP counties. Conversely, low GDP counties witnessed a significant increase in total toxic releases intensity primarily driven by fugitive air emissions intensity. This effect became significant from the second year post-treatment. However, the effects on point air emissions intensity, surface water discharge, land releases, and surface impoundment intensities are muted for low GDP counties. There is no evidence of significant pre-trends.


    \section{Offsite and POTWs Toxic Release Intensities}\label{sec:offsite-and-potws-toxic-release-intensities}

    \subsection{Offsite Descriptive Statistics}\label{subsec:offsite-descriptive-statistics}
    \begin{table}[H]
    \centering
    \caption{Summary Statistics (Offsite)}
    \label{tab:sumstat-offsite}
    \begin{tabular}{lrrrrr}
        \toprule \toprule
        Variable                                     & Obs     & Mean   & StdDev  & Min & Max       \\ \midrule
        total releases intensity                     & 1179754 & 257.85 & 2453.53 & 0   & 125639.48 \\
        total land releases intensity                & 1179754 & 196.20 & 2037.73 & 0   & 125639.48 \\
        total land releases other intensity          & 1179754 & 1.24   & 22.82   & 0   & 1637.57   \\
        total landfills intensity                    & 1179754 & 172.99 & 1958.09 & 0   & 125639.48 \\
        total surface impoundment intensity          & 1179754 & 0.63   & 58.77   & 0   & 7517.19   \\
        total underground injection intensity        & 1179754 & 18.44  & 557.48  & 0   & 59894.46  \\
        total wastewater releases intensity          & 1179754 & 6.00   & 125.39  & 0   & 15101.70  \\
        total releases (metal solidify) intensity    & 1179754 & 61.22  & 1507.92 & 0   & 84868.80  \\
        total releases (storage) intensity           & 1179754 & 0.87   & 34.06   & 0   & 6755.87   \\
        total releases (other mgt) intensity         & 1179754 & 5.18   & 99.26   & 0   & 11850.66  \\
        total releases (to-land) treatment intensity & 1179754 & 2.91   & 90.79   & 0   & 7875.12   \\
        total releases (unknown) intensity           & 1179754 & 6.30   & 59.87   & 0   & 3610.69   \\
        total releases (waste broker) intensity      & 1179754 & 8.28   & 115.33  & 0   & 6738.71   \\ \bottomrule\bottomrule
    \end{tabular}
\end{table}

    \begin{table}[H]
    \centering
    \caption{Summary Statistics (POTWs)}
    \label{tab:sumstat-potws}
    \begin{tabular}{lrrrrr}
        \toprule\toprule
        Variable                               & Obs    & Mean  & StdDev & Min & Max      \\ \midrule
        total releases intensity               & 308943 & 17.87 & 404.45 & 0   & 27648.29 \\
        underground releases intensity         & 308943 & 6.69  & 288.78 & 0   & 27648.29 \\
        underground releases intensity (other) & 308943 & 11.18 & 253.74 & 0   & 26548.32 \\ \bottomrule \bottomrule
    \end{tabular}
\end{table}


    \subsection{Offsite Toxic Releases Intensities}\label{subsec:offsite-toxic-releases-intensities}
    Figure~\ref{fig:baseline-offsite-total-releases-intensity} reports the event studies for total offsite releases intensities including the intensities of others, unknown, waste broker, metal solidification, and storage releases to offsite facilities. Unknown sources are toxic releases reported in Form-R but with no defined chemical source, other toxic offsite releases are those reported in Form-R but not included in the environmental protection agency release pathways. It shows that total offsite releases intensities including other forms of toxic releases are increasing but with a two-year anticipation. Additionally, total release intensity meant for offsite storage appear to be declining with a one-year anticipation. There are no statistically significant effects on the intensities of releases to waste broker (intended for further waste management), metal solidification, and from unknown sources.
    \begin{figure}[H]
    \centering
    \includegraphics[width=1\textwidth, height=0.5\textheight,keepaspectratio]{fig_sdid_total_releases_offsite}
    \caption{Total Offsite Releases Intensity}
    \label{fig:baseline-offsite-total-releases-intensity}
    \begin{minipage}{12cm}
        \vspace{0.05in}
        NOTES: The event study model of equation~\ref{eq:baseline-offsite-total-releases-intensity} is $P_{f,c,i,cp,s,t}^{offsite} = \sum_{{e = 2011},{e \neq 2013}}^{2017} \beta Treated_{s,t}^e + \delta X_{v,c,t-1} + \omega F_{f,t} + \gamma_{f} + \phi_{cp} + \eta_{c,t} + \left[\lambda_{t} + \theta_{f,h} + \sigma_{s} + \zeta_{c} \right] + \varepsilon_{f,c,i,cp,s,t}$. Three-way clustered robust standard errors are reported in parentheses, and clustered at the toxic chemical, industry and state levels.
    \end{minipage}
\end{figure}

    Figure~\ref{fig:total-offsite-land-releases-intensity} reports the event studies for total offsite land releases intensity including others, industrial waste water, surface impoundment, to-land treatment, landfills, and underground injection. Whereas others, and surface impoundment show an instant significant positive difference with a two-year anticipation, landfills intensity declined in the second post-treatment period. There are no statistically significant difference on the intensities of total land releases, industrial waste water, to-land treatment, and underground injection.
    \begin{figure}[H]
    \centering
    \includegraphics[width=1\textwidth, height=0.5\textheight,keepaspectratio]{fig_sdid_total_land_releases_offsite}
    \caption{Total Offsite Total Releases Intensity}
    \label{fig:baseline-offsite-land-releases-intensity}
    \begin{minipage}{12cm}
        \vspace{0.05in}
        NOTES: The event study model of equation~\ref{eq:baseline-offsite-land-releases-intensity} is $L_{f,c,i,cp,s,t}^{offsite} = \sum_{{e = 2011},{e \neq 2013}}^{2017} \beta Treated_{s,t}^e + \delta X_{v,c,t-1} + \omega F_{f,t} + \gamma_{f} + \phi_{cp} + \eta_{c,t} + \left[\lambda_{t} + \theta_{f,h} + \sigma_{s} + \zeta_{c} \right] + \varepsilon_{f,c,i,cp,s,t}$. Three-way clustered robust standard errors are reported in parentheses, and clustered at the toxic chemical, industry and state levels.
    \end{minipage}
\end{figure}

    \subsection{Offsite: POTWs Toxic Releases Intensities}\label{subsec:potws-toxic-releases-intensities}
    Figure~\ref{fig:baseline-potws-total-releases-intensity} reports the event studies for the intensities of the total toxic releases, including underground and other underground releases, transferred to POTWs. Also, total offsite POTW treatment and waste management activities are reported. There is a significant positive difference in total offsite release intensities, one year later, driven by other underground release intensity. However, the total offsite treatment declined two years later, with a declining but insignificant waste management activities.
    \begin{figure}[H]
    \centering
    \includegraphics[width=1\textwidth, height=0.5\textheight,keepaspectratio]{C:/Users/david/OneDrive/Documents/ULMS/PhD/Thesis/chapter3/src/climate_change/latex/fig_sdid_total_releases_potws}
    \caption{Total POTWs Releases Intensity and Waste Management}
    \label{fig:baseline-potws-total-releases-intensity}
    \begin{minipage}{\columnwidth}
        \vspace{0.05in}
        \tiny NOTES: The event study model is $P_{f,cp,c,t}^{POTWs} = \sum_{{e = 2011},{e \neq 2013}}^{2017} \beta Treated_{s,t}^e + \delta X_{v,c,t-1} + \omega F_{f,t} + \lambda_{t} + \gamma_{f} + \phi_{cp} + \zeta_{c} + \eta_{c,t} + \theta_{cp,t} + \varepsilon_{f,cp,c,t}$. Three-way clustered robust standard errors are reported in parentheses, and clustered at the toxic chemical, industry and state levels.
    \end{minipage}
\end{figure}


    \section{State-Level Preliminary Analyses}\label{sec:state-level-preliminary-analyses}
    The state-level preliminary analyses begin in this section by examining the effect of raising MW on manufacturing industry costs (labour and materials), employment, production workers and hours, and outputs. To rule out any possible treatment selection, I estimate this state-level equation:
    \begin{equation}
        Treated_{s,t}^e = \beta Z_{f,sp,t} + \lambda_{t} + \phi_{sp} + \delta_{s} + \zeta_{sp,t} + \epsilon_{f,sp,t}\label{eq:state-treatment-selection}
    \end{equation}

    Where $Treated_{s,t}^e = 1[t - G_{s,t}]$ denotes treated states that are $e$-periods away from the initial treatment date, and $G_{s,t}$ is the vector of initial treatment dates. $Z_{f,sp,t}$ is the vector of facilities by state-pair covariates, and $\beta$ is the vector of coefficients. Albeit, the year fixed effects, $\lambda_{t}$, nets out any inflationary effects, city-region inflation is explicitly controlled for in the model. Cross-border state-pair, $\phi_{sp}$, and state, $\delta_{s}$, fixed effects are controlled for to account for within cross-border state-pair and state differences that may affect the MW policy. Finally, I control for cross-border state-pair linear trends, $\zeta_{sp,t}$, to account for time-varying common shocks affecting the evolution of the MW policy in paired cross-border states. The result (reported in Table~\ref{tab:state-treatment-selection}) showed no significant treatment selection effects in the following covariates: lagged values of county-level gross domestic product (GDP), GDP per capita, annual average establishments; inflation; average number of establishments, industry ownership, and coverage.
    % Please add the following required packages to your document preamble:
% \usepackage{booktabs}
% \usepackage{graphicx}
\begin{table}[H]
    \centering
    \caption{Treatment Selection}
    \label{tab:state-treatment-selection}
    \resizebox{\columnwidth}{!}{%
        \begin{tabular}{@{}llllll@{}}
            \toprule\toprule
            $Treated^{e}$                         & 1                 & 2                 & 3                & 4                & 5              \\ \midrule
            $GDP_{1}$                             & -0.000 (0.000)    & -0.000 (0.000)    & -0.000 (0.000)   & -0.000 (0.000)   & -0.000 (0.000) \\
            $GDPPC_{1}$                           & 0.001 (0.002)     & 0.001 (0.002)     & -0.000 (0.001)   & -0.000 (0.001)   & -0.000 (0.001) \\
            $Personal\_Income_{1}$                & 0.000 (0.000)     & 0.000 (0.000)     & 0.000 (0.000)    & 0.000 (0.000)    & -0.000 (0.000) \\
            $Average\_Number\_Establishments_{1}$ & -0.001 (0.001)    & -0.000 (0.001)    & 0.000 (0.000)    & 0.000 (0.000) & 0.000 (0.000)  \\
            $Inflation_{1}$                       & 0.010 (0.007)     & NA                & NA               & NA               & NA             \\
            Entire Facility (0,1)                 & -0.646*** (0.148) & -0.663*** (0.144) & -0.273** (0.082) & -0.222** (0.079) & -0.075 (0.071) \\
            Private Facility (0,1)                & 0.094 (0.060)     & 0.094 (0.059)     & -0.014** (0.005) & -0.005 (0.006)   & 0.006 (0.007)  \\
            Federal Facility (0,1)                & -0.086 (0.075)    & -0.087 (0.077)    & 0.006 (0.035)    & -0.003 (0.031)   & -0.041 (0.028) \\
            controls                              & Yes               & Yes               & Yes              & Yes              & Yes            \\
            year FE                               & No                & Yes               & Yes              & Yes              & Yes            \\
            state FE                              & No                & No                & Yes              & Yes              & Yes            \\
            border-state FE                       & No                & No                & No               & Yes              & Yes            \\
            border-state LTs                      & No                & No                & No               & No               & Yes            \\ \midrule
            Observations                          & 1,893,689         & 1,893,689         & 1,893,689        & 1,893,689        & 1,893,689      \\
            $R^2$                                 & 0.082             & 0.103             & 0.545            & 0.558            & 0.726          \\ \bottomrule \bottomrule
        \end{tabular}%
    }
    \begin{minipage}{\columnwidth}
        \vspace{0.05in}
        \tiny NOTES: Robust standard errors clustered at the state level are reported in parentheses. ***, **, and * denote significance levels at the less than $1\%$, $5\%$ and $10\%$, respectively.

    \end{minipage}
\end{table}

    \subsection{State-Level Baseline Results: Industry Costs}\label{subsec:state-level-baseline-results-industry-costs}
    In what follows, I estimate the state-level wage responses of manufacturing industry employees in the baseline. The baseline model is given by:
    \begin{equation}
        C_{i,sp,t} = \beta Treated_{s,t}^e + \delta X_{v,s,t-1} + \omega F_{f,t} + \lambda_{t} + \sigma_{s} + \phi_{sp} + \zeta_{sp,t} + \epsilon_{i,sp,t},\label{eq:state-baseline-wages}
    \end{equation}

    where $C_{i,sp,t}$ is the vector of industry costs (hourly wages, total payroll and material costs) of manufacturing industry, $i$ in cross-border state pairs, $sp$ in the year, $t$. $Treated_{s,t}^e = \textbf{1}[t - G_{s,t}]$ is unity for the treated states that are $e$-periods away from the vector of initial treatment dates, $G_{s,t}$ and zero for the control states. $X_{v,s,t-1}$ denotes lagged values of county-level GDP per capita, annual average establishments, and city-region inflation~\citep{gopalan2021state, dube2010minimum, clemens2019making}. $F_{f,t}$ contains facility dummies on industry ownership.

    I control for year fixed effects, $\lambda_{t}$ to account for time varying differences in the MW policy as well as trending inflation. Cross-border state-pair, $\phi_{sp}$ and state, $\sigma_{s}$ fixed effects are controlled for to account for within state-pair and state differences that may affect the MW policy such as within state industry compositions and political climate. Finally, $\zeta_{sp,t}$ is the cross-border state-pair linear trends to control for the evolution of common shocks in cross-border state-pairs. Standard errors are clustered at the state level as there are possibilities that changes in MW may be correlated within a state, and it is the level of the treatment.
    % Please add the following required packages to your document preamble:
% \usepackage{booktabs}
\begin{table}[H]
    \centering
    \caption{Effect of the MW Policy on Industry Costs}
    \label{tab:state-baseline-industry-costs}
    \begin{tabular}{@{}lllllll@{}}
        \toprule\toprule
        Industry costs & \multicolumn{2}{c}{Hourly wage} & \multicolumn{2}{c}{Total payroll (log)} & \multicolumn{2}{c}{Material cost (log)} \\
        \cmidrule(lr){2-3}\cmidrule(lr){4-5}\cmidrule(lr){6-7}
        & 1         & 2         & 3         & 4         & 5         & 6         \\ \midrule
        $Treated^{e}$    & 0.847**   & 0.704     & -0.007    & 0.040     & 0.036     & 0.081     \\
        & (0.362)   & (0.508)   & (0.062)   & (0.046)   & (0.123)   & (0.110)   \\
        cohort 2014      & 1.070**   & 0.802     & -0.010    & -0.012    & -0.011    & 0.057     \\
        & (0.516)   & (0.780)   & (0.058)   & (0.042)   & (0.159)   & (0.129)   \\
        cohort 2015      & 0.483*    & 0.566     & -0.001    & 0.130     & 0.113     & 0.124     \\
        & (0.277)   & (0.397)   & (0.129)   & (0.099)   & (0.193)   & (0.203)   \\
        cohort 2017      & 0.364     & -2.87***  & -0.140*** & -0.485*** & 0.348***  & -0.268*** \\
        & (0.401)   & (0.214)   & (0.038)   & (0.025)   & (0.057)   & (0.022)   \\
        controls         & Yes       & Yes       & Yes       & Yes       & Yes       & Yes       \\
        year FE          & Yes       & Yes       & Yes       & Yes       & Yes       & Yes       \\
        state FE         & Yes       & Yes       & Yes       & Yes       & Yes       & Yes       \\
        border-state FE  & No        & Yes       & No        & Yes       & No        & Yes       \\
        border-state LTs & No        & Yes       & No        & Yes       & No        & Yes       \\ \midrule
        Observations     & 1,893,689 & 1,893,689 & 1,893,689 & 1,893,689 & 1,893,689 & 1,893,689 \\
        $R^2$            & 0.296     & 0.393     & 0.162     & 0.254     & 0.309     & 0.396     \\
        Baseline Mean    & 26.56     & 26.56     & 2962.68   & 2962.68   & 61328.05  & 61328.05  \\ \bottomrule \bottomrule
    \end{tabular}
    \begin{minipage}{\columnwidth}
        \vspace{0.05in}
        \tiny NOTES: These results are obtained from estimating model~\ref{eq:baseline-wages}. Robust standard errors clustered at the state level are reported in parentheses. ***, **, and * denote significance levels at the less than $1\%$, $5\%$ and $10\%$, respectively.
    \end{minipage}
\end{table}

    The average treatment effect on the treated (ATT) is captured by $\beta$, which is the difference in the average effect of raising the MW floor on manufacturing industry costs in treated states relative to adjacent control states. The effects measured by $\beta$ are recovered using the~\citet{sun2021estimating} staggered difference-in-differences estimator, an improvement over the TWFE estimator.

    The results on industry costs are reported in Table~\ref{tab:state-baseline-industry-costs}. I find that a higher MW policy increases manufacturing industry wages (labour costs) in treated states relative to adjacent control states. Manufacturing industry wages per hour rose by $\$0.70$. This result is close to the county-level estimate, and still higher than that documented in~\citet{gopalan2021state} for the industry-wide effect on hourly wages of $\$0.48$. I further document an increase in total payroll in treated states relative to adjacent control states by $4.4$ percentage points (ppts), as well as an increase in cost of materials in manufacturing industries.

    The cohorts specific effects reveal that except for the $2017$ cohort where labour cost declined, it is positive for the $2014$ and $2015$ cohorts. Similarly, I find an increase in total payroll for the $2015$ cohort, and an increase in material cost for the $2014$ cohort. However, I find a declining effect in total payroll and material cost for the $2017$ cohort---Maine.
    \begin{figure}[H]
    \centering
    \includegraphics[width=1\textwidth,keepaspectratio]{C:/Users/david/OneDrive/Documents/ULMS/PhD/Thesis/chapter3/src/climate_change/latex/fig_sdid_industry_costs_state}
    \caption{Manufacturing Industry Costs}
    \label{fig:state-baseline-manufacturing-industry-costs}
    \begin{minipage}{\columnwidth}
        \vspace{0.05in}
        \tiny NOTES: The event study model of equation~\ref{eq:baseline-wages} is $C_{i,sp,t} = \sum_{{e = -3},{e \neq -1}}^{3} \beta Treated_{s,t}^e = \textbf{1}[t - G_{s,t}] + \delta X_{v,s,t-1} + \omega F_{f,t} + \lambda_{t} + \sigma_{s} + \phi_{sp} + \zeta_{sp,t} + \epsilon_{i,sp,t}$. Standard errors are clustered at the state level. de Chaisemartin and D'Haultfoeuille Decomposition: $\sum dCDH_{ATTs}^{weights(+)} = 1$ and $\sum dCDH_{ATTs}^{weights(-)} = 0$.
    \end{minipage}
\end{figure}
    \begin{figure}[H]
    \centering
    \includegraphics[width=1\textwidth,keepaspectratio]{C:/Users/david/OneDrive/Documents/ULMS/PhD/Thesis/chapter3/src/climate_change/latex/fig_sdid_industry_cost_heter_state}
    \caption{Heterogeneous Effects of the MW Policy on Manufacturing Industry Costs}
    \label{fig:baseline-manufacturing-industry-cost-heter}
    \begin{minipage}{\columnwidth}
        \vspace{0.05in}
        \tiny NOTES: The event study model of equation~\ref{eq:baseline-wages} is $C_{i,sp,t} = \sum_{{e = -3},{e \neq -1}}^{3} \beta (Treated^{e} \cdot D)_{i,s,t} + \psi (Treated^{e})_{s,t} + \vartheta (Treated \cdot D)_{i,s,t} + \mu (Post \cdot D)_{i,s,t} + \tau Treated_{s,t} + \rho D_{i,s,t} + \alpha Post_{t} + \delta X_{v,s,t-1} + \omega F_{f,t} + \lambda_{t} + \sigma_{s} + \phi_{sp} + \zeta_{sp,t} + \epsilon_{i,sp,t}$. Standard errors are clustered at the state level. $D_{i,s,t}$ is unity for low-skilled and zero for high-skilled workers; $Treated_{s,t}$ is unity for treated and zero for control states; and $Post_{t}$ is unity for post-treatment and zero for pre-treatment periods.
    \end{minipage}
\end{figure}

    Due to loss of statistical power at the state-level, the dynamic effects appear to be insignificant, however, they are consistent with the direction of impact in Figure~\ref{fig:baseline-manufacturing-industry-costs}. It shows increases in hourly wages, total payrolls, and cost of materials in the post-treatment periods.

    \subsection{State-Level Baseline Results: Employment and Hours}\label{subsec:state-level-baseline-results-employment-and-hours}
    This subsection estimates the state-level effect of raising MW on employment of manufacturing industry workers and their production hours. The model is given by:
    \begin{equation}
        E_{i,sp,t} = \beta Treated_{s,t}^e + \delta X_{v,s,t-1} + \omega F_{f,t} + \lambda_{t} + \sigma_{s} + \phi_{sp} + \zeta_{sp,t} + \epsilon_{i,sp,t},\label{eq:state-baseline-emp-hours}
    \end{equation}
    where $E_{i,sp,t}$ is the vector of employment, total production workers, and hours in manufacturing industry, $i$ in cross-border state pairs, $sp$ in the year, $t$. Standard errors are clustered at the state level.
    % Please add the following required packages to your document preamble:
% \usepackage{booktabs}
% \usepackage{graphicx}
\begin{table}[H]
    \centering
    \caption{Effect of the MW Policy on Employment and Production Workers' Hours}
    \label{tab:state-baseline-employment-hours}
    \resizebox{\columnwidth}{!}{%
        \begin{tabular}{@{}lllllll@{}}
            \toprule\toprule
            & \multicolumn{2}{c}{Employment (log)} & \multicolumn{2}{c}{Production Workers (log)} & \multicolumn{2}{c}{Production Hours (log)} \\
            \cmidrule(lr){2-3} \cmidrule(lr){4-5} \cmidrule(lr){6-7}
            employment \& hours & 1         & 2         & 3         & 4         & 5         & 6         \\ \midrule
            treated             & -0.053    & -0.002    & -0.084    & -0.029    & -0.082    & -0.027    \\
            & (0.064)   & (0.039)   & (0.066)   & (0.039)   & (0.066)   & (0.039)   \\
            cohort 2014         & -0.060    & -0.058**  & -0.101    & -0.095*** & -0.101    & -0.095*** \\
            & (0.058)   & (0.026)   & (0.064)   & (0.028)   & (0.067)   & (0.029)   \\
            cohort 2015         & -0.043    & 0.094     & -0.058    & 0.083     & -0.052    & 0.090     \\
            & (0.134)   & (0.094)   & (0.131)   & (0.091)   & (0.127)   & (0.089)   \\
            cohort 2017         & 0.106***  & -0.366*** & 0.114***  & -0.318*** & 0.136***  & -0.289*** \\
            & (0.027)   & (0.031)   & (0.023)   & (0.034)   & (0.025)   & (0.034)   \\
            controls            & Yes       & Yes       & Yes       & Yes       & Yes       & Yes       \\
            year FE             & Yes       & Yes       & Yes       & Yes       & Yes       & Yes       \\
            state FE            & Yes       & Yes       & Yes       & Yes       & Yes       & Yes       \\
            border-state FE     & No        & Yes       & No        & Yes       & No        & Yes       \\
            border-state LTs    & No        & Yes       & No        & Yes       & No        & Yes       \\ \midrule
            Observations        & 1,893,689 & 1,893,689 & 1,893,689 & 1,893,689 & 1,893,689 & 1,893,689 \\
            $R^2$               & 0.136     & 0.211     & 0.123     & 0.195     & 0.124     & 0.199     \\
            Baseline Mean       & 44.99     & 44.99     & 31.42     & 31.42     & 64.82     & 64.82     \\ \bottomrule \bottomrule
        \end{tabular}%
    }
    \begin{minipage}{\columnwidth}
        \vspace{0.05in}
        \tiny NOTES: These results are obtained from estimating model~\ref{eq:baseline-emp-hours}. Robust standard errors clustered at the state level are reported in parentheses. ***, **, and * denote significance levels at the less than $1\%$, $5\%$ and $10\%$, respectively.
    \end{minipage}
\end{table}

    The average treatment effect on the treated (ATT) is captured by $\beta$, which is the difference in the average effect of raising the MW floor on manufacturing industry employment, production workers, and hours in treated states relative to adjacent control states.

    The results are reported in Table~\ref{tab:state-baseline-employment-hours}. It reports results consistent with the county-level. It shows no significant changes in overall manufacturing industry employment, total production workers, and hours following an MW policy in the treated states relative to adjacent control states. Particularly, the size of the effect suggests a sharp null effect of the MW policy on manufacturing industry employment including production workers and hours, after accounting for time-varying common shocks to border states. These results are consistent with the labour market literature assuming monopsonistic competition~\citep{card2000minimum, aaronson2018industry, cengiz2019effect, wong2019minimum, dustmann2022reallocation}.

    However, I find heterogeneity in the cohort-specific effects. While there are disemployment effects in the $2014$ and $2017$ cohorts, there is a positive effect on employment in the $2015$ cohort. These disemployment effects as shown in Figure~\ref{fig:baseline-manufacturing-industry-employment-heter} are dominated in low-skilled workers, low-profit and labour-intensive industries, and driven by the declining number of production workers and hours due to higher labour costs. These disemployment effects explain the decline in total payroll and material cost in the $2017$ cohort in subsection~\ref{subsec:state-level-baseline-results-industry-costs}. On the other hand, the positive employment effect is dominated in high-skilled workers, high-profit and capital-intensive industries.

    Similarly, the dynamic treatment effects in Figure~\ref{fig:state-baseline-employment-hours} show an insignificant positive post-treatment increases in manufacturing industry employment (including production workers and hours) followed by a decline in the third year.~\citet{neumark2019econometrics} argues that cross-border studies may be biased against detecting disemployment effects due to worker mobility spillovers. To test if my results are influenced by cross-state mobility, potentially violating the stable unit treatment assumption, I use the interaction of the staggered difference-in-differences coefficient with the distance between population centers. The principle here is that worker mobility to states/states with higher MW decreases as geographic distance to that state/state increases. Table~\ref{tab:state-baseline-cross-county-state-mobility} indicates that worker mobility does not affect the overall baseline results, including the positive employment effect specific to the $2015$ cohort. However, the disemployment effects observed for the $2014$ and $2017$ cohorts are explained by workers' unwillingness to commute to distant states with higher MW. This supports the hypothesis that worker mobility decreases with increasing distance to higher MW regions. This indicates that the higher MW policy heightens the reluctance of low-skilled manufacturing industry workers to commute to regions with higher MW, as the demand for high-skilled manufacturing industry workers increase. There is no evidence of pre-trends.
    \begin{figure}[H]
    \centering
    \includegraphics[width=1\textwidth, keepaspectratio]{C:/Users/david/OneDrive/Documents/ULMS/PhD/Thesis/chapter3/src/climate_change/latex/fig_sdid_emp_hours_state}
    \caption{Industry Employment, Production Workers and Hours}
    \label{fig:state-baseline-employment-hours}
    \begin{minipage}{\columnwidth}
        \vspace{0.05in}
        \tiny NOTES: The event study model of equation~\ref{eq:baseline-emp-hours} is $E_{i,sp,t} = \sum_{{e = -3},{e \neq -1}}^{3} \beta Treated_{s,t}^e = \textbf{1}[t - G_{s,t}] + \delta X_{v,s,t-1} + \omega F_{f,t} + \lambda_{t} + \sigma_{s} + \phi_{sp} + \zeta_{sp,t} + \epsilon_{i,sp,t}$. Standard errors are clustered at the state level. de Chaisemartin and D'Haultfoeuille Decomposition: $\sum dCDH_{ATTs}^{weights(+)} = 1$ and $\sum dCDH_{ATTs}^{weights(-)} = 0$.
    \end{minipage}
\end{figure}
    \begin{figure}[H]
    \centering
    \includegraphics[width=1\textwidth,keepaspectratio]{C:/Users/david/OneDrive/Documents/ULMS/PhD/Thesis/chapter3/src/climate_change/latex/fig_sdid_emp_hours_heter_state}
    \caption{Heterogeneous Effects of the MW Policy on Employment and Hours}
    \label{fig:baseline-manufacturing-industry-employment-heter}
    \begin{minipage}{\columnwidth}
        \vspace{0.05in}
        \tiny NOTES: The event study model of equation~\ref{eq:baseline-wages} is $E_{i,sp,t} = \sum_{{e = -3},{e \neq -1}}^{3} \beta (Treated^{e} \cdot D)_{i,s,t} + \psi (Treated^{e})_{s,t} + \vartheta (Treated \cdot D)_{i,s,t} + \mu (Post \cdot D)_{i,s,t} + \tau Treated_{s,t} + \rho D_{i,s,t} + \alpha Post_{t} + \delta X_{v,s,t-1} + \omega F_{f,t} + \lambda_{t} + \sigma_{s} + \phi_{sp} + \zeta_{sp,t} + \epsilon_{i,sp,t}$. Standard errors are clustered at the state level. $D_{i,s,t}$ is unity for low-skilled and zero for high-skilled workers; $Treated_{s,t}$ is unity for treated and zero for control states; and $Post_{t}$ is unity for post-treatment and zero for pre-treatment periods.
    \end{minipage}
\end{figure}

    \subsection{State-Level Baseline Results: Industrial Output}\label{subsec:state-level-baseline-results-industrial-output}
    This subsection estimates the effect of raising MW on outputs of the manufacturing industry. The model is given by:
    \begin{equation}
        Y_{i,sp,t} = \beta Treated_{s,t}^e + \delta X_{v,s,t-1} + \omega F_{f,t} + \lambda_{t} + \sigma_{s} + \phi_{sp} + \zeta_{sp,t} + \epsilon_{i,sp,t},\label{eq:state-baseline-output}
    \end{equation}
    where $Y_{i,sp,t}$ is the vector of manufacturing industry output, output per hour and output per worker (labour productivity), in manufacturing industry, $i$ in cross-border state pairs, $sp$ in the year, $t$. Standard errors are clustered at the state level.
    % Please add the following required packages to your document preamble:
% \usepackage{booktabs}
% \usepackage{graphicx}
\begin{table}[H]
    \centering
    \caption{Effect of the MW policy on Manufacturing Industry Output}
    \label{tab:baseline-industry-output}
    \resizebox{\columnwidth}{!}{%
        \begin{tabular}{@{}lllllll@{}}
            \toprule\toprule
            Industry outputs (log) & \multicolumn{2}{c}{Output} & \multicolumn{2}{c}{Output per Hour} & \multicolumn{2}{c}{Output per Worker} \\
            \cmidrule(lr){2-3} \cmidrule(lr){4-5} \cmidrule(lr){6-7}
            & 1         & 2         & 3         & 4         & 5         & 6         \\ \midrule
            treated          & 0.032     & 0.130*    & 0.114     & 0.157**   & 0.085     & 0.132**   \\
            & (0.103)   & (0.069)   & (0.084)   & (0.058)   & (0.078)   & (0.051)   \\
            cohort 2014      & -0.007    & 0.077     & 0.094     & 0.173*    & 0.053     & 0.135     \\
            & (0.140)   & (0.086)   & (0.132)   & (0.092)   & (0.124)   & (0.080)   \\
            cohort 2015      & 0.095     & 0.221*    & 0.147***  & 0.132***  & 0.138***  & 0.128***  \\
            & (0.141)   & (0.119)   & (0.037)   & (0.041)   & (0.034)   & (0.038)   \\
            cohort 2017      & 0.200***  & -0.467*** & 0.064     & -0.179*** & 0.094**   & -0.102*** \\
            & (0.063)   & (0.020)   & (0.053)   & (0.019)   & (0.044)   & (0.017)   \\
            controls         & Yes       & Yes       & Yes       & Yes       & Yes       & Yes       \\
            year FE          & Yes       & Yes       & Yes       & Yes       & Yes       & Yes       \\
            state FE         & Yes       & Yes       & Yes       & Yes       & Yes       & Yes       \\
            border-state FE  & No        & Yes       & No        & Yes       & No        & Yes       \\
            border-state LTs & No        & Yes       & No        & Yes       & No        & Yes       \\ \midrule
            Observations     & 1,893,689 & 1,893,689 & 1,893,689 & 1,893,689 & 1,893,689 & 1,893,689 \\
            $R^2$            & 0.245     & 0.353     & 0.327     & 0.419     & 0.338     & 0.433     \\
            Baseline Mean    & 177.13    & 177.13    & 2.62      & 2.62      & 3.65      & 3.65      \\ \bottomrule\bottomrule
        \end{tabular}%
    }
    \begin{minipage}{\columnwidth}
        \vspace{0.05in}
        \tiny NOTES: These results are obtained from estimating model~\ref{eq:baseline-output}. Robust standard errors clustered at the state level are reported in parentheses. ***, **, and * denote significance levels at the less than $1\%$, $5\%$ and $10\%$, respectively.
    \end{minipage}
\end{table}

    The average treatment effect on the treated (ATT) is captured by $\beta$, which is the difference in the average effect of raising the MW floor on manufacturing industry output, output per hour, and output per worker in treated states relative to adjacent control states.

    The results are reported in Table~\ref{tab:state-baseline-industry-output}. Following an MW policy, I still document large statistically significant increases in manufacturing industry outputs by $13ppts$ in treated states for the $2014$ and $2015$ cohorts. Additionally, the output per hour and labour productivity rose by $14.4$ and $12.7$ (ppts), respectively. Thus, suggesting that for every $\$0.70/hr$ increase in wages, output per hour and labour productivity rise by those margins. Back of the envelop calculation shows that manufacturing industry output per hour and labour productivity increases by an additional $0.157 \cdot 2.62 \cdot \left(\frac{\$100m}{\$1m}\right) = \$41.13$ units per hour and $0.132 \cdot 3.65 \cdot \left(\frac{\$100m}{\$1000}\right) = \$48,180$ units per worker, respectively. The rising output is dominated amongst high-skilled workers, while labour productivity is dominated amongst low-profit industries.

    The cohort-specific effects reveal significant and substantially strong increases in total output, output per hour and output per worker across all cohorts, except for the $2017$ cohort. The decline in overall output for the $2017$ cohort is still explained by the decline in employment and number of production hours in labour-intensive industries since low-skilled workers are less likely to commute to distant higher MW states. The state-level results are marginally largely than the county-level.

    Figure~\ref{fig:state-baseline-industry-output} records consistent results. The MW policy caused instantaneous significant increases in manufacturing industry outputs, output per hour, and output per worker in treated states relative to adjacent control states. The effects on output per hour and output per worker persists throughout the spectrum. Importantly, there is no evidence of significant pre-trends.
    \begin{figure}[H]
    \centering
    \includegraphics[width=1\textwidth, keepaspectratio]{C:/Users/david/OneDrive/Documents/ULMS/PhD/Thesis/chapter3/src/climate_change/latex/fig_sdid_output_state}
    \caption{Manufacturing Industry Output: Output per Hour and Output per Worker}
    \label{fig:baseline-industry-output}
    \begin{minipage}{\columnwidth}
        \vspace{0.05in}
        \tiny NOTES: The event study model of equation~\ref{eq:baseline-wages} is $Y_{i,sp,t} = \sum_{{e = -3},{e \neq -1}}^{3} \beta Treated_{s,t}^e = \textbf{1}[t - G_{s,t}] + \delta X_{v,s,t-1} + \omega F_{f,t} + \lambda_{t} + \sigma_{s} + \phi_{sp} + \zeta_{sp,t} + \epsilon_{i,sp,t}$. Standard errors are clustered at the state level. de Chaisemartin and D'Haultfoeuille Decomposition: $\sum dCDH_{ATTs}^{weights(+)} = 1$ and $\sum dCDH_{ATTs}^{weights(-)} = 0$.
    \end{minipage}
\end{figure}
    \begin{figure}[H]
    \centering
    \includegraphics[width=1\textwidth,keepaspectratio]{C:/Users/david/OneDrive/Documents/ULMS/PhD/Thesis/chapter3/src/climate_change/latex/fig_sdid_output_heter_state}
    \caption{Heterogeneous Effects of the MW Policy on Outputs}
    \label{fig:state-baseline-manufacturing-industry-output-heter}
    \begin{minipage}{\columnwidth}
        \vspace{0.05in}
        \tiny NOTES: The event study model of equation~\ref{eq:baseline-wages} is $Y_{i,sp,t} = \sum_{{e = -3},{e \neq -1}}^{3} \beta (Treated^{e} \cdot D)_{i,s,t} + \psi (Treated^{e})_{s,t} + \vartheta (Treated \cdot D)_{i,s,t} + \mu (Post \cdot D)_{i,s,t} + \tau Treated_{s,t} + \rho D_{i,s,t} + \alpha Post_{t} + \delta X_{v,s,t-1} + \omega F_{f,t} + \lambda_{t} + \sigma_{s} + \phi_{sp} + \zeta_{sp,t} + \epsilon_{i,sp,t}$. Standard errors are clustered at the state level. $D_{i,s,t}$ is unity for low-skilled and zero for high-skilled workers; $Treated_{s,t}$ is unity for treated and zero for control states; and $Post_{t}$ is unity for post-treatment and zero for pre-treatment periods.
    \end{minipage}
\end{figure}

    \subsection{State-Level Baseline Results: Industry Profits}\label{subsec:state-level-baseline-results-industry-profits}
    Figure~\ref{fig:state-baseline-manufacturing-industry-profits} records consistent results with the county-level. Higher MW policies leads to higher profits for the manufacturing industry. Essential, this is driven by higher output levels amid higher labour costs and labour productivity.
    \begin{figure}[H]
    \centering
    \includegraphics[width=1\textwidth,keepaspectratio]{C:/Users/david/OneDrive/Documents/ULMS/PhD/Thesis/chapter3/src/climate_change/latex/fig_sdid_profits_state}
    \caption{Manufacturing Industry Profits}
    \label{fig:baseline-manufacturing-industry-profits}
    \begin{minipage}{\columnwidth}
        \vspace{0.05in}
        \tiny NOTES: The event study model of equation~\ref{eq:baseline-wages} is $R_{i,sp,t} = \sum_{{e = -3},{e \neq -1}}^{3} \beta Treated_{s,t}^e = \textbf{1}[t - G_{s,t}] + \delta X_{v,s,t-1} + \omega F_{f,t} + \lambda_{t} + \sigma_{s} + \phi_{sp} + \zeta_{sp,t} + \epsilon_{i,sp,t}$; where $R_{i,sp,t}$ is the profit and margin vector. Standard errors are clustered at the state level. de Chaisemartin and D'Haultfoeuille Decomposition: $\sum dCDH_{ATTs}^{weights(+)} = 1$ and $\sum dCDH_{ATTs}^{weights(-)} = 0$.
    \end{minipage}
\end{figure}

    \subsection{State-Level Baseline Robustness}\label{subsec:state-level-baseline-robustness}
    I conduct several robustness exercises to confirm the robustness of the baseline industry results presented above.

    \subsubsection{State-Level Standard Errors} Standard errors are clustered at the facility, zipcode, industry NAICS codes, and state levels. I document that the state-level results are not sensitive to these alternative clustering at the facility and NAICS levels. See Tables~\ref{tab:state-baseline-cost-robustness},~\ref{tab:state-baseline-employ-robustness} and~\ref{tab:state-baseline-output-robustness}.
    % Please add the following required packages to your document preamble:
% \usepackage{booktabs}
% \usepackage{graphicx}
\begin{table}[H]
    \centering
    \caption{Manufacturing Industry Costs: Alternative Clustering of the SEs}
    \label{tab:baseline-cost-robustness}
    \resizebox{\columnwidth}{!}{%
        \begin{tabular}{@{}lllllll@{}}
            \toprule  \toprule
            & \multicolumn{2}{c}{Hourly Wage} & \multicolumn{2}{c}{Total Payroll (log)} & \multicolumn{2}{c}{Material Cost (log)} \\
            \cmidrule(lr){2-3} \cmidrule(lr){4-5} \cmidrule(lr){6-7}
            Dependent Var.    & 1         & 2         & 3         & 4         & 5         & 6         \\ \midrule
            $Treated^{e}$     & 0.704     & 0.704     & 0.040     & 0.040     & 0.081     & 0.081     \\
            & (0.560)   & (0.513)   & (0.060)   & (0.055)   & (0.115)   & (0.100)   \\
            cohort 2014       & 0.802     & 0.802     & -0.012    & -0.012    & 0.057     & 0.057     \\
            & (0.812)   & (0.701)   & (0.068)   & (0.062)   & (0.154)   & (0.125)   \\
            cohort 2015       & 0.566     & 0.566     & 0.130     & 0.130     & 0.124     & 0.124     \\
            & (0.635)   & (0.654)   & (0.113)   & (0.116)   & (0.167)   & (0.166)   \\
            cohort 2017       & -2.87*    & -2.87*    & -0.485    & -0.485    & -0.268    & -0.268    \\
            & (1.68)    & (1.73)    & (0.318)   & (0.320)   & (0.259)   & (0.254)   \\
            controls          & Yes       & Yes       & Yes       & Yes       & Yes       & Yes       \\
            year FE           & Yes       & Yes       & Yes       & Yes       & Yes       & Yes       \\
            state FE          & Yes       & Yes       & Yes       & Yes       & Yes       & Yes       \\
            border-state FE   & Yes       & Yes       & Yes       & Yes       & Yes       & Yes       \\
            border-state LTs  & Yes       & Yes       & Yes       & Yes       & Yes       & Yes       \\
            clustered at the: & facility  & industry  & facility  & industry  & facility  & industry  \\ \midrule
            Observations      & 1,893,689 & 1,893,689 & 1,893,689 & 1,893,689 & 1,893,689 & 1,893,689 \\
            $R^2$             & 0.393     & 0.393     & 0.254     & 0.254     & 0.396     & 0.396     \\ \bottomrule \bottomrule
        \end{tabular}%
    }
    \begin{minipage}{\columnwidth}
        \vspace{0.05in}
        \tiny NOTES: These results are obtained from estimating model~\ref{eq:baseline-emp-hours}. Robust standard errors clustered at the state level are reported in parentheses. ***, **, and * denote significance levels at the less than $1\%$, $5\%$ and $10\%$, respectively.
    \end{minipage}
\end{table}
    % Please add the following required packages to your document preamble:
% \usepackage{booktabs}
% \usepackage{graphicx}
\begin{table}[H]
    \centering
    \caption{Employment and Hours: Alternative Clustering of the SEs}
    \label{tab:state-baseline-employ-robustness}
    \resizebox{\columnwidth}{!}{%
        \begin{tabular}{@{}lllllll@{}}
            \toprule\toprule
            & \multicolumn{2}{c}{Employment} & \multicolumn{2}{c}{Production Workers} & \multicolumn{2}{c}{Production Hours} \\
            \cmidrule(lr){2-3} \cmidrule(lr){4-5} \cmidrule(lr){6-7}
            Dependent Var.    & 1         & 2         & 3         & 4         & 5         & 6         \\ \midrule
            $Treated^{e}$     & -0.002    & -0.002    & -0.029    & -0.029    & -0.027    & -0.027    \\
            & (0.053)   & (0.050)   & (0.053)   & (0.049)   & (0.054)   & (0.049)   \\
            cohort 2014       & -0.058    & -0.058    & -0.095*   & -0.095*   & -0.095*   & -0.095*   \\
            & (0.058)   & (0.054)   & (0.059)   & (0.056)   & (0.060)   & (0.056)   \\
            cohort 2015       & 0.094     & 0.094     & 0.083     & 0.083     & 0.090     & 0.090     \\
            & (0.105)   & (0.108)   & (0.105)   & (0.106)   & (0.105)   & (0.106)   \\
            cohort 2017       & -0.366    & -0.366    & -0.318    & -0.318    & -0.289    & -0.289    \\
            & (0.340)   & (0.342)   & (0.366)   & (0.366)   & (0.358)   & (0.357)   \\
            controls          & Yes       & Yes       & Yes       & Yes       & Yes       & Yes       \\
            year FE           & Yes       & Yes       & Yes       & Yes       & Yes       & Yes       \\
            state FE          & Yes       & Yes       & Yes       & Yes       & Yes       & Yes       \\
            border-state FE   & Yes       & Yes       & Yes       & Yes       & Yes       & Yes       \\
            border-state LTs  & Yes       & Yes       & Yes       & Yes       & Yes       & Yes       \\ \midrule
            clustered at the: & facility  & industry  & facility  & industry  & facility  & industry  \\
            Observations      & 1,893,689 & 1,893,689 & 1,893,689 & 1,893,689 & 1,893,689 & 1,893,689 \\
            $R^2$             & 0.211     & 0.211     & 0.195     & 0.195     & 0.199     & 0.199     \\ \bottomrule \bottomrule
        \end{tabular}%
    }
    \begin{minipage}{\columnwidth}
        \vspace{0.05in}
        \tiny NOTES: These results are obtained from estimating model~\ref{eq:baseline-emp-hours}. Robust standard errors clustered at the state level are reported in parentheses. ***, **, and * denote significance levels at the less than $1\%$, $5\%$ and $10\%$, respectively.
    \end{minipage}
\end{table}
    % Please add the following required packages to your document preamble:
% \usepackage{booktabs}
% \usepackage{graphicx}
\begin{table}[H]
    \centering
    \caption{Output and Labour Productivity: Alternative Clustering of the SEs}
    \label{tab:state-baseline-output-robustness}
    \resizebox{\columnwidth}{!}{%
        \begin{tabular}{@{}lllllll@{}}
            \toprule\toprule
            Output (log) & \multicolumn{2}{c}{Output} & \multicolumn{2}{c}{Output per Hour} & \multicolumn{2}{c}{Output per Worker} \\
            \cmidrule(lr){2-3} \cmidrule(lr){4-5} \cmidrule(lr){6-7}
            Dependent Var.    & 1         & 2         & 3         & 4         & 5         & 6         \\ \midrule
            $Treated^{e}$     & 0.130*    & 0.130*    & 0.157***  & 0.157***  & 0.132**   & 0.132***  \\
            & (0.080)   & (0.075)   & (0.058)   & (0.058)   & (0.053)   & (0.050)   \\
            cohort 2014       & 0.077     & 0.077     & 0.173**   & 0.173**   & 0.135*    & 0.135*    \\
            & (0.104)   & (0.092)   & (0.086)   & (0.080)   & (0.080)   & (0.072)   \\
            cohort 2015       & 0.221*    & 0.221*    & 0.132**   & 0.132**   & 0.128**   & 0.128**   \\
            & (0.125)   & (0.133)   & (0.060)   & (0.066)   & (0.052)   & (0.058)   \\
            cohort 2017       & -0.467*   & -0.467*   & -0.179    & -0.179    & -0.102    & -0.102    \\
            & (0.279)   & (0.281)   & (0.197)   & (0.196)   & (0.160)   & (0.158)   \\
            controls          & Yes       & Yes       & Yes       & Yes       & Yes       & Yes       \\
            year FE           & Yes       & Yes       & Yes       & Yes       & Yes       & Yes       \\
            state FE          & Yes       & Yes       & Yes       & Yes       & Yes       & Yes       \\
            border-state FE   & Yes       & Yes       & Yes       & Yes       & Yes       & Yes       \\
            border-state LTs  & Yes       & Yes       & Yes       & Yes       & Yes       & Yes       \\ \midrule
            clustered at the: & facility  & industry  & facility  & industry  & facility  & industry  \\
            Observations      & 1,893,689 & 1,893,689 & 1,893,689 & 1,893,689 & 1,893,689 & 1,893,689 \\
            $R^2$             & 0.353     & 0.353     & 0.419     & 0.419     & 0.433     & 0.433     \\ \bottomrule \bottomrule
        \end{tabular}%
    }
    \begin{minipage}{\columnwidth}
        \vspace{0.05in}
        \tiny NOTES: These results are obtained from estimating model~\ref{eq:baseline-output}. Robust standard errors clustered at the state level are reported in parentheses. ***, **, and * denote significance levels at the less than $1\%$, $5\%$ and $10\%$, respectively.
    \end{minipage}
\end{table}

    \subsubsection{Cross-State Worker Mobility}\label{subsubsec:cross-state-worker-mobility}
    The results here rejects the hypothesis of cross-worker state mobility.
    % Please add the following required packages to your document preamble:
% \usepackage{booktabs}
% \usepackage{graphicx}
\begin{table}[H]
    \centering
    \caption{Potential Cross County/State Mobility}
    \label{tab:baseline-cross-county-state-mobility}
    \resizebox{\columnwidth}{!}{%
        \begin{tabular}{@{}lllllll@{}}
            \toprule\toprule
            & \multicolumn{2}{c}{Employment (log)} & \multicolumn{2}{c}{Production Workers (log)} & \multicolumn{2}{c}{Production Hours (log)} \\
            \cmidrule(lr){2-3} \cmidrule(lr){4-5} \cmidrule(lr){6-7}
            employment \& hours          & 1         & 2         & 3         & 4         & 5         & 6         \\ \midrule
            $Treated^{e} \cdot distance$ & -0.005    & -0.003    & -0.006    & -0.004    & -0.006*   & -0.004    \\
            & (0.003)   & (0.003)   & (0.003)   & (0.003)   & (0.003)   & (0.003)   \\
            $Treated^{e}$                & -0.017    & 0.021     & -0.003    & -0.004    & 0.012     & 0.021     \\
            & (0.078)   & (0.057)   & (0.074)   & (0.030)   & (0.073)   & (0.051)   \\
            $distance \cdot$ cohort 2014 & 0.000     & -0.001    & -0.000    & -0.001*   & -0.000    & -0.001*   \\
            & (0.001)   & (0.001)   & (0.001)   & (0.001)   & (0.001)   & (0.001)   \\
            $distance \cdot$ cohort 2015 & -0.014    & -0.007    & -0.015    & 0.008     & -0.016*   & -0.008    \\
            & (0.009)   & (0.007)   & (0.009)   & (0.008)   & (0.009)   & (0.007)   \\
            $distance \cdot$ cohort 2017 & 0.017***  & 0.003     & 0.020***  & 0.009     & 0.021***  & 0.010     \\
            & (0.004)   & (0.008)   & (0.002)   & (0.010)   & (0.002)   & (0.010)   \\
            controls                     & Yes       & Yes       & Yes       & Yes       & Yes       & Yes       \\
            year FE                      & Yes       & Yes       & Yes       & Yes       & Yes       & Yes       \\
            county FE                    & Yes       & Yes       & Yes       & Yes       & Yes       & Yes       \\
            border-county FE             & No        & Yes       & No        & Yes       & No        & Yes       \\
            border-county LTs            & No        & Yes       & No        & Yes       & No        & Yes       \\ \midrule
            Observations                 & 1,893,689 & 1,893,689 & 1,893,689 & 1,893,689 & 1,893,689 & 1,893,689 \\
            $R^2$                        & 0.138     & 0.220     & 0.124     & 0.203     & 0.123     & 0.207     \\
            Baseline Mean                & 44.99     & 44.99     & 31.42     & 31.42     & 64.82     & 64.82     \\ \bottomrule \bottomrule
        \end{tabular}%
    }
    \begin{minipage}{\columnwidth}
        \vspace{0.05in}
        \tiny NOTES: These results are obtained from estimating model $E_{i,cp,t} = \beta (Treated^{e} \cdot D)_{h,s,t} + \psi (Treated^{e})_{s,t} + \vartheta (Treated \cdot D)_{h,s,t} + \mu (Post \cdot D)_{h,s,t} + \tau Treated_{s,t} + \rho D_{h,s,t} + \alpha Post_{t} + \delta X_{v,c,t-1} + \omega F_{f,t} + \lambda_{t} + \sigma_{h} + \phi_{cp} + \zeta_{cp,t} + \epsilon_{i,cp,t}$. Robust standard errors clustered at the state level are reported in parentheses. ***, **, and * denote significance levels at the less than $1\%$, $5\%$ and $10\%$, respectively.
    \end{minipage}
\end{table}

%======================================================================================================================%
    \begin{appendices}
        \renewcommand\thesection{\Roman{section}} % Use Roman numerals for section numbers in appendices
        \renewcommand\thesubsection{\Alph{subsection}} % Use Alphabets for sub-section numbers in appendices


        \section{Distribution of Industries and Pollution Emissions Intensities}\label{sec:appendix-distribution-of-industries-and-pollution-emissions-intensities}
%        \begin{figure}[H]
    \centering
    \includegraphics[width=0.85\textwidth]{C:/Users/david/OneDrive/Documents/ULMS/PhD/Thesis/chapter3/src/climate_change/latex/fig_naics_distribution}
    \caption{Distribution of Manufacturing Industries in the Sample}
    \label{fig:naics-manufacturing-industries}
\end{figure}
        \begin{figure}[H]
    \centering
    \includegraphics[width = 0.8\textwidth]{fig_releases_distribution}
    \caption{Distribution of Total Onsite Releases Intensity across Manufacturing Industries}
    \label{fig:releases-distribution}
\end{figure}
        \begin{figure}[H]
    \centering
    \includegraphics[width = 0.8\textwidth]{C:/Users/david/OneDrive/Documents/ULMS/PhD/Thesis/chapter3/src/climate_change/latex/fig_air_emissions_distribution_naics}
    \caption{Distribution of Total Onsite Air Emissions Intensity across Manufacturing Industries}
    \label{fig:air-emissions-distribution-naics}
\end{figure}
        \begin{figure}[H]
    \centering
    \includegraphics[width = 0.8\textwidth]{fig_water_distribution_naics}
    \caption{Distribution of Total Onsite Surface Water Discharge Intensity across Manufacturing Industries}
    \label{fig:water-discharge-distribution-naics}
\end{figure}
        \begin{figure}[H]
    \centering
    \includegraphics[width = 0.8\textwidth]{C:/Users/david/OneDrive/Documents/ULMS/PhD/Thesis/chapter3/src/climate_change/latex/fig_land_releases_distribution_naics}
    \caption{Distribution of Total Onsite Land Releases Intensity across Manufacturing Industries}
    \label{fig:land-releases-distribution-naics}
\end{figure}
        \begin{figure}[H]
    \centering
    \includegraphics[width = 0.8\textwidth]{fig_releases_distribution_states}
    \caption{Distribution of Total Onsite Releases Intensity between the Treated and Control States}
    \label{fig:releases-distribution}
\end{figure}
        \begin{figure}[H]
    \centering
    \includegraphics[width = 0.8\textwidth]{fig_air_emissions_distribution_state}
    \caption{Distribution of Total Air Emission Intensity between the Treated and Control States.}
    \label{fig:air-emissions-distribution}
\end{figure}
        \begin{figure}[H]
    \centering
    \includegraphics[width = 0.8\textwidth]{C:/Users/david/OneDrive/Documents/ULMS/PhD/Thesis/chapter3/src/climate_change/latex/fig_water_discharge_distribution_state}
    \caption{Distribution of Total Surface Water Discharge Intensity between the Treated and Control States}
    \label{fig:water-discharge-distribution}
\end{figure}
        \begin{figure}[H]
    \centering
    \includegraphics[width = 0.8\textwidth]{fig_land_releases_distribution_state}
    \caption{Distribution of Total Onsite Land Releases Intensity between the Treated and Control States}
    \label{fig:land-releases-distribution}
\end{figure}
        \begin{figure}[H]
    \centering
    \includegraphics[width = 0.8\textwidth]{fig_releases_distribution_carcinogenic}
    \caption{Distribution of Average Total Onsite Carcinogenic Releases Intensity between the Treated and Control States}
    \label{fig:releases-distribution-carcinogenic}
\end{figure}
        \begin{figure}[H]
    \centering
    \includegraphics[width = 0.8\textwidth]{C:/Users/david/OneDrive/Documents/ULMS/PhD/Thesis/chapter3/src/climate_change/latex/fig_releases_distribution_caa}
    \caption{Distribution of Average Total Onsite CAA Releases Intensity between the Treated and Control States}
    \label{fig:releases-distribution-caa}
\end{figure}
        \begin{figure}[H]
    \centering
    \includegraphics[width = 0.8\textwidth]{fig_releases_distribution_haps}
    \caption{Distribution of Total Onsite HAPs Releases Intensity between the Treated and Control States}
    \label{fig:releases-distribution-haps}
\end{figure}
        \begin{figure}[H]
    \centering
    \includegraphics[width = 0.8\textwidth]{fig_releases_distribution_pbts}
    \caption{Distribution of Average Total Onsite PBT Releases Intensity between the Treated and Control States}
    \label{fig:releases-distribution-pbts}
\end{figure}

    \end{appendices}
%----------------------------------------------------------------------------------------------------------------------%
    \bibliographystyle{elsarticle-harv} % Bibliography style
    \bibliography{emissions} % Bibliography file
%======================================================================================================================%
\end{document}
%======================================================================================================================%