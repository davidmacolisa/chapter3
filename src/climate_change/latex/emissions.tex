\documentclass[12pt, english]{article}
%======================================================================================================================%
% Preamble
\usepackage[
    backend = biber,
    style = apa,
    citestyle = authoryear-comp,
    sorting = ydnt,
    mincitenames = 1,
    maxcitenames = 2,
    uniquelist = minyear
]{biblatex}
\addbibresource{emissions.bib}
%\AtBeginBibliography{\small}

\usepackage{hyperref}
\hypersetup{colorlinks = true, citecolor = blue, linkcolor = blue, urlcolor = blue, hypertexnames = true}
\newcommand{\shortlink}[1]{\href{https://www.#1}{\texttt{#1}}}

\DeclareCiteCommand{\cite} % Ensures author-year hyperlink applies to \cite
{\usebibmacro{prenote}}
{\usebibmacro{citeindex}%
\printtext[bibhyperref]{\usebibmacro{cite}}}
{\multicitedelim}
{\usebibmacro{postnote}}

\DeclareCiteCommand{\parencite}[\mkbibparens] % Ensures author-year hyperlink applies to \parencite
{\usebibmacro{prenote}}
{\usebibmacro{citeindex}%
\printtext[bibhyperref]{\usebibmacro{cite}}}
{\multicitedelim}
{\usebibmacro{postnote}}

\DeclareCiteCommand{\cite} % Ensures that year appears in parentheses
{\usebibmacro{prenote}}
{\usebibmacro{citeindex}%
\printtext[bibhyperref]{\printnames{labelname} \mkbibparens{\printfield{year}}}}
{\multicitedelim}
{\usebibmacro{postnote}}

\usepackage{authblk}
\usepackage{times}
\usepackage{parskip}
\usepackage{graphicx}
\usepackage{amsmath}
\usepackage{amsfonts}
\usepackage{mathrsfs}
\usepackage{float}
\usepackage{geometry}
\geometry{papersize = {9in, 11in}, left = 2.5cm, right = 2.5cm, top = 2.5cm, bottom = 2.5cm}
\usepackage{scrextend}
\usepackage[doublespacing]{setspace}
\usepackage{textcomp}
\usepackage{csquotes}

% Tables Packages
\usepackage{booktabs}
\usepackage{titlesec} %for modified section numbering
\setcounter{secnumdepth}{4}  % How deep you want to number
\titleformat{\paragraph}
{\normalfont\normalsize\bfseries}{\theparagraph}{1em}{}
\titlespacing*{\paragraph}
{0pt}{3.25ex plus 1ex minus 0.2ex}{1.5ex plus 0.2ex}

%Appendix
\usepackage[toc, page]{appendix}
\usepackage{titleps}
\usepackage{caption}
\usepackage{supertabular}

%======================================================================================================================%
% Title Page
\title{{Unforeseen Minimum Wage Consequences}}
\author[1]{D.O. Ekeocha}
\affil[1]{
    University of Liverpool Management School \\
    \texttt{davidmac.ekeocha@liverpool.ac.uk}
}
%\affil[2]{University of Liverpool Management School}
%\affil[3]{University of Liverpool Management School}

\date{\today}

%======================================================================================================================%
\begin{document}
    \maketitle
    \newpage
    \tableofcontents
    \newpage
    \listoffigures
    \newpage
    \listoftables

    \newpage
    \begin{abstract}
        \noindent I document unforeseen minimum wage consequences.
    \end{abstract}


    \section{Introduction}\label{sec:introduction}
    The effects of raising the minimum wage (MW) have garnered interest in the literature, reaching mixed conclusions for employment given assumed labour market conditions. For perfect competition, some studies have recorded disemployment effects of an MW raise on low-wage workers~\parencite{stigler1946economics, hamermesh1982minimum, neumark1992employment, brown1999minimum, machin2004minimum, neumark2000minimum, borjas2010labor}, whereas in monopsonistic markets, limited or a sharp null employment effect is documented~\parencite{lester1960employment, card1993minimum, card2000minimum, aaronson2018industry, cengiz2019effect, wong2019minimum, dustmann2022reallocation}. Other studies have recorded heterogeneous employment effects~\parencite{okudaira2019minimum, medrano2023minimum, meer2023effects, gregory2022minimum}.

    On the other hand, the symmetry between raising MW and labour costs or wages of low-wage workers is well established~\parencite{medrano2023minimum,clemens2023important}. Career progressions and high labour demand, higher job/worker effort and labour productivity, especially for incumbent workers in the $40th$ percentile of the wage distribution relative to workers in higher percentiles, largely drive the increase in wages of low-wage workers~\parencite{riley2017raising, kim2019minimum, wong2019minimum, baek2021impact, zhao2021effects, seok2022macroeconomic, ku2022does, coviello2022minimum, alexandre2022minimum}. They argued that firms' adjustments in total factor productivity, lower hiring and layoff and monitoring incentive explain the increased labour productivity, and that the increase in worker effort can offset about $50\%$ of projected rise in labour cost. Furthermore,~\cite{harasztosi2019pays} argued that the cost-effect of raising MW is not only borne by employers and workers, but that firms tend to pass some of these burdens to consumers via higher prices.

    Interestingly, the question regarding the existence of other pathways through which firms pass-through the burdens of higher-cost-induced MW is still nascent. These studies focused on Chinese manufacturing firms to document higher emissions and pollution emissions intensity induced MW raise~\parencite{li2023does, zhang2023unintended}.~\footnote{\tiny Two hypotheses present opposing views on the direction of environmental impact of MW. The labour-technology mix which argues that firms react to labour cost induced MW by adopting clean production/operation technologies and labour-savings to increase energy efficiency, labour productivity thereby decreasing their pollution intensity. Conversely, the crowding-out effect hypothesis argues in favour of increased firms' pollution intensity. Given MW, firms struggle financially and to maintain pre-reform operation/production capacity while managing cost, it crowds-out clean production technologies and green innovative practices leading to increased pollution intensity. Thus, to understand how labour-market shocks translate to environmental consequences becomes an empirical exercise.} They argued that this effect is more prominent in financially constrained firms such that the labour cost pass-through of MW shifts firms energy usage towards cost-effective but crude energy sources, as well as reduced pollution abatement inputs and declining green innovation in their production functions. Albeit these studies used border-county design, they identified their effect using endogenous changes in MW, possibly indexed to inflation, thereby masking their causal claims of rising pollution emissions intensity. Hence, their results are indeed not causal as it is not clear whether the documented pollution responses are due to an increase in MW or inflation.

    However, this study leverages state-level exogenous changes in MW over time and administrative plant-level toxic emissions data (including air, land and water) from the environmental protection agency (EPA) to examine toxic responses of manufacturing firms in the United States (US) to the MW policy. The evidence on the environmental consequences of labour market shocks is particularly important for better health, environmental and labour market policy designs.~\cite{shapiro2018pollution} developed a model to historically explain the fall in the US pollution emissions in the manufacturing sector and revealed several findings including outsourcing the production of pollution-intensive goods to other countries like China and Mexico; environmental regulations such as pollution tax have the most significant negative impact on US pollution intensity through increased investment in effective abatement technologies; and rising labour productivity decreases pollution intensity, thus decreasing pollution emissions. Moreover, some literature has found causal evidence on the effects of air pollution $(PM_{2.5})$ and water pollution on cancer in humans and aquatic animals, respectively~\parencite{turner2020outdoor, turner2017ambient, baines2021linking}.~\cite{coneus2012pollution} found reductions in infant birth weights and increasing bronchitis and respiratory illnesses in toddlers due to increasing carbon monoxides and $O_{3}$ emissions in Germany.

    Two hypotheses exist on possible transmission mechanisms of environmental consequences of raising MW: the factor substitution and crowding out effect hypotheses~\parencite{zhang2023unintended}. The factor substitution hypothesis argues for the reductions in firms pollution emissions intensity. To maximize profits amid rising labour costs due to a raised MW floor, firms adjust their factor inputs by switching to automated production processes from labour-intensive manual processes. Thus, replacing manual labour with machines and technologies, and increasing capital per worker while prioritizing resource allocations toward improving production functions through research~\parencite{harasztosi2019pays,hau2020firm, geng2022minimum,dai2023minimum, li2020labor}. This will further increase labour productivity and total factor productivity~\parencite{riley2017raising}. Thus, raising MW will cause firms to use efficient capital intensive methods, reducing energy intensity and ultimately decreasing emissions per unit of output.~\footnote{\tiny Similarly, the possiblity exists that firms may replace expensive factor inputs with cheaper but crude energy inputs with higher pollution emission potentials, in their production process, which in turn increases energy and pollution emission intensities.} Conversely, the crowding out effect hypothesis argues in favour of increased firms' pollution intensity. These studies have evinced that this effect is attributed to the declining pressure on firms' profitability and heavier financial constraints resulting from wage hikes~\parencite{draca2011minimum, bell2018minimum, du2022minimum}. Consequently, constrained financial resources limit firms' investment in pollution abatement activities (as they focus on core production), leading to reduced pollution removal and higher emission intensity.~\footnote{\tiny In a different way, innovation is influenced by resource scarcity, leading to a preference for certain types of solutions. Technological advancement continues unabated, but resource constraints necessitate a balance between different types of solutions. Labour deficits, often caused by increases in the minimum wage, promote the development of labour-saving technologies, which can impede the progress of labour-intensive green technologies~\parencite{acemoglu2010does}. Firms often prioritize improvements in production efficiency over environmental considerations due to limited resources for innovation. Interestingly, some environmentally friendly technologies, such as end-of-pipe treatments, require significant labour. This emphasis on efficient automation could divert resources away from the development of clean energy and waste management technologies, potentially resulting in increased pollution emissions intensity.}

    To document the causal effect of raising MW on pollution emission intensity, I use a staggered difference-in-differences framework exploiting copious exogenous changes in state-level MW ($\geq \$0.5$ per hour) with clearly defined before- and after-periods. I exploit increases in Arkansas, California, Delaware, Maine, Massachusetts, Maryland, Michigan, Minnesota, Nebraska, New Jersey, New York, South Dakota and West Virginia in $2014$, $2015$ and $2017$. Hereafter, the treated states. I match each treated state to a set of adjacent control states that never implemented any MW policy from $2012-2017$~\parencite{gopalan2021state}. I further restrict the sample to border-counties~\parencite{dube2010minimum}. The identifying assumption is that in the absence of any MW policy, economic conditions in adjacent cross-border counties would have evolved in parallel. Given this assumption, I show that the pre-treatment trends between the treated and control states are similar prior to the MW policy. Figure~\ref{fig:releases-plots-treatment} shows that the treated states experienced significant increases in toxic releases intensities (including air emissions and surface water discharge) in $2014$, $2015$ and $2017$. %TODO: Explain more

%    Third, following the results from the above exercises, I revisit the nascent literature on the impact of MW on firms output per capita and profitability~\parencite{van2006minimum, draca2011minimum, bell2018minimum} documenting declined profits and firms overall value in the UK. To date, the methods used in the literature is the traditional difference-in-differences design which under-performs especially in multiple period settings that the studies cover~\parencite{callaway2021difference, goodman2021difference}. I extend this literature by considering firms in the US and employing the advantages of a staggered design to account for the dynamic raising of MW across US states and dynamic firms adjustment as a result. This has not been accounted for before in the literature.

    The following sections are organized as follows. Section~\ref{sec:policy-context} provides the policy context of the study. Section~\ref{sec:data} discusses the data and descriptive statistics. Section~\ref{sec:industry-costs-employment-and-outputs} presents the effect on industry labour costs and employment, and robustness. Section~\ref{sec:onsite-toxic-releases} discusses the effects on onsite toxic releases intensities. Section~\ref{sec:robustness-exercises} discusses robustness exercises. Section~\ref{sec:heterogeneous-effects} focuses on the heterogeneous effects. Section~\ref{sec:mechanism-analyses} investigates the potential mechanisms. Section~\ref{sec:conclusions} concludes with policy implications.


    \section{Policy Context}\label{sec:policy-context}
    This section discusses the exogenous state minimum wage changes exploited in the covered sample period $(2011-2017)$, for the causal identification.

    \subsection{Minimum Wage Changes across States}\label{subsec:minimum-wage-changes-across-states}
    About $27$ states raised MW between $2011$ and $2017$. Except for Nevada, the majority of the changes in MW between $2011-2013$ are attributed to inflation. However, following the labour union protests in $2012$ for higher MW, many states responded by instituting a one- or multiphased large MW changes~\parencite{lathrop2021raises}. Specifically, $13$ states implemented a statutory MW raise of at least $\$0.5$ in $2014$, $2015$ and $2017$. Pertinently, there was no federal MW raise in the covered sample period. Table~\ref{tab:states-mw-changes} of Appendix I presents the state-level changes in MW for the sample period.

    \subsection{Selecting the Treated and Control States}\label{subsec:selecting-the-treated-and-control-states}
    The identification strategy focuses on the copious state-level MW changes. Particularly, I restrict the sample to MW changes that meet the following conditions: $(i)$ states with a large MW raise of $\geq$ $\$0.5/hour$ in one year and had never raised their MW since $2012$; $(ii)$ subsequent raises must either be equal to or greater than $\$0.5/hour$ in the post year, or the sum of post-multiphase raises (in any $2$-years) must be equal or greater than the first initial raise for that state. These conditions ensure that the exploited MW changes are not indexed to inflation but statutorily driven. Thirteen $(13)$ states meet the above conditions. They include Arkansas, California, Delaware, Maine, Massachusetts, Maryland, Michigan, Minnesota, Nebraska, New Jersey, New York, South Dakota, and West Virginia.~\footnote{\tiny New Jersey and South Dakota do not meet the second condition but are included in the treated sample. Hence, I explicitly control for US city average inflation in the model to net out any inflationary effects. Controlling for the year fixed effects in the model nets out this inflationary effect.} Table~\ref{tab:states-mw-adjustments-t-and-c} provides a summary of the minimum wage adjustments in the sample. There are four increases of $\$0.75/hour$, four increases of $\$1.00/hour$, four increases of $\geq \$1.00/hour$ with a max of $\$1.85/hour$ and one increase of $\$0.5/hour$ in the sample. The average initial MW change is about $\$1.02/hour$ $(7.6\%)$. Following the first MW raise till the end of the covered period, there are seven post-initial total MW changes between $\$1.00/hour \leq \Delta MW \leq \$1.75/hour$, four total post-initial MW changes between $\$2.00/hour \leq \Delta MW \leq \$2.50/hour$ and two total post-initial MW changes between $\$3.00/hour \leq \Delta MW \leq \$3.35/hour$. These have a total of post-initial MW average of $\$1.95/hour$, about $(7.6\%)$ total average MW changes between $2011$ and $2017$. Further, the total percentage change in MW from $2011$ to $2017$ for each treated state range between $13.79$ and $54.38\%$.
    %\usepackage{booktabs}
\begin{table}[H]
    \centering
    \caption{Exogenous State-level MW Adjustments}
    \label{tab:states-mw-adjustments-t-and-c}
    \resizebox{\columnwidth}{!}{%
        \begin{tabular}{lrrrrrrlrr}
            \toprule\toprule
            treated states & MW $\Delta$ year & MW $\Delta$ amount & $\sum_{1}^{2}\Delta MW$ & total MW $\Delta$ amount & start MW & end MW & control states & \# of border counties (T) & \# of border counties (C) \\ \midrule\midrule
            MN             & 2014             & 1.85               & 1.50                    & 3.35                     & 6.16     & 9.51   & (IA, ND, WI)   & 42                        & 54                        \\
            MA             & 2015             & 1.00               & 2.00                    & 3.00                     & 8.00     & 11.00  & (NH)           & 8                         & 6                         \\
            CA             & 2014             & 1.00               & 1.50                    & 2.50                     & 8.00     & 10.50  & (NV)           & 10                        & 11                        \\
            NY             & 2014             & 0.75               & 1.70                    & 2.45                     & 7.25     & 9.70   & (PA)           & 27                        & 16                        \\
            AR             & 2015             & 1.25               & 1.00                    & 2.25                     & 6.25     & 8.50   & (OK, TX)       & 19                        & 12                        \\
            MD             & 2015             & 1.00               & 1.00                    & 2.00                     & 7.25     & 9.25   & (PA, VA)       & 21                        & 39                        \\
            NE             & 2015             & 0.75               & 1.00                    & 1.75                     & 7.25     & 9.00   & (IA, KS, WY)   & 28                        & 24                        \\
            ME             & 2017             & 1.50               & 0.00                    & 1.50                     & 7.50     & 9.00   & (NH)           & 5                         & 7                         \\
            MI             & 2014             & 0.75               & 0.75                    & 1.50                     & 7.40     & 8.90   & (IL, IN, WI)   & 32                        & 39                        \\
            WV             & 2015             & 0.75               & 0.75                    & 1.50                     & 7.25     & 8.75   & (KY, PA, VA)   & 32                        & 28                        \\
            SD             & 2015             & 1.25               & 0.15                    & 1.40                     & 7.25     & 8.65   & (IA, ND, WY)   & 27                        & 19                        \\
            NJ             & 2014             & 1.00               & 0.19                    & 1.19                     & 7.25     & 8.44   & (PA)           & 16                        & 13                        \\
            DE             & 2014             & 0.50               & 0.50                    & 1.00                     & 7.25     & 8.25   & (PA)           & 5                         & 6                         \\ \bottomrule\bottomrule
        \end{tabular}
    }
    \begin{minipage}{17.5cm}
        \vspace{0.01in}
        \tiny NOTES: This table summarizes the exogenous state-level MW changes from $2012$ to $2017$. There are thirteen $(13)$ treated and $(14)$ control states. The definition of treated and control states is given in sub-section~\ref{subsec:selecting-the-treated-and-control-states}. MW $\Delta$ year represents the year in which a treated state first raised its MW. MW $\Delta$ amount corresponds to the first MW raised amount for that year. $\sum_{1}^{2}\Delta MW$ denotes the sum of any post-two-year MW raises after the first initial raise. Total MW $\Delta$ amount corresponds to the total MW raised amount till the end of the sample. Start (End) is the MW at the start (end) of the sample. Control states are the set of control states for each treated states. These states never raised MW between $2012$ and $2017$. \# of border counties (T) is the number of counties in a treated state that border at least one county in a control state. And \# of border counties (C) is the number of counties in a control state that border at least one county in a treated state. * means not used in the POTWs sample.
    \end{minipage}
\end{table}


    Furthermore, I match each treated state to adjacent control states that never raised their MW between $2012$ and $2017$, and follow~\cite{dube2010minimum} and~\cite{gopalan2021state} to limit the sample to border counties in treated and control states.~\footnote{\tiny Other recent papers to use this identification strategy include~\cite{aaronson2018industry},~\cite{dube2019fairness},~\cite{jardim2018minimum}, and~\cite{zhang2019distributional}. The control states in the sample include Iowa, Illinois, Indiana, Kansas, Kentucky, North Dakota, New Hampshire, Nevada, Oklahoma, Pennsylvania, Texas, Virginia, Wisconsin, and Wyoming. Importantly, Georgia, Idaho, New Mexico, North Carolina, and Utah are removed from the list of control states as they are not adjacent to any treated states.} Table~\ref{tab:states-mw-adjustments-t-and-c} shows that there are $13$ treated and $14$ control states in the sample. The last two columns further show that there are a total of $272$ treated and $274$ control border counties. Figures~\ref{fig:border-state-map} and~\ref{fig:border-county-map} show the geographical locations of the treated and control states and counties, respectively.
    \begin{figure}[H]
    \centering
    \includegraphics[width=0.85\textwidth, height=0.4\textheight]{border_state_map}
    \caption{Map of Treated and Control States}
    \label{fig:border-state-map}
\end{figure}
    \begin{figure}[H]
    \centering
    \includegraphics[width=0.85\textwidth, height=0.4\textheight]{border_county_map}
    \caption{Map of Treated and Control Counties}
    \label{fig:border-county-map}
\end{figure}

    Each treated border county is paired with a cross-border control county, described as the pair of the adjacent treated and control counties. The identifying assumption in a cross-border county pair is that the evolution of economic conditions for the pairs is symmetric and parallel, but the MW levels vary discontinuously at the border. To address the concern raised in~\cite{neumark2014revisiting} on the validity of border counties as counterfactuals, Figures~\ref{fig:county-level-macroeconomic-trends-in-border-counties} and ~\ref{fig:state-level-macroeconomic-trends-in-border-states} present a comparison of economic conditions before the first initial year of the MW change. Along most of the observable pre-treatment variables, the trends appear to be statistically parallel for the treated and control border counties and states.~\footnote{\tiny Tables~\ref{tab:descriptive-statistics-control-border-counties} and~\ref{tab:descriptive-statistics-control-border-states} of Appendix I presents a comparison of the means of the pre-treatment variables for the treated and control border counties. For most of the variables, the results show no substantial differences in their means for the treated and control border counties. The state-level descriptive statistics uses state-level aggregated dataset.}


    \section{The Data}\label{sec:data}
    For the empirical analysis, the novel data used come from five different sources and combined with the administrative facility level toxic release inventory (TRI) data from the environmental protection agency (EPA) for the United States.

    \subsection{Toxic Release Inventory Data}\label{subsec:toxic-release-inventory-data}
    I collect TRI-form-R data from EPA, which contains inventory of toxic chemical releases that are either manufactured, processed, otherwise used, and/or managed at private, state and federal industrial facilities across the US States. To file a TRI reporting form R, a facility must have at least ten full-time employees, and manufactures (including import) or processes more than $25,000$ pounds or otherwise uses more than $10,000$ pounds of a TRI-listed chemical during a calendar year. TRI data reflect, among other things, quantities of chemicals managed by facilities as waste, including those quantities released into the environment (as air emissions, water and land pollution), treated, burned for energy, recycled, and transferred from one facility to another for release or further management. It provides facility-level information based on $5$-digit zip codes identifying the exact location of the facility within each city, county, and states in the US.

    I collate onsite facility panel information as well as offsite transfers and publicly owned treatment works (POTWs) toxic chemical releases and waste management practices. There are $167$ different toxic chemical releases consistently reported by the same facilities from $2011-2017$ belonging to $213$ different NAICS manufacturing industries.~\footnote{\tiny These manufacturing industries are categorized into: Beverage and tobacco product, chemical, computer and electronic product, food, forging and stamping, furniture and related products, household appliances, leather and allied products, machinery, miscelleneous, non-metallic mineral product, paper, petroleum and coal products, plastics and rubber products, primary metal, printing and related support activities, textile mills, textile product mills, transportation equipment, and wood product manufacturing.}

    \subsection{Wage Data}\label{subsec:wage-data}
    I use the industry level wage data from the national bureau of economic research-centre for economic studies (NBER-CES). The NBER-CES data contains complete information on industry-level production workers and their wages, production workers' hours, and total payroll for manufacturing industries. Further, using this information I construct the industry-level production workers' wage per hour and wages per worker.

    \subsection{Other Industry and Macroeconomic Data}\label{subsec:other-industry-and-macroeconomic-data}
    The industry level data are from the NBER-CES and contains other variables including employment, number of total revenues, material costs, energy use, value added and total factor productivity, etc., for only manufacturing industries in the US. The macroeconomic data are at the county sourced from the quarterly census of employment and wages (QCEW) of the US Bureau of Labour Statistics (BLS) and include variables such as the average number of establishments and industry ownership. The county-level gross domestic product is sourced from the bureau of economic analysis (BEA). The inflation data is got from the BLS consumer price index historical dataset for all urban consumers.

    \subsection{Joining the Datasets and the Sample}\label{subsec:joining-the-datasets-and-the-sample}
    I begin by joining the US zip-code and county-level geographic shapefiles (by zip-codes) to the TRI data to get the corresponding FIPS codes. These were then used to join the BEA, BLS and QCEW data by year, FIPS and $6$-digit NAICS codes. The resulting data were further merged to the NBER-CES dataset by their NAICS codes. Finally, this merged data was then joined with the prepared US geographic adjacent county shapefile that has information on county-level population and county distance to a state border. I prepared these shapefiles in the spirit of~\cite{dube2010minimum} and~\cite{gopalan2021state}, where each treated county or state is matched to at least one adjacent cross-border county or state, yielding cross-border county/state pairs. This final novel dataset is used for the empirical analysis of this paper.
%    Table~\ref{variable-definitions} describes all the variables used in the analysis.

    From the merged dataset, I subset an unbalanced panel sample of onsite facilities, as well as their transfers to offsite and POTWs locations. The onsite data sample size is $1,893,689$ and consists of $1276$ manufacturing facilities belonging to $213$ NAICS codes (in $20$ manufacturing industries) and a panel of $167$ toxic chemicals. Figure~\ref{fig:naics-manufacturing-industries} of Appendix~\ref{sec:appendix-distribution-of-industries-and-pollution-emissions-intensities} shows the distribution of these manufacturing industries, which reports that chemical, forging and stamping, primary metals, petroleum and coal products, transportation equipment, and machinery manufacturing industries are the most common in the sample. This is an administrative facility-level panel located in a $5$-digit zip-codes in both the treated $(13)$ and adjacent control $(14)$ border states. A total of $27$ states and the number of cross-border counties are as described in section~\ref{subsec:selecting-the-treated-and-control-states}. The offsite and POTWs samples are subsets of the onsite sample. The unbalanced offsite panel sample size is $1,179,754$ and POTWs sample is $308,943$. The original panel of onsite facilities across time reduced to $680$ and $236$ in the offsite and POTWs samples, respectively. The number of toxic chemicals in the offsite and POTWs samples are $125$ and $69$, respectively. See Table~\ref{tab:analyzed-chemicals} of Appendix~\ref{sec:appendix-list-of-toxic-chemicals-trends-and-mechanisms} for the list of analysed chemicals. The number of NAICS industries in the offsite and POTWs samples are $171$ and $101$. Finally, there are $14$ and $9$ control states in the offsite and POTWs samples; $13$ and $12$ treated states in the offsite and POTWs samples, respectively (see Table~\ref{tab:states-mw-adjustments-t-and-c}). Importantly, the number of onsite treated and control states remained unchanged in the offsite sample.

    A total of $827$ offsite facilities are located in $1056$ zip-codes in $573$ cities of $299$ counties in $45$ US states, whereas the $114$ POTWs sites are found in $202$ zip-codes in $123$ cities of $63$ US counties in $24$ US states.~\footnote{\tiny The offsite states include Alabama, Arkansas, Arizona, California, Colorado, Connecticut, Delaware, Florida, Georgia, Iowa, Idaho, Illinois, Indiana, Kansas, Kentucky, Louisiana, Massachussetts, Maryland, Maine, Michigan, Minnesota, Missouri, North Carolina, North Dakota, Nebraska, New Hampshire, New Jersey, New Mexico, Nevada, New York, Ohio, Okhlahoma, Oregon, Pennsylvania, Rhode Island, South Carolina, South Dakota,Tennesse, Texas, Utah, Virginia, Vermont, Wisconsin, and West Virginia. The POTWs states include Arkansas, California, District of Columbia, Delaware, Iowa, Illinois, Indiana, Massachussetts, Maryland, Maine, Michigan, Minnesota, North Dakota, Nebraska, New Hampshire, New Jersey, New York, Pennsylvania, Rhode Island, Texas, Virginia, Wisconsin, West Virginia, and Wyoming.}

    \subsection{Descriptive Statistics}\label{subsec:descriptive-statistics}
    The summary statistics are reported in Tables~\ref{tab:sumstat-onsite},~\ref{tab:sumstat-offsite} and~\ref{tab:sumstat-potws}. They show the facility average onsite releases, as well as those transferred offsite and to POTWs locations for further waste management.
    \begin{table}[H]
    \centering
    \caption{Summary Statistics (Onsite)}
    \label{tab:sumstat-onsite}
    \resizebox{\textwidth}{!}{
        \begin{tabular}{lrrrrr}
            \toprule \toprule
            Variable                                                & Obs     & Mean      & StdDev     & Min     & Max        \\ \midrule
            GDP per capita $(\$1000's)$                             & 1893689 & 44.99     & 42.29      & 2.90    & 365.80     \\
            industry employment (1000's)                            & 1893689 & 44.99     & 42.29      & 2.9     & 365.80     \\
            annual average establishments                           & 1893689 & 5.44      & 12.38      & 0.0     & 330.00     \\
            population (county) (1000's)                            & 1893689 & 693432.18 & 1247538.81 & 1466.00 & 5194675.00 \\
            city region average consumer price index $(\$)$         & 1893689 & 235.46    & 6.47       & 224.94  & 245.12     \\
            federal.facility                                        & 1893689 & 0.00      & 0.01       & 0.00    & 1.00       \\
            chemical ancillary use                                  & 1893689 & 0.25      & 0.43       & 0.0     & 1.00       \\
            chemical formulation component                          & 1893689 & 0.32      & 0.47       & 0.0     & 1.00       \\
            chemical manufacturing aid                              & 1893689 & 0.11      & 0.31       & 0.0     & 1.00       \\
            max number of chemicals at facility                     & 1893689 & 3.89      & 1.43       & 1.0     & 19.00      \\
            imported chemicals at facility                          & 1893689 & 0.07      & 0.25       & 0.0     & 1.00       \\
            produced chemicals at facility                          & 1893689 & 0.24      & 0.42       & 0.0     & 1.00       \\
            production ratio or activity index                      & 1893689 & 1.59      & 178.58     & 0.0     & 117229.00  \\
            total releases intensity (lbs)                          & 1893689 & 87.99     & 1065.44    & 0.00    & 122005.98  \\
            total air emissions intensity (lbs)                     & 1893689 & 60.02     & 616.84     & 0.00    & 40743.89   \\
            total fugitive air emissions intensity (lbs)            & 1893689 & 10.95     & 163.84     & 0.00    & 21484.45   \\
            total point air emissions intensity (lbs)               & 1893689 & 49.07     & 537.76     & 0.00    & 31559.41   \\
            total land releases intensity (lbs)                     & 1893689 & 7.93      & 701.61     & 0.00    & 122005.98  \\
            total underground injection intensity (lbs)             & 1893689 & 4.80      & 697.74     & 0.00    & 122005.98  \\
            total landfills intensity (lbs)                         & 1893689 & 1.43      & 53.40      & 0.00    & 6892.31    \\
            total releases to-land treatment intensity (lbs)        & 1893689 & 0.66      & 34.58      & 0.00    & 6006.01    \\
            total surface impoundment intensity (lbs)               & 1893689 & 0.03      & 2.40       & 0.00    & 929.15     \\
            total land releases intensity, others (lbs)             & 1893689 & 1.01      & 30.47      & 0.00    & 2299.05    \\
            total surface water discharge intensity (lbs)           & 1893689 & 20.04     & 475.39     & 0.00    & 41422.43   \\
            total number of receiving streams, onsite (lbs)         & 1893689 & 0.39      & 0.50       & 0.00    & 4.00       \\
            total release intensity, from catastrophic events (lbs) & 1893689 & 4.36      & 249.63     & 0.00   & 42103.29  \\
            total industry payroll $(\$1m)$                         & 1893689 & 2962.68   & 2630.57    & 127.00  & 16647.90   \\
            production workers (1000's)                             & 1893689 & 31.42     & 31.71      & 1.40    & 280.60     \\
            production hours (1m)                                   & 1893689 & 64.82     & 63.90      & 3.10    & 561.50     \\
%            production workers' wages $(\$1m)$                      & 1893689 & 1744.54   & 1583.39    & 64.90   & 10351.60    \\
            production workers' wages per hour                      & 1893689 & 26.56     & 7.35       & 12.24   & 54.35      \\
%            production workers' wages per worker                    & 1893689 & 55.28     & 17.85      & 24.47   & 114.48      \\
            cost of materials $(\$1m)$                              & 1893689 & 61328.05  & 162337.03  & 271.20  & 690771.20  \\
            industry value added (output) $(\$100m)$                & 1893689 & 177.13    & 272.06     & 3.00    & 1180.37    \\
            output per hour                                         & 1893689 & 2.62      & 2.82       & 0.44    & 34.28      \\
            output per worker                                       & 1893689 & 3.65      & 3.99       & 0.63    & 44.31      \\
            industry employment (1000's)                            & 1893689 & 44.99     & 42.29      & 2.90    & 365.80     \\ \bottomrule\bottomrule
        \end{tabular}
    }
\end{table}

    \begin{table}[H]
    \centering
    \caption{Summary Statistics (Offsite)}
    \label{tab:sumstat-offsite}
    \begin{tabular}{lrrrrr}
        \toprule \toprule
        Variable                                     & Obs     & Mean   & StdDev  & Min & Max       \\ \midrule
        total releases intensity                     & 1179754 & 257.85 & 2453.53 & 0   & 125639.48 \\
        total land releases intensity                & 1179754 & 196.20 & 2037.73 & 0   & 125639.48 \\
        total land releases other intensity          & 1179754 & 1.24   & 22.82   & 0   & 1637.57   \\
        total landfills intensity                    & 1179754 & 172.99 & 1958.09 & 0   & 125639.48 \\
        total surface impoundment intensity          & 1179754 & 0.63   & 58.77   & 0   & 7517.19   \\
        total underground injection intensity        & 1179754 & 18.44  & 557.48  & 0   & 59894.46  \\
        total wastewater releases intensity          & 1179754 & 6.00   & 125.39  & 0   & 15101.70  \\
        total releases (metal solidify) intensity    & 1179754 & 61.22  & 1507.92 & 0   & 84868.80  \\
        total releases (storage) intensity           & 1179754 & 0.87   & 34.06   & 0   & 6755.87   \\
        total releases (other mgt) intensity         & 1179754 & 5.18   & 99.26   & 0   & 11850.66  \\
        total releases (to-land) treatment intensity & 1179754 & 2.91   & 90.79   & 0   & 7875.12   \\
        total releases (unknown) intensity           & 1179754 & 6.30   & 59.87   & 0   & 3610.69   \\
        total releases (waste broker) intensity      & 1179754 & 8.28   & 115.33  & 0   & 6738.71   \\ \bottomrule\bottomrule
    \end{tabular}
\end{table}

    \begin{table}[H]
    \centering
    \caption{Summary Statistics (POTWs)}
    \label{tab:sumstat-potws}
    \begin{tabular}{lrrrrr}
        \toprule\toprule
        Variable                               & Obs    & Mean  & StdDev & Min & Max      \\ \midrule
        total releases intensity               & 308943 & 17.87 & 404.45 & 0   & 27648.29 \\
        underground releases intensity         & 308943 & 6.69  & 288.78 & 0   & 27648.29 \\
        underground releases intensity (other) & 308943 & 11.18 & 253.74 & 0   & 26548.32 \\ \bottomrule \bottomrule
    \end{tabular}
\end{table}

%    \begin{table}[H]
    \centering
    \caption{Summary Statistics of Onsite Mechanisms}
    \label{tab:sumstat-onsite-mechanisms}
%    \scalebox{0.8}{
    \resizebox{\textwidth}{!}{
        \begin{tabular}{lrrrrr}
            \toprule\toprule
            Variable                                          & Obs     & Mean     & SD        & Min & Max      \\ \midrule
%            total waste management                            & 1893689 & 83667.25 & 894161.45 & 0   & 45000000 \\
            biological treatment                              & 1893689 & 0.06     & 0.24      & 0   & 1        \\
            physical treatment                                & 1893689 & 0.18     & 0.38      & 0   & 1        \\
            incineration or thermal treatment                 & 1893689 & 0.13     & 0.34      & 0   & 1        \\
%            industrial boiler energy recovery method          & 1893689 & 0.01     & 0.11      & 0   & 1        \\
            recycling quantity                                & 1893689 & 22386.84 & 427110.33 & 0   & 44938800 \\
            recycling to reuse in production process          & 1893689 & 0.04     & 0.19      & 0   & 1        \\
            source reduction activities                       & 1893689 & 0.25     & 0.43      & 0   & 1        \\
            chemical purity modification                      & 1893689 & 0.00     & 0.03      & 0   & 1        \\
            clean fuel substitution                           & 1893689 & 0.01     & 0.12      & 0   & 1        \\
            organic solvent substitution                      & 1893689 & 0.00     & 0.02      & 0   & 1        \\
            new technology or technique in production process & 1893689 & 0.00     & 0.06      & 0   & 1        \\
            recirculation in production process               & 1893689 & 0.00     & 0.04      & 0   & 1        \\
            recycling in production process                   & 1893689 & 0.03     & 0.16      & 0   & 1        \\
            product quality analysis                          & 1893689 & 0.00     & 0.02      & 0   & 1        \\
            operating practices training                      & 1893689 & 0.00     & 0.03      & 0   & 1        \\
            changing size of storage containers               & 1893689 & 0.00     & 0.02      & 0   & 1        \\
            improved material handling                        & 1893689 & 0.00     & 0.02      & 0   & 1        \\ \bottomrule
        \end{tabular}
    }
%    }
\end{table}


    \section{Industry Costs, Employment and Outputs}\label{sec:industry-costs-employment-and-outputs}
    The empirical analyses begin in this section by examining the effect of raising MW on manufacturing industry costs (labour and materials), employment, production workers and hours, and outputs.~\footnote{\tiny To rule out any possible treatment selection, I estimate this equation at both the county and state level: $Treated_{cp,s,t}^e = \beta Z_{f,i,cp,s,t} + \lambda_{t} + \phi_{cp} + \delta_{s} + \zeta_{cp,t} + \epsilon_{cp,s,t}$. Where $Treated_{cp,s,t}^e = 1[t - P_{s,t}]$ denotes treated states that are $e$-periods away from the initial treatment date, and $P_{s,t}$ is the vector of initial treatment dates. $Z_{f,cp,s,t}$ is the vector of facilities by county-pair and state-level covariates, and $\beta$ is the vector of coefficients. Albeit, the year fixed effects, $\lambda_{t}$, nets out any inflationary effects, city-region inflation is explicitly controlled for in the model. Cross-border county pair, $\phi_{cp}$, and state, $\delta_{s}$, fixed effects are controlled for to account for within cross-border county pair and state differences that may affect the MW policy. Finally, I control for cross-border county pair linear trends, $\zeta_{cp,t}$, to account for the evolution of the MW policy in paired cross-border counties. The result (reported in Table~\ref{tab:treatment-selection}) showed no significant treatment selection effects in the following covariates: lagged values of county-level gross domestic product (GDP), GDP per capita, annual average establishments, and population; inflation; and industry level plant and equipment capital costs. Others include industry ownership binary variables (federal, state and private).}

    \subsection{Baseline Results: Industry Costs}\label{subsec:baseline-results-industry-costs}
    In what follows, I estimate wage responses of manufacturing industry employees in the baseline. The baseline model is given by:
    \begin{equation}
        C_{i,cp,t} = \beta Treated_{s,t}^e + \delta X_{v,c,t-1} + \omega F_{f,t} + \lambda_{t} + \sigma_{c} + \phi_{cp} + \zeta_{cp,t} + \epsilon_{i,cp,t},\label{eq:baseline-wages}
    \end{equation}
    % Please add the following required packages to your document preamble:
% \usepackage{booktabs}
\begin{table}[H]
    \centering
    \caption{Effect of the MW Policy on Industry Costs}
    \label{tab:baseline-industry-costs}
    \begin{tabular}{@{}lllllll@{}}
        \toprule\toprule
        Industry costs & \multicolumn{2}{c}{Hourly wage} & \multicolumn{2}{c}{Total payroll (log)} & \multicolumn{2}{c}{Material cost (log)} \\
        \cmidrule(lr){2-3}\cmidrule(lr){4-5}\cmidrule(lr){6-7}
        & 1         & 2         & 3         & 4         & 5         & 6         \\ \midrule
        $Treated^{e}$     & 0.397     & 0.889*    & -0.015    & 0.043*    & -0.031    & 0.129*    \\
        & (0.358)   & (0.452)   & (0.031)   & (0.025)   & (0.113)   & (0.069)   \\
        cohort 2014       & 0.582     & 1.197     & -0.030    & 0.004     & -0.081    & 0.187*    \\
        & (0.523)   & (0.707)   & (0.032)   & (0.039)   & (0.174)   & (0.047)   \\
        cohort 2015       & 0.091     & 0.382*    & 0.010     & 0.115***  & 0.051     & 0.035     \\
        & (0.260)   & (0.217)   & (0.062)   & (0.015)   & (0.094)   & (0.077)   \\
        cohort 2017       & -0.265    & -0.208    & -0.113*** & -0.613*** & 0.024     & -0.343*** \\
        & (0.226)   & (0.323)   & (0.019)   & (0.126)   & (0.031)   & (0.100)   \\
        controls          & Yes       & Yes       & Yes       & Yes       & Yes       & Yes       \\
        year FE           & Yes       & Yes       & Yes       & Yes       & Yes       & Yes       \\
        county FE         & Yes       & Yes       & Yes       & Yes       & Yes       & Yes       \\
        border-county FE  & No        & Yes       & No        & Yes       & No        & Yes       \\
        border-county LTs & No        & Yes       & No        & Yes       & No        & Yes       \\ \midrule
        Observations      & 1,893,689 & 1,893,689 & 1,893,689 & 1,893,689 & 1,893,689 & 1,893,689 \\
        $R^2$             & 0.581     & 0.624     & 0.404     & 0.440     & 0.577     & 0.619     \\
        Baseline Mean     & 26.56     & 26.56     & 2962.68   & 2962.68   & 61328.05  & 61328.05  \\ \bottomrule \bottomrule
    \end{tabular}
    \begin{minipage}{\columnwidth}
        \vspace{0.05in}
        \tiny NOTES: These results are obtained from estimating model~\ref{eq:baseline-wages}. Robust standard errors clustered at the state level are reported in parentheses. ***, **, and * denote significance levels at the less than $1\%$, $5\%$ and $10\%$, respectively.
    \end{minipage}
\end{table}

    where $C_{i,cp,t}$ is the vector of industry costs (hourly wages, total payroll and material costs) of manufacturing industry, $i$ in cross-border county pairs, $cp$ in the year, $t$. $Treated_{s,t}^e = \textbf{1}[t - P_{s,t}]$ is unity for the treated states that are $e$-periods away from the vector of initial treatment dates, $P_{s,t}$ and zero for the control states. $X_{v,c,t-1}$ denotes lagged values of county-level GDP per capita, annual average establishments, population and city-region inflation~\parencite{gopalan2021state, dube2010minimum, clemens2019making}. $F_{f,t}$ contains facility-level dummies on industry ownership. I control for year fixed effects, $\lambda_{t}$ to account for time varying differences in the MW policy as well as trending inflation. Cross-border county pair, $\phi_{cp}$ and county, $\sigma_{c}$ fixed effects are controlled for to account for within county pair and county differences that may affect the MW policy such as within county industry compositions and political climate. Finally, $\zeta_{cp,t}$ is the cross-border county pair linear trends to control for the evolution of common shocks in cross-border county pairs. Standard errors are clustered at the state level as there are possibilities that changes in MW may be correlated within a state.

    The average treatment effect on the treated (ATT) is captured by $\beta$, which is the difference in the average effect of raising the MW floor on manufacturing industry costs in treated counties relative to adjacent control counties. The effects measured by $\beta$ are recovered using the~\cite{sun2021estimating} staggered difference-in-differences estimator, an improvement over the TWFE estimator. I report the decomposition results from~\cite{de2020two} investigating possible negative weights in the TWFE estimator when the treatment assignment is staggered. Albeit the results from both estimators are largely similar, the analyses in this paper are guided by the coefficients of the staggered difference-in-differences estimator. The results on industry costs are reported in Table~\ref{tab:baseline-industry-costs}. I find that a higher MW policy increases manufacturing industry wages (labour costs) in treated counties relative to adjacent control counties. Manufacturing industry wages per hour rose by $\$0.89$. This result is almost twice as strong than that documented in~\cite{gopalan2021state} for the industry-wide effect on hourly wages of $\$0.48$. This suggests that the manufacturing industry is strongly affected by the MW policy relative to other industries in the US. Hence, the bite in industry costs due to a higher MW floor is higher in the manufacturing industry compared to others.~\footnote{\tiny Figure~\ref{fig:baseline-manufacturing-industry-costs-skilled} of Appendix~\ref{sec:appendix-baseline-robustness-tables-and-figures} illustrates that the impact on manufacturing industry labor costs is more pronounced for high-skilled workers compared to low-skilled workers. The figure shows an immediate increase in hourly wages. While this increase is short-lived for low-skilled workers, it persists for up to three years post-treatment for high-skilled workers. Similarly, total payroll for high-skilled workers increases, whereas it substantially declines for low-skilled workers in the second year post-treatment. These findings indicate that the labor cost effect of raising the minimum wage is stronger for high-skilled workers in manufacturing industries. Notably, low-skilled workers, defined as those below the 30th percentile of the wage distribution and comprising $18.2\%$ of workers, further demonstrate that the manufacturing industry workforce is skewed towards high-skilled workers.} Further, I document an increase in total payroll in treated counties relative to adjacent control counties by $4.4$ percentage points (ppts), as well as an increase in cost of materials in manufacturing industries by $12.9ppts$. The cohorts specific effects reveal that except for the $2015$ cohort where labour cost rose by $\$0.38$, it is mute for both the $2014$ and $2017$ cohorts. This suggests that the bite of raising MW is strongest for states that first raised their MW floor in $2015$. Similarly, I find an increase in total payroll for the $2015$ cohort, and an increase in material cost for the $2014$ cohort. However, I find a declining effect in total payroll and material cost for the $2017$ cohort---Maine. The disemployment effect in~\ref{subsec:baseline-results-employment} for the $2017$ cohort explain this declining wage-effect.
    \begin{figure}[H]
    \centering
    \includegraphics[width=1\textwidth,keepaspectratio]{fig_sdid_industry_costs}
    \caption{Manufacturing Industry Costs}
    \label{fig:baseline-manufacturing-industry-costs}
    \begin{minipage}{14cm}
        \vspace{0.05in}
        NOTES: The event study model of equation~\ref{eq:baseline-wages} is $C_{i,cp,t} = \sum_{{e = -3},{e \neq -1}}^{3} \beta Treated_{s,t}^e = \textbf{1}[t - G_{s,t}] + \delta X_{v,c,t-1} + \omega P_{f,t} + \lambda_{t} + \sigma_{c} + \phi_{cp} + \zeta_{cp,t} + \epsilon_{i,cp,t}$. Standard errors are clustered at the state level. de Chaisemartin and D'Haultfoeuille Decomposition: $\sum dCDH_{ATTs}^{weights(+)} = 1$ and $\sum dCDH_{ATTs}^{weights(-)} = 0$.
    \end{minipage}
\end{figure}

    Figure~\ref{fig:baseline-manufacturing-industry-costs} reports the dynamic effects. It shows increases in hourly wages in the first and second year following an MW policy. Moreover, I find instantaneous increases in the manufacturing industry's total payrolls and cost of materials, persisting up to three years after initially raising MW. There are no significant evidence of pre-trends. The timing and size of the effect are consistent with my data and setting.

    \subsection{Baseline Results: Employment}\label{subsec:baseline-results-employment}
    This subsection estimates the effect of raising MW on employment of manufacturing industry workers and their production hours. The model is given by:
    \begin{equation}
        E_{i,cp,t} = \beta Treated_{s,t}^e + \delta X_{v,c,t-1} + \omega F_{f,t} + \lambda_{t} + \sigma_{c} + \phi_{cp} + \zeta_{cp,t} + \epsilon_{i,cp,t},\label{eq:baseline-emp-hours}
    \end{equation}
    where $E_{i,cp,t}$ is the vector of employment, total production workers and hours in manufacturing industry, $i$ in cross-border county pairs, $cp$ in the year, $t$. Standard errors are clustered at the state level.
    % Please add the following required packages to your document preamble:
% \usepackage{booktabs}
% \usepackage{graphicx}
\begin{table}[H]
    \centering
    \caption{Effect of the MW Policy on Employment and Production Workers' Hours}
    \label{tab:baseline-employment-hours}
    \resizebox{\columnwidth}{!}{%
        \begin{tabular}{@{}lllllll@{}}
            \toprule\toprule
            & \multicolumn{2}{c}{Employment (log)} & \multicolumn{2}{c}{Production Workers (log)} & \multicolumn{2}{c}{Production Hours (log)} \\
            \cmidrule(lr){2-3} \cmidrule(lr){4-5} \cmidrule(lr){6-7}
            employment \& hours & 1         & 2         & 3         & 4         & 5         & 6         \\ \midrule
            treated             & -0.043    & -0.002    & -0.066*   & -0.023    & -0.064*   & -0.019    \\
            & (0.031)   & (0.025)   & (0.035)   & (0.033)   & (0.035)   & (0.034)   \\
            cohort 2014         & -0.063**  & -0.058    & -0.097*** & -0.097*   & -0.096*** & -0.094*   \\
            & (0.027)   & (0.039)   & (0.032)   & (0.050)   & (0.033)   & (0.053)   \\
            cohort 2015         & -0.010    & 0.095***  & -0.014    & 0.104***  & -0.012    & 0.111***  \\
            & (0.034)   & (0.018)   & (0.070)   & (0.024)   & (0.070)   & (0.024)   \\
            cohort 2017         & -0.117*** & -0.552*** & -0.088*** & -0.568*** & -0.068*** & -0.556*** \\
            & (0.019)   & (0.136)   & (0.022)   & (0.156)   & (0.023)   & (0.144)   \\
            controls            & Yes       & Yes       & Yes       & Yes       & Yes       & Yes       \\
            year FE             & Yes       & Yes       & Yes       & Yes       & Yes       & Yes       \\
            county FE           & Yes       & Yes       & Yes       & Yes       & Yes       & Yes       \\
            border-county FE    & No        & Yes       & No        & Yes       & No        & Yes       \\
            border-county LTs   & No        & Yes       & No        & Yes       & No        & Yes       \\ \midrule
            Observations        & 1,893,689 & 1,893,689 & 1,893,689 & 1,893,689 & 1,893,689 & 1,893,689 \\
            $R^2$               & 0.358     & 0.393     & 0.342     & 0.378     & 0.350     & 0.385     \\
            Baseline Mean       & 44.99     & 44.99     & 31.42     & 31.42     & 64.82     & 64.82     \\ \bottomrule \bottomrule
        \end{tabular}%
    }
    \begin{minipage}{18cm}
        \vspace{0.05in}
        These results are obtained from estimating model~\ref{eq:baseline-emp-hours}. Robust standard errors clustered at the state level are reported in parentheses. ***, **, and * denote significance levels at the less than $1\%$, $5\%$ and $10\%$, respectively.
    \end{minipage}
\end{table}

    The average treatment effect on the treated (ATT) is captured by $\beta$, which is the difference in the average effect of raising the MW floor on manufacturing industry employment, production workers and hours in treated counties relative to adjacent control counties. The results are reported in Table~\ref{tab:baseline-employment-hours}. It shows no significant changes in overall manufacturing industry employment, total production workers and hours following an MW policy in the treated counties relative to adjacent control counties. Particularly, the size of the effect suggests a sharp null effect of the MW policy on manufacturing industry employment including production workers and hours, after accounting for time-varying common shocks to border counties. The results are consistent with the labour market literature assuming monopsonistic competition~\parencite{card2000minimum, aaronson2018industry, cengiz2019effect, wong2019minimum, dustmann2022reallocation}. However, I find heterogeneity in the cohort-specific effects. While there are disemployment effects in the $2014$ and $2017$ cohorts, there is a significant positive effect on employment in the $2015$ cohort. These disemployment effects are driven by the declining production hours due to higher hourly wages. These disemployment effects explain the decline in total payroll and material cost for the $2017$ cohort in subsection~\ref{subsec:baseline-results-industry-costs}. Similarly, the dynamic treatment effects in Figure~\ref{fig:baseline-employment-hours} show an instantaneous increase in manufacturing industry employment (including production workers and hours) followed by a decline in the third year after the initial raise in MW.\footnote{\tiny~\cite{neumark2019econometrics} argues that cross-border studies may be biased against detecting disemployment effects due to worker mobility spillovers. To test if my results are influenced by cross-county mobility, potentially violating the stable unit treatment assumption, I use the interaction of the staggered difference-in-differences coefficient with the distance between population centers. The principle here is that worker mobility to counties/states with higher minimum wages (MW) decreases as geographic distance to that county/state increases. Table~\ref{tab:baseline-cross-county-state-mobility} in Appendix~\ref{sec:appendix-baseline-robustness-tables-and-figures} indicates that worker mobility does not affect the overall baseline results, including the positive employment effect specific to the 2015 cohort. However, the disemployment effects observed for the 2014 and 2017 cohorts are explained by workers' unwillingness to commute to distant counties/states with higher MW. This supports the hypothesis that worker mobility decreases with increasing distance to higher MW regions. Furthermore, Figure~\ref{fig:baseline-employment-hours-skilled} of Appendix~\ref{sec:appendix-baseline-robustness-tables-and-figures} illustrates that the disemployment effect is more pronounced for low-skilled workers in the second and third years following an initial minimum wage increase. Conversely, high-skilled workers experienced an immediate employment boost, followed by a decline in the third year. Similar trends are observed among both low- and high-skilled production workers, attributed to analogous changes in their working hours. This indicates that raising the minimum wage heightens the reluctance of low-skilled manufacturing workers to commute to regions with higher minimum wages, consequently increasing the demand for high-skilled manufacturing industry workers.} There is no evidence of pre-trends.
    \begin{figure}[H]
    \centering
    \includegraphics[width=1\textwidth, keepaspectratio]{fig_sdid_emp_hours}
    \caption{Industry Employment and Production Workers Hours}
    \label{fig:baseline-employment-hours}
    \begin{minipage}{14cm}
        \vspace{0.05in}
        NOTES: The event study model of equation~\ref{eq:baseline-emp-hours} is $E_{i,cp,t} = \sum_{{e = -3},{e \neq -1}}^{3} \beta Treated_{s,t}^e = \textbf{1}[t - G_{s,t}] + \delta X_{v,c,t-1} + \omega F_{f,t} + \lambda_{t} + \sigma_{c} + \phi_{cp} + \zeta_{cp,t} + \epsilon_{i,cp,t}$. Standard errors are clustered at the state level. de Chaisemartin and D'Haultfoeuille Decomposition: $\sum dCDH_{ATTs}^{weights(+)} = 1$ and $\sum dCDH_{ATTs}^{weights(-)} = 0$.
    \end{minipage}
\end{figure}

    \subsection{Baseline Results: Industrial Output}\label{subsec:baseline-results-industrial-output}
    This subsection estimates the effect of raising MW on outputs of the manufacturing industry. The model is given by:
    \begin{equation}
        Y_{i,cp,t} = \beta Treated_{s,t}^e + \delta X_{v,c,t-1} + \omega F_{f,t} + \lambda_{t} + \sigma_{c} + \phi_{cp} + \zeta_{cp,t} + \epsilon_{i,cp,t},\label{eq:baseline-output}
    \end{equation}
    where $Y_{i,cp,t}$ is the vector of manufacturing industry output, output per hour and output per worker (labour productivity), in manufacturing industry, $i$ in cross-border county pairs, $cp$ in the year, $t$. Standard errors are clustered at the state level.
    % Please add the following required packages to your document preamble:
% \usepackage{booktabs}
% \usepackage{graphicx}
\begin{table}[H]
    \centering
    \caption{Effect of the MW policy on Manufacturing Industry Output}
    \label{tab:baseline-industry-output}
    \resizebox{\columnwidth}{!}{%
        \begin{tabular}{@{}lllllll@{}}
            \toprule\toprule
            Industry outputs (log) & \multicolumn{2}{c}{Output} & \multicolumn{2}{c}{Output per Hour} & \multicolumn{2}{c}{Output per Worker} \\
            \cmidrule(lr){2-3} \cmidrule(lr){4-5} \cmidrule(lr){6-7}
            & 1         & 2         & 3         & 4         & 5         & 6         \\ \midrule
            treated           & -0.020    & 0.125***  & 0.045     & 0.144***  & 0.024     & 0.127***  \\
            & (0.080)   & (0.032)   & (0.083)   & (0.038)   & (0.082)   & (0.032)   \\
            cohort 2014       & -0.078    & 0.122**   & 0.018     & 0.216***  & -0.015    & 0.180***  \\
            & (0.123)   & (0.050)   & (0.131)   & (0.059)   & (0.129)   & (0.049)   \\
            cohort 2015       & 0.079     & 0.135***  & 0.090***  & 0.024     & 0.089***  & 0.039*    \\
            & (0.063)   & (0.016)   & (0.026)   & (0.028)   & (0.020)   & (0.020)   \\
            cohort 2017       & -0.086*** & 0.447***  & -0.018    & 0.108**   & 0.032     & 0.105***  \\
            & (0.025)   & (0.101)   & (0.026)   & (0.045)   & (0.025)   & (0.037)   \\
            controls          & Yes       & Yes       & Yes       & Yes       & Yes       & Yes       \\
            year FE           & Yes       & Yes       & Yes       & Yes       & Yes       & Yes       \\
            county FE         & Yes       & Yes       & Yes       & Yes       & Yes       & Yes       \\
            border-county FE  & No        & Yes       & No        & Yes       & No        & Yes       \\
            border-county LTs & No        & Yes       & No        & Yes       & No        & Yes       \\ \midrule
            Observations      & 1,893,689 & 1,893,689 & 1,893,689 & 1,893,689 & 1,893,689 & 1,893,689 \\
            $R^2$             & 0.504     & 0.548     & 0.590     & 0.630     & 0.602     & 0.648     \\
            Baseline Mean     & 177.13    & 177.13    & 2.62      & 2.62      & 3.65      & 3.65      \\ \bottomrule\bottomrule
        \end{tabular}%
    }
    \begin{minipage}{18cm}
        \vspace{0.05in}
        These results are obtained from estimating model~\ref{eq:baseline-output}. Robust standard errors clustered at the state level are reported in parentheses. ***, **, and * denote significance levels at the less than $1\%$, $5\%$ and $10\%$, respectively.
    \end{minipage}
\end{table}

    The average treatment effect on the treated (ATT) is captured by $\beta$, which is the difference in the average effect of raising the MW floor on manufacturing industry output, output per hour and output per worker in treated counties relative to adjacent control counties. The results are reported in Table~\ref{tab:baseline-industry-output}. Following an MW policy, I document large statistically significant increases in manufacturing industry outputs by $12.50ppts$ in treated counties. Specifically, the output per hour and labour productivity rose by $14.4$ and $12.7$ (ppts), respectively. Thus, suggesting that for every $\$0.89/hr$ increase in wages, output per hour and labour productivity rise by those margins.~\footnote{\tiny Back of the envelop calculation shows that manufacturing industry output per hour and labour productivity increases by an additional $0.144 \cdot 2.62 \cdot \left(\frac{\$100m}{\$1m}\right) = \$37.73$ units per hour and $0.117 \cdot 3.65 \cdot \left(\frac{\$100m}{\$1000}\right) = \$42,705$ units per worker, respectively.} The cohort-specific effect reveals significant and substantially strong increases in total output, output per hour and output per worker across all cohorts, except for the $2017$ cohort. The decline in overall output for the $2017$ cohort is explained by the decline in their hours and employment since workers are less likely to commute to distant higher MW counties/states.

    Figure~\ref{fig:baseline-industry-output} records consistent results. The MW policy caused instantaneous significant increases in manufacturing industry outputs, output per hour and output per worker in treated counties relative to adjacent control counties. The effects on output per hour and output per worker persists throughout the spectrum. Importantly, there is no evidence of significant pre-trends.
    \begin{figure}[H]
    \centering
    \includegraphics[width=1\textwidth, keepaspectratio]{C:/Users/david/OneDrive/Documents/ULMS/PhD/Thesis/chapter3/src/climate_change/latex/fig_sdid_output}
    \caption{Manufacturing Industry Output: Output per Hour and Output per Worker}
    \label{fig:baseline-industry-output}
    \begin{minipage}{\columnwidth}
        \vspace{0.05in}
        \tiny NOTES: The event study model of equation~\ref{eq:baseline-wages} is $Y_{i,cp,t} = \sum_{{e = -3},{e \neq -1}}^{3} \beta Treated_{s,t}^e = \textbf{1}[t - G_{s,t}] + \delta X_{v,c,t-1} + \omega F_{f,t} + \lambda_{t} + \sigma_{c} + \phi_{cp} + \zeta_{cp,t} + \epsilon_{i,cp,t}$. Standard errors are clustered at the state level. de Chaisemartin and D'Haultfoeuille Decomposition: $\sum dCDH_{ATTs}^{weights(+)} = 1$ and $\sum dCDH_{ATTs}^{weights(-)} = 0$.
    \end{minipage}
\end{figure}

    \subsection{Baseline Robustness}\label{subsec:baseline-robustness}
    I conduct several robustness exercises to confirm the robustness of the baseline industry results presented above.

    \subsubsection{Standard Errors} Standard errors are clustered at the facility, zipcode, industry NAICS codes, and county levels. I document that the results are not sensitive to these alternative clustering. See Tables~\ref{tab:baseline-cost-robustness},~\ref{tab:baseline-employ-robustness} and~\ref{tab:baseline-output-robustness}.
    % Please add the following required packages to your document preamble:
% \usepackage{booktabs}
% \usepackage{graphicx}
\begin{table}[H]
    \centering
    \caption{Manufacturing Industry Costs: Alternative Clustering of the SEs}
    \label{tab:baseline-cost-robustness}
    \resizebox{\columnwidth}{!}{%
        \begin{tabular}{@{}lllllllllllll@{}}
            \toprule  \toprule
            & \multicolumn{4}{c}{Hourly Wage} & \multicolumn{4}{c}{Total Payroll (log)} & \multicolumn{4}{c}{Material Cost (log)} \\
            \cmidrule(lr){2-5} \cmidrule(lr){6-9} \cmidrule(lr){10-13}
            industry costs    & 1         & 2         & 3         & 4         & 5         & 6         & 7         & 8         & 9         & 10        & 11        & 12        \\ \midrule
            $Treated^{e}$     & 0.889**   & 0.889**   & 0.889**   & 0.889**   & 0.043*    & 0.043     & 0.043     & 0.043     & 0.129*    & 0.129*    & 0.129*    & 0.129*    \\
            & (0.449)   & (0.404)   & (0.426)   & (0.415)   & (0.026)   & (0.038)   & (0.034)   & (0.037)   & (0.071)   & (0.069)   & (0.072)   & (0.072)   \\
            cohort 2014       & 1.200*    & 1.200**   & 1.200*    & 1.200*    & 0.004     & 0.004     & 0.004     & 0.004*    & 0.187*    & 0.187*    & 0.187*    & 0.187*    \\
            & (0.707)   & (0.593)   & (0.656)   & (0.629)   & (0.037)   & (0.049)   & (0.044)   & (0.049)   & (0.108)   & (0.107)   & (0.106)   & (0.106)   \\
            cohort 2015       & 0.382**   & 0.382     & 0.382     & 0.382     & 0.115***  & 0.115*    & 0.115**   & 0.115*    & 0.035     & 0.035     & 0.035     & 0.035     \\
            & (0.192)   & (0.379)   & (0.314)   & (0.357)   & (0.031)   & (0.065)   & (0.053)   & (0.058)   & (0.060)   & (0.077)   & (0.074)   & (0.074)   \\
            cohort 2017       & -0.208    & -0.208    & -0.208    & -0.208    & -0.612*** & -0.612*** & -0.612*** & -0.612*** & -0.343*** & -0.343*** & -0.343*** & -0.343*** \\
            & (0.436)   & (1.270)   & (1.000)   & (0.996)   & (0.166)   & (0.199)   & (0.199)   & (0.200)   & (0.131)   & (0.212)   & (0.181)   & (0.180)   \\
            controls          & Yes       & Yes       & Yes       & Yes       & Yes       & Yes       & Yes       & Yes       & Yes       & Yes       & Yes       & Yes       \\
            year FE           & Yes       & Yes       & Yes       & Yes       & Yes       & Yes       & Yes       & Yes       & Yes       & Yes       & Yes       & Yes       \\
            county FE         & Yes       & Yes       & Yes       & Yes       & Yes       & Yes       & Yes       & Yes       & Yes       & Yes       & Yes       & Yes       \\
            border-county FE  & Yes       & Yes       & Yes       & Yes       & Yes       & Yes       & Yes       & Yes       & Yes       & Yes       & Yes       & Yes       \\
            border-county LTs & Yes       & Yes       & Yes       & Yes       & Yes       & Yes       & Yes       & Yes       & Yes       & Yes       & Yes       & Yes       \\\midrule
            clustered at the: & county    & industry  & zipcode   & facility  & county    & industry  & zipcode   & facility  & county    & industry  & zipcode   & facility  \\
            Observations      & 1,893,689 & 1,893,689 & 1,893,689 & 1,893,689 & 1,893,689 & 1,893,689 & 1,893,689 & 1,893,689 & 1,893,689 & 1,893,689 & 1,893,689 & 1,893,689 \\
            $R^2$             & 0.624     & 0.624     & 0.624     & 0.624     & 0.440     & 0.440     & 0.440     & 0.440     & 0.619     & 0.619     & 0.619     & 0.619     \\ \bottomrule \bottomrule
        \end{tabular}%
    }
    \begin{minipage}{\columnwidth}
        \vspace{0.05in}
        \tiny NOTES: These results are obtained from estimating model~\ref{eq:baseline-emp-hours}. Robust standard errors clustered at the state level are reported in parentheses. ***, **, and * denote significance levels at the less than $1\%$, $5\%$ and $10\%$, respectively.
    \end{minipage}
\end{table}
    % Please add the following required packages to your document preamble:
% \usepackage{booktabs}
% \usepackage{graphicx}
\begin{table}[H]
    \centering
    \caption{Employment and Hours: Alternative Clustering of the SEs}
    \label{tab:baseline-employ-robustness}
    \resizebox{\columnwidth}{!}{%
        \begin{tabular}{@{}lllllllllllll@{}}
            \toprule\toprule
            & \multicolumn{4}{c}{Employment} & \multicolumn{4}{c}{Production Workers} & \multicolumn{4}{c}{Production Hours} \\
            \cmidrule(lr){2-5} \cmidrule(lr){6-9} \cmidrule(lr){10-13}
            employment/hours (log) & 1         & 2         & 3         & 4         & 5         & 6         & 7         & 8         & 9         & 10        & 11        & 12        \\ \midrule
            $Treated^{e}$          & -0.002    & -0.002    & -0.002    & -0.002    & -0.023    & -0.023    & -0.023    & -0.023    & -0.019    & -0.019    & -0.019    & -0.019    \\
            & (0.024)   & (0.037)   & (0.033)   & (0.036)   & (0.030)   & (0.039)   & (0.038)   & (0.039)   & (0.032)   & (0.039)   & (0.038)   & (0.040)   \\
            cohort 2014            & -0.058*   & -0.058    & -0.058    & -0.058    & -0.097**  & -0.097**  & -0.097**  & -0.097*   & -0.094**  & -0.094**  & -0.094*   & -0.094*   \\
            & (0.034)   & (0.043)   & (0.041)   & (0.045)   & (0.044)   & (0.046)   & (0.049)   & (0.052)   & (0.047)   & (0.047)   & (0.051)   & (0.054)   \\
            cohort 2015            & 0.095***  & 0.095     & 0.095*    & 0.095     & 0.104***  & 0.104     & 0.104*    & 0.104*    & 0.111***  & 0.111*    & 0.111*    & 0.111*    \\
            & (0.030)   & (0.068)   & (0.056)   & (0.062)   & (0.034)   & (0.068)   & (0.059)   & (0.063)   & (0.036)   & (0.066)   & (0.058)   & (0.061)   \\
            cohort 2017            & -0.552*** & -0.552*** & -0.552*** & -0.552*** & -0.568*** & -0.568**  & -0.568**  & -0.568**  & -0.556*** & -0.556**  & -0.556**  & -0.556**  \\
            & (0.179)   & (0.208)   & (0.211)   & (0.212)   & (0.208)   & (0.244)   & (0.241)   & (0.241)   & (0.191)   & (0.221)   & (0.229)   & (0.227)   \\
            controls               & Yes       & Yes       & Yes       & Yes       & Yes       & Yes       & Yes       & Yes       & Yes       & Yes       & Yes       & Yes       \\
            year FE                & Yes       & Yes       & Yes       & Yes       & Yes       & Yes       & Yes       & Yes       & Yes       & Yes       & Yes       & Yes       \\
            county FE              & Yes       & Yes       & Yes       & Yes       & Yes       & Yes       & Yes       & Yes       & Yes       & Yes       & Yes       & Yes       \\
            border-county FE       & Yes       & Yes       & Yes       & Yes       & Yes       & Yes       & Yes       & Yes       & Yes       & Yes       & Yes       & Yes       \\
            border-county LTs      & Yes       & Yes       & Yes       & Yes       & Yes       & Yes       & Yes       & Yes       & Yes       & Yes       & Yes       & Yes       \\ \midrule
            clustered at the:      & county    & industry  & zipcode   & facility  & county    & industry  & zipcode   & facility  & county    & industry  & zipcode   & facility  \\
            Observations           & 1,893,689 & 1,893,689 & 1,893,689 & 1,893,689 & 1,893,689 & 1,893,689 & 1,893,689 & 1,893,689 & 1,893,689 & 1,893,689 & 1,893,689 & 1,893,689 \\
            $R^2$                  & 0.393     & 0.393     & 0.393     & 0.393     & 0.378     & 0.378     & 0.378     & 0.378     & 0.385     & 0.385     & 0.385     & 0.385     \\ \bottomrule \bottomrule
        \end{tabular}%
    }
    \begin{minipage}{\columnwidth}
        \vspace{0.05in}
        \tiny NOTES: These results are obtained from estimating model~\ref{eq:baseline-emp-hours}. Robust standard errors clustered at the state level are reported in parentheses. ***, **, and * denote significance levels at the less than $1\%$, $5\%$ and $10\%$, respectively.
    \end{minipage}
\end{table}
    % Please add the following required packages to your document preamble:
% \usepackage{booktabs}
% \usepackage{graphicx}
\begin{table}[H]
    \centering
    \caption{Output and Labour Productivity: Alternative Clustering of the SEs}
    \label{tab:baseline-output-robustness}
    \resizebox{\columnwidth}{!}{%
        \begin{tabular}{@{}lllllllllllll@{}}
            \toprule\toprule
            Output (log) & \multicolumn{4}{c}{Output} & \multicolumn{4}{c}{Output per Hour} & \multicolumn{4}{c}{Output per Worker} \\
            \cmidrule(lr){2-5} \cmidrule(lr){6-9} \cmidrule(lr){10-13} & 1         & 2         & 3         & 4         & 5         & 6         & 7         & 8         & 9         & 10        & 11        & 12        \\ \midrule
            $Treated^{e}$                                              & 0.125***  & 0.125**   & 0.125***  & 0.125**   & 0.144***  & 0.144***  & 0.144***  & 0.144***  & 0.127***  & 0.127***  & 0.127***  & 0.127***  \\
            & (0.038)   & (0.051)   & (0.047)   & (0.049)   & (0.044)   & (0.045)   & (0.048)   & (0.047)   & (0.037)   & (0.041)   & (0.043)   & (0.043)   \\
            cohort 2014                                                & 0.122**   & 0.122*    & 0.122*    & 0.122*    & 0.216***  & 0.216***  & 0.216***  & 0.216***  & 0.180***  & 0.180***  & 0.180***  & 0.180***  \\
            & (0.057)   & (0.071)   & (0.067)   & (0.070)   & (0.065)   & (0.066)   & (0.073)   & (0.071)   & (0.055)   & (0.059)   & (0.066)   & (0.064)   \\
            cohort 2015                                                & 0.135***  & 0.135*    & 0.135**   & 0.135**   & 0.024     & 0.024     & 0.024     & 0.024     & 0.039     & 0.039     & 0.039     & 0.039     \\
            & (0.037)   & (0.073)   & (0.059)   & (0.061)   & (0.043)   & (0.046)   & (0.043)   & (0.045)   & (0.032)   & (0.042)   & (0.038)   & (0.041)   \\
            cohort 2017                                                & -0.447*** & -0.447**  & -0.447**  & -0.447**  & 0.108*    & 0.108     & 0.108     & 0.108     & 0.105**   & 0.105*    & 0.105     & 0.105     \\
            & (0.133)   & (0.212)   & (0.180)   & (0.180)   & (0.060)   & (0.069)   & (0.099)   & (0.098)   & (0.048)   & (0.062)   & (0.092)   & (0.088)   \\
            controls                                                   & Yes       & Yes       & Yes       & Yes       & Yes       & Yes       & Yes       & Yes       & Yes       & Yes       & Yes       & Yes       \\
            year FE                                                    & Yes       & Yes       & Yes       & Yes       & Yes       & Yes       & Yes       & Yes       & Yes       & Yes       & Yes       & Yes       \\
            county FE                                                  & Yes       & Yes       & Yes       & Yes       & Yes       & Yes       & Yes       & Yes       & Yes       & Yes       & Yes       & Yes       \\
            border-county FE                                           & Yes       & Yes       & Yes       & Yes       & Yes       & Yes       & Yes       & Yes       & Yes       & Yes       & Yes       & Yes       \\
            border-county LTs                                          & Yes       & Yes       & Yes       & Yes       & Yes       & Yes       & Yes       & Yes       & Yes       & Yes       & Yes       & Yes       \\ \midrule
            clustered at the:                                          & county    & industry  & zipcode   & facility  & county    & industry  & zipcode   & facility  & county    & industry  & zipcode   & facility  \\
            Observations                                               & 1,893,689 & 1,893,689 & 1,893,689 & 1,893,689 & 1,893,689 & 1,893,689 & 1,893,689 & 1,893,689 & 1,893,689 & 1,893,689 & 1,893,689 & 1,893,689 \\
            $R^2$                                                      & 0.549     & 0.549     & 0.549     & 0.549     & 0.630     & 0.630     & 0.630     & 0.630     & 0.648     & 0.648     & 0.648     & 0.648     \\ \bottomrule\bottomrule
        \end{tabular}%
    }
    \begin{minipage}{\columnwidth}
        \vspace{0.05in}
        \tiny NOTES: These results are obtained from estimating model~\ref{eq:baseline-output}. Robust standard errors clustered at the state level are reported in parentheses. ***, **, and * denote significance levels at the less than $1\%$, $5\%$ and $10\%$, respectively.
    \end{minipage}
\end{table}

    \subsubsection{State-level Results} I repeat all the above analysis at the state-level. The results persist even at this level, and are presented in the online supplementary material.

    I have established that raising MW indeed increases manufacturing industry labour cost, cost of materials and outputs. Further, the hypothesis of null overall employment effects subsists with heterogeneity in specific cohorts. While cross-county worker mobility is responsible for the disemployment effects, the positive employment effect for the $2015$ cohort is entirely due to hiring of more high-skilled workers caused by higher MW policy.

    In what follows, I examine the following topical questions: $(i)$ is this increased cost-burden due to higher MW passed onto the environment, in terms of higher pollution emissions per $\$100m$ units of output? $(ii)$ is the environmental impact heterogeneous across cohorts and manufacturing industries?


    \section{Onsite Toxic Releases}\label{sec:onsite-toxic-releases}
    I begin by estimating the effect of raising MW on total onsite toxic releases intensity generated at manufacturing industry facilities. The baseline model is given by:
    \begin{equation}
        P_{f,cp,c,t} = \beta Treated_{s,t}^e + \delta X_{v,c,t-1} + \omega F_{f,t} + \lambda_{t} + \gamma_{f} + \phi_{cp} + \zeta_{c} + \eta_{c,t} + \theta_{cp,t} + \varepsilon_{f,cp,c,t},\label{eq:baseline-total-onsite-releases-intensity}
    \end{equation}
    where $P_{f,cp,c,t}$ is the total onsite releases intensity at manufacturing industry facility, $f$ in cross-border county pairs, $cp$ through toxic chemical use, $c$ in the year, $t$. Total onsite releases intensity is the sum of pounds weights (lbs) of total onsite toxic air emissions, land releases, and surface water discharge intensities. $Treated_{s,t}^e = \textbf{1}[t - P_{s,t}]$, $X_{v,c,t-1}$ and $\lambda_{t}$ are as defined in subsection~\ref{subsec:baseline-results-industry-costs}. $F_{f,t}$ contains facility-level variables regarding the maximum number of toxic chemicals at the facility at any point in time, production/activity ratio index of the facility, and the following dummies: industry ownership, and toxic chemical attributes, that is, whether the chemical was produced at or imported to the facility, used as a formulation or article component, used as a manufacturing aid or for ancillary purposes. Further I control for environmental policy variables such as the clean air act (CAA) of $1970$ and amended in $1990$ to capture hazardous air pollutants (HAPs), proxied by the dummy of CAA and HAPs regulated chemicals, and the toxic substances control act (TSCA) of $1976$ proxied by the dummy of persistent bio-accumulative chemicals. I control for facility-level fixed effect, $\gamma_{f}$ to account for within facility-level differences in the management of toxic releases intensities at the facility; cross-border county pairs, $\phi_{cp}$ fixed effect as defined in subsection~\ref{subsec:baseline-results-industry-costs}. $\zeta_{c}$ is the toxic chemical-fixed effect that controls for within chemical usage or mixture or compound differences in the production functions of manufacturing facilities; and $\eta_{c,t}$ toxic chemical linear trends which controls for the evolution of common time varying shocks that may affect the nature of chemical usage at manufacturing industry facilities. $\theta_{cp,t}$ is the border-county linear trends to control for common time varying shocks that affect border counties such as change in county governments and national economic conditions. Finally, $\varepsilon_{f,cp,c,t}$ is the idiosyncratic error term. I employ a three-way clustering of the standard errors at the chemical use, industry, and state levels, since the changes in MW may be correlated within a state, within manufacturing industries in the state, and in the nature of toxic chemical usage in the industry.
    % Please add the following required packages to your document preamble:
% \usepackage{booktabs}
% \usepackage{graphicx}
\begin{table}[H]
    \centering
    \caption{Effect of the MW policy on Total Onsite Toxic Releases Intensity}
    \label{tab:baseline-total-onsite-releases-intensity}
    \resizebox{\columnwidth}{!}{%
        \begin{tabular}{@{}llll@{}}
            \toprule\toprule
            Total releases intensity (log) & 1         & 2         & 3         \\ \midrule
            $Treated^{e}$                  & 0.120**   & 0.120**   & 0.109**   \\
            & (0.052)   & (0.052)   & (0.049)   \\
            cohort 2014                    & 0.077     & 0.077     & 0.090**   \\
            & (0.055)   & (0.055)   & (0.044)   \\
            cohort 2015                    & 0.191***  & 0.191***  & 0.139*    \\
            & (0.073)   & (0.073)   & (0.077)   \\
            cohort 2017                    & 0.030     & 0.030     & 0.223**   \\
            & (0.061)   & (0.061)   & (0.090)   \\
            controls                       & Yes       & Yes       & Yes       \\
            year FE                        & Yes       & Yes       & Yes       \\
            facility FE                    & Yes       & Yes       & Yes       \\
            border-county FE               & No        & Yes       & Yes       \\
            toxic chemical FE              & No        & No        & Yes       \\
            toxic chemical LTs             & No        & No        & Yes       \\\midrule
            Observations                   & 1,893,689 & 1,893,689 & 1,893,689 \\
            $R^2$                          & 0.520     & 0.520     & 0.720     \\
            Baseline Mean                  & 87.99     & 87.99     & 87.99     \\ \bottomrule\bottomrule
        \end{tabular}%
    }
    \begin{minipage}{18cm}
        \vspace{0.05in}
        These results are obtained from estimating model~\ref{eq:baseline-total-onsite-releases-intensity}. Three-way clustered robust standard errors are reported in parentheses, and clustered at the toxic chemical, industry and state levels. ***, **, and * denote significance levels at the less than $1\%$, $5\%$ and $10\%$, respectively.
    \end{minipage}
\end{table}

    The overall and cohort-specific average treatment effect on the treated (ATT) is captured by $\beta$, which is the difference in the average effect of raising the MW floor on total onsite releases intensity at manufacturing industry facilities in treated counties relative to adjacent control counties. The results on total onsite releases intensity are reported in Table~\ref{tab:baseline-total-onsite-releases-intensity}. The results show that a higher MW policy increases total onsite releases per $\$100m$ units of manufacturing industry output by $11.9ppts$ in treated counties relative to adjacent control counties. This translates to a $10.47lbs$ additional increase in total onsite releases intensity for every $\$0.89/hr$ increase in the wages of manufacturing industry workers.~\footnote{\tiny Alternatively, this means that for every $\$14.41ppts$ and $12.73ppts$ increase in the manufacturing industry output per hour and per worker, respectively, due to higher MW, total onsite releases intensity increases by $9.55lbs$.} Furthermore, the cohort-specific ATTs corroborate this strong increase in total onsite releases intensity, but strongest for the $2014$ and $2015$ cohorts. This effect persists even after controlling for time-varying common shocks that may affect border counties and the toxic chemical usage in the manufacturing industry.
    \begin{figure}[H]
    \centering
    \includegraphics[width=1\textwidth, height=0.5\textheight,keepaspectratio]{fig_sdid_total_releases_onsite_int}
    \caption{Total Onsite Releases Intensity}
    \label{fig:baseline-total-onsite-releases-intensity}
    \begin{minipage}{18cm}
        \vspace{0.05in}
        NOTES: The event study model of equation~\ref{eq:baseline-total-onsite-releases-intensity} is $P_{f,cp,c,t} = \sum_{{e = -3},{e \neq -1}}^{3} \beta Treated_{s,t}^e + \delta X_{v,c,t-1} + \omega F_{f,t} + \lambda_{t} + \gamma_{f} + \phi_{cp} + \zeta_{c} + \eta_{c,t} + \varepsilon_{f,cp,c,t}$. Three-way clustered robust standard errors are reported in parentheses, and clustered at the toxic chemical, industry and state levels. The test for the presence of pre-trends shows and F-statistic of $0.1461$ $(0.9296)$, p-value (in parentheses) using F-Statistic $(\chi^2): \sum_{-3}^{-2} \beta_{e} = 0$. de Chaisemartin and D'Haultfoeuille Decomposition: $\sum dCDH_{ATTs}^{weights(+)} = 1$ and $\sum dCDH_{ATTs}^{weights(-)} = 0$.
    \end{minipage}
\end{figure}

    Moreover, Figure~\ref{fig:baseline-total-onsite-releases-intensity} shows an instantaneous increase of $7.7ppts$ in total onsite releases intensity, reaching $23.5ppts$ in two years after the initial raise in the MW and then declined to $13.2ppts$ in the third year, for treated counties relative to adjacent control counties. There is no significant evidence of pre-trends. As a robustness check, the ATTs from the alternative TWFE estimator show consistent effects.

    \subsection{Air Emissions Intensity}\label{subsec:air-emission-intensity}
    Here, I estimate the effect of raising MW on onsite air emission intensity. The model is given by:
    \begin{equation}
        A_{f,cp,c,t} = \beta Treated_{s,t}^e + \delta X_{v,c,t-1} + \omega F_{f,t} + \lambda_{t} + \gamma_{f} + \phi_{cp} + \zeta_{c} + \eta_{c,t} + \theta_{cp,t} + \varepsilon_{f,cp,c,t},\label{eq:baseline-onsite-air-emission-intensity}
    \end{equation}
    where $A_{f,cp,c,t}$ is the total onsite air emission intensity from manufacturing industry facility, $f$ in cross-border county pairs, $cp$ through toxic chemical, $c$ in the year, $t$. Total onsite air emission intensity is the sum of point/stack air emissions and fugitive air emissions. Point air emissions usually involve the release of pollutants or substances into the atmosphere from industrial or commercial sources through a designated stack, chimney, or venting mechanism. These emissions occur as a result of combustion processes, industrial operations, or other activities that involve the burning, processing, or handling of materials. Stack emissions are typically regulated and monitored to ensure compliance with environmental regulations and standards. Fugitive air emissions on the other hand, refer to the release of pollutants or substances into the atmosphere from various industrial, commercial, or other sources that are not captured by a stack/point, duct, or other venting mechanism. These emissions typically occur during the handling, storage, processing, or transportation of materials and can originate from leaks, spills, evaporation, or other unintended releases. Examples of emitted pollutants include volatile organic compounds, hazardous air pollutants, and particulate matter, \textit{inter alia}. Monitoring and controlling stack emissions are critical for minimizing air pollution, protecting public health, and reducing environmental impacts.
    % Please add the following required packages to your document preamble:
% \usepackage{booktabs}
% \usepackage{graphicx}
\begin{table}[H]
    \centering
    \caption{Effect of the MW policy on Air Emissions Intensity}
    \label{tab:baseline-onsite-air-emissions-intensity}
    \resizebox{\columnwidth}{!}{%
        \begin{tabular}{@{}lllllll@{}}
            \toprule\toprule
            & \multicolumn{2}{c}{Total} & \multicolumn{2}{c}{Point} & \multicolumn{2}{c}{Fugitive} \\
            \cmidrule(lr){2-3}  \cmidrule(lr){4-5} \cmidrule(lr){6-7}
            Air emissions intensity (log) & 1         & 2         & 3         & 4         & 5         & 6         \\ \midrule
            $Treated^{e}$                 & 0.101**   & 0.088**   & 0.061*    & 0.035     & 0.076     & 0.072*    \\
            & (0.047)   & (0.039)   & (0.036)   & (0.026)   & (0.050)   & (0.041)   \\
            cohort 2014                   & 0.100*    & 0.097**   & 0.107***  & 0.104***  & 0.018     & -0.000    \\
            & (0.053)   & (0.042)   & (0.047)   & (0.038)   & (0.051)   & (0.040)   \\
            cohort 2015                   & 0.103     & 0.122     & -0.014    & -0.081    & 0.175**   & 0.192***  \\
            & (0.067)   & (0.076)   & (0.072)   & (0.050)   & (0.042)   & (0.062)   \\
            cohort 2017                   & -0.045    & 0.076*    & 0.001     & 0.154**   & -0.087**  & -0.009    \\
            & (0.064)   & (0.040)   & (0.062)   & (0.073)   & (0.042)   & (0.051)   \\
            controls                      & Yes       & Yes       & Yes       & Yes       & Yes       & Yes       \\
            year FE                       & Yes       & Yes       & Yes       & Yes       & Yes       & Yes       \\
            facility FE                   & Yes       & Yes       & Yes       & Yes       & Yes       & Yes       \\
            border-county FE              & Yes       & Yes       & Yes       & Yes       & Yes       & Yes       \\
            toxic chemical FE             & No        & Yes       & No        & Yes       & No        & Yes       \\
            toxic chemical LTs            & No        & Yes       & No        & Yes       & No        & Yes       \\ \midrule
            Observations                  & 1,893,689 & 1,893,689 & 1,893,689 & 1,893,689 & 1,893,689 & 1,893,689 \\
            $R^2$                         & 0.522     & 0.739     & 0.516     & 0.712     & 0.472     & 0.660     \\
            Baseline Mean                 & 60.02     & 60.02     & 49.07     & 49.07     & 10.95     & 10.95     \\ \bottomrule\bottomrule
        \end{tabular}%
    }
    \begin{minipage}{\columnwidth}
        \vspace{0.05in}
        NOTES: These results are obtained from estimating model~\ref{eq:baseline-onsite-air-emission-intensity}. Three-way clustered robust standard errors are reported in parentheses, and clustered at the toxic chemical, industry and state levels. ***, **, and * denote significance levels at the less than $1\%$, $5\%$ and $10\%$, respectively.
    \end{minipage}
\end{table}

    The parameter of interest is $\beta$ which measures the overall and cohort-specific ATT of the MW policy on air emissions intensity at manufacturing industry facilities in treated counties relative to adjacent control counties. The results are presented in Table~\ref{tab:baseline-onsite-air-emissions-intensity}. I find an increase of $10.2ppts$ in total air emissions of toxic chemicals per $\$100m$ units of manufacturing industry outputs in treated counties relative to adjacent control counties. Albeit the reported effect size is smaller than the size of the effect in~\cite{zhang2023unintended}, the result is similar to their conclusions for Chinese manufacturing industries. The focus on only specific air emitted toxic chemicals may provide tangible insights regarding the distance in the effect size here relative to the Chinese paper. Back of the envelop calculation shows that for every $\$0.89$ increase in wages per hour, air emissions intensity increases by $6.12lbs$ per $\$100m$ units in manufacturing industry outputs. Further, the reported effect is predominantly driven by both point and fugitive air emissions intensities, strongest for the $2014$ and $2015$ cohorts, respectively.

    Moreover, the results of the dynamic effects in Figure~\ref{fig:baseline-onsite-air-emission-intensity} show an instantaneous increase of $5.7ppts$ in total air emissions intensity, reaching $16.6ppts$ two years after the initial MW raise and then dropping to $15.8ppts$ in the third year. Similarly I find the significant increases in point air emission intensity between the second and third year of post treatment, while that of fugitive air emission intensity is significant only in the first and second year after the initial MW raise. No significant pre-trends are recorded.
    \begin{figure}[H]
    \centering
    \includegraphics[width=1\textwidth, height=0.5\textheight,keepaspectratio]{C:/Users/david/OneDrive/Documents/ULMS/PhD/Thesis/chapter3/src/climate_change/latex/fig_sdid_onsite_air_emissions_int}
    \caption{Total Onsite Releases Intensity}
    \label{fig:baseline-onsite-air-emission-intensity}
    \begin{minipage}{\columnwidth}
        \vspace{0.05in}
        \tiny NOTES: The event study model of equation~\ref{eq:baseline-onsite-air-emission-intensity} is $A_{f,cp,c,t} = \sum_{{e = -3},{e \neq -1}}^{3} \beta Treated_{s,t}^e + \omega F_{f,t} + \lambda_{t} + \gamma_{f} + \phi_{cp} + \zeta_{c} + \eta_{c,t} + \theta_{cp,t} + \varepsilon_{f,cp,c,t}$. Three-way clustered robust standard errors are reported in parentheses, and clustered at the toxic chemical, industry and state levels. de Chaisemartin and D'Haultfoeuille Decomposition: $\sum dCDH_{ATTs}^{weights(+)} = 1$ and $\sum dCDH_{ATTs}^{weights(-)} = 0$.
    \end{minipage}
\end{figure}

    \subsection{Water Discharge Intensity}\label{subsec:water-discharge-intensity}
    Next, I estimate the effect of raising MW on total onsite surface water discharge intensity. The model is given by:
    \begin{equation}
        W_{f,cp,c,t} = \beta Treated_{s,t}^e + \delta X_{v,c,t-1} + \omega F_{f,t} + \lambda_{t} + \gamma_{f} + \phi_{cp} + \zeta_{c} + \eta_{c,t} + \theta_{cp,t} + \varepsilon_{f,cp,c,t},\label{eq:baseline-onsite-water-discharge-intensity}
    \end{equation}
    where $W_{f,cp,c,t}$ is the total onsite surface water discharge intensity from a manufacturing industry facility, $f$ in cross-border county pairs, $cp$ through a toxic chemical, $c$ in the year, $t$. Surface water discharge intensity is the total weight of toxic chemicals per $\$100m$ units of output in the form of contaminants, or wastewater from industrial or commercial facilities that are released into surface water bodies such as streams. This discharge can occur through various pathways, including direct discharge through pipes, drains, or outfalls, as well as through runoff from facility surfaces or surrounding areas. Surface water discharge at facilities is regulated by environmental agencies (the clean water act of $1972$) to protect water quality, safeguard human health, and preserve aquatic ecosystems. Facilities are often required to obtain permits, monitor their discharge, and implement pollution control measures to minimize the impact of their activities on surface water resources.
    % Please add the following required packages to your document preamble:
% \usepackage{booktabs}
% \usepackage{graphicx}
\begin{table}[H]
    \centering
    \caption{Effect of the MW policy on Onsite Surface Water Discharge Intensity}
    \label{tab:baseline-onsite-water-disc-int}
    \resizebox{\columnwidth}{!}{%
        \begin{tabular}{@{}lllll@{}}
            \toprule\toprule
            Surface water discharge intensity (log) & \multicolumn{2}{c}{Total} & \multicolumn{2}{c}{Number of Receiving Streams} \\
            \cmidrule(lr){2-3}  \cmidrule(lr){4-5}
            & 1         & 2         & 3         & 4         \\ \midrule
            $Treated^{e}$      & 0.028     & 0.026     & -0.007    & -0.013    \\
            & (0.032)   & (0.033)   & (0.010)   & (0.011)   \\
            cohort 2014        & -0.037    & -0.028    & -0.019    & 0.029*    \\
            & (0.024)   & (0.025)   & (0.014)   & (0.015)   \\
            cohort 2015        & 0.136**   & 0.117**   & 0.012     & 0.014     \\
            & (0.064)   & (0.059)   & (0.011)   & (0.013)   \\
            cohort 2017        & 0.069**   & 0.073**   & 0.006     & 0.003     \\
            & (0.032)   & (0.033)   & (0.034)   & (0.035)   \\
            controls           & Yes       & Yes       & Yes       & Yes       \\
            year FE            & Yes       & Yes       & Yes       & Yes       \\
            facility FE        & Yes       & Yes       & Yes       & Yes       \\
            border-county FE   & Yes       & Yes       & Yes       & Yes       \\
            toxic chemical FE  & No        & Yes       & No        & Yes       \\
            toxic chemical LTs & No        & Yes       & No        & Yes       \\\midrule
            Observations       & 1,893,689 & 1,893,689 & 1,893,689 & 1,893,689 \\
            $R^2$              & 0.297     & 0.585     & 0.695     & 0.792     \\
            Baseline Mean      & 20.04     & 20.04     & 0.39      & 0.39      \\ \bottomrule\bottomrule
        \end{tabular}%
    }
    \begin{minipage}{18cm}
        \vspace{0.05in}
        NOTES: These results are obtained from estimating model~\ref{eq:baseline-onsite-water-discharge-intensity}. Three-way clustered robust standard errors are reported in parentheses, and clustered at the toxic chemical, industry and state levels. ***, **, and * denote significance levels at the less than $1\%$, $5\%$ and $10\%$, respectively.
    \end{minipage}
\end{table}

    The average treatment effect on the treated (ATT) is captured by $\beta$, which is the difference in the average effect of raising the MW floor on total surface water discharge intensity at manufacturing industry facilities in treated counties relative to adjacent control counties. The results on total surface water discharge are reported in Table~\ref{tab:baseline-onsite-water-disc-int}. I find no statistically significant differences in the overall effect of raising MW on surface water discharge intensity. However, I find heterogeneous effect between the  $2014$ and $2015$ cohorts. Whereas raising MW reduces surface water discharge for the $2014$ cohort, and opposite is documented for the $2015$ cohort. Similarly, the results in Figure~\ref{fig:baseline-onsite-water-discharge-intensity} jumped to $8.4ppts$ two years after the initial treatment. I find no evidence of significant pre-trends. However, the effect on the total number of receiving streams are not necessarily causal as there is significant pre-trends.
    \begin{figure}[H]
    \centering
    \includegraphics[width=1\textwidth, height=0.5\textheight,keepaspectratio]{C:/Users/david/OneDrive/Documents/ULMS/PhD/Thesis/chapter3/src/climate_change/latex/fig_sdid_onsite_water_discharge_int}
    \caption{Total Onsite Releases Intensity}
    \label{fig:baseline-onsite-water-discharge-intensity}
    \begin{minipage}{18cm}
        \vspace{0.05in}
        \tiny NOTES: The event study model of equation~\ref{eq:baseline-onsite-water-discharge-intensity} is $W_{f,cp,c,t} = \sum_{{e = -3},{e \neq -1}}^{3} \beta Treated_{s,t}^e + \delta X_{v,c,t-1} + \omega F_{f,t} + \lambda_{t} + \gamma_{f} + \phi_{cp} + \zeta_{c} + \eta_{c,t} + \theta_{cp,t} + \varepsilon_{f,cp,c,t}$. Three-way clustered robust standard errors are reported in parentheses, and clustered at the toxic chemical, industry and state levels. de Chaisemartin and D'Haultfoeuille Decomposition: $\sum dCDH_{ATTs}^{weights(+)} = 1$ and $\sum dCDH_{ATTs}^{weights(-)} = 0$.
    \end{minipage}
\end{figure}

    \subsection{Land Releases Intensity}\label{subsec:land-releases-intensity}
    Lastly, I estimate the effect of raising MW on total onsite land releases intensity. The model is given by:
    \begin{equation}
        L_{f,cp,c,t} = \beta Treated_{s,t}^e + \delta X_{v,c,t-1} + \omega F_{f,t} + \lambda_{t} + \gamma_{f} + \phi_{cp} + \zeta_{c} + \eta_{c,t} + \theta_{cp,t} + \varepsilon_{f,cp,c,t},\label{eq:baseline-onsite-land-releases-intensity}
    \end{equation}
    where $L_{f,cp,c,t}$ is the total onsite land releases intensity from manufacturing industry facility, $f$ in cross-border county pairs, $cp$ through toxic chemical, $c$ in the year, $t$. Total onsite land releases intensity is the total weight of toxic chemicals per $\$100m$ units of output that are released to land.
    % Please add the following required packages to your document preamble:
% \usepackage{booktabs}
% \usepackage{graphicx}
\begin{table}[H]
    \centering
    \caption{Effect of the MW policy on Onsite Land Releases Intensity}
    \label{tab:baseline-onsite-land-releases-intensity}
    \resizebox{\columnwidth}{!}{%
        \begin{tabular}{@{}lllllllllllll@{}}
            \toprule\toprule
            Land releases intensity (log) & \multicolumn{2}{c}{Total} & \multicolumn{2}{c}{Underground Injection} & \multicolumn{2}{c}{Landfills} & \multicolumn{2}{c}{To-Land Treatment} & \multicolumn{2}{c}{Surface Impoundment} & \multicolumn{2}{c}{Land Releases (Others)} \\
            \cmidrule(lr){2-3}  \cmidrule(lr){4-5}  \cmidrule(lr){6-7}  \cmidrule(lr){8-9}  \cmidrule(lr){10-11}  \cmidrule(lr){12-13}
            & 1         & 2         & 3         & 4         & 5         & 6         & 7         & 8         & 9         & 10        & 11        & 12        \\ \midrule
            $Treated^{e}$      & -0.025*   & -0.018    & 0.000     & 0.001     & -0.005    & -0.003    & 0.002     & 0.004     & 0.009*    & 0.015* & -0.030** & -0.034** \\
            & (0.015)   & (0.017)   & (0.000)   & (0.001)   & (0.003)   & (0.004)   & (0.007)   & (0.008)   & (0.005)   & (0.008) & (0.014) & (0.015)               \\
            cohort 2014        & -0.014    & -0.003    & -0.001    & 0.000     & -0.001    & -0.002    & 0.003     & 0.005     & 0.011*    & 0.020*    & -0.027* & -0.030* \\
            & (0.017)   & (0.021)   & (0.001)   & (0.000)   & (0.002)   & (0.003)   & (0.010)   & (0.012)   & (0.006)   & (0.009) & (0.015) & (0.016)               \\
            cohort 2015        & -0.042    & -0.044    & -0.001    & 0.000     & -0.012*   & -0.011    & 0.001     & 0.002     & 0.005*    & 0.007*    & -0.036 & -0.040 \\
            & (0.029)   & (0.033)   & (0.001)   & (0.000)   & (0.003)   & (0.009)   & (0.002)   & (0.005)   & (0.003)   & (0.004) & (0.030) & (0.034)               \\
            cohort 2017        & 0.025**   & 0.061**   & 0.002     & 0.002     & -0.001    & 0.002     & 0.000     & 0.000     & 0.014*    & 0.025*    & 0.008 & 0.032 \\
            & (0.011)   & (0.031)   & (0.001)   & (0.001)   & (0.002)   & (0.004)   & (0.002)   & (0.002)   & (0.008)   & (0.014) & (0.005) & (0.028)               \\
            controls           & Yes       & Yes       & Yes       & Yes       & Yes       & Yes       & Yes       & Yes       & Yes       & Yes       & Yes       & Yes       \\
            year FE            & Yes       & Yes       & Yes       & Yes       & Yes       & Yes       & Yes       & Yes       & Yes       & Yes       & Yes       & Yes       \\
            facility FE        & Yes       & Yes       & Yes       & Yes       & Yes       & Yes       & Yes       & Yes       & Yes       & Yes       & Yes       & Yes       \\
            border-county FE   & Yes       & Yes       & Yes       & Yes       & Yes       & Yes       & Yes       & Yes       & Yes       & Yes       & Yes       & Yes       \\
            toxic chemical FE  & No        & Yes       & No        & Yes       & No        & Yes       & No        & Yes       & No        & Yes       & No        & Yes       \\
            toxic chemical LTs & No        & Yes       & No        & Yes       & No        & Yes       & No        & Yes       & No        & Yes       & No        & Yes       \\\midrule
            Observations       & 1,893,689 & 1,893,689 & 1,893,689 & 1,893,689 & 1,893,689 & 1,893,689 & 1,893,689 & 1,893,689 & 1,893,689 & 1,893,689 & 1,893,689 & 1,893,689 \\
            $R^2$              & 0.345     & 0.500     & 0.380     & 0.382     & 0.438     & 0.466     & 0.312     & 0.323     & 0.080     & 0.126     & 0.234     & 0.592     \\
            Baseline Mean      & 7.93      & 7.93      & 4.80      & 4.80      & 1.43      & 1.43      & 0.66      & 0.66      & 0.03      & 0.03      & 1.01      & 1.01      \\ \bottomrule\bottomrule
        \end{tabular}%
    }
    \begin{minipage}{\columnwidth}
        \vspace{0.05in}
        These results are obtained from estimating model~\ref{eq:baseline-onsite-land-releases-intensity}. Three-way clustered robust standard errors are reported in parentheses, and clustered at the toxic chemical, industry and state levels. ***, **, and * denote significance levels at the less than $1\%$, $5\%$ and $10\%$, respectively.
    \end{minipage}
\end{table}

    The average treatment effect on the treated (ATT) is captured by $\beta$, which is the difference in the average effect of raising the MW floor on total surface water discharge intensity at manufacturing industry facilities in treated counties relative to adjacent control counties. The results on total land releases intensity are reported in Table~\ref{tab:baseline-onsite-land-releases-intensity}. Except for the total surface impoundment intensity, I find no statistically and economically significant effect on the intensities of total land releases, underground injection, landfills and to-land treatment (used for land fertilization). A surface impoundment is a type of containment structure used for the storage and management of liquid wastes, such as industrial wastewater, hazardous chemicals, or contaminated water. It consists of an excavated depression or basin lined with impermeable materials such as clay or synthetic liners to prevent the leakage of liquids into the surrounding soil and groundwater. The results show that a higher MW floor increases in total surface impoundment intensity by $1.1ppts$ in treated counties relative to adjacent control counties. Similarly, the cohort-specific effect shows an increase total land releases for the $2017$ cohort by $5.1ppts$ and increases in total surface impoundment intensity by$1.3ppts$ for the $2014$ cohort only. The event study shows an instantaneous increases surface impoundment intensity persisting throughout the post-treatment periods, reaching $3.5ppts$. I record no significant evidence of pre-trends. However, I found no causal evidence on the increases in other total land releases given significant pre-trends.
    \begin{figure}[H]
    \centering
    \includegraphics[width=1\textwidth, height=0.5\textheight,keepaspectratio]{fig_sdid_total_land_releases_onsite_int}
    \caption{Total Onsite Land Releases Intensity}
    \label{fig:baseline-onsite-land-releases-intensity}
    \begin{minipage}{\columnwidth}
        \vspace{0.05in}
        \tiny NOTES: The event study model of equation~\ref{eq:baseline-onsite-land-releases-intensity} is $L_{f,cp,c,t} = \sum_{{e = -3},{e \neq -1}}^{3} \beta Treated_{s,t}^e + \delta X_{v,c,t-1} + \omega F_{f,t} + \lambda_{t} + \gamma_{f} + \phi_{cp} + \zeta_{c} + \eta_{c,t} + \theta_{cp,t} + \varepsilon_{f,cp,c,t}$. Three-way clustered robust standard errors are reported in parentheses, and clustered at the toxic chemical, industry and state levels. de Chaisemartin and D'Haultfoeuille Decomposition: $\sum dCDH_{ATTs}^{weights(+)} = 1$ and $\sum dCDH_{ATTs}^{weights(-)} = 0$.
    \end{minipage}
\end{figure}

%    \section{Offsite and POTWs Toxic Releases}\label{sec:offsite-and-potws-toxic-releases}
%    \begin{figure}[H]
    \centering
    \includegraphics[width=1\textwidth, height=0.5\textheight,keepaspectratio]{fig_sdid_total_releases_offsite}
    \caption{Total Offsite Releases Intensity}
    \label{fig:baseline-offsite-total-releases-intensity}
    \begin{minipage}{12cm}
        \vspace{0.05in}
        NOTES: The event study model of equation~\ref{eq:baseline-offsite-total-releases-intensity} is $P_{f,c,i,cp,s,t}^{offsite} = \sum_{{e = 2011},{e \neq 2013}}^{2017} \beta Treated_{s,t}^e + \delta X_{v,c,t-1} + \omega F_{f,t} + \gamma_{f} + \phi_{cp} + \eta_{c,t} + \left[\lambda_{t} + \theta_{f,h} + \sigma_{s} + \zeta_{c} \right] + \varepsilon_{f,c,i,cp,s,t}$. Three-way clustered robust standard errors are reported in parentheses, and clustered at the toxic chemical, industry and state levels.
    \end{minipage}
\end{figure}
%    \begin{figure}[H]
    \centering
    \includegraphics[width=1\textwidth, height=0.5\textheight,keepaspectratio]{fig_sdid_total_land_releases_offsite}
    \caption{Total Offsite Total Releases Intensity}
    \label{fig:baseline-offsite-land-releases-intensity}
    \begin{minipage}{12cm}
        \vspace{0.05in}
        NOTES: The event study model of equation~\ref{eq:baseline-offsite-land-releases-intensity} is $L_{f,c,i,cp,s,t}^{offsite} = \sum_{{e = 2011},{e \neq 2013}}^{2017} \beta Treated_{s,t}^e + \delta X_{v,c,t-1} + \omega F_{f,t} + \gamma_{f} + \phi_{cp} + \eta_{c,t} + \left[\lambda_{t} + \theta_{f,h} + \sigma_{s} + \zeta_{c} \right] + \varepsilon_{f,c,i,cp,s,t}$. Three-way clustered robust standard errors are reported in parentheses, and clustered at the toxic chemical, industry and state levels.
    \end{minipage}
\end{figure}
%    \begin{figure}[H]
    \centering
    \includegraphics[width=1\textwidth, height=0.5\textheight,keepaspectratio]{C:/Users/david/OneDrive/Documents/ULMS/PhD/Thesis/chapter3/src/climate_change/latex/fig_sdid_total_releases_potws}
    \caption{Total POTWs Releases Intensity and Waste Management}
    \label{fig:baseline-potws-total-releases-intensity}
    \begin{minipage}{\columnwidth}
        \vspace{0.05in}
        \tiny NOTES: The event study model is $P_{f,cp,c,t}^{POTWs} = \sum_{{e = 2011},{e \neq 2013}}^{2017} \beta Treated_{s,t}^e + \delta X_{v,c,t-1} + \omega F_{f,t} + \lambda_{t} + \gamma_{f} + \phi_{cp} + \zeta_{c} + \eta_{c,t} + \theta_{cp,t} + \varepsilon_{f,cp,c,t}$. Three-way clustered robust standard errors are reported in parentheses, and clustered at the toxic chemical, industry and state levels.
    \end{minipage}
\end{figure}


    \section{Robustness Exercises}\label{sec:robustness-exercises}
    I conduct several robustness exercises to raise the credibility of the above results. This ranges from placebo exercises, alternative clustering of the standard errors, economic growth effect on pollution emission intensity, and removal of states with the highest total emission intensity.

    \subsection{Placebo Exercise}\label{subsec:placebo-exercise}
    To increase the credibility in the reported emission intensity results; I model the placebo effect of raising the MW on total releases intensity using total onsite releases intensity from catastrophic events. Such events include chemical spills, fire, explosions and natural disasters at specific manufacturing facilities. Catastrophic events are expected to be uncorrelated with the MW policy. Hence, I do not expect to see any statistically significant effect in the releases per $\$100m$ units of output, from catastrophic events. I estimate the following model:
    \begin{equation}
        P_{f,cp,c,t}^{catastrophic} = \beta Treated_{s,t}^e + \delta X_{v,c,t-1} + \omega F_{f,t} + \lambda_{t} + \gamma_{f} + \phi_{cp} + \zeta_{c} + \eta_{c,t} + \theta_{cp,t} + \varepsilon_{f,cp,c,t},\label{eq:robustness-placebo}
    \end{equation}
    % Please add the following required packages to your document preamble:
% \usepackage{booktabs}
% \usepackage{graphicx}
\begin{table}[H]
    \centering
    \caption{Effect of the MW policy on Total Onsite Toxic Releases Intensity, from Catastrophic Events}
    \label{tab:robustness-placebo}
    \resizebox{\columnwidth}{!}{%
        \begin{tabular}{@{}llll@{}}
            \toprule\toprule
            Total releases intensity, from catastrophic events (log) & 1         & 2         & 3         \\ \midrule
            $Treated^{e}$                                            & 0.008     & 0.004     & -0.010    \\
            & (0.049)   & (0.050)   & (0.058)   \\
            cohort 2014                                              & 0.068     & 0.086     & 0.024     \\
            & (0.075)   & (0.074)   & (0.082)   \\
            cohort 2015                                              & -0.092*** & -0.132*** & -0.066    \\
            & (0.034)   & (0.038)   & (0.053)   \\
            cohort 2017                                              & -0.135    & -0.120    & -0.328    \\
            & (0.239)   & (0.001)   & (0.338)   \\
            controls                                                 & Yes       & Yes       & Yes       \\
            year FE                                                  & Yes       & Yes       & Yes       \\
            facility FE                                              & Yes       & Yes       & Yes       \\
            border-county FE                                         & Yes       & Yes       & Yes       \\
            toxic chemical FE                                        & No        & Yes       & Yes       \\
            toxic chemical LTs                                       & No        & Yes       & Yes       \\
            border-county LTs                                        & No        & No        & Yes       \\ \midrule
            Observations                                             & 1,893,689 & 1,893,689 & 1,893,689 \\
            $R^2$                                                    & 0.774     & 0.803     & 0.819     \\
            Baseline Mean                                            & 4.36      & 4.36      & 4.36      \\ \bottomrule \bottomrule
        \end{tabular}%
    }
    \begin{minipage}{\columnwidth}
        \vspace{0.05in}
        \tiny NOTES: These results are obtained from estimating model~\ref{eq:robustness-placebo}. Three-way clustered robust standard errors are reported in parentheses, and clustered at the toxic chemical, industry and state levels. ***, **, and * denote significance levels at the less than $1\%$, $5\%$ and $10\%$, respectively.
    \end{minipage}
\end{table}

    where $P_{f,cp,c,t}$ is the total onsite releases intensity from catastrophic events at manufacturing industry facility, $f$ in cross-border county pairs, $cp$ through toxic chemical use, $c$ in the year, $t$. The ATT measures the difference in the average effect of raising the MW on total onsite releases intensity from catastrophic events at manufacturing industry facilities in treated counties relative to adjacent control counties. The results are presented in Table~\ref{tab:robustness-placebo}. Indeed, the result is consistent with the null hypothesis of the effect of raising MW on total releases intensity, from catastrophic events. This effect remains unchanged even after controlling for the nature of toxic chemical usage and time varying common shocks affecting toxic chemical use in column $2$. Additionally, Figure~\ref{fig:baseline-placebo} shows the dynamic placebo effect on total releases intensity from catastrophic events and no evidence of significant pre-trends.
    \begin{figure}[H]
    \centering
    \includegraphics[width=1\textwidth, height=0.5\textheight,keepaspectratio]{fig_sdid_total_releases_onsite_catastrophicevents_int}
    \caption{Total Onsite Releases Intensity, from Catastrophic Events}
    \label{fig:baseline-placebo}
    \begin{minipage}{\columnwidth}
        \vspace{0.05in}
        \tiny NOTES: The event study model of equation~\ref{eq:robustness-placebo} is $P_{f,cp,c,t}^{catastrophic} = \sum_{{e = -5},{e \neq -1}}^{3}\beta Treated_{s,t}^e + \delta X_{v,c,t-1} + \omega F_{f,t} + \lambda_{t} + \gamma_{f} + \phi_{cp} + \zeta_{c} + \eta_{c,t} + \theta_{cp,t} + \varepsilon_{f,cp,c,t}$. Three-way clustered robust standard errors are reported in parentheses, and clustered at the toxic chemical, industry and state levels. The test for the presence of pre-trends shows and F-statistic of $0.6159$ $(0.7349)$, p-value (in parentheses) using F-Statistic $(\chi^2): \sum_{-3}^{-2} \beta_{e} = 0$. de Chaisemartin and D'Haultfoeuille Decomposition: $\sum dCDH_{ATTs}^{weights(+)} = 1$ and $\sum dCDH_{ATTs}^{weights(-)} = 0$.
    \end{minipage}
\end{figure}

    \subsection{Alternative Clustering of the SEs}\label{subsec:alternative-clustering-of-the-ses}
    The results in Tables~\ref{tab:robustness-ses-clustering-total-releases-onsite},~\ref{tab:robustness-ses-clustering-total-air-releases-onsite},~\ref{tab:robustness-ses-clustering-water-discharge-onsite-intensity}, and~\ref{tab:robustness-ses-clustering-total-land-releases-intensity} remain insensitive to alternative clustering of the standard errors.
    % Please add the following required packages to your document preamble:
% \usepackage{booktabs}
% \usepackage{graphicx}
\begin{table}[H]
    \centering
    \caption{Total Onsite Releases Intensity: Alternative clustering of the SEs}
    \label{tab:robustness-ses-clustering-total-releases-onsite}
    \resizebox{\columnwidth}{!}{%
        \begin{tabular}{@{}lllllllllllll@{}}
            \toprule\toprule
            Total releases intensity (log) & 1         & 2         & 3         & 4         & 5         & 6         & 7                    & 8                    & 9                    & 10                & 11                & 12                \\ \midrule
            $Treated^{e}$                        & 0.108**   & 0.108**   & 0.108**   & 0.108**   & 0.108**   & 0.108*    & 0.108*               & 0.108*               & 0.108**              & 0.108**         & 0.108**          & 0.108**          \\
            & (0.052)   & (0.050)   & (0.054)   & (0.055)   & (0.049)   & (0.063)   & (0.062)              & (0.062)              & (0.049)              & (0.049)          & (0.052)          & (0.055)          \\
            cohort 2014                    & 0.090     & 0.090*    & 0.090     & 0.0897    & 0.090**   & 0.090     & 0.090                & 0.090                & 0.090**              & 0.090**           & 0.090          & 0.090          \\
            & (0.056)   & (0.054)   & (0.062)   & (0.087)   & (0.044)   & (0.081)   & (0.080)              & (0.080)              & (0.044)              & (0.044)          & (0.081)          & (0.080)          \\
            cohort 2015                    & 0.139     & 0.139     & 0.139     & 0.139     & 0.139*    & 0.139     & 0.139                & 0.139                & 0.139*               & 0.139*            & 0.139          & 0.139          \\
            & (0.092)   & (0.092)   & (0.086)   & (0.055)   & (0.077)   & (0.097)   & (0.096)              & (0.096)              & (0.077)              & (0.077)          & (0.097)          & (0.096)          \\
            cohort 2017                    & 0.223     & 0.223     & 0.223     & 0.223     & 0.223**   & 0.223***  & 0.223***             & 0.223***             & 0.223**              & 0.223**         & 0.223***          & 0.223***          \\
            & (0.147)   & (0.151)   & (0.174)   & (0.141)   & (0.090)   & (0.060)   & (0.060)              & (0.060)              & (0.090)              & (0.090)          & (0.061)          & (0.060)          \\
            controls                       & Yes       & Yes       & Yes       & Yes       & Yes       & Yes       & Yes                  & Yes                  & Yes                  & Yes               & Yes               & Yes               \\
            year FE                        & Yes       & Yes       & Yes       & Yes       & Yes       & Yes       & Yes                  & Yes                  & Yes                  & Yes               & Yes               & Yes               \\
            facility FE                    & Yes       & Yes       & Yes       & Yes       & Yes       & Yes       & Yes                  & Yes                  & Yes                  & Yes               & Yes               & Yes               \\
            border-county FE               & Yes       & Yes       & Yes       & Yes       & Yes       & Yes       & Yes                  & Yes                  & Yes                  & Yes               & Yes               & Yes               \\
            toxic chemical FE              & Yes       & Yes       & Yes       & Yes       & Yes       & Yes       & Yes                  & Yes                  & Yes                  & Yes               & Yes               & Yes               \\
            toxic chemical LTs             & Yes       & Yes       & Yes       & Yes       & Yes       & Yes       & Yes                  & Yes                  & Yes                  & Yes               & Yes               & Yes               \\ \midrule
            clustered at the:              & facility  & zipcode   & county    & industry  & chemical  & state     & facility \& chemical & facility \& industry & chemical \& industry & chemical \& state & facility \& state & industry \& state \\
            Observations                   & 1,893,689 & 1,893,689 & 1,893,689 & 1,893,689 & 1,893,689 & 1,893,689 & 1,893,689            & 1,893,689            & 1,893,689            & 1,893,689         & 1,893,689         & 1,893,689         \\
            $R^2$                          & 0.720     & 0.720     & 0.720     & 0.720     & 0.720     & 0.720     & 0.720                & 0.720                & 0.720                & 0.720             & 0.720             & 0.720             \\ \bottomrule \bottomrule
        \end{tabular}%
    }
    \begin{minipage}{18cm}
        \vspace{0.05in}
        These results are obtained from estimating model~\ref{eq:baseline-total-onsite-releases-intensity}. ***, **, and * denote significance levels at the less than $1\%$, $5\%$ and $10\%$, respectively.
    \end{minipage}
\end{table}
    % Please add the following required packages to your document preamble:
% \usepackage{booktabs}
% \usepackage{graphicx}
\begin{table}[H]
    \centering
    \caption{Total Onsite Air Emissions Intensity: Alternative clustering of the SEs}
    \label{tab:robustness-ses-clustering-total-air-releases-onsite}
    \resizebox{\columnwidth}{!}{%
        \begin{tabular}{@{}lllllllllllll@{}}
            \toprule \toprule
            Total air emissions intensity (log) & 1         & 2         & 3         & 4         & 5         & 6         & 7                    & 8                    & 9                    & 10                & 11                & 12                \\ \midrule
            $Treated^{e}$                       & 0.088**   & 0.088**   & 0.088**   & 0.088*    & 0.088**   & 0.088*    & 0.088**              & 0.088**              & 0.088**              & 0.088**           & 0.088**            & 0.088*           \\
            & (0.041)   & (0.039)   & (0.042)   & (0.046)   & (0.039)   & (0.048)   & (0.041)              & (0.041)              & (0.039)              & (0.039)          & (0.041)          & (0.046)          \\
            cohort 2014                         & 0.097*    & 0.097**   & 0.097*    & 0.097*    & 0.097**   & 0.097     & 0.097*               & 0.097*               & 0.097**              & 0.097**           & 0.097*            & 0.097*           \\
            & (0.051)   & (0.049)   & (0.056)   & (0.058)   & (0.042)   & (0.069)   & (0.051)              & (0.051)              & (0.042)              & (0.042)          & (0.051)          & (0.058)          \\
            cohort 2015                         & 0.071     & 0.071     & 0.071     & 0.071*    & 0.071*    & 0.071     & 0.071                & 0.071                & 0.071                & 0.071             & 0.071             & 0.071             \\
            & (0.056)   & (0.056)   & (0.051)   & (0.055)   & (0.052)   & (0.048)   & (0.056)              & (0.056)              & (0.051)              & (0.052)          & (0.056)          & (0.055)          \\
            cohort 2017                         & 0.122     & 0.122     & 0.122     & 0.122*    & 0.122*    & 0.122***  & 0.122                & 0.122                & 0.122                & 0.122             & 0.122            & 0.122           \\
            & (0.111)   & (0.115)   & (0.131)   & (0.116)   & (0.076)   & (0.035)   & (0.111)              & (0.111)              & (0.076)              & (0.076)          & (0.111)          & (0.116)          \\
            controls                            & Yes       & Yes       & Yes       & Yes       & Yes       & Yes       & Yes                  & Yes                  & Yes                  & Yes               & Yes               & Yes               \\
            year FE                             & Yes       & Yes       & Yes       & Yes       & Yes       & Yes       & Yes                  & Yes                  & Yes                  & Yes               & Yes               & Yes               \\
            facility FE                         & Yes       & Yes       & Yes       & Yes       & Yes       & Yes       & Yes                  & Yes                  & Yes                  & Yes               & Yes               & Yes               \\
            border-county FE                    & Yes       & Yes       & Yes       & Yes       & Yes       & Yes       & Yes                  & Yes                  & Yes                  & Yes               & Yes               & Yes               \\
            toxic chemical FE                   & Yes       & Yes       & Yes       & Yes       & Yes       & Yes       & Yes                  & Yes                  & Yes                  & Yes               & Yes               & Yes               \\
            toxic chemical LTs                  & Yes       & Yes       & Yes       & Yes       & Yes       & Yes       & Yes                  & Yes                  & Yes                  & Yes               & Yes               & Yes               \\\midrule
            clustered at the:                   & facility  & zipcode   & county    & industry  & chemical  & state     & facility \& chemical & facility \& industry & chemical \& industry & chemical \& state & facility \& state & industry \& state \\
            Observations                        & 1,893,689 & 1,893,689 & 1,893,689 & 1,893,689 & 1,893,689 & 1,893,689 & 1,893,689            & 1,893,689            & 1,893,689            & 1,893,689         & 1,893,689         & 1,893,689         \\
            $R^2$                               & 0.739     & 0.739     & 0.739     & 0.739     & 0.739     & 0.739     & 0.739                & 0.739                & 0.739                & 0.739             & 0.739             & 0.739             \\ \bottomrule\bottomrule
        \end{tabular}%
    }
    \begin{minipage}{\columnwidth}
        \vspace{0.05in}
        \tiny NOTES: These results are obtained from estimating model~\ref{eq:baseline-onsite-air-emission-intensity}. ***, **, and * denote significance levels at the less than $1\%$, $5\%$ and $10\%$, respectively.
    \end{minipage}
\end{table}
    % Please add the following required packages to your document preamble:
% \usepackage{booktabs}
% \usepackage{graphicx}% \usepackage{graphicx}
\begin{table}[H]
    \centering
    \caption{Total Onsite Surface Water Discharge Intensity: Alternative clustering of the SEs}
    \label{tab:robustness-ses-clustering-water-discharge-onsite-intensity}
    \resizebox{\columnwidth}{!}{%
        \begin{tabular}{@{}lllllllllllll@{}}
            \toprule\toprule
            Total surface water discharge intensity (log) & 1         & 2         & 3         & 4         & 5         & 6         & 7                    & 8                    & 9                    & 10                & 11                & 12                \\ \midrule
            $Treated^{e}$                                 & 0.014     & 0.014     & 0.014     & 0.014     & 0.014     & 0.014     & 0.014                & 0.014                & 0.014                & 0.014             & 0.014             & 0.014             \\
            & (0.020)   & (0.020)   & (0.020)   & (0.022)   & (0.030)   & (0.018)   & (0.020)              & (0.020)              & (0.030)              & (0.030)          & (0.020)          & (0.022)          \\
            cohort 2014                                   & -0.047**  & -0.047**  & -0.047**  & -0.047*   & -0.047    & -0.047**  & -0.047**             & -0.047**               & -0.047*               & -0.047           & -0.047**         & -0.047*      \\
            & (0.024)   & (0.024)   & (0.023)   & (0.028)   & (0.028)   & (0.021)   & (0.024)              & (0.024)              & (0.028)              & (0.028)          & (0.024)          & (0.028)          \\
            cohort 2015                                   & 0.116***  & 0.116***  & 0.116***  & 0.116***  & 0.116***  & 0.116***  & 0.116***             & 0.116***               & 0.116***              & 0.116***             & 0.116***         & 0.116***      \\
            & (0.030)   & (0.030)   & (0.032)   & (0.028)   & (0.041)   & (0.032)   & (0.030)              & (0.030)              & (0.041)              & (0.041)          & (0.030)          & (0.028)          \\
            cohort 2017                                   & 0.060     & 0.060     & 0.060     & 0.060     & 0.060     & 0.060     & 0.060                & 0.060                & 0.060                & 0.060             & 0.060             & 0.060             \\
            & (0.055)   & (0.055)   & (0.046)   & (0.044)   & (0.043)   & (0.036)   & (0.055)              & (0.055)              & (0.043)              & (0.043)          & (0.055)          & (0.044)          \\
            controls                                      & Yes       & Yes       & Yes       & Yes       & Yes       & Yes       & Yes                  & Yes                  & Yes                  & Yes               & Yes               & Yes               \\
            year FE                                       & Yes       & Yes       & Yes       & Yes       & Yes       & Yes       & Yes                  & Yes                  & Yes                  & Yes               & Yes               & Yes               \\
            facility FE                                   & Yes       & Yes       & Yes       & Yes       & Yes       & Yes       & Yes                  & Yes                  & Yes                  & Yes               & Yes               & Yes               \\
            border-county FE                              & Yes       & Yes       & Yes       & Yes       & Yes       & Yes       & Yes                  & Yes                  & Yes                  & Yes               & Yes               & Yes               \\
            toxic chemical FE                             & Yes       & Yes       & Yes       & Yes       & Yes       & Yes       & Yes                  & Yes                  & Yes                  & Yes               & Yes               & Yes               \\
            toxic chemical LTs                            & Yes       & Yes       & Yes       & Yes       & Yes       & Yes       & Yes                  & Yes                  & Yes                  & Yes               & Yes               & Yes               \\
            border-county LTs                             & Yes       & Yes       & Yes       & Yes       & Yes       & Yes       & Yes                  & Yes                  & Yes                  & Yes               & Yes               & Yes               \\ \midrule
            clustered at the:                             & facility  & zipcode   & county    & industry  & chemical  & state     & facility \& chemical & facility \& industry & chemical \& industry & chemical \& state & facility \& state & industry \& state \\
            Observations                                  & 1,893,689 & 1,893,689 & 1,893,689 & 1,893,689 & 1,893,689 & 1,893,689 & 1,893,689            & 1,893,689            & 1,893,689            & 1,893,689         & 1,893,689         & 1,893,689         \\
            $R^2$                                         & 0.592     & 0.592     & 0.592     & 0.592     & 0.592     & 0.592     & 0.592                & 0.592                & 0.592                & 0.592             & 0.592             & 0.592             \\ \bottomrule \bottomrule
        \end{tabular}%
    }
    \begin{minipage}{\columnwidth}
        \vspace{0.05in}
        \tiny NOTES: These results are obtained from estimating model~\ref{eq:baseline-onsite-water-discharge-intensity}. ***, **, and * denote significance levels at the less than $1\%$, $5\%$ and $10\%$, respectively.
    \end{minipage}
\end{table}
    % Please add the following required packages to your document preamble:
% \usepackage{booktabs}
% \usepackage{graphicx}
\begin{table}[H]
    \centering
    \caption{Total Onsite Land Releases Intensity: Alternative Clustering of SEs}
    \label{tab:robustness-ses-clustering-total-land-releases-intensity}
    \resizebox{\columnwidth}{!}{%
        \begin{tabular}{@{}lllllllllllll@{}}
            \toprule\toprule
            Total onsite land releases intensity (log) & 1         & 2         & 3         & 4         & 5         & 6         & 7                    & 8                    & 9                    & 10                & 11                & 12                \\ \midrule
            treated                                    & 0.0095    & 0.0095    & 0.0095    & 0.0095    & 0.0095    & 0.0095    & 0.0095               & 0.0095               & 0.0095               & 0.0095            & 0.0095            & 0.0095            \\
            & (0.0179)  & (0.0179)  & (0.0174)  & (0.0187)  & (0.0112)  & (0.0119)  & (0.0179)             & (0.0179)             & (0.0111)             & (0.0111)          & (0.0179)          & (0.0187)          \\
            controls                                   & Yes       & Yes       & Yes       & Yes       & Yes       & Yes       & Yes                  & Yes                  & Yes                  & Yes               & Yes               & Yes               \\
            year FE                                    & Yes       & Yes       & Yes       & Yes       & Yes       & Yes       & Yes                  & Yes                  & Yes                  & Yes               & Yes               & Yes               \\
            facility FE                                & Yes       & Yes       & Yes       & Yes       & Yes       & Yes       & Yes                  & Yes                  & Yes                  & Yes               & Yes               & Yes               \\
            border-county FE                             & Yes       & Yes       & Yes       & Yes       & Yes       & Yes       & Yes                  & Yes                  & Yes                  & Yes               & Yes               & Yes               \\
            toxic chemical FE                          & Yes       & Yes       & Yes       & Yes       & Yes       & Yes       & Yes                  & Yes                  & Yes                  & Yes               & Yes               & Yes               \\
            toxic chemical LTs                         & Yes       & Yes       & Yes       & Yes       & Yes       & Yes       & Yes                  & Yes                  & Yes                  & Yes               & Yes               & Yes               \\\midrule
            clustered at the:                          & facility  & zipcode   & county    & industry  & chemical  & state     & facility \& chemical & facility \& industry & chemical \& industry & chemical \& state & facility \& state & industry \& state \\
            Observations                               & 1,893,689 & 1,893,689 & 1,893,689 & 1,893,689 & 1,893,689 & 1,893,689 & 1,893,689            & 1,893,689            & 1,893,689            & 1,893,689         & 1,893,689         & 1,893,689         \\
            $R^2$                                      & 0.4997    & 0.4997    & 0.4997    & 0.4997    & 0.4997    & 0.4997    & 0.4997               & 0.4997               & 0.4997               & 0.4997            & 0.4997            & 0.4997            \\ \bottomrule\bottomrule
        \end{tabular}%
    }
    \begin{minipage}{18cm}
        \vspace{0.05in}
        These results are obtained from estimating model~\ref{eq:baseline-onsite-land-releases-intensity}. ***, **, and * denote significance levels at the less than $1\%$, $5\%$ and $10\%$, respectively.
    \end{minipage}
\end{table}

%    \subsection{Removing Highest Emitting States}\label{subsec:removing-highest-emitting-states}


    \section{Heterogeneous Effects}\label{sec:heterogeneous-effects}

    \subsection{Financial Constraints and Production Technology}\label{subsec:financial-constraints-and-production-technology}
    According to theory, financial constraints and industrial production technology are crucial in modeling industrial pollution responses. Financial constraints are proxied by the ratio of revenue to profit (the profit margin). A high profit margin suggests the manufacturing industry is less financially constrained, while a low profit margin indicates greater financial constraint. Production technology is proxied by the ratio of total payroll or production workers' wages to revenue. A high payroll or wages to revenue ratio indicates a labor-intensive industry, whereas a low ratio suggests a capital-intensive industry. This subsection examines whether the documented effects vary based on these classifications. To investigate the differential impact of raising MW on onsite total release intensities given manufacturing industry's financial constraints and production technology, I estimate the following model:
    \begin{align}
        P_{f,cp,c,t}^{fintech} &= \beta (Treated^{e} \cdot D)_{f,s,t} + \psi (Treated^{e})_{s,t} + \vartheta (Treated \cdot D)_{f,s,t} + \mu (Post \cdot D)_{f,s,t} \nonumber \\
        &\quad + \tau Treated_{s,t} + \rho D_{f,s,t} + \alpha Post_{t} + \delta X_{v,c,t-1} + \omega F_{f,t} + \lambda_{t} + \gamma_{f} + \phi_{cp} \nonumber \\
        &\quad + \zeta_{c} + \eta_{c,t} + \theta_{cp,t} + \varepsilon_{f,cp,c,t},\label{eq:heterogeneous-onsite-releases-intensity-fintech}
    \end{align}
    where $P_{f,cp,c,t}^{fintech}$ is the vector of total onsite releases intensities (air, water and land) of toxic chemicals at a manufacturing industry facility, $f$ in a cross-border county pair, $cp$ through a toxic chemical, $c$ in the year, $t$. $Treated_{s,t}$ is a dummy that is equal to $1$ for the treated states, and $0$ the control states. $Post_{t}$ is a dummy that is equal to $1$ if the year $t$ is a post-treatment year, and $0$ otherwise. And $D_{f,s,t}$ is a separate dummy of financial constraints or production technology of manufacturing facilities, $f$ in state, $s$ in the year, $t$ and $0$ otherwise. It is unity for less financially constrained industries, and $0$ for more financially constrained industries; also, it is unity for labour intensive manufacturing industries and $0$ for capital intensive. Others are as defined in Section~\ref{sec:onsite-toxic-releases}.

    The parameters of interest here is the triple-differences parameter $\beta$ which measures the differential average effects of higher MW on onsite total releases intensities at manufacturing facilities in treated counties relative to adjacent control counties given either their financial constraint or technological classification. That is, the separate differential impacts on either less financially constrained or labour-intensive industries. $\psi$ captures the relative differential impact on industries with greater financial constraint or uses capital-intensive production technology. $\beta + \psi$ captures the total relative change on either less financially constrained or labour-intensive industries. To have a causal interpretation, I assume a weaker parallel trends assumption and is only required to hold for one of the groups in triple differences~\parencite{olden2022triple}.~\cite{zhang2023unintended} hypothesised that more financially constrained firms and labour-intensive industries are more responsive to higher MW relative capital-intensive industries. Hence, I test whether the effects of raising MW on total onsite releases intensities are dominated in labour-intensive industries. The results are presented in Tables~\ref{tab:heterogeneous-onsite-releases-int-finc} and~\ref{tab:heterogeneous-onsite-releases-int-tech-payroll}.

    \paragraph{Financial Constraints:}
    For the less financially constrained manufacturing industries, the triple-differences parameter $\beta$, as shown in Table~\ref{tab:heterogeneous-onsite-releases-int-finc}, is negative and insignificant. The observed decline in toxic release intensity is largely attributable to the significant reduction in the $2015$ cohort. The overall relative change, $\beta_{2015} + \psi$, for the $2015$ cohort indicates a substantial decrease of $49.2ppts$ per $\$100$ million units of output. This suggests increased efficiency due to higher MW among less financially constrained manufacturing industries. This reduction is primarily driven by the declining total air emissions intensities, both in point and fugitive emissions, particularly in the $2015$ and $2014$ cohorts, respectively. Similarly, there is a notable decline in land release intensity within the $2014$ cohort while the surface impoundment intensity remains generally muted.
    % Please add the following required packages to your document preamble:
% \usepackage{booktabs}
% \usepackage{graphicx}

\begin{table}[H]
    \centering
    \caption{Onsite Releases Intensity given Financial Constraints}
    \label{tab:heterogeneous-onsite-releases-int-finc}
    \resizebox{\columnwidth}{!}{%
        \begin{tabular}{@{}llllllll@{}}
            \toprule\toprule
            Onsite releases intensity (log) & total     & air emissions & point air & fugitive air & water discharge & land releases & surface impoundment \\ \midrule
            $Treated^{e}$                   & -0.051    & -0.002        & 0.103     & -0.074*      & -0.022          & -0.053        & 0.004               \\
            & (0.067)   & (0.055)       & (0.067)   & (0.039)      & (0.040)         & (0.035)       & (0.003)             \\
            $Treated$                       & 0.136***  & 0.087**       & 0.009     & 0.072**      & 0.026           & 0.013         & 0.009*              \\
            & (0.042)   & (0.039)       & (0.038)   & (0.031)      & (0.031)         & (0.010)       & (0.005)             \\
            cohort 2014                     & 0.083     & 0.107*        & 0.238***  & -0.097**     & 0.017           & -0.060*       & 0.003               \\
            & (0.080)   & (0.064)       & (0.075)   & (0.043)      & (0.039)         & (0.032)       & (0.003)             \\
            cohort 2015                     & -0.628*** & -0.470***     & -0.478*** & 0.026        & -0.192          & -0.024        & 0.008               \\
            & (0.190)   & (0.151)       & (0.148)   & (0.110)      & (0.211)         & (0.064)       & (0.008)             \\
            cohort 2017                     & 0.173     & 0.177         & 0.133     & 0.388***     & 0.149           & -0.037        & 0.005               \\
            & (0.220)   & (0.214)       & (0.205)   & (0.144)      & (0.112)         & (0.071)       & (0.007)             \\
            controls                        & Yes       & Yes           & Yes       & Yes          & Yes             & Yes           & Yes                 \\
            year FE                         & Yes       & Yes           & Yes       & Yes          & Yes             & Yes           & Yes                 \\
            facility FE                     & Yes       & Yes           & Yes       & Yes          & Yes             & Yes           & Yes                 \\
            border-county FE                & Yes       & Yes           & Yes       & Yes          & Yes             & Yes           & Yes                 \\
            toxic chemical FE               & Yes       & Yes           & Yes       & Yes          & Yes             & Yes           & Yes                 \\
            toxic chemical LTs              & Yes       & Yes           & Yes       & Yes          & Yes             & Yes           & Yes                 \\
            border-county LTs               & Yes       & Yes           & Yes       & Yes          & Yes             & Yes           & Yes                 \\\midrule
            $R^2$                           & 0.727     & 0.746         & 0.718     & 0.670        & 0.593           & 0.507         & 0.159               \\
            Obsservations                   & 1,893,689 & 1,893,689     & 1,893,689 & 1,893,689    & 1,893,689       & 1,893,689     & 1,893,689           \\ \bottomrule \bottomrule
        \end{tabular}%
    }
    \begin{minipage}{\columnwidth}
        \vspace{0.05in}
        \tiny NOTES: These results are obtained from estimating equation~\ref{eq:heterogeneous-onsite-releases-intensity-fintech}. Three-way clustered robust standard errors are reported in parentheses, and clustered at the toxic chemical, industry and state levels. ***, **, and * denote significance levels at the less than $1\%$, $5\%$ and $10\%$, respectively.
    \end{minipage}
\end{table}

    In contrast, for the more financially constrained industries, the results show a significant positive effect of $13.6ppts$. This positive effect surpasses the baseline estimate of $11.9ppts$, indicating that toxic release intensity is highest among highly financially constrained manufacturing industries due to a higher MW floor. This increase is driven by rises in total air emissions intensities, particularly in fugitive emissions. There is limited evidence for increases in land release intensity, but significant increases in surface impoundment intensity are observed. Finally, I find limited evidence of differential changes in surface water discharge intensity due to a higher MW policy in both less and more financially constrained manufacturing industries.

    Figure~\ref{fig:heterogeneous-onsite-releases-intensities-finc} illustrates the dynamic effects of raising the MW on various pollution intensities. For less financially constrained manufacturing industries, the effects on total onsite releases, air emissions, and surface water discharge intensities are generally insignificant. However, there is an instantaneous increase in point air emissions intensity, followed by another increase three years later, and a similar pattern is observed for surface impoundment intensity. Conversely, fugitive air emissions intensity declines three years later, while land releases intensity decreases one year after the MW increase.
    \begin{figure}[H]
    \centering
    \includegraphics[width=1\textwidth, height=0.5\textheight,keepaspectratio]{fig_sdid_total_onsite_releases_int_finc}
    \caption{Triple-Differences: Onsite Total Releases Intensities given Financial Constraint}
    \label{fig:heterogeneous-onsite-releases-intensities-finc}
    \begin{minipage}{\columnwidth}
        \vspace{0.05in}
        \tiny NOTES: The event study model of equation~\ref{eq:heterogeneous-onsite-releases-intensity-fintech} is $P_{f,cp,c,t}^{fintech} = \sum_{{e = -3},{e \neq -1}}^{3} \beta (Treated^{e} \cdot D)_{f,s,t} + \psi (Treated^{e})_{s,t} + \vartheta (Treated \cdot D)_{f,s,t} + \mu (Post \cdot D)_{f,s,t} + \tau Treated_{s,t} + \rho D_{f,s,t} + \alpha Post_{t} + \delta X_{v,c,t-1} + \omega F_{f,t} + \lambda_{t} + \gamma_{f} + \phi_{cp} + \zeta_{c} + \eta_{c,t} + \theta_{cp,t} + \varepsilon_{f,cp,c,t}$. Three-way clustered robust standard errors are reported in parentheses, and clustered at the toxic chemical, industry and state levels. HRPR means high revenue to profit ratio or high profit margin---less financially constrained. LRPR means low revenue to profit ratio or low profit margin---more financially constrained.
    \end{minipage}
\end{figure}

    In contrast, significant increases in onsite toxic releases, air emissions (both point and fugitive), and surface water discharge intensities are prominent in manufacturing industries with greater financial constraints. These effects are immediate for total toxic releases and air emissions intensities and persist for three years. Specifically, point air emissions intensity shows significant increases from the second to the third year post-treatment, while fugitive air emissions intensity rises significantly in the first and second year post-treatment. The effect on surface water discharge intensity emerges two years later. No significant effects are recorded for surface impoundment intensity, and there are no notable pre-trends.

    \paragraph{Production Technology:}
    Table~\ref{tab:heterogeneous-onsite-releases-int-tech-payroll} reveals that the decreasing differential effect on toxic release intensity due to a higher MW is most pronounced in labor-intensive manufacturing industries, registering a reduction of $20.2ppts$, consistent across all cohorts. This decline is primarily driven by reductions in total air emissions (both point and fugitive), especially in the $2015$ and $2014$ cohorts, respectively. A similar pattern is observed in land release intensity, particularly surface impoundment intensity, notably from the $2014$ and $2017$ cohorts.
    % Please add the following required packages to your document preamble:
% \usepackage{booktabs}
% \usepackage{graphicx}
\begin{table}[H]
    \centering
    \caption{Onsite Releases Intensity given Production Technology (Payroll)}
    \label{tab:heterogeneous-onsite-releases-int-tech-payroll}
    \resizebox{\columnwidth}{!}{%
        \begin{tabular}{@{}llllllll@{}}
            \toprule\toprule
            Onsite releases intensity (log) & total     & air emissions & point air & fugitive air & water discharge & land releases & surface impoundment \\ \midrule
            $Treated^{e}$                   & -0.202*** & -0.111**      & -0.081*   & -0.005       & -0.060          & -0.026        & -0.014**            \\
            & (0.065)   & (0.047)       & (0.043)   & (0.037)      & (0.038)         & (0.029)       & (0.006)             \\
            $Treated$                       & 0.205***  & 0.115**       & 0.048     & 0.060*       & 0.067           & 0.020         & 0.019**             \\
            & (0.064)   & (0.045)       & (0.040)   & (0.033)      & (0.050)         & (0.014)       & (0.009)             \\
            cohort 2014                     & -0.139**  & -0.051        & -0.087*   & 0.096*       & -0.058          & -0.038**      & -0.018**            \\
            & (0.066)   & (0.057)       & (0.052)   & (0.049)      & (0.036)         & (0.018)       & (0.008)             \\
            cohort 2015                     & -0.310*** & -0.210***     & -0.068    & -0.178***    & -0.064          & -0.004        & -0.005              \\
            & (0.100)   & (0.069)       & (0.057)   & (0.055)      & (0.048)         & (0.060)       & (0.004)             \\
            cohort 2017                     & -0.404*   & -0.298        & -0.302    & 0.024        & -0.095          & -0.038        & -0.042**            \\
            & (0.227)   & (0.201)       & (0.205)   & (0.132)      & (0.070)         & (0.034)       & (0.021)             \\
            controls                        & Yes       & Yes           & Yes       & Yes          & Yes             & Yes           & Yes                 \\
            year FE                         & Yes       & Yes           & Yes       & Yes          & Yes             & Yes           & Yes                 \\
            facility FE                     & Yes       & Yes           & Yes       & Yes          & Yes             & Yes           & Yes                 \\
            border-county FE                & Yes       & Yes           & Yes       & Yes          & Yes             & Yes           & Yes                 \\
            toxic chemical FE               & Yes       & Yes           & Yes       & Yes          & Yes             & Yes           & Yes                 \\
            toxic chemical LTs              & Yes       & Yes           & Yes       & Yes          & Yes             & Yes           & Yes                 \\
            border-county LTs               & Yes       & Yes           & Yes       & Yes          & Yes             & Yes           & Yes                 \\ \midrule
            $R^2$                           & 0.727     & 0.745         & 0.718     & 0.670        & 0.592           & 0.507         & 0.162               \\
            Obsservations                   & 1,893,689 & 1,893,689     & 1,893,689 & 1,893,689    & 1,893,689       & 1,893,689     & 1,893,689           \\ \bottomrule \bottomrule
        \end{tabular}%
    }
    \begin{minipage}{\columnwidth}
        \vspace{0.05in}
        \tiny NOTES: These results are obtained from estimating equation~\ref{eq:heterogeneous-onsite-releases-intensity-fintech}. Three-way clustered robust standard errors are reported in parentheses, and clustered at the toxic chemical, industry and state levels. ***, **, and * denote significance levels at the less than $1\%$, $5\%$ and $10\%$, respectively.
    \end{minipage}
\end{table}

    Conversely, the increasing effects on total toxic release intensity are more pronounced in capital-intensive manufacturing industries. This increase is driven by significant rises in total air emissions intensity, particularly fugitive emissions, and surface impoundment intensities. There is limited evidence for changes in surface water discharge intensity.

    The dynamic effects depicted in Figure~\ref{fig:heterogeneous-onsite-releases-intensities-payroll-tech} further illustrate that the negative effect on toxic release intensity is most significant in labor-intensive manufacturing industries, emerging instantly and persisting for up to two years. This effect is driven by a decline in total air emissions intensity, particularly from point sources, surface water discharge, and surface impoundment intensities. The dynamic effect on fugitive air emissions and land release intensities remains generally muted.
    \begin{figure}[H]
    \centering
    \includegraphics[width=1\textwidth, height=0.5\textheight,keepaspectratio]{fig_sdid_total_onsite_releases_int_tech_payroll}
    \caption{Triple-Differences: Onsite Total Releases Intensities given Production Technology (Payroll)}
    \label{fig:heterogeneous-onsite-releases-intensities-payroll-tech}
    \begin{minipage}{\columnwidth}
        \vspace{0.05in}
        \tiny NOTES: The event study model of equation~\ref{eq:heterogeneous-onsite-releases-intensity-fintech} is $P_{f,cp,c,t}^{fintech} = \sum_{{e = -3},{e \neq -1}}^{3} \beta (Treated^{e} \cdot D)_{f,s,t} + \psi (Treated^{e})_{s,t} + \vartheta (Treated \cdot D)_{f,s,t} + \mu (Post \cdot D)_{f,s,t} + \tau Treated_{s,t} + \rho D_{f,s,t} + \alpha Post_{t} + \delta X_{v,c,t-1} + \omega F_{f,t} + \lambda_{t} + \gamma_{f} + \phi_{cp} + \zeta_{c} + \eta_{c,t} + \theta_{cp,t} + \varepsilon_{f,cp,c,t}$. Three-way clustered robust standard errors are reported in parentheses, and clustered at the toxic chemical, industry and state levels. HEIs mean highest emitting industries. HPRR means high total payroll to revenues ratio---labour-intensive manufacturing industries. LPRR means low total payroll to revenues ratio---capital-intensive manufacturing industries.
    \end{minipage}
\end{figure}

    In contrast, the increasing effect on toxic release intensity is most pronounced in capital-intensive manufacturing industries, appearing immediately and persisting for up to three years. This increase is driven by rising total air emissions intensities (both point and fugitive), surface water discharge, and surface impoundment intensities. There is limited evidence of changes in total land release intensity. As expected, there are no significant pre-trends.

    \subsection{Highest Emitting Industries}\label{subsec:highest-emitting-industries}
    There are manufacturing industries that release and emit more toxic chemicals than others. This subsection investigates if the documented effect is heterogeneous and entirely driven by such highest emitting manufacturing industries. They include the chemical, food, leather and allied products and wood manufacturing industries (see their distributions in Figures~\ref{fig:releases-distribution-naics},~\ref{fig:air-emissions-distribution-naics},~\ref{fig:water-discharge-distribution-naics} and~\ref{fig:land-releases-distribution-naics} of Appendix~\ref{sec:appendix-distribution-of-industries-and-pollution-emissions-intensities}). To investigate the differential effect of higher MW on onsite total releases intensity of highest emitting manufacturing industries (HEIs), I estimate the following model:
    \begin{align}
        P_{f,cp,c,t}^{heis} &= \beta (Treated^{e} \cdot D)_{f,s,t} + \psi (Treated^{e})_{s,t} + \vartheta (Treated \cdot D)_{f,s,t} + \mu (Post \cdot D)_{f,s,t} \nonumber \\
        &\quad + \tau Treated_{s,t} + \rho D_{f,s,t} + \alpha Post_{t} + \delta X_{v,c,t-1} + \omega F_{f,t} + \lambda_{t} + \gamma_{f} + \phi_{cp} \nonumber \\
        &\quad + \zeta_{c} + \eta_{c,t} + \theta_{cp,t} + \varepsilon_{f,cp,c,t},\label{eq:heterogeneous-onsite-releases-intensity-heis}
    \end{align}
    where $P_{f,cp,c,t}^{heis}$ is the vector of total onsite releases intensity (air, water and land) of toxic chemicals at a manufacturing industry facility, $f$ in a cross-border county pair, $cp$ through a toxic chemical, $c$ in the year, $t$. $Treated_{s,t}$ is a dummy that is equal to $1$ for the treated states, and $0$ the control states. $Post_{t}$ is a dummy that is equal to $1$ if the year $t$ is a post-treatment year, and $0$ otherwise. And $D_{f,s,t}$ is a dummy that is unity for the set of HEIs, $f$ in state, $s$ in the year, $t$ and $0$ low and lowest emitting manufacturing industries (LEIs).
    % Please add the following required packages to your document preamble:
% \usepackage{booktabs}
% \usepackage{graphicx}
\begin{table}[H]
    \centering
    \caption{Onsite Releases Intensity given Highest Emitting Manufacturing Industries}
    \label{tab:heterogeneous-onsite-releases-int-heis}
    \resizebox{\columnwidth}{!}{%
        \begin{tabular}{@{}llllllll@{}}
            \toprule\toprule
            Onsite releases intensity (log) & total     & air emissions & point air & fugitive air & water discharge & land releases & surface impoundment \\ \midrule
            $Treated^{e} \cdot D$           & 0.360***  & 0.224***      & 0.032     & 0.235***     & 0.094           & 0.002         & -0.011**            \\
            & (0.116)   & (0.077)       & (0.066)   & (0.067)      & (0.087)         & (0.019)       & (0.005)             \\
            $Treated^{e}$                   & 0.020     & 0.023         & 0.048     & -0.011       & 0.007           & 0.007         & 0.021**             \\
            & (0.039)   & (0.039)       & (0.036)   & (0.030)      & (0.011)         & (0.013)       & (0.010)             \\
            cohort 2014 $\cdot D$           & 0.350***  & 0.305***      & 0.259**   & 0.077        & -0.079          & 0.041         & -0.011**            \\
            & (0.113)   & (0.114)       & (0.107)   & (0.077)      & (0.067)         & (0.044)       & (0.005)             \\
            cohort 2015 $\cdot D$           & 0.369***  & 0.149**       & -0.178*   & 0.381***     & 0.254**         & -0.034        & -0.012              \\
            & (0.141)   & (0.069)       & (0.091)   & (0.090)      & (0.122)         & (0.021)       & (0.007)             \\
            cohort 2017 $\cdot D$           & 0.052     & 0.032         & 0.166     & 0.145        & 0.033           & 0.025         & -0.002              \\
            & (0.174)   & (0.179)       & (0.141)   & (0.142)      & (0.091)         & (0.025)       & (0.009)             \\
            controls                        & Yes       & Yes           & Yes       & Yes          & Yes             & Yes           & Yes                 \\
            year FE                         & Yes       & Yes           & Yes       & Yes          & Yes             & Yes           & Yes                 \\
            facility FE                     & Yes       & Yes           & Yes       & Yes          & Yes             & Yes           & Yes                 \\
            border-county FE                & Yes       & Yes           & Yes       & Yes          & Yes             & Yes           & Yes                 \\
            toxic chemical FE               & Yes       & Yes           & Yes       & Yes          & Yes             & Yes           & Yes                 \\
            toxic chemical LTs              & Yes       & Yes           & Yes       & Yes          & Yes             & Yes           & Yes                 \\\midrule
            Observations                    & 1,893,689 & 1,893,689     & 1,893,689 & 1,893,689    & 1,893,689       & 1,893,689     & 1,893,689           \\
            $R^2$                           & 0.720     & 0.739         & 0.712     & 0.661        & 0.585           & 0.501         & 0.127               \\ \bottomrule \bottomrule
        \end{tabular}%
    }
    \begin{minipage}{\columnwidth}
        \vspace{0.05in}
        \tiny NOTES: These results are obtained from estimating this equation: $P_{f,cp,c,t}^{heis} = \beta (Treated^{e} \cdot D)_{f,s,t} + \psi (Treated^{e})_{s,t} + \vartheta (Treated \cdot D)_{f,s,t} + \mu (Post \cdot D)_{f,s,t} + \tau Treated_{s,t} + \rho D_{f,s,t} + \alpha Post_{t} + \delta X_{v,c,t-1} + \omega F_{f,t} + \lambda_{t} + \gamma_{f} + \phi_{cp} + \zeta_{c} + \eta_{c,t} + \varepsilon_{f,cp,c,t}$. Three-way clustered robust standard errors are reported in parentheses, and clustered at the toxic chemical, industry and state levels. ***, **, and * denote significance levels at the less than $1\%$, $5\%$ and $10\%$, respectively.
    \end{minipage}
\end{table}

    The parameter of interest here is the triple-differences parameter $\beta$ which measures the differential impact on onsite total releases intensities due to a higher MW policy for HEIs. $\psi$ measures the relative change in onsite total releases intensity for non-carcinogenic chemicals. And $\beta + \psi$ measures the total relative change in onsite total releases intensities for HEIs. The results are presented in Table~\ref{tab:heterogeneous-onsite-releases-int-heis}. I find that the increasing toxic release intensity is predominantly observed in higher-emitting manufacturing industries, particularly within the $2014$ and $2015$ cohorts. This increase is primarily driven by rises in total air emissions (both point and fugitive) and surface water discharge intensities, especially notable in these cohorts. No significant effect is recorded on surface impoundment intensity. Conversely, in lower-emitting manufacturing industries, except for a notable positive impact on surface impoundment intensities, the overall impact on total toxic releases, including all air emissions and surface water discharge intensities, remains relatively muted. Additionally, there is limited evidence of significant changes in land release intensity across the industries.
    \begin{figure}[H]
    \centering
    \includegraphics[width=1\textwidth, height=0.5\textheight,keepaspectratio]{fig_sdid_total_onsite_releases_int_EMITT}
    \caption{Triple-Differences: Onsite Total Releases Intensity given Highest Emitting Manufacturing Industries}
    \label{fig:heterogeneous-onsite-releases-intensity-emitt}
    \begin{minipage}{18cm}
        \vspace{0.05in}
        NOTES: The event study model of equation~\ref{eq:heterogeneous-onsite-releases-intensity-heis} is $G_{f,cp,c,t}^{gdp} = \sum_{{e = -3},{e \neq -1}}^{3} \beta (Treated^{e} \cdot D)_{f,s,t} + \psi (Treated^{e})_{s,t} + \vartheta (Treated \cdot D)_{f,s,t} + \mu (Post \cdot D)_{f,s,t} + \tau Treated_{s,t} + \rho D_{f,s,t} + \alpha Post_{t} + \delta X_{v,c,t-1} + \omega F_{f,t} + \lambda_{t} + \gamma_{f} + \phi_{cp} + \zeta_{c} + \eta_{c,t} + \varepsilon_{f,cp,c,t}$. Three-way clustered robust standard errors are reported in parentheses, and clustered at the toxic chemical, industry and state levels. HEIs mean highest emitting industries.
    \end{minipage}
\end{figure}

    Figure~\ref{fig:heterogeneous-onsite-releases-intensity-emitt} illustrates the dynamic effects of higher MW policies. The results indicate that the positive effects on toxic release intensities are most pronounced in the highest and lowest emitting manufacturing industries. These effects are especially evident in air emissions (both point and fugitive), surface water discharge, and land release intensities. The impact emerges immediately and persists for up to three years. For HEIs, the effect on surface impoundment intensity declines, whereas, for LEIs, both land releases and surface impoundment intensities increase instantaneously and persist for up to three years.

    \subsection{Economic Growth Patterns}\label{subsec:economic-growth-patterns}
    In this subsection, I check whether the increasing onsite releases intensity due to a higher MW floor is peculiar to treated counties with high economic growth patterns. Theory suggests that economic growth patterns are correlated with pollutant emissions with a turning point at higher economic growth~\parencite{grossman1995economic, shapiro2018pollution}. To investigate the differential effect of higher MW on onsite total releases intensity, I estimate the following model:
    \begin{align}
        P_{f,cp,c,t}^{gdp} &= \beta (Treated^{e} \cdot D)_{h,s,t} + \psi (Treated^{e})_{s,t} + \vartheta (Treated \cdot D)_{h,s,t} + \mu (Post \cdot D)_{h,s,t} \nonumber \\
        &\quad + \tau Treated_{s,t} + \rho D_{h,s,t} + \alpha Post_{t} + \delta X_{v,c,t-1} + \omega F_{f,t} + \lambda_{t} + \gamma_{f} + \phi_{cp} \nonumber \\
        &\quad+ \zeta_{c} + \eta_{c,t} + \theta_{cp,t} + \varepsilon_{f,cp,c,t},\label{eq:heterogeneous-onsite-releases-intensity-gdp}
    \end{align}
    where $P_{f,cp,c,t}^{gdp}$ is the vector of total onsite releases intensity (air e.g., point and fugitive, water and land) in a high GDP county at a manufacturing industry facility, $f$  in a cross-border county pair, $cp$ through a toxic chemical, $c$ in the year, $t$. $Treated_{s,t}$ is a dummy that is equal to $1$ for the treated states, and $0$ the control states. $Post_{t}$ is a dummy that is equal to $1$ if the year $t$ is a post-treatment year, and $0$ otherwise. And $D_{h,s,t}$ is a dummy that is unity for a high gross domestic product (GDP) of county, $h$ in state, $s$ in the year, $t$ and $0$ otherwise (i.e., low GDP). High GDP is defined as those counties with GDP above the median quantile of the GDP distribution of all counties in $2013$.
    % Please add the following required packages to your document preamble:
% \usepackage{booktabs}
% \usepackage{graphicx}
\begin{table}[H]
    \centering
    \caption{Onsite Releases Intensity given GDP Patterns}
    \label{tab:heterogeneous-onsite-releases-int-gdp}
    \resizebox{\columnwidth}{!}{%
        \begin{tabular}{@{}llllllll@{}}
            \toprule\toprule
            Onsite releases intensity (log) & total     & air emissions & point air & fugitive air & water discharge & land releases & surface impoundment \\ \midrule
            $Treated^{e} \cdot D$           & 0.037     & 0.094         & 0.087     & 0.036        & -0.036          & -0.009        & 0.005               \\
            & (0.082)   & (0.061)       & (0.059)   & (0.047)      & (0.042)         & (0.030)       & (0.004)             \\
            $Treated^{e}$                   & 0.050     & 0.024         & -0.027    & 0.034        & 0.012           & 0.003         & 0.012*              \\
            & (0.063)   & (0.057)       & (0.042)   & (0.056)      & (0.036)         & (0.010)       & (0.007)             \\
            cohort 2014 $\cdot D$           & -0.056    & 0.094         & 0.089     & 0.011        & -0.152          & 0.016         & 0.012**             \\
            & (0.144)   & (0.078)       & (0.064)   & (0.074)      & (0.099)         & (0.016)       & (0.006)             \\
            cohort 2015 $\cdot D$           & 0.187*    & 0.094         & 0.083     & 0.111*       & 0.148**         & -0.050        & -0.008              \\
            & (0.110)   & (0.067)       & (0.092)   & (0.062)      & (0.072)         & (0.061)       & (0.006)             \\
            controls                        & Yes       & Yes           & Yes       & Yes          & Yes             & Yes           & Yes                 \\
            year FE                         & Yes       & Yes           & Yes       & Yes          & Yes             & Yes           & Yes                 \\
            facility FE                     & Yes       & Yes           & Yes       & Yes          & Yes             & Yes           & Yes                 \\
            border-county FE                & Yes       & Yes           & Yes       & Yes          & Yes             & Yes           & Yes                 \\
            toxic chemical FE               & Yes       & Yes           & Yes       & Yes          & Yes             & Yes           & Yes                 \\
            toxic chemical LTs              & Yes       & Yes           & Yes       & Yes          & Yes             & Yes           & Yes                 \\\midrule
            Observations                    & 1,893,689 & 1,893,689     & 1,893,689 & 1,893,689    & 1,893,689       & 1,893,689     & 1,893,689           \\
            $R^2$                           & 0.720     & 0.739         & 0.712     & 0.660        & 0.586           & 0.500         & 0.127               \\ \bottomrule \bottomrule
        \end{tabular}%
    }
    \begin{minipage}{\columnwidth}
        \vspace{0.05in}
        \tiny NOTES: These results are obtained from estimating model~\ref{eq:heterogeneous-onsite-releases-intensity-gdp}. Three-way clustered robust standard errors are reported in parentheses, and clustered at the toxic chemical, industry and state levels. ***, **, and * denote significance levels at the less than $1\%$, $5\%$ and $10\%$, respectively.
    \end{minipage}
\end{table}

    The parameter of interest here is the triple-differences parameter $\beta$ which measures the differential impact on onsite total releases intensity due to a higher MW policy in high GDP counties. $\psi$ measures the differential change in onsite total releases intensity in low GDP counties. And $\beta + \psi$ measures the overall relative change in onsite total releases intensity in high GDP counties. The results are presented in Table~\ref{tab:heterogeneous-onsite-releases-int-gdp}. Notice that the $2017$ cohort is contained in the $\psi$ parameter, as their $2013$ GDP is less than the median.The triple-difference coefficient is negligible for high GDP counties, whereas it is significant and positive at $7.9$ percentage points for low GDP counties. This indicates limited evidence of a differential increase in total onsite release intensity between high and low GDP counties, including across specific cohorts. The overall relative change $(\beta + \psi)$ in total onsite release intensity attributable to higher minimum wage in high GDP counties is $9.1ppts$. Similarly, there is limited evidence of differential increases in total air emissions (both point and fugitive), land releases, and surface water discharge intensities between low and high GDP counties. An exception is the $2015$ cohort, which experienced a significant increase in surface water discharge intensity. The increase in onsite toxic release intensity for low GDP counties is primarily driven by significant rises in surface impoundment intensity. Additionally, the $2014$ cohort shows a notable increase, with an overall relative change of $0.8ppts$.
    \begin{figure}[H]
    \centering
    \includegraphics[width=1\textwidth, height=0.5\textheight,keepaspectratio]{fig_sdid_total_onsite_releases_int_GDP}
    \caption{Triple-Differences: Onsite Total Releases Intensity given Growth Patterns}
    \label{fig:heterogeneous-onsite-releases-intensity-gdp}
    \begin{minipage}{\columnwidth}
        \vspace{0.05in}
        \tiny NOTES: The event study model of equation~\ref{eq:heterogeneous-onsite-releases-intensity-gdp} is $G_{f,cp,c,t}^{gdp} = \sum_{{e = -3},{e \neq -1}}^{3} \beta (Treated^{e} \cdot D)_{f,s,t} + \psi (Treated^{e})_{s,t} + \vartheta (Treated \cdot D)_{f,s,t} + \mu (Post \cdot D)_{f,s,t} + \tau Treated_{s,t} + \rho D_{f,s,t} + \alpha Post_{t} + \delta X_{v,c,t-1} + \omega F_{f,t} + \lambda_{t} + \gamma_{f} + \phi_{cp} + \zeta_{c} + \eta_{c,t} + \theta_{cp,t} + \varepsilon_{f,cp,c,t}$. Three-way clustered robust standard errors are reported in parentheses, and clustered at the toxic chemical, industry and state levels.
    \end{minipage}
\end{figure}

    Figure~\ref{fig:heterogeneous-onsite-releases-intensity-gdp} illustrates dynamic trends across counties, showing limited evidence of a differential increase in total onsite releases intensity for high GDP counties relative to low GDP counties, which experienced a significant increase in the second year after initially raising the minimum wage. This limited evidence for high GDP counties is also observed for total air emissions intensities (both point and fugitive). Conversely, the positive effect on low GDP counties is driven by increases in total air emissions intensities, especially from fugitive emissions in the first and second years post-treatment. There is significant evidence of an increase in surface water discharge intensity for high GDP counties in the third year post-treatment, but this effect is muted for low GDP counties. Land releases intensity declined for low GDP counties in the second year post-treatment, with no significant change for high GDP counties. Lastly, there are significant differential increases in total surface impoundment intensity for low GDP counties three years after the initial minimum wage increase. There is no evidence of significant pre-trends.

    \subsection{Industry Concentration}\label{subsec:industry-concentration}
    In theory, companies operating in highly concentrated industries (HCIs), which face less competition, benefit from economies of scale and maintain dominant market positions~\parencite{baumol1982contestable}. These firms can absorb the costs associated with minimum wage increases due to their large production scales. Moreover, they can transfer the increased labour costs to consumers through higher sales prices or to suppliers by negotiating lower purchase prices. Consequently,~\cite{zhang2023unintended} argued that HCI industries will exhibit a smaller increase in total releases intensities in response to MW hikes. I test whether high concentration (less competitiveness) implies smaller increases in total releases intensities in response to MW increases. To gauge market concentration, I employ the Herfindahl-Hirschman Index (HHI)~\parencite{zhang2023unintended, weinstock1982using}. This index is derived each year by summing the squared market shares of all firms within a six-digit NAICS industry codes. Utilizing revenue data to compute the HHI, I then average these values over the entire sample period and classify industries based on the median HHI value. A lower Herfindahl-Hirschman Index (HHI) indicates that the industry is less concentrated and exhibits greater competitiveness. These are industries below the median HHI value. Those above the median HHI value are classified as high-concentrated industries and exhibit less competition. To investigate the differential effect of higher MW on onsite total releases intensity given industry concentration, I estimate the following model:
    \begin{align}
        P_{f,cp,c,t}^{ind-conc} &= \beta (Treated^{e} \cdot D)_{f,s,t} + \psi (Treated^{e})_{s,t} + \vartheta (Treated \cdot D)_{f,s,t} + \mu (Post \cdot D)_{f,s,t} \nonumber \\
        &\quad + \tau Treated_{s,t} + \rho D_{f,s,t} + \alpha Post_{t} + \delta X_{v,c,t-1} + \omega F_{f,t} + \lambda_{t} + \gamma_{f} + \phi_{cp} \nonumber \\
        &\quad + \zeta_{c} + \eta_{c,t} + \theta_{cp,t} + \varepsilon_{f,cp,c,t},\label{eq:heterogeneous-onsite-releases-intensity-lcis}
    \end{align}
    where $P_{f,cp,c,t}^{ind-conc}$ is the vector of total onsite releases intensity (air, water and land) of toxic chemicals at a low-concentrated manufacturing industry facility, $f$ in a cross-border county pair, $cp$ through a toxic chemical, $c$ in the year, $t$. $Treated_{s,t}$ is a dummy that is equal to $1$ for the treated states, and $0$ the control states. $Post_{t}$ is a dummy that is equal to $1$ if the year $t$ is a post-treatment year, and $0$ otherwise. And $D_{f,s,t}$ is a dummy that is unity for a low-concentrated manufacturing industry facility, $f$ in state, $s$ in the year, $t$ and $0$ for facilities in high-concentrated industries.
    % Please add the following required packages to your document preamble:
% \usepackage{booktabs}
% \usepackage{graphicx}
\begin{table}[H]
    \centering
    \caption{Onsite Releases Intensity given Industry Concentration}
    \label{tab:heterogeneous-onsite-releases-int-lcis}
    \resizebox{\columnwidth}{!}{%
        \begin{tabular}{@{}llllllll@{}}
            \toprule \toprule
            Onsite total releases intensity (log) & total     & air emissions & point air & fugitive air & water discharge & land releases & surface impoundment \\ \midrule
            $Treated^{e} \cdot D$                 & -0.040    & -0.025        & -0.082    & 0.054        & 0.017           & 0.028         & 0.000               \\
            & (0.068)   & (0.061)       & (0.062)   & (0.054)      & (0.027)         & (0.028)       & (0.004)             \\
            $Treated^{e}$                         & 0.085     & 0.054         & 0.045     & 0.031        & -0.001          & 0.016         & 0.009***            \\
            & (0.064)   & (0.060)       & (0.057)   & (0.050)      & (0.034)         & (0.011)       & (0.004)             \\
            cohort 2014 $\cdot D$                 & -0.103    & -0.044        & -0.092    & 0.015        & -0.073**        & -0.020        & 0.003               \\
            & (0.083)   & (0.082)       & (0.082)   & (0.067)      & (0.034)         & (0.037)       & (0.007)             \\
            cohort 2015 $\cdot D$                 & 0.030     & -0.003        & -0.072    & 0.097*       & 0.120**         & -0.037        & -0.003              \\
            & (0.088)   & (0.076)       & (0.073)   & (0.056)      & (0.050)         & (0.035)       & (0.003)             \\
            cohort 2017 $\cdot D$                 & 0.062     & 0.018         & -0.042    & 0.098        & 0.050           & 0.036         & -0.004              \\
            & (0.137)   & (0.108)       & (0.099)   & (0.104)      & (0.065)         & (0.034)       & (0.005)             \\
            controls                              & Yes       & Yes           & Yes       & Yes          & Yes             & Yes           & Yes                 \\
            year FE                               & Yes       & Yes           & Yes       & Yes          & Yes             & Yes           & Yes                 \\
            facility FE                           & Yes       & Yes           & Yes       & Yes          & Yes             & Yes           & Yes                 \\
            border-county FE                      & Yes       & Yes           & Yes       & Yes          & Yes             & Yes           & Yes                 \\
            toxic chemical FE                     & Yes       & Yes           & Yes       & Yes          & Yes             & Yes           & Yes                 \\
            toxic chemical LTs                    & Yes       & Yes           & Yes       & Yes          & Yes             & Yes           & Yes                 \\
            border-county LTs                     & Yes       & Yes           & Yes       & Yes          & Yes             & Yes           & Yes                 \\\midrule
            Observations                          & 1,893,689 & 1,893,689     & 1,893,689 & 1,893,689    & 1,893,689       & 1,893,689     & 1,893,689           \\
            $R^2$                                 & 0.728     & 0.746         & 0.719     & 0.670        & 0.595           & 0.507         & 0.159               \\ \bottomrule\bottomrule
        \end{tabular}%
    }
    \begin{minipage}{\columnwidth}
        \vspace{0.05in}
        \tiny NOTES: These results are obtained from estimating model~\ref{eq:heterogeneous-onsite-releases-intensity-lcis}. Three-way clustered robust standard errors are reported in parentheses, and clustered at the toxic chemical, industry and state levels. ***, **, and * denote significance levels at the less than $1\%$, $5\%$ and $10\%$, respectively.
    \end{minipage}
\end{table}

    The parameter of interest here is the triple-differences parameter $\beta$ which measures the differential impact on onsite total releases intensity due to a higher MW policy for manufacturing facilities in low-concentrated industries. $\psi$ measures the relative change in onsite total releases intensity for manufacturing facilities in high-concentrated industries. And $\beta + \psi$ measures the total relative change in onsite total releases intensity for manufacturing facilities in low-concentrated industries. The results are presented in Table~\ref{tab:heterogeneous-onsite-releases-int-lcis}. The differential impact of higher MW on total onsite release intensity in low-concentration manufacturing facilities significantly decreases by $9.9ppts$, with the most pronounced effect observed in the $2014$ cohorts. This reduction is primarily driven by decreases in surface water discharge and land release intensities, including surface impoundment intensity, in the $2014$ and $2017$ cohorts. Limited evidence is found regarding total air emission intensities (both point and fugitive) in low-concentrated manufacturing facilities. However, in contrast to highly-concentrated manufacturing facilities, total air release intensity is significantly impacted by higher MW policy, increasing by $12.4ppts$. This finding contradicts~\cite{zhang2023unintended}. I argue that in higher output volumes, the environmental impact of higher MW remains significant, particularly in fugitive air emissions and land release (including surface impoundment) intensities.
    \begin{figure}[H]
    \centering
    \includegraphics[width=1\textwidth, height=0.5\textheight,keepaspectratio]{fig_sdid_total_onsite_releases_int_lowindconc}
    \caption{Triple-Differences: Onsite Total Releases Intensity Industry Concentration}
    \label{fig:heterogeneous-onsite-releases-intensity-lcis}
    \begin{minipage}{18cm}
        \vspace{0.05in}
        \tiny NOTES: The event study model of equation~\ref{eq:heterogeneous-onsite-releases-intensity-lcis} is $G_{f,c,i,cp,s,t}^{ind-conc} = \sum_{{e = -3},{e \neq -1}}^{3} \beta (Treated^{e} \cdot D)_{f,s,t} + \psi (Treated^{e})_{s,t} + \vartheta (Treated \cdot D)_{f,s,t} + \mu (Post \cdot D)_{f,s,t} + \tau Treated_{s,t} + \rho D_{f,s,t} + \alpha Post_{t} + \delta X_{v,c,t-1} + \omega F_{f,t} + \lambda_{t} + \gamma_{f} + \phi_{cp} + \zeta_{c} + \eta_{c,t} + \theta_{cp,t} + \varepsilon_{f,cp,c,t}$. Three-way clustered robust standard errors are reported in parentheses, and clustered at the toxic chemical, industry and state levels.
    \end{minipage}
\end{figure}

    Similar patterns are observed in the dynamic effects presented in Figure~\ref{fig:heterogeneous-onsite-releases-intensity-lcis}. In highly competitive industries, total toxic release intensity decreases one year after, and again in the third year of raising the minimum wage (MW), primarily due to reductions in surface water discharges and land release intensities, particularly surface impoundment. Conversely, in low competitive industries, total toxic release intensity increases immediately and consistently, driven by fugitive air emissions and surface impoundment intensities. There is no evidence of significant pre-trends.

    \subsection{Carcinogenic Chemicals}\label{subsec:carcinogenic-chemicals}
    In this subsection, I check whether the MW policy is potentially carcinogenic. To investigate the differential effect of higher MW on onsite total releases intensity of carcinogenic chemicals, I estimate the following model:
    \begin{align}
        P_{f,cp,c,t}^{carcinogen} &= \beta (Treated^{e} \cdot D)_{f,s,t} + \psi (Treated^{e})_{s,t} + \vartheta (Treated \cdot D)_{f,s,t} + \mu (Post \cdot D)_{f,s,t} \nonumber \\
        &\quad + \tau Treated_{s,t} + \rho D_{f,s,t} + \alpha Post_{t} + \delta X_{v,c,t-1} + \omega F_{f,t} + \lambda_{t} + \gamma_{f} + \phi_{cp} \nonumber \\
        &\quad + \zeta_{c} + \eta_{c,t} + \theta_{cp,t} + \varepsilon_{f,cp,c,t},\label{eq:heterogeneous-onsite-releases-intensity-carcinogens}
    \end{align}
    where $P_{f,cp,c,t}^{carcinogen}$ is the vector of total onsite releases intensity (air, water and land) of toxic chemicals at a manufacturing industry facility, $f$ in a cross-border county pair, $cp$ through a toxic chemical, $c$ in the year, $t$. $Treated_{s,t}$ is a dummy that is equal to $1$ for the treated states, and $0$ the control states. $Post_{t}$ is a dummy that is equal to $1$ if the year $t$ is a post-treatment year, and $0$ otherwise. And $D_{f,s,t}$ is a dummy that is unity for a carcinogenic chemical at manufacturing facility, $f$ in state, $s$ in the year, $t$ and $0$ otherwise. Carcinogenic chemicals are toxic chemicals that can cause cancer in both humans and animals alike. Examples include benzene, formaldehyde, arsenic, and vinyl chloride, etc.
    % Please add the following required packages to your document preamble:
% \usepackage{booktabs}
% \usepackage{graphicx}
\begin{table}[H]
    \centering
    \caption{Onsite Releases Intensity for Carcinogenic Chemicals}
    \label{tab:heterogeneous-onsite-releases-int-carcinogens}
    \resizebox{\columnwidth}{!}{%
        \begin{tabular}{@{}llllllll@{}}
            \toprule\toprule
            Onsite releases intensity (log) & total     & air emissions & point air & fugitive air & water discharge & land releases & surface impoundment \\ \midrule
            $Treated^{e} \cdot D$           & 0.012     & 0.063         & 0.106     & -0.006       & -0.121*         & -0.040        & -0.015**            \\
            & (0.095)   & (0.090)       & (0.085)   & (0.070)      & (0.065)         & (0.040)       & (0.007)             \\
            $Treated^{e}$                   & 0.106*    & 0.037         & -0.030    & 0.043        & 0.069*          & 0.022         & 0.014**             \\
            & (0.056)   & (0.045)       & (0.040)   & (0.041)      & (0.038)         & (0.014)       & (0.007)             \\
            cohort 2014 $\cdot D$           & -0.037    & -0.025        & 0.024     & -0.087       & -0.089          & -0.053        & -0.013*             \\
            & (0.108)   & (0.107)       & (0.098)   & (0.072)      & (0.062)         & (0.065)       & (0.007)             \\
            cohort 2015 $\cdot D$           & 0.105     & 0.231*        & 0.264**   & 0.146        & -0.180**        & -0.014        & -0.019**            \\
            & (0.127)   & (0.123)       & (0.113)   & (0.117)      & (0.090)         & (0.030)       & (0.008)             \\
            cohort 2017 $\cdot D$           & 0.132     & 0.267         & 0.174     & 0.345**      & -0.168          & 0.074         & -0.011              \\
            & (0.298)   & (0.259)       & (0.239)   & (0.170)      & (0.151)         & (0.047)       & (0.013)             \\
            controls                        & Yes       & Yes           & Yes       & Yes          & Yes             & Yes           & Yes                 \\
            year FE                         & Yes       & Yes           & Yes       & Yes          & Yes             & Yes           & Yes                 \\
            facility FE                     & Yes       & Yes           & Yes       & Yes          & Yes             & Yes           & Yes                 \\
            border-county FE                & Yes       & Yes           & Yes       & Yes          & Yes             & Yes           & Yes                 \\
            toxic chemical FE               & Yes       & Yes           & Yes       & Yes          & Yes             & Yes           & Yes                 \\
            toxic chemical LTs              & Yes       & Yes           & Yes       & Yes          & Yes             & Yes           & Yes                 \\
            border-county LTs               & Yes       & Yes           & Yes       & Yes          & Yes             & Yes           & Yes                 \\\midrule
            Observations                    & 1,893,689 & 1,893,689     & 1,893,689 & 1,893,689    & 1,893,689       & 1,893,689     & 1,893,689           \\
            $R^2$                           & 0.727     & 0.746         & 0.719     & 0.670        & 0.594           & 0.507         & 0.159               \\ \bottomrule\bottomrule
        \end{tabular}%
    }
    \begin{minipage}{\columnwidth}
        \vspace{0.05in}
        \tiny NOTES: These results are obtained from estimating model~\ref{eq:heterogeneous-onsite-releases-intensity-carcinogens}. Three-way clustered robust standard errors are reported in parentheses, and clustered at the toxic chemical, industry and state levels. ***, **, and * denote significance levels at the less than $1\%$, $5\%$ and $10\%$, respectively.
    \end{minipage}
\end{table}

    The parameter of interest here is the triple-differences parameter $\beta$ which measures the differential impact on onsite total releases intensity due to a higher MW policy for carcinogenic chemicals at manufacturing facilities. $\psi$ measures the relative change in onsite total releases intensity for non-carcinogenic chemicals. And $\beta + \psi$ measures the total relative change in onsite total releases intensity for carcinogenic chemicals. The results are presented in Table~\ref{tab:heterogeneous-onsite-releases-int-carcinogens}. There is limited evidence on the differential impact of higher MW on total carcinogenic release intensity, particularly in air emissions, including point and fugitive emissions. However, cohort-specific effects suggest that high MW may potentially increase carcinogenic releases for the 2015 and 2017 cohorts. Additionally, there is a significant differential decline in surface water discharge and surface impoundment intensities of carcinogenic chemicals, notably in the 2014 and 2015 cohorts. In contrast, the intensity of non-carcinogenic chemical releases increases, particularly for surface water discharge and impoundment intensities. Limited evidence is found for air emissions (both point and fugitive) and land release intensities of non-carcinogenic chemicals.
    \begin{figure}[H]
    \centering
    \includegraphics[width=1\textwidth, height=0.5\textheight,keepaspectratio]{fig_sdid_total_onsite_releases_int_carcinogens}
    \caption{Triple-Differences: Onsite Total Releases Intensity for Carcinogens}
    \label{fig:heterogeneous-onsite-releases-intensity-carcinogens}
    \begin{minipage}{18cm}
        \vspace{0.05in}
        NOTES: The event study model of equation~\ref{eq:heterogeneous-onsite-releases-intensity-carcinogens} is $G_{f,cp,c,t}^{carcinogen} = \sum_{{e = -3},{e \neq -1}}^{3} \beta (Treated^{e} \cdot D)_{h,s,t} + \psi (Treated^{e})_{s,t} + \vartheta (Treated \cdot D)_{h,s,t} + \mu (Post \cdot D)_{h,s,t} + \tau Treated_{s,t} + \rho D_{h,s,t} + \alpha Post_{t} + \delta X_{v,c,t-1} + \omega F_{f,t} + \lambda_{t} + \gamma_{f} + \phi_{cp} + \eta_{c,t} + \zeta_{c} + \varepsilon_{f,cp,c,t}$. Three-way clustered robust standard errors are reported in parentheses, and clustered at the toxic chemical, industry and state levels.
    \end{minipage}
\end{figure}

    The dynamic effects shown in Figure~\ref{fig:heterogeneous-onsite-releases-intensity-carcinogens} indicate a similar pattern. There is limited evidence of differential total toxic release intensity for carcinogenic chemicals, particularly in fugitive air emissions and land releases. Point air emissions intensity increases instantaneously but becomes negligible within one to three years after a statutory MW raise. Additionally, there is an immediate and consistent significant decline in carcinogenic surface water discharge and surface impoundment intensities. Conversely, non-carcinogenic total onsite toxic release intensity significantly increases, especially for fugitive air emissions, surface water discharge, and surface impoundment intensities, predominantly observed two to three years post-treatment. No significant pre-trends are evident.

    \subsection{Clean Air Act Chemicals or Hazardous Air Pollutants}\label{subsec:clean-air-act-chemicals-haps}
    This section investigates the question: does higher MW have any significant differential impact on total onsite releases intensities of clean air act chemicals (CAA) and hazardous air pollutants (HAPs)? CAA chemicals are toxic chemicals heavily regulated under the CAA of $1970$. The CAA is a comprehensive federal law enacted by the United States Congress to address air pollution control and improve air quality standards across the nation. The primary goals of the CAA are to protect public health and the environment by regulating the emission of harmful air pollutants. HAPs, also known as air toxics, are pollutants that are known or suspected to cause serious health and environmental effects. These pollutants are regulated under the CAA Amendments of $1990$.~\footnote{\tiny  Examples include heavy metals (mercury, lead, cadmium, and chromium are examples of heavy metals that can be released into the air from industrial processes, combustion of fossil fuels, and waste incineration), VOCs (e.g., include benzene, toluene, xylene, and formaldehyde, which are easily emitted into the air from industrial sources and consumer products. Exposure to VOCs can cause respiratory problems, neurological effects, and contribute to the formation of ground-level ozone and smog), polycyclic aromatic hydrocarbons (PAHs) from industrial processes (e.g., include organic compounds formed during the incomplete combustion of fossil fuels, wood, and other organic materials. Some PAHs are known carcinogens and can also cause developmental and reproductive effects), persistent organic compounds (organic compounds that resist degradation in the environment and can accumulate in living organisms. Examples include dioxins, and certain pesticides such as dichlorodiphenyltrichloroethane (DDT) and PCBs. These chemicals can travel long distances through air and water, posing risks to ecosystems and human health), and chlorinated compounds (such as chloroform and vinyl chloride, are byproducts of industrial processes, including chemical manufacturing and waste incineration. Exposure to these chemicals can cause liver and kidney damage, as well as neurological effects). Regulation of hazardous air pollutants under the Clean Air Act involves the development of technology-based standards for industrial sources to control emissions of these pollutants. The EPA establishes Maximum Achievable Control Technology (MACT) standards for specific source categories, such as chemical plants, petroleum refineries, and pulp and paper mills, to reduce emissions of hazardous air pollutants to the maximum extent feasible. Facilities subject to MACT standards are required to install pollution control equipment and implement management practices to minimize emissions of hazardous air pollutants. Compliance with MACT standards helps protect public health and the environment by reducing exposure to toxic air pollutants and preventing adverse health effects.} To investigate the above question, I estimate the following model:
    \begin{align}
        P_{f,cp,c,t}^{caa-haps} &= \beta (Treated^{e} \cdot D)_{f,s,t} + \psi (Treated^{e})_{s,t} + \vartheta (Treated \cdot D)_{f,s,t} + \mu (Post \cdot D)_{f,s,t} \nonumber \\
        &\quad + \tau Treated_{s,t} + \rho D_{f,s,t} + \alpha Post_{t} + \delta X_{v,c,t-1} + \omega F_{f,t} + \lambda_{t} + \gamma_{f} + \phi_{cp} \nonumber \\
        &\quad + \zeta_{c} + \eta_{c,t} + \theta_{cp,t} + \varepsilon_{f,cp,c,t},\label{eq:heterogeneous-onsite-releases-intensity-caahaps}
    \end{align}
    where $P_{f,cp,c,t}^{caa-haps}$ is the vector of total onsite releases intensity (air, water and land) of toxic chemicals at a manufacturing industry facility, $f$ in a cross-border county pair, $cp$ through a toxic chemical, $c$ in the year, $t$. $Treated_{s,t}$ is a dummy that is equal to $1$ for the treated states, and $0$ the control states. $Post_{t}$ is a dummy that is equal to $1$ if the year $t$ is a post-treatment year, and $0$ otherwise. And $D_{f,s,t}$ is a dummy that is unity for a CAA chemical or HAP at manufacturing facility, $f$ in state, $s$ in the year, $t$ and $0$ otherwise.
    % Please add the following required packages to your document preamble:
% \usepackage{booktabs}
% \usepackage{graphicx}
\begin{table}[H]
    \centering
    \caption{Onsite Releases Intensity for CAA and HAPs Chemicals}
    \label{tab:heterogeneous-onsite-releases-int-caa-haps}
    \resizebox{\columnwidth}{!}{%
        \begin{tabular}{@{}llllllll@{}}
            \toprule\toprule
            Onsite releases intensity (log) & total     & air emissions & point air & fugitive air & water discharge & land releases & surface impoundment \\ \midrule
            $Treated^{e} \cdot D$           & 0.076     & 0.212***      & 0.110     & 0.125**      & -0.174***       & -0.064        & -0.002              \\
            & (0.094)   & (0.082)       & (0.074)   & (0.059)      & (0.060)         & (0.049)       & (0.011)             \\
            $Treated^{e}$                   & 0.069     & -0.039        & -0.050    & 0.053        & 0.107**         & 0.035*        & 0.013**             \\
            & (0.067)   & (0.059)       & (0.055)   & (0.041)      & (0.044)         & (0.018)       & (0.005)             \\
            cohort 2014 $\cdot D$           & 0.039     & 0.173         & 0.116     & 0.094        & -0.203***       & -0.098        & -0.002              \\
            & (0.117)   & (0.107)       & (0.097)   & (0.065)      & (0.063)         & (0.073)       & (0.011)             \\
            cohort 2015 $\cdot D$           & 0.134     & 0.275**       & 0.092     & 0.180        & -0.126          & -0.006        & -0.003              \\
            & (0.129)   & (0.130)       & (0.121)   & (0.129)      & (0.087)         & (0.039)       & (0.010)             \\
            cohort 2017 $\cdot D$           & 0.875***  & 0.839***      & 0.922***  & 0.175        & 0.155           & 0.018         & -0.006              \\
            & (0.280)   & (0.267)       & (0.285)   & (0.206)      & (0.281)         & (0.043)       & (0.012)             \\
            controls                        & Yes       & Yes           & Yes       & Yes          & Yes             & Yes           & Yes                 \\
            year FE                         & Yes       & Yes           & Yes       & Yes          & Yes             & Yes           & Yes                 \\
            facility FE                     & Yes       & Yes           & Yes       & Yes          & Yes             & Yes           & Yes                 \\
            border-county FE                & Yes       & Yes           & Yes       & Yes          & Yes             & Yes           & Yes                 \\
            toxic chemical FE               & Yes       & Yes           & Yes       & Yes          & Yes             & Yes           & Yes                 \\
            toxic chemical LTs              & Yes       & Yes           & Yes       & Yes          & Yes             & Yes           & Yes                 \\
            border-county LTs               & Yes       & Yes           & Yes       & Yes          & Yes             & Yes           & Yes                 \\ \midrule
            Observations                    & 1,893,689 & 1,893,689     & 1,893,689 & 1,893,689    & 1,893,689       & 1,893,689     & 1,893,689           \\
            $R^2$                           & 0.727     & 0.746         & 0.718     & 0.671        & 0.595           & 0.507         & 0.159               \\ \bottomrule \bottomrule
        \end{tabular}%
    }
    \begin{minipage}{\columnwidth}
        \vspace{0.05in}
        \tiny NOTES: These results are obtained from estimating model~\ref{eq:heterogeneous-onsite-releases-intensity-caahaps}. Three-way clustered robust standard errors are reported in parentheses, and clustered at the toxic chemical, industry and state levels. ***, **, and * denote significance levels at the less than $1\%$, $5\%$ and $10\%$, respectively.
    \end{minipage}
\end{table}

    The parameter of interest here is the triple-differences parameter $\beta$ which measures the differential impact on onsite total releases intensity due to a higher MW policy for CAA chemicals or HAPs at manufacturing facilities. $\psi$ measures the relative change in onsite total releases intensity for non-CAA-HAPs chemicals. And $\beta + \psi$ measures the overall relative change in onsite total releases intensity for CAA chemicals or HAPs. The results are presented in Table~\ref{tab:heterogeneous-onsite-releases-int-caa-haps}.For chemicals regulated under the CAA-HAPs framework, the results demonstrate a substantial and consistent differential reduction in the intensity of total toxic releases, driven primarily by decreases in both point and fugitive air emissions and surface water discharge. This reduction is significant across all cohorts. Conversely, there is limited evidence of an increase in land releases, including surface impoundment intensities, with the overall relative change in these intensities remaining close to baseline estimates. In contrast, for non-CAA-HAPs chemicals, there is a notable and significant differential increase in the intensity of total toxic releases, driven by increases in both point and fugitive air emissions, surface water discharge, and surface impoundment intensities. While there is limited evidence of changes in total land releases intensity, this increase is significant across all cohorts, except for surface impoundment intensity. These findings offer insights into the behaviour of manufacturing industries in response to stringent regulatory policies for CAA-HAPs chemicals, suggesting that industries may prioritize reducing emissions from regulated CAA or HAPs chemicals at the expense of increasing emissions from non-regulated chemicals.
    \begin{figure}[H]
    \centering
    \includegraphics[width=1\textwidth, height=0.5\textheight,keepaspectratio]{fig_sdid_total_onsite_releases_int_caahaps}
    \caption{Triple-Differences: Onsite Total Releases Intensity for CAA and HAPs Chemicals}
    \label{fig:heterogeneous-onsite-releases-intensity-caa-haps}
    \begin{minipage}{\columnwidth}
        \vspace{0.05in}
        \tiny NOTES: The event study model of equation~\ref{eq:heterogeneous-onsite-releases-intensity-caahaps} is $G_{f,cp,c,t}^{caa} = \sum_{{e = -3},{e \neq -1}}^{3} \beta (Treated^{e} \cdot D)_{f,s,t} + \psi (Treated^{e})_{s,t} + \vartheta (Treated \cdot D)_{f,s,t} + \mu (Post \cdot D)_{f,s,t} + \tau Treated_{s,t} + \rho D_{f,s,t} + \alpha Post_{t} + \delta X_{v,c,t-1} + \omega F_{f,t} + \lambda_{t} + \gamma_{f} + \phi_{cp} + \zeta_{c} + \eta_{c,t} + \theta_{cp,t} + \varepsilon_{f,cp,c,t}$. Three-way clustered robust standard errors are reported in parentheses, and clustered at the toxic chemical, industry and state levels.
    \end{minipage}
\end{figure}
    The dynamic results depicted in Figure~\ref{fig:heterogeneous-onsite-releases-intensity-caa-haps} reveal consistent effect patterns. For CAA-HAPs chemicals, there are significant decreases in the intensity of total toxic releases, including fugitive air emissions and surface water discharge, occurring immediately and persisting throughout the post-treatment period. Limited evidence is found for changes in point air emissions and surface impoundment intensities. In contrast, for non-CAA-HAPs chemicals, there are significant differential increases in the intensity of total toxic releases, driven by increases in both point and fugitive air emissions, surface water discharge, and surface impoundment intensities, also occurring immediately and persisting throughout the post-treatment period. Limited evidence is found for changes in total land releases intensity. No significant pre-trends are observed. These findings highlight the dynamic responses of these manufacturing facilities to high MW policies for CAA regulated and non-CAA regulated toxic chemicals.

    \subsection{Persistent Bio-accumulative Toxic Chemicals}\label{subsec:persistent-bioaccumulative-toxic-chemicals}
    This sections asks the question: does a higher MW regime exert significant differential impact on persistent bio-accumulative toxic chemicals (PBTs)?~\footnote{\tiny PBTs are a group of chemicals characterized by their persistence in the environment, ability to accumulate in living organisms, and toxicity. These chemicals pose significant risks to human health and the environment due to their long-term effects and potential for biomagnification in food chains. Examples of PBTs include certain persistent organic pollutants (POPs), such as polychlorinated biphenyls (PCBs), dioxins, and certain pesticides like dichlorodiphenyltrichloroethane (DDT). These chemicals have been widely used in industrial processes, agriculture, and consumer products. Due to their persistence, bioaccumulation potential, and toxicity, PBTs are of particular concern to environmental regulators and policymakers. Efforts to reduce PBT emissions and exposure often involve regulatory measures, such as bans or restrictions on their use, as well as pollution prevention and cleanup programs to mitigate their impact on human health and the environment. PBTs are heavily regulated under the Toxic Substances Control Act (TSCA) of $1976$ by the US EPA. Some of these regulatory actions by the EPA on manufacturing facilities on the use of PBTs include chemical testing, reporting and recording keeping, restriction and bans, risk management assessment, labelling and notification, etc.} To investigate the above question, I estimate the following model:
    \begin{align}
        P_{f,cp,c,t}^{pbts} &= \beta (Treated^{e} \cdot D)_{f,s,t} + \psi (Treated^{e})_{s,t} + \vartheta (Treated \cdot D)_{f,s,t} + \mu (Post \cdot D)_{f,s,t} \nonumber \\
        &\quad + \tau Treated_{s,t} + \rho D_{f,s,t} + \alpha Post_{t} + \delta X_{v,c,t-1} + \omega F_{f,t} + \lambda_{t} + \gamma_{f} + \phi_{cp} \nonumber \\
        &\quad + \zeta_{c} + \eta_{c,t} + \theta_{cp,t} + \varepsilon_{f,cp,c,t},\label{eq:heterogeneous-onsite-releases-intensity-pbts}
    \end{align}
    where $P_{f,cp,c,t}^{pbts}$ is the vector of total onsite releases intensity (air, water and land) of toxic chemicals at a manufacturing facility, $f$ in a cross-border county pair, $cp$ through a toxic chemical, $c$ in the year, $t$. $Treated_{s,t}$ is a dummy that is equal to $1$ for the treated states, and $0$ the control states. $Post_{t}$ is a dummy that is equal to $1$ if the year $t$ is a post-treatment year, and $0$ otherwise. And $D_{f,s,t}$ is a dummy that is unity for a PBT chemical at manufacturing facility, $f$ in state, $s$ in the year, $t$ and $0$ otherwise.
    % Please add the following required packages to your document preamble:
% \usepackage{booktabs}
% \usepackage{graphicx}
\begin{table}[H]
    \centering
    \caption{Onsite Releases Intensity for PBTs}
    \label{tab:heterogeneous-onsite-releases-int-pbts}
    \resizebox{\columnwidth}{!}{%
        \begin{tabular}{@{}llllllll@{}}
            \toprule\toprule
            Onsite releases intensity (log) & total     & air emissions & point air & fugitive air & water discharge & land releases & surface impoundment \\ \midrule
            $Treated^{e} \cdot D$           & -0.032    & -0.122        & -0.176*   & 0.037        & 0.138           & -0.076**      & -0.007              \\
            & (0.104)   & (0.105)       & (0.092)   & (0.104)      & (0.092)         & (0.035)       & (0.009)             \\
            $Treated^{e}$                   & 0.147***  & 0.101**       & 0.061*    & 0.042        & 0.032           & 0.019         & 0.012*              \\
            & (0.048)   & (0.040)       & (0.034)   & (0.033)      & (0.032)         & (0.014)       & (0.006)             \\
            cohort 2014 $\cdot D$           & 0.020     & -0.153*       & -0.183*   & 0.074        & 0.254**         & -0.095*       & -0.000              \\
            & (0.139)   & (0.090)       & (0.095)   & (0.100)      & (0.120)         & (0.054)       & (0.009)             \\
            cohort 2015 $\cdot D$           & -0.117    & -0.063        & -0.159    & -0.030       & -0.067          & -0.056        & -0.019              \\
            & (0.142)   & (0.186)       & (0.135)   & (0.189)      & (0.094)         & (0.048)       & (0.012)             \\
            cohort 2017 $\cdot D$           & -1.220*** & -0.839***     & -1.090*** & 0.074        & -0.625**        & -0.070        & -0.011              \\
            & (0.318)   & (0.286)       & (0.324)   & (0.196)      & (0.269)         & (0.053)       & (0.015)             \\
            controls                        & Yes       & Yes           & Yes       & Yes          & Yes             & Yes           & Yes                 \\
            year FE                         & Yes       & Yes           & Yes       & Yes          & Yes             & Yes           & Yes                 \\
            facility FE                     & Yes       & Yes           & Yes       & Yes          & Yes             & Yes           & Yes                 \\
            border-county FE                & Yes       & Yes           & Yes       & Yes          & Yes             & Yes           & Yes                 \\
            toxic chemical FE               & Yes       & Yes           & Yes       & Yes          & Yes             & Yes           & Yes                 \\
            toxic chemical LTs              & Yes       & Yes           & Yes       & Yes          & Yes             & Yes           & Yes                 \\
            border-county LTs               & Yes       & Yes           & Yes       & Yes          & Yes             & Yes           & Yes                 \\\midrule
            Observations                    & 1,893,689 & 1,893,689     & 1,893,689 & 1,893,689    & 1,893,689       & 1,893,689     & 1,893,689           \\
            $R^2$                           & 0.727     & 0.746         & 0.719     & 0.670        & 0.594           & 0.507         & 0.159               \\ \bottomrule\bottomrule
        \end{tabular}%
    }
    \begin{minipage}{\columnwidth}
        \vspace{0.05in}
        \tiny NOTES: These results are obtained from estimating model~\ref{eq:heterogeneous-onsite-releases-intensity-pbts}. Three-way clustered robust standard errors are reported in parentheses, and clustered at the toxic chemical, industry and state levels. ***, **, and * denote significance levels at the less than $1\%$, $5\%$ and $10\%$, respectively.
    \end{minipage}
\end{table}

    The parameter of interest here is the triple-differences parameter $\beta$ which measures the differential impact on onsite total releases intensity due to a higher MW policy for PBTs chemicals at manufacturing facilities. $\psi$ measures the relative change in onsite total releases intensity for non-PBTs chemicals. And $\beta + \psi$ measures the overall relative change in onsite total releases intensity for PBTs chemicals. The results are presented in Table~\ref{tab:heterogeneous-onsite-releases-int-pbts}. For PBT chemicals, there is limited evidence of changes in total releases intensity, including total air (both fugitive and point source) emissions, surface water discharge, and surface impoundment intensities. However, cohort-specific effects for $2014$ and $2017$ indicate significant differential declines in the intensities of total releases and air emissions, including point sources, as well as land releases. Overall, point air emissions and land releases intensities for PBT chemicals show a significant decline. The effects on surface water discharge intensity for PBT chemicals are heterogeneous. In contrast, for non-PBT chemicals, significant differential increases are observed in total releases intensity, driven by increases in total air emissions (point sources) and surface impoundment intensities, except for fugitive emissions, surface water discharge, and land releases intensities. Consequently, the adverse environmental impacts of higher MW are more pronounced in non-PBT chemicals.
    \begin{figure}[H]
    \centering
    \includegraphics[width=1\textwidth, height=0.5\textheight,keepaspectratio]{fig_sdid_total_onsite_releases_int_pbts}
    \caption{Triple-Differences: Onsite Total Releases Intensity for PBTs}
    \label{fig:heterogeneous-onsite-releases-intensity-pbts}
    \begin{minipage}{\columnwidth}
        \vspace{0.05in}
        \tiny NOTES: The event study model of equation~\ref{eq:heterogeneous-onsite-releases-intensity-pbts} is $G_{f,cp,c,t}^{pbts} = \sum_{{e = -3},{e \neq -1}}^{3} \beta (Treated^{e} \cdot D)_{f,s,t} + \psi (Treated^{e})_{s,t} + \vartheta (Treated \cdot D)_{f,s,t} + \mu (Post \cdot D)_{f,s,t} + \tau Treated_{s,t} + \rho D_{f,s,t} + \alpha Post_{t} + \delta X_{v,c,t-1} + \omega F_{f,t} + \lambda_{t} + \gamma_{f} + \phi_{cp} + \zeta_{c} + \eta_{c,t} + \theta_{cp,t} + \varepsilon_{f,cp,c,t}$. Three-way clustered robust standard errors are reported in parentheses, and clustered at the toxic chemical, industry and state levels.
    \end{minipage}
\end{figure}

    The dynamic effects illustrated in Figure~\ref{fig:heterogeneous-onsite-releases-intensity-pbts} reveal similar patterns. For PBT chemicals, the intensities of total air emissions from point sources and land releases (including surface impoundment) decline immediately and persist for up to three years. Conversely, surface water discharge intensities increase instantly and remain elevated for up to three years, suggesting potential adverse effects on the aquatic ecosystem due to higher MW policy. Limited evidence is found for changes in fugitive air emissions and overall air emissions intensities. In contrast, the dynamic effects indicate that the increasing adverse environmental impact is more pronounced for non-PBT chemicals, as evidenced by rises in overall toxic releases, air emissions (both point source and fugitive), surface water discharge, and surface impoundment intensities. These effects are immediate and consistent throughout the post-treatment periods. Limited evidence is found for total land releases intensity, and no significant pre-trends are observed.


    \section{Mechanism Analyses}\label{sec:mechanism-analyses}
    In this section, I investigate the potential transmission mechanisms of the above results through two broad lenses: onsite waste management and source reduction activities. From theory, the analyses are examined based on the financial constraints and type of production technology, $(D^{fintech}_{f,s,t})$ across manufacturing industries, as defined in subsection~\ref{subsec:financial-constraints-and-production-technology}.

    \subsection{Onsite Waste Management Activities}\label{subsec:onsite-waste-management-activities}
    Onsite waste management activities refers to the handling and management of already generated toxic wastes at onsite manufacturing facilities. These include treatment, energy recovery, and recycling. Treatment involves processes used to change the physical, chemical, or biological composition of a waste to make it less hazardous or easier to manage. It uses the following methods: biological, physical and incineration methods.~\footnote{\tiny Biological treatment necessitates highly skilled microbiologists and environmental engineers to monitor and maintain optimal conditions for the biodegradation of hazardous wastes. Incineration demands expert personnel to guarantee complete combustion, manage residual ash safely, and ensure environmental compliance. Similarly, physical treatment requires proficient labor to execute filtration, sedimentation, and adsorption processes for contaminant removal from waste.} Similarly, energy recovery involves processes that use waste materials as a source of energy through methods such as industrial boiler. Industrial boiler energy recovery can reduce the volume of disposed waste and offset the use of fossil fuels. It requires high-skilled operators to manage fuel feed, combustion processes, and maintenance. Lastly, recycling activities to reuse or reclaim materials from toxic chemical waste streams such as depolymerization of plastics, requires specialised knowledge in chemistry, material science, and engineering. To investigate these onsite waste management activities, I estimate the following model:
    \begin{align}
        M_{f,cp,c,t}^{wma} &= \beta (Treated^{e} \cdot D^{fintech})_{f,s,t} + \psi (Treated^{e})_{s,t} + \vartheta (Treated \cdot D^{fintech})_{f,s,t} + \mu (Post \cdot D^{fintech})_{f,s,t} \nonumber \\
        &\quad + \tau Treated_{s,t} + \rho D_{f,s,t}^{fintech} + \alpha Post_{t} + \delta X_{v,c,t-1} + \omega F_{f,t} + \lambda_{t} + \gamma_{f} + \phi_{cp} \nonumber \\
        &\quad + \zeta_{c} + \eta_{c,t} + \theta_{cp,t} + \varepsilon_{f,cp,c,t},\label{eq:mechanisms-waste-management}
    \end{align}
    where $M_{f,cp,c,t}^{wma}$ is the vector of logged total onsite waste management activities (including treatment, energy recovery and recycling) of already generated toxic wastes at manufacturing industry facility, $f$ in cross-border county pairs, $cp$ through toxic chemical use, $c$ in the year, $t$, and dummies of associated methods. The ATT is captured by $\beta$, which is the differential average effects of higher MW on total onsite waste management activities at manufacturing facilities in treated counties relative to adjacent control counties given either their financial constraint or production technology classification. That is, the separate differential impacts on either less financially constrained or labour-intensive industries. $\psi$ captures the relative differential impact on industries with greater financial constraint or uses capital-intensive production technology. $\beta + \psi$ captures the total differential impact on either less financially constrained or labour-intensive industries. The results on total onsite waste management activities are reported in Table~\ref{tab:mechanisms-onsite-waste-management-activities}.
    % Please add the following required packages to your document preamble:
% \usepackage{booktabs}
% \usepackage{graphicx}
\begin{table}[H]
    \centering
    \caption{Onsite Waste Management Activities}
    \label{tab:mechanisms-onsite-waste-management-activities}
    \resizebox{\columnwidth}{!}{%
        \begin{tabular}{@{}llllllllllll@{}}
            \toprule\toprule
            & \multicolumn{5}{c}{Production Technology} & \multicolumn{5}{c}{Financially Constrained} \\ \cmidrule(lr){2-6} \cmidrule(lr){7-11}
            Waste Management Activities (log) & \multicolumn{3}{c}{Treatment} & \multicolumn{2}{c}{Energy Recovery \& Recycling} & \multicolumn{3}{c}{Treatment} & \multicolumn{2}{c}{Recycling} \\ \cmidrule(lr){3-4} \cmidrule(lr){5-6} \cmidrule(lr){8-9} \cmidrule(lr){10-11}
            & total     & biological & incineration & boiler    & recycling & total     & biological & physical  & recycling & reuse \\ \midrule
            $Treated^{e} \cdot D$ & -0.115    & 0.014*     & 0.018**      & 0.006**   & -0.251    & 0.033     & -0.012     & -0.038*   & 0.009 & -0.002 \\
            & (0.181)   & (0.008)    & (0.008)      & (0.003)   & (0.179)   & (0.190)   & (0.015)    & (0.023)   & (0.146)   & (0.015)   \\
            $Treated^{e}$         & -0.230*   & -0.022***  & -0.011       & -0.002    & -0.160**  & -0.252*   & -0.013**   & -0.007    & -0.222** & -0.018*** \\
            & (0.119)   & (0.008)    & (0.010)      & (0.002)   & (0.067)   & (0.147)   & (0.005)    & (0.013)   & (0.100)   & (0.006)   \\
            controls              & Yes       & Yes        & Yes          & Yes       & Yes       & Yes       & Yes        & Yes       & Yes       & Yes       \\
            year FE               & Yes       & Yes        & Yes          & Yes       & Yes       & Yes       & Yes        & Yes       & Yes       & Yes       \\
            facility FE           & Yes       & Yes        & Yes          & Yes       & Yes       & Yes       & Yes        & Yes       & Yes       & Yes       \\
            border-county FE      & Yes       & Yes        & Yes          & Yes       & Yes       & Yes       & Yes        & Yes       & Yes       & Yes       \\
            toxic chemical FE     & Yes       & Yes        & Yes          & Yes       & Yes       & Yes       & Yes        & Yes       & Yes       & Yes       \\
            toxic chemical LTs    & Yes       & Yes        & Yes          & Yes       & Yes       & Yes       & Yes        & Yes       & Yes       & Yes       \\
            border-county LTs     & Yes       & Yes        & Yes          & Yes       & Yes       & Yes       & Yes        & Yes       & Yes       & Yes       \\ \midrule
            Observations          & 1,893,689 & 1,893,689  & 1,893,689    & 1,893,689 & 1,893,689 & 1,893,689 & 1,893,689 & 1,893,689 & 1,893,689 & 1,893,689 \\
            $R^2$                 & 0.779     & 0.615      & 0.668        & 0.683     & 0.751     & 0.780     & 0.616      & 0.644     & 0.751     & 0.006     \\ \bottomrule\bottomrule
        \end{tabular}%
    }
    \begin{minipage}{\columnwidth}
        \vspace{0.05in}
        \tiny \textbf{NOTES:} These results are obtained from estimating model~\ref{eq:mechanisms-waste-management}. ***, **, and * denote significance levels at the less than $1\%$, $5\%$, and $10\%$, respectively. The result for the production technology column is based on total payroll to revenue ratio.
    \end{minipage}
\end{table}

    The results indicate that rising toxic release intensities from high financially constrained, capital-intensive manufacturing industries, caused by higher MW, is partly transmitted through significant overall decline in their waste management activities. This decline is primarily due to substantial reductions in biological, physical, and incineration treatment methods, industrial boiler energy recovery, and recycling activities, particularly to reuse in production processes. To maximize profits amid higher MW, as shown in Figure~\ref{fig:baseline-manufacturing-industry-profits} of Appendix~\ref{sec:appendix-list-of-toxic-chemicals-trends-and-mechanisms}, these industries prioritize production efficiency and economies of scale through automation, often disregarding environmental impacts. Consequently, output and labor productivity increase in Figure~\ref{fig:baseline-industry-output}, while the aforementioned waste management activities decline in Table~\ref{tab:mechanisms-onsite-waste-management-activities}. These waste management activities and methods require specialized knowledge for safe execution and environmental compliance. Although employment and hours for high-skilled workers have increased, economies of scale are favored to maximize profits amid the cost burden of these high-skilled workers in post-treatment periods (see Figures~\ref{fig:baseline-employment-hours-skilled} and~\ref{fig:baseline-manufacturing-industry-costs-skilled}). Given their financial constraints, capital-intensive manufacturing industries struggle to balance profit maximization with reducing toxic waste release intensities. Thus, they tend to increase non-CAA-HAPs and PBT toxic chemical releases (see~ Figures~\ref{fig:heterogeneous-onsite-releases-intensity-caa-haps} and~\ref{fig:heterogeneous-onsite-releases-intensity-pbts}).

    Furthermore, the results show that declining CAA-HAPs and PBT toxic chemical release intensities from less financially constrained, labour-intensive manufacturing industries are partly driven by significant increases in onsite biological and incineration treatment methods for already generated toxic wastes, as indicated in Table~\ref{tab:mechanisms-onsite-waste-management-activities}. Given their high financial capacity, to maximize profits amid higher MW, these industries balance production efficiency with environmental compliance. Consequently, employment and production hours of high-skilled workers increase, leading to higher output and labour productivity. Additionally, these industries are better equipped to absorb the rising costs of high-skilled labour while maintaining environmental compliance through enhanced onsite waste management activities, such as biological and incineration treatments that require specialized skills.

    \subsection{Onsite Source Reduction Activities}\label{subsec:onsite-source-reduction-activities}
    Onsite source reduction refers to any practice that minimizes or eliminates the generation of toxic wastes or pollutants at the very beginning, before they enter any waste stream or get released into the environment. This is also referred to as Pollution Prevention (P2). These activities aim to minimize or eliminate the creation of waste materials or the use of hazardous substances in manufacturing processes. They include: $(i)$ material substitution and modification which involves replacement of raw materials including changing input purity, replacing fuel type, and organic solvents with environmentally preferable alternatives; $(ii)$ process and equipment modifications which involve improvements to industrial processes and/or associated equipment including implementation of new processes that produce less waste, direct reuse of chemicals, or technological changes impacting synthesis, formulation, fabrication, and assembly, and surface treatment such as cleaning, degreasing, surface preparation, and finishing. All aimed to improve efficiency and reduce waste generation; $(iv)$ inventory management includes improvements in procurement in terms of storage container sizes, and handling of chemicals and materials as they move through a facility to optimize their use and prevent spills and leaks during operation; and $(v)$ operating practices in terms of personnel training and product quality analysis to eliminate or minimize waste generation. To investigate onsite source reduction activities, I estimate the following model:
    \begin{align}
        M_{f,cp,c,t}^{sra} &= \beta (Treated^{e} \cdot D^{fintech})_{f,s,t} + \psi (Treated^{e})_{s,t} + \vartheta (Treated \cdot D^{fintech})_{f,s,t} + \mu (Post \cdot D^{fintech})_{f,s,t} \nonumber \\
        &\quad + \tau Treated_{s,t} + \rho D_{f,s,t}^{fintech} + \alpha Post_{t} + \delta X_{v,c,t-1} + \omega F_{f,t} + \lambda_{t} + \gamma_{f} + \phi_{cp} \nonumber \\
        &\quad + \zeta_{c} + \eta_{c,t} + \theta_{cp,t} + \varepsilon_{f,cp,c,t},\label{eq:mechanisms-source-reduction}
    \end{align}
    where $M_{f,cp,c,t}^{sra}$ is the vector of onsite source reduction activities to reduce the generation of toxic wastes at manufacturing industry facility, $f$ in cross-border county pairs, $cp$ through toxic chemical use, $c$ in industry, in the year, $t$. The ATT is captured by $\beta$, which is the difference in the average effect of higher MW on onsite source reduction activities at manufacturing facilities in treated counties relative to adjacent control counties given either their financial constraint or production technology classification. That is, the separate differential impacts on either less financially constrained or labour-intensive industries. $\psi$ captures the relative differential impact on industries with greater financial constraint or are more capital-intensive. $\beta + \psi$ captures the total differential impact on either less financially constrained or labour-intensive industries.
    The results on total onsite source reduction activities are reported in Tables~\ref{tab:production-technology-onsite-source-reduction-activities-prr} and~\ref{tab:finc-onsite-source-reduction-activities-prr}.

    The second part of the potential transmission mechanisms of the effect of higher MW policy on onsite toxic release intensities involves the source reduction activities of manufacturing industries. The results show that for highly financially constrained, capital-intensive industries, there is a significant decline in their overall onsite source reduction activities due to higher MW. This decline is driven by reductions in green material substitutions and modifications, process and equipment modifications, operating practices and activities, and inventory management. Specifically, the decline in green material substitutions and modifications is attributed to decreasing chemical purity and organic solvent use, which require specialized skilled workers such as chemists, lab technicians, chemical engineers, and environmental scientists. Additionally, there is a shift away from clean fuels to cheaper alternatives, as indicated by declining energy cost intensity, leading to higher energy consumption per $\$100m$ units of output. The decline in green processes and equipment modifications is driven by reductions in recycling, recirculation, total factor productivity, and research and development of new green technologies used in the production process. Declining operating practices, activities, and inventory management are influenced by a reduction in product quality analysis, operating training, and improved material handling. As higher MW leads to an increasing cost burden and employment/hours of high-skilled workers, output, labor productivity, and profits rise. However, due to the higher cost burden, these financially constrained, capital-intensive industries prioritize economies of scale through automation for profit maximization over environmental considerations. Consequently, they focus on core production activities while neglecting investments in the discussed source reduction activities, resulting in higher generation intensities of non-CAA-HAPs and PBT toxic chemicals wastes.

    Conversely, further results show that for less financially constrained, labour intensive manufacturing industries, there is a corresponding significant increase in the following green onsite reduction activities: material substitution and modifications, process and equipment modifications, operating training and activities, and inventory management. Particularly, green material substitution and modifications is attributed to increasing chemical purity and organic solvent use, which require specialized skilled workers such as chemists, lab technicians, chemical engineers, and environmental scientists. Additionally, there is a shift towards more expensive clean fuels, as indicated by increasing energy cost intensity, leading to lower energy consumption per $\$100m$ units of output. The increase in green processes and equipment modifications is driven by increases in recycling, recirculation, total factor productivity, and research and development of new green technologies used in the production process. Increasing operating practices, activities, and inventory management are influenced by an increase in product quality analysis, size of chemical storage containers and improved material handling.~\footnote{\tiny The results further show that the increasing air emission intensities (both point and fugitive sources) for less financially constrained manufacturing industries in the $2014$ and $2017$ cohorts are transmitted by the decline in physical treatment of already generated toxic wastes, and the decline in recycling and operating trainings to prevent the generation of toxic wastes.} As higher MW leads to an increasing cost burden and employment/hours of high-skilled workers, output, labor productivity, and profits rise. However, due to their high financial capacity, these manufacturing industries prioritize balancing production efficiency and environmental considerations. Hence, they are able to better absorb the rising cost burden of high-skilled workers while investing in pollution reduction activities to reduce the generation of toxic wastes from CAA-HAPs and PBT regulated chemicals in production processes.



    % Please add the following required packages to your document preamble:
% \usepackage{graphicx}
\begin{table}[H]
    \centering
    \caption{Production Technology: Onsite Source Reduction Activities}
    \label{tab:production-technology-onsite-source-reduction-activities-prr}
    \resizebox{\columnwidth}{!}{%
        \begin{tabular}{@{}lllllllllll@{}}
            \toprule\toprule
            Source Reduction Activities & \multicolumn{4}{c}{Material Substitution} & \multicolumn{2}{c}{Process Modification}   & \multicolumn{2}{c}{Operations Activities}   & \multicolumn{2}{c}{Inventory Management} \\
            \cmidrule(lr){3-5} \cmidrule(lr){6-7} \cmidrule(lr){8-9} \cmidrule(lr){10-11}
            & all       & chemical purity & clean fuel & energy cost intensity & new green tech & recycling & prod quality analysis & operating trainings & container size change & imp material handle \\ \midrule
            $Treated^{e} \cdot D$ & 0.021     & 0.003***        & 0.034***   & 0.034***              & 0.005***       & 0.044***  & 0.004***              & 0.001               & 0.003**               & 0.001               \\
            & (0.023)   & (0.001)         & (0.006)    & (0.012)               & (0.002)        & (0.012)   & (0.001)               & (0.001)             & (0.001)               & (0.001)             \\
            $Treated^{e}$         & -0.072*** & -0.001*         & -0.029***  & -0.002                & -0.005***      & -0.035*** & -0.001**              & -0.002**            & 0.001                 & -0.003***           \\
            & (0.020)   & (0.000)         & (0.005)    & (0.009)               & (0.002)        & (0.007)   & (0.000)               & (0.001)             & (0.001)               & (0.001)             \\
            controls              & Yes       & Yes             & Yes        & Yes                   & Yes            & Yes       & Yes                   & Yes                 & Yes                   & Yes                 \\
            year FE               & Yes       & Yes             & Yes        & Yes                   & Yes            & Yes       & Yes                   & Yes                 & Yes                   & Yes                 \\
            facility FE           & Yes       & Yes             & Yes        & Yes                   & Yes            & Yes       & Yes                   & Yes                 & Yes                   & Yes                 \\
            border-county FE      & Yes       & Yes             & Yes        & Yes                   & Yes            & Yes       & Yes                   & Yes                 & Yes                   & Yes                 \\
            toxic chemical FE     & Yes       & Yes             & Yes        & Yes                   & Yes            & Yes       & Yes                   & Yes                 & Yes                   & Yes                 \\
            toxic chemical LTs    & Yes       & Yes             & Yes        & Yes                   & Yes            & Yes       & Yes                   & Yes                 & Yes                   & Yes                 \\
            border-county LTs     & Yes       & Yes             & Yes        & Yes                   & Yes            & Yes       & Yes                   & Yes                 & Yes                   & Yes                 \\\midrule
            Observations          & 1,893,689 & 1,893,689       & 1,893,689  & 1,893,689             & 1,893,689      & 1,893,689 & 1,893,689             & 1,893,689           & 1,893,689             & 1,893,689           \\
            $R^2$                 & 0.699     & 0.147           & 0.781      & 0.987                 & 0.354          & 0.503     & 0.154                 & 0.167               & 0.260                 & 0.433               \\ \bottomrule\bottomrule
        \end{tabular}%
    }
    \begin{minipage}{\columnwidth}
        \vspace{0.05in}
        \tiny NOTES: These results are obtained from estimating model~\ref{eq:mechanisms-source-reduction}. ***, **, and * denote significance levels at the less than $1\%$, $5\%$ and $10\%$, respectively. SRA means source reduction activities. This results are based on using total payroll to revenue ratio as a measure of labour v. capital intensive manufacturing industries.
    \end{minipage}
\end{table}
    % Please add the following required packages to your document preamble:
% \usepackage{booktabs}
% \usepackage{graphicx}
\begin{table}[H]
    \centering
    \caption{Financial Constraints: Onsite Source Reduction Activities}
    \label{tab:finc-onsite-source-reduction-activities-prr}
    \resizebox{\columnwidth}{!}{%
        \begin{tabular}{@{}lllllllllllll@{}}
            \toprule\toprule
            Source Reduction Activities & \multicolumn{5}{c}{Material Substitution} & \multicolumn{5}{c}{Process Modification}   & \multicolumn{1}{c}{Operations Activities}   & \multicolumn{1}{c}{Inventory Management} \\
            \cmidrule(lr){3-6} \cmidrule(lr){7-11} \cmidrule(lr){12-12} \cmidrule(lr){13-13}
            & all       & material  & organic solvent & clean fuel & energy cost intensity & new tech  & recycling & recirculation & r and d   & tfp       & operating trainings & imp material handle \\ \midrule
            $Treated^{e} \cdot D$ & -0.030    & 0.008***  & 0.001*          & -0.008     & 0.070***              & 0.014***  & -0.017**  & 0.006**       & -0.006**  & 0.090***  & -0.007***           & 0.003*              \\
            & (0.029)   & (0.002)   & (0.000)         & (0.008)    & (0.018)               & (0.002)   & (0.008)   & (0.003)       & (0.003)   & (0.008)   & (0.002)             & (0.002)             \\
            $Treated^{e}$         & -0.059**  & 0.004*    & -0.002***       & -0.012***  & -0.030***             & -0.009*** & -0.018*** & -0.003        & 0.002*    & -0.004    & -0.002**            & -0.009***           \\
            & (0.023)   & (0.002)   & (0.001)         & (0.003)    & (0.008)               & (0.002)   & (0.007)   & (0.002)       & (0.001)   & (0.003)   & (0.001)             & (0.003)             \\
            controls              & Yes       & Yes       & Yes             & Yes        & Yes                   & Yes       & Yes       & Yes           & Yes       & Yes       & Yes                 & Yes                 \\
            year FE               & Yes       & Yes       & Yes             & Yes        & Yes                   & Yes       & Yes       & Yes           & Yes       & Yes       & Yes                 & Yes                 \\
            facility FE           & Yes       & Yes       & Yes             & Yes        & Yes                   & Yes       & Yes       & Yes           & Yes       & Yes       & Yes                 & Yes                 \\
            border-county FE      & Yes       & Yes       & Yes             & Yes        & Yes                   & Yes       & Yes       & Yes           & Yes       & Yes       & Yes                 & Yes                 \\
            toxic chemical FE     & Yes       & Yes       & Yes             & Yes        & Yes                   & Yes       & Yes       & Yes           & Yes       & Yes       & Yes                 & Yes                 \\
            toxic chemical LTs    & Yes       & Yes       & Yes             & Yes        & Yes                   & Yes       & Yes       & Yes           & Yes       & Yes       & Yes                 & Yes                 \\
            border-county LTs     & Yes       & Yes       & Yes             & Yes        & Yes                   & Yes       & Yes       & Yes           & Yes       & Yes       & Yes                 & Yes                 \\ \midrule
            Observations          & 1,893,689 & 1,893,689 & 1,893,689       & 1,893,689  & 1,893,689             & 1,893,689 & 1,893,689 & 1,893,689     & 1,893,689 & 1,893,689 & 1,893,689           & 1,893,689           \\
            $R^2$                 & 0.700     & 0.233     & 0.091           & 0.779      & 0.988                 & 0.354     & 0.503     & 0.231         & 0.954     & 0.738     & 0.167               & 0.425               \\ \bottomrule\bottomrule
        \end{tabular}%
    }
    \begin{minipage}{\columnwidth}
        \vspace{0.05in}
        \tiny NOTES: These results are obtained from estimating model~\ref{eq:mechanisms-source-reduction}. ***, **, and * denote significance levels at the less than $1\%$, $5\%$ and $10\%$, respectively. SRA means source reduction activities.
    \end{minipage}
\end{table}


    \section{Conclusions}\label{sec:conclusions}
    This paper use precise administrative toxic release inventory and manufacturing industry payroll data to document the unintended environmental consequences of a higher MW policy. In the preliminaries, the study elicits causal evidence of the higher cost burdens (both low- and high-skilled/wage workers) due to a higher MW floor, followed by rising outputs, labour productivity, and profits. Albeit I find overall null effects across employment and production workers hours, the cohort-specific effects are more nuanced. While the disemployment is more pronounced among low-skilled or low-wage workers, the positive employment effect is dominated amongst the high-skilled or high-wage workers. I find that the low-skilled/wage workers' disemployment effect is primarily driven by the reluctance of low-skilled/wage workers to commute to distant higher MW counties/states; and the increasing demand for high-skilled/wage workers' is independent of cross-county/state worker mobility. The preliminary findings have notable implications on the environmental consequences of the higher MW policy

    The main results on the environmental consequences of higher MW are more nuanced as manufacturing industries adjust towards capital or labour intensive technologies given their financial standings. The increasing toxic release intensities is primarily driven by higher air emissions (both point and fugitive sources), surface water discharge, land releases (particularly surface impoundment) intensities, especially in non-CAA-HAPs and PBT regulated toxic chemicals. This high pollution effect is dominated in high financially constrained, capital-intensive industries and highest emitting manufacturing industries, including chemical, food, wood, and leather and allied product manufacturing industries in less competitive environments. There is also evidence of higher MW being potentially carcinogenic in point air emission sources. Conversely, the decreasing effect of toxic release intensities is dominated in less financially constrained, labour-intensive manufacturing industries in more competitive environments, especially in CAA-HAPs and PBT regulated chemicals.

    The potential mechanisms of these effects are conditional on the type of production technology and financial capacity of manufacturing industries. The mechanism analyses reveal that declining waste management activities of already generated toxic wastes are partly responsible for the increased toxic release intensities. Specifically, reductions in biological, physical, incineration treatment methods, industrial boiler energy recovery method, and recycling activities. Additionally, decreasing onsite source reduction activities that prevent the generation of toxic wastes from production processes explain most of the increases in toxic release intensities. These declining source reduction activities include green material substitution and modifications, green process and equipment modifications, operating practices and activities, and inventory management. Overall, this decline in waste management and source reduction activities are responsible for the increased toxic release intensities due to higher MW in high-financially-constrained-capital-intensive manufacturing industries. Hence, the US policymakers should focus on providing incentives for the abatement and management of toxic chemical releases for this group of manufacturing industries. Conversely, the increase in these waste management and source reduction activities are responsible for the decreasing toxic release intensities in less-financially-constrained-labour-intensive manufacturing industries.

    \newpage
%======================================================================================================================%
    \begin{appendices}
        \renewcommand\thesection{\Roman{section}} % Use Roman numerals for section numbers in appendices
        \renewcommand\thesubsection{\Alph{subsection}} % Use Alphabets for sub-section numbers in appendices


        \section{State Minimum Wage and Balance Tests}\label{sec:appendix-state-minimum-wage-and-balance-tests}
        % Please add the following required packages to your document preamble:
% \usepackage{booktabs}
% \usepackage{graphicx}
\begin{table}[H]
    \centering
    \caption{Minimum Wage Changes in US States from $2011-2017$}
    \label{tab:states-mw-changes}
    \scalebox{0.7}{
        \resizebox{\columnwidth}{!}{%
            \begin{tabular}{@{}llllllllll@{}}
                \toprule \toprule
                states         & 2011 & 2012 & 2013  & 2014 & 2015 & 2016 & 2017 & start MW & end MW \\ \midrule
                Alaska         & 0    & 0    & 0     & 0    & 1    & 1    & 0.05 & 7.75     & 9.8    \\
                Arkansas       & 0    & 0    & 0     & 0    & 1.25 & 0.5  & 0.5  & 6.25     & 8.5    \\
                Arizona        & 0.1  & 0.3  & 0.15  & 0.1  & 0.15 & 0    & 1.95 & 7.35     & 10     \\
                California     & 0    & 0    & 0     & 1    & 0    & 1    & 0.5  & 8        & 10.5   \\
                Colorado       & 0.12 & 0.28 & 0.14  & 0.22 & 0.23 & 0.08 & 0.99 & 7.36     & 9.3    \\
                Connecticut    & 0    & 0    & 0     & 0.45 & 0.45 & 0.45 & 0.5  & 8.25     & 10.1   \\
                Delaware       & 0    & 0    & 0     & 0.5  & 0.5  & 0    & 0    & 7.25     & 8.25   \\
                Florida        & 0    & 0.42 & 0.12  & 0.14 & 0.12 & 0    & 0.05 & 7.21     & 8.1    \\
                Georgia        & 0    & 0    & 0     & 0    & 0    & 0    & 0    & 5.15     & 5.15   \\
                Hawaii         & 0    & 0    & 0     & 0    & 0.5  & 0.75 & 0.75 & 7.25     & 9.25   \\
                Iowa           & 0    & 0    & 0     & 0    & 0    & 0    & 0    & 7.25     & 7.25   \\
                Idaho          & 0    & 0    & 0     & 0    & 0    & 0    & 0    & 7.25     & 7.25   \\
                Illinois       & 0    & 0    & 0     & 0    & 0    & 0    & 0    & 8.25     & 8.25   \\
                Indiana        & 0    & 0    & 0     & 0    & 0    & 0    & 0    & 7.25     & 7.25   \\
                Kansas         & 0    & 0    & 0     & 0    & 0    & 0    & 0    & 7.25     & 7.25   \\
                Kentucky       & 0    & 0    & 0     & 0    & 0    & 0    & 0    & 7.25     & 7.25   \\
                Massachusetts  & 0    & 0    & 0     & 0    & 1    & 1    & 1    & 8        & 11     \\
                Maryland       & 0    & 0    & 0     & 0    & 1    & 0.5  & 0.5  & 7.25     & 9.25   \\
                Maine          & 0    & 0    & 0     & 0    & 0    & 0    & 1.5  & 7.5      & 9      \\
                Michigan       & 0    & 0    & 0     & 0.75 & 0    & 0.35 & 0.4  & 7.4      & 8.9    \\
                Minnesota      & 0    & 0    & -0.01 & 1.85 & 1    & 0.5  & 0    & 6.16     & 9.5    \\
                Missouri       & 0    & 0    & 0.1   & 0.15 & 0.15 & 0    & 0.05 & 7.25     & 7.7    \\
                Montana        & 0.1  & 0.3  & 0.15  & 0.1  & 0.15 & 0    & 0.1  & 7.35     & 8.15   \\
                North Carolina & 0    & 0    & 0     & 0    & 0    & 0    & 0    & 7.25     & 7.25   \\
                North Dakota   & 0    & 0    & 0     & 0    & 0    & 0    & 0    & 7.25     & 7.25   \\
                Nebraska       & 0    & 0    & 0     & 0    & 0.75 & 1    & 0    & 7.25     & 9      \\
                New Hampshire  & 0    & 0    & 0     & 0    & 0    & 0    & 0    & 7.25     & 7.25   \\
                New Jersey     & 0    & 0    & 0     & 1    & 0.13 & 0    & 0.06 & 7.25     & 8.44   \\
                New Mexico     & 0    & 0    & 0     & 0    & 0    & 0    & 0    & 7.5      & 7.5    \\
                Nevada         & 0.7  & 0    & 0     & 0    & 0    & 0    & 0    & 8.25     & 8.25   \\
                New York       & 0    & 0    & 0     & 0.75 & 0.75 & 0.25 & 0.7  & 7.25     & 9.7    \\
                Ohio           & 0.1  & 0.3  & 0.15  & 0.1  & 0.15 & 0    & 0.05 & 7.4      & 8.1    \\
                Oklahoma       & 0    & 0    & 0     & 0    & 0    & 0    & 0    & 7.25     & 7.25   \\
                Oregon         & 0.1  & 0.3  & 0.15  & 0.15 & 0.15 & 0.5  & 0.5  & 8.5      & 10.25  \\
                Pennsylvania   & 0    & 0    & 0     & 0    & 0    & 0    & 0    & 7.25     & 7.25   \\
                Rhode Island   & 0    & 0    & 0.35  & 0.25 & 1    & 0.6  & 0    & 7.4      & 9.6    \\
                South Dakota   & 0    & 0    & 0     & 0    & 1.25 & 0.05 & 0.1  & 7.25     & 8.65   \\
                Texas          & 0    & 0    & 0     & 0    & 0    & 0    & 0    & 7.25     & 7.25   \\
                Utah           & 0    & 0    & 0     & 0    & 0    & 0    & 0    & 7.25     & 7.25   \\
                Virgina        & 0    & 0    & 0     & 0    & 0    & 0    & 0    & 7.25     & 7.25   \\
                Vermont        & 0.09 & 0.31 & 0.14  & 0.13 & 0.42 & 0.45 & 0.4  & 8.15     & 10     \\
                Washington     & 0.12 & 0.37 & 0.15  & 0.13 & 0.15 & 0    & 1.53 & 8.67     & 11     \\
                Wisconsin      & 0    & 0    & 0     & 0    & 0    & 0    & 0    & 7.25     & 7.25   \\
                West Virginia  & 0    & 0    & 0     & 0    & 0.75 & 0.75 & 0    & 7.25     & 8.75   \\
                Wyoming        & 0    & 0    & 0     & 0    & 0    & 0    & 0    & 5.15     & 5.15   \\ \bottomrule\bottomrule
            \end{tabular}%
        }
    }

\end{table}
        \begin{table}[H]
    \centering
    \caption{Descriptive Statistics: Treated v. Control Border Counties}
    \label{tab:descriptive-statistics-control-border-counties}
    \begin{tabular}{lrrrr}
        \toprule \toprule
        Variable                                     & Mean  & SD     & T     & C     \\ \midrule
        GDP per capita (1000's)                      & 44.92 & 8.56   & 45.07 & 44.89 \\
        industry employment (1000's)                 & 43.18 & 39.26  & 46.88 & 42.48 \\
        annual average establishments                & 4.88  & 12.57  & 3.19  & 5.20  \\
        chemical ancillary use (onsite)              & 0.21  & 0.41   & 0.28  & 0.20  \\
        chemical formulation component (onsite)      & 0.32  & 0.47   & 0.31  & 0.33  \\
        chemical manufacturing aid (onsite)          & 0.10  & 0.30   & 0.15  & 0.09  \\
        max number of chemicals at facility (onsite) & 3.86  & 1.52   & 3.86  & 3.87  \\
        entire facility (onsite)                     & 1.00  & 0.02   & 1.00  & 1.00  \\
        private facility (onsite)                    & 1.00  & 0.01   & 1.00  & 1.00  \\
        imported chemicals at facility (onsite)      & 0.06  & 0.23   & 0.11  & 0.04  \\
        produced chemicals at facility (onsite)      & 0.19  & 0.39   & 0.33  & 0.16  \\
        production ratio or activity index (onsite)  & 3.07  & 485.56 & 13.70 & 1.06  \\ \bottomrule\bottomrule
    \end{tabular}
    \begin{minipage}{13.5cm}
        \vspace{0.05in}
        \tiny NOTES: The table contains county-level descriptive statistics as of the year immediately before the first initial MW change. The sample is restricted to border counties in treated and control states (See Table~\ref{tab:states-mw-adjustments-t-and-c}).
    \end{minipage}
\end{table}

        \begin{table}[H]
    \centering
    \caption{Descriptive Statistics: Treated v. Control Border States}
    \label{tab:descriptive-statistics-control-border-states}
    \begin{tabular}{lrrrr}
        \toprule \toprule
        Variable                                     & Mean  & SD      & T     & C     \\ \midrule
        GDP per capita (1000's)                      & 45.67 & 8.92    & 43.95 & 46.14 \\
        industry employment (1000's)                 & 46.25 & 47.71   & 48.87 & 45.52 \\
        annual average establishments                & 9.14  & 23.47   & 3.92  & 10.59 \\
        chemical ancillary use (onsite)              & 0.19  & 0.39    & 0.23  & 0.18  \\
        chemical formulation component (onsite)      & 0.24  & 0.43    & 0.24  & 0.24  \\
        chemical manufacturing aid (onsite)          & 0.10  & 0.30    & 0.12  & 0.09  \\
        max number of chemicals at facility (onsite) & 3.78  & 1.61    & 3.69  & 3.80  \\
        entire facility (onsite)                     & 1.00  & 0.04    & 1.00  & 1.00  \\
        private facility (onsite)                    & 1.00  & 0.01    & 1.00  & 1.00  \\
        imported chemicals at facility (onsite)      & 0.06  & 0.25    & 0.08  & 0.06  \\
        produced chemicals at facility (onsite)      & 0.27  & 0.44    & 0.33  & 0.25  \\
        production ratio or activity index (onsite)  & 12.07 & 1130.57 & 51.29 & 1.19  \\ \bottomrule\bottomrule
    \end{tabular}
    \begin{minipage}{13.5cm}
        \vspace{0.05in}
        \tiny NOTES: The table contains state-level descriptive statistics as of the year immediately before the first initial MW change. The sample is restricted to border counties in treated and control states (See Table~\ref{tab:states-mw-adjustments-t-and-c}).
    \end{minipage}
\end{table}

        \begin{figure}[H]
    \centering
    \includegraphics[width=1\textwidth, height=0.45\textheight]{fig_pre_evolution}
    \caption{County-level Macroeconomic Trends in Border Counties}
    \label{fig:county-level-macroeconomic-trends-in-border-counties}
    \begin{minipage}{18cm}
        \vspace{0.05in}
        {NOTES: This figure is obtained from estimating this equation $y_{c,t} = \sum_{t = 2011}^{2013} \beta_{c,t} (Treated \cdot B)_{s,t} + \lambda_{t} + \Phi_{c,p} + \zeta_{cp,t} + \epsilon_{c,t}$. Where $y_{c,t}$ is the vector of observables. Treated is the grouping variable that is unity for the treated states and zero for the control states. $B_{t}$ is a dummy variable with three levels of time, $2011$, $2012$, and $2013$. $\beta_{c,t}$ is the parameter vector of coefficients. $\lambda_{t}$ is the year fixed effects; $\Phi_{cp}$ is the border-county pair fixed effects; and $\zeta_{cp,t}$ is the border-county-pair-year fixed effects. $\epsilon_{c,t}$ is the error term. Robust standard errors are clustered at the state level. Row one shows the plots for the county level regressions and row two shows the plots for the state level regressions. \par}
    \end{minipage}
\end{figure}
        \begin{figure}[H]
    \centering
    \includegraphics[width=1\textwidth, height=0.45\textheight]{C:/Users/david/OneDrive/Documents/ULMS/PhD/Thesis/chapter3/src/climate_change/latex/fig_pre_evolution_state}
    \caption{State-level Macroeconomic Trends in Border States}
    \label{fig:state-level-macroeconomic-trends-in-border-states}
    \begin{minipage}{14cm}
        \vspace{0.05in}
        \tiny NOTES: This figure is obtained from estimating this equation $y_{s,t} = \sum_{t = 2011}^{2013} \beta (Treated \cdot B)_{s,t} + \lambda_{t} + \Phi_{sp} + \zeta_{sp,t} + \epsilon_{s,t}$. Where $y_{s,t}$ is the vector of observables. Treated is the grouping variable that is unity for the treated states and zero for the control states. $B_{t}$ is a dummy variable with three levels of time, $2011$, $2012$, and $2013$. $\beta$ is the state-level parameter vector of coefficients. $\lambda_{t}$ is the year fixed effects; $\Phi_{sp}$ is the border-state pair fixed effects; and $\zeta_{sp,t}$ is the border-state-pair-year fixed effects. $\epsilon_{s,t}$ is the error term. Robust standard errors are clustered at the state level.
    \end{minipage}
\end{figure}


        \section{List of Toxic Chemicals, Trends and Mechanisms}\label{sec:appendix-list-of-toxic-chemicals-trends-and-mechanisms}
        \begin{table}[H]
    \centering
    \caption{Analyzed Chemicals}
    \label{tab:analyzed-chemicals}
%    \scalebox{0.35}{
    \resizebox{\textwidth}{!}{
        \begin{tabular}{llllllllllll}
            \toprule\toprule
            chemical name                                                              & classification & attribute             & onsite & offsite & potw & chemical name                                                                                                      & classification & attribute & onsite & offsite & potw\\
            \midrule
            1-Chloro-1,1-difluoroethane (HCFC-142b)                                    & TRI            & formulation component & yes    & NA      & NA   & Dimethylamine dicamba & TRI & others & yes & NA & NA\\
            1,2-Dibromoethane                                                          & TRI            & carcinogenic          & yes    & yes     & NA   & Dioxin and dioxin-like compounds                                                                                   & DIOXIN & carcinogenic & yes & yes & NA\\
            1,2-Dichloroethane                                                         & TRI            & carcinogenic          & yes    & yes     & NA   & Epichlorohydrin                                                                                                    & TRI            & carcinogenic          & yes & yes & NA\\
            1,2,4-Trimethylbenzene                                                     & TRI            & formulation component & yes    & yes     & yes  & Ethyl acrylate                                                                                                     & TRI            & carcinogenic & yes & yes & yes\\
            1,3-Butadiene                                                              & TRI            & carcinogenic          & yes    & NA      & NA   & Ethylbenzene                                                                                                       & TRI            & carcinogenic          & yes    & yes     & yes  \\
            1,3-Phenylenediamine                                                       & TRI            & formulation component & yes    & NA      & NA   & Ethylene                                                                                                           & TRI            & formulation component & yes & NA & NA\\
            1,4-Dioxane                                                                & TRI            & carcinogenic          & yes    & yes     & NA   & Ethylene glycol                                                                                                    & TRI            & clean air act         & yes    & yes & yes\\
            2-Ethoxyethanol                                                            & TRI            & formulation component & yes    & yes     & NA   & Ethylene oxide                                                                                                     & TRI            & carcinogenic & yes & NA & NA\\
            2-Phenylphenol                                                             & TRI            & formulation component & yes    & NA      & NA   & Fomesafen                                                                                                          & TRI            & formulation component & yes & yes & yes\\
            2,2-Bis(bromomethyl)-1,3-propanediol                                       & TRI            & carcinogenic          & yes    & NA      & NA   & Formaldehyde                                                                                                       & TRI & carcinogenic & yes & yes & yes\\
            2,2-Dichloro-1,1,1-trifluoroethane (HCFC-123)                              & TRI            & manufacturing aid     & yes    & yes     & NA   & Formic acid & TRI & formulation component & yes & yes & NA\\
            2,4-D                                                                      & TRI            & carcinogenic          & yes    & yes     & NA   & Hexachlorobenzene                                                                                                  & PBT            & carcinogenic          & yes    & yes     & yes  \\
            2,4-D 2-ethylhexyl ester                                                   & TRI            & carcinogenic          & yes    & yes     & NA   & Hydrazine                                                                                                          & TRI            & carcinogenic          & yes & NA & NA\\
            2,4-Dimethylphenol                                                         & TRI            & ancillary use         & yes    & NA      & NA   & Hydrochloric acid (acid aerosols including mists, vapors, gas, fog, and other airborne forms of any particle size) & TRI & clean air act & yes & NA & NA\\
            2,4-Dinitrotoluene                                                         & TRI            & carcinogenic          & yes    & NA      & NA   & Hydrogen cyanide                                                                                                   & TRI            & article component & yes & yes & NA\\
            2,6-Dinitrotoluene                                                         & TRI            & carcinogenic          & yes    & NA      & NA   & Hydrogen fluoride                                                                                                  & TRI            & clean air act & yes & yes & yes\\
            3-Iodo-2-propynyl butylcarbamate                                           & TRI            & formulation component & yes    & yes     & NA   & Hydroquinone & TRI & clean air act & yes & yes & yes\\
            4,4'-Isopropylidenediphenol                                                & TRI            & formulation component & yes    & yes     & NA   & Lead                                                                                                               & PBT            & carcinogenic & yes & yes & yes\\
            Acetaldehyde                                                               & TRI            & carcinogenic          & yes    & yes     & NA   & Lead compounds                                                                                                     & PBT            & clean air act         & yes    & yes & yes\\
            Acetonitrile                                                               & TRI            & clean air act         & yes    & yes     & NA   & Lithium carbonate                                                                                                  & TRI            & metal restricted & yes & yes & NA\\
            Acetophenone                                                               & TRI            & clean air act         & yes    & yes     & NA   & Maleic anhydride                                                                                                   & TRI            & clean air act         & yes & yes & yes\\
            Acrolein                                                                   & TRI            & clean air act         & yes    & NA      & NA   & Manganese                                                                                                          & TRI            & clean air act         & yes    & yes     & yes  \\
            Acrylamide                                                                 & TRI            & carcinogenic          & yes    & yes     & NA   & Manganese compounds                                                                                                & TRI            & clean air act         & yes & yes & yes\\
            Acrylic acid                                                               & TRI            & clean air act         & yes    & yes     & NA   & Mercury                                                                                                            & PBT            & clean air act         & yes    & yes     & yes  \\
            Acrylonitrile                                                              & TRI            & carcinogenic          & yes    & yes     & yes  & Mercury compounds                                                                                                  & PBT            & clean air act         & yes & yes & yes\\
            Allyl alcohol                                                              & TRI            & article component     & yes    & yes     & NA   & Methanol                                                                                                           & TRI            & clean air act         & yes    & yes & yes\\
            Aluminum (fume or dust)                                                    & TRI            & metal restricted      & yes    & yes     & NA   & Methoxone                                                                                                          & TRI            & carcinogenic & yes & NA & NA\\
            Aluminum oxide (fibrous forms)                                             & TRI            & metal restricted      & yes    & yes     & NA   & Methyl acrylate                                                                                                    & TRI & formulation component & yes & yes & NA\\
            Ammonia                                                                    & TRI            & formulation component & yes    & yes     & yes  & Methyl isobutyl ketone                                                                                             & TRI            & carcinogenic & yes & yes & yes\\
            Anthracene                                                                 & TRI            & formulation component & yes    & yes     & yes  & Methyl methacrylate                                                                                                & TRI            & clean air act & yes & yes & yes\\
            Antimony                                                                   & TRI            & clean air act         & yes    & yes     & yes  & Methyl tert-butyl ether                                                                                            & TRI            & clean air act & yes & yes & yes\\
            Antimony compounds                                                         & TRI            & clean air act         & yes    & yes     & yes  & Molybdenum trioxide                                                                                                & TRI            & metal restricted & yes & yes & NA\\
            Arsenic                                                                    & TRI            & carcinogenic          & yes    & NA      & NA   & n-Butyl alcohol                                                                                                    & TRI            & formulation component & yes & yes & yes\\
            Arsenic compounds                                                          & TRI            & clean air act         & yes    & yes     & NA   & n-Hexane                                                                                                           & TRI            & clean air act         & yes    & yes & yes\\
            Asbestos (friable)                                                         & TRI            & carcinogenic          & yes    & NA      & NA   & N-Methyl-2-pyrrolidone                                                                                             & TRI            & formulation component & yes & yes & yes\\
            Atrazine                                                                   & TRI            & formulation component & yes    & yes     & yes  & N-Methylolacrylamide                                                                                               & TRI            & others                & yes & yes & NA\\
            Barium                                                                     & TRI            & metal restricted      & yes    & NA      & NA   & N,N-Dimethylaniline                                                                                                & TRI            & clean air act         & yes    & NA & NA\\
            Barium compounds (except for barium sulfate (CAS No. 7727-43-7))           & TRI            & metal restricted & yes & yes & yes & N,N-Dimethylformamide & TRI & clean air act & yes & yes & yes\\
            Benzal chloride                                                            & TRI            & others                & yes    & yes     & NA   & Naphthalene                                                                                                        & TRI            & carcinogenic          & yes    & yes     & yes  \\
            Benzene                                                                    & TRI            & carcinogenic          & yes    & yes     & yes  & Nickel                                                                                                             & TRI            & carcinogenic          & yes    & yes     & yes  \\
            Benzo[g,h,i]perylene                                                       & PBT            & clean air act         & yes    & yes     & yes  & Nickel compounds                                                                                                   & TRI            & carcinogenic & yes & yes & yes\\
            Benzoyl peroxide                                                           & TRI            & formulation component & yes    & yes     & NA   & Nitrate compounds (water dissociable; reportable only when in aqueous solution) & TRI & formulation component & yes & yes & yes\\
            Benzyl chloride                                                            & TRI            & clean air act         & yes    & NA      & NA   & Nitric acid                                                                                                        & TRI            & formulation component & yes & yes & yes\\
            Biphenyl                                                                   & TRI            & clean air act         & yes    & NA      & NA   & Nitrobenzene                                                                                                       & TRI            & carcinogenic          & yes    & NA      & NA   \\
            Boron trichloride                                                          & TRI            & metal restricted      & yes    & NA      & NA   & o-Xylene                                                                                                           & TRI            & clean air act         & yes    & yes & NA\\
            Bromomethane                                                               & TRI            & clean air act         & yes    & NA      & NA   & Ozone                                                                                                              & TRI            & ancillary use         & yes    & NA      & NA   \\
            Butyl acrylate                                                             & TRI            & formulation component & yes    & yes     & yes  & p-Xylene                                                                                                           & TRI            & clean air act         & yes & NA & NA\\
            Cadmium                                                                    & TRI            & carcinogenic          & yes    & yes     & NA   & Pentachlorophenol                                                                                                  & TRI            & carcinogenic          & yes    & yes     & NA   \\
            Cadmium compounds                                                          & TRI            & carcinogenic          & yes    & yes     & yes  & Peracetic acid                                                                                                     & TRI            & formulation component & yes & yes & NA\\
            Carbon disulfide                                                           & TRI            & clean air act         & yes    & NA      & NA   & Phenanthrene                                                                                                       & TRI            & clean air act         & yes    & yes & yes\\
            Carbonyl sulfide                                                           & TRI            & clean air act         & yes    & NA      & NA   & Phenol                                                                                                             & TRI            & clean air act         & yes    & yes     & yes  \\
            Catechol                                                                   & TRI            & carcinogenic          & yes    & NA      & NA   & Phosphorus (yellow or white)                                                                                       & TRI            & clean air act & yes & NA & NA\\
            Certain glycol ethers                                                      & TRI            & clean air act         & yes    & yes     & yes  & Phthalic anhydride                                                                                                 & TRI            & clean air act & yes & yes & yes\\
            Chlorine                                                                   & TRI            & clean air act         & yes    & NA      & NA   & Polychlorinated biphenyls                                                                                          & PBT            & carcinogenic          & yes & NA & NA\\
            Chlorine dioxide                                                           & TRI            & article component     & yes    & NA      & NA   & Polycyclic aromatic compounds                                                                                      & PBT & carcinogenic & yes & yes & yes\\
            Chlorobenzene                                                              & TRI            & clean air act         & yes    & yes     & NA   & Propiconazole                                                                                                      & TRI            & formulation component & yes & yes & yes\\
            Chloroethane                                                               & TRI            & clean air act         & yes    & yes     & NA   & Propylene                                                                                                          & TRI            & formulation component & yes & yes & NA\\
            Chloroform                                                                 & TRI            & carcinogenic          & yes    & yes     & yes  & Propylene oxide                                                                                                    & TRI            & carcinogenic          & yes    & NA      & NA   \\
            Chloromethane                                                              & TRI            & clean air act         & yes    & NA      & NA   & Pyridine                                                                                                           & TRI            & article component     & yes    & yes & NA\\
            Chromium                                                                   & TRI            & clean air act         & yes    & yes     & yes  & sec-Butyl alcohol                                                                                                  & TRI            & formulation component & yes & yes & yes\\
            Chromium compounds (except for chromite ore mined in the Transvaal Region) & TRI            & clean air act & yes & yes & yes & Selenium compounds & TRI & clean air act & yes & yes & NA\\
            Cobalt                                                                     & TRI            & carcinogenic          & yes    & yes     & yes  & Silver                                                                                                             & TRI            & metal restricted      & yes    & yes     & yes  \\
            Cobalt compounds                                                           & TRI            & clean air act         & yes    & yes     & yes  & Silver compounds                                                                                                   & TRI            & metal restricted & yes & yes & yes\\
            Copper                                                                     & TRI            & metal restricted      & yes    & yes     & yes  & Sodium dimethyldithiocarbamate                                                                                     & TRI            & formulation component & yes & NA & NA\\
            Copper compounds                                                           & TRI            & metal restricted      & yes    & yes     & yes  & Sodium nitrite                                                                                                     & TRI            & metal restricted & yes & yes & yes\\
            Creosote                                                                   & TRI            & carcinogenic          & yes    & yes     & NA   & Styrene                                                                                                            & TRI            & carcinogenic          & yes    & yes     & yes  \\
            Cresol (mixed isomers)                                                     & TRI            & clean air act         & yes    & yes     & NA   & Sulfuric acid (acid aerosols including mists, vapors, gas, fog, and other airborne forms of any particle size) & TRI & formulation component & yes & NA & NA\\
            Cumene                                                                     & TRI            & carcinogenic          & yes    & yes     & NA   & tert-Butyl alcohol                                                                                                 & TRI            & formulation component & yes & NA & NA\\
            Cumene hydroperoxide                                                       & TRI            & manufacturing aid     & yes    & yes     & NA   & Tetrabromobisphenol A                                                                                              & PBT            & formulation component & yes & yes & yes\\
            Cyanide compounds                                                          & TRI            & clean air act         & yes    & yes     & yes  & Tetrachloroethylene                                                                                                & TRI            & carcinogenic & yes & yes & NA\\
            Cyclohexane                                                                & TRI            & formulation component & yes    & yes     & yes  & Thiabendazole                                                                                                      & TRI            & formulation component & yes & yes & yes\\
            Decabromodiphenyl oxide                                                    & TRI            & formulation component & yes    & yes     & NA   & Thiram                                                                                                             & TRI            & article component & yes & yes & yes\\
            Di(2-ethylhexyl) phthalate                                                 & TRI            & carcinogenic          & yes    & yes     & yes  & Toluene                                                                                                            & TRI            & clean air act & yes & yes & yes\\
            Diaminotoluene (mixed isomers)                                             & TRI            & carcinogenic          & yes    & NA      & NA   & Toluene-2,4-diisocyanate & TRI & carcinogenic & yes & yes & NA\\
            Dibenzofuran                                                               & TRI            & clean air act         & yes    & yes     & NA   & Toluene diisocyanate (mixed isomers)                                                                               & TRI & carcinogenic & yes & yes & NA\\
            Dibutyl phthalate                                                          & TRI            & clean air act         & yes    & NA      & NA   & Trichloroethylene                                                                                                  & TRI            & carcinogenic & yes & yes & NA\\
            Dichloromethane                                                            & TRI            & carcinogenic          & yes    & yes     & yes  & Triethylamine                                                                                                      & TRI            & clean air act         & yes & yes & NA\\
            Dicyclopentadiene                                                          & TRI            & formulation component & yes    & yes     & yes  & Trifluralin                                                                                                        & PBT            & clean air act & yes & NA & NA\\
            Diethanolamine                                                             & TRI            & clean air act         & yes    & yes     & NA   & Vanadium compounds                                                                                                 & TRI            & metal restricted & yes & yes & NA\\
            Diglycidyl resorcinol ether                                                & TRI            & carcinogenic          & yes    & NA      & NA   & Vinyl acetate                                                                                                      & TRI            & carcinogenic & yes & yes & NA\\
            Diisocyanates                                                              & TRI            & clean air act         & yes    & yes     & NA   & Vinyl chloride                                                                                                     & TRI            & carcinogenic          & yes    & yes & NA\\
            Dimethyl phthalate                                                         & TRI            & clean air act         & yes    & yes     & NA   & Xylene (mixed isomers)                                                                                             & TRI            & clean air act & yes & yes & yes\\
            Dimethylamine                                                              & TRI            & formulation component & yes    & yes     & NA   & Zinc (fume or dust)                                                                                                & TRI            & metal restricted & yes & yes & NA\\
            Dimethylamine dicamba                                                      & TRI            & others                & yes    & NA      & NA   & Zinc compounds                                                                                                     & TRI            & metal restricted      & yes & yes & yes\\ \bottomrule\bottomrule
        \end{tabular}
    }
%    }
    \begin{minipage}
        \linewidth
        \vspace{0.01in}
        \tiny NOTES: NA means absent in that sample; POTW means publicly owned treatment works.
    \end{minipage}
\end{table}

        \begin{figure}[H]
    \centering
    \includegraphics[width=\textwidth]{C:/Users/david/OneDrive/Documents/ULMS/PhD/Thesis/chapter3/src/climate_change/latex/motivation_plots}
    \caption{Trends in Releases Intensities by Treatment Status}
    \label{fig:releases-plots-treatment}
\end{figure}
        % Please add the following required packages to your document preamble:
% \usepackage{booktabs}
% \usepackage{graphicx}
\begin{table}[H]
    \centering
    \caption{Onsite Releases Intensity given Production Technology (Wages)}
    \label{tab:heterogeneous-onsite-releases-int-tech-wages}
    \resizebox{\columnwidth}{!}{%
        \begin{tabular}{@{}llllllll@{}}
            \toprule\toprule
            Onsite releases intensity (log) & total     & air emissions & point air & fugitive air & water discharge & land releases & surface impoundment \\ \midrule
            $Treated^{e}$                   & -0.179*** & -0.082*       & -0.042    & -0.008       & -0.054*         & -0.026        & 0.005               \\
            & -0.123.   & (0.046)       & (0.040)   & (0.035)      & (0.033)         & (0.031)       & (0.006)             \\
            $Treated$                       & 0.203***  & 0.109**       & 0.033     & 0.063*       & 0.067           & 0.018         & 0.005**             \\
            & (0.064)   & (0.045)       & (0.038)   & (0.034)      & (0.050)         & (0.014)       & (0.003)             \\
            cohort 2014                     & -0.123*   & -0.057        & -0.057    & 0.045        & -0.054          & 0.006         & 0.008               \\
            & (0.067)   & (0.055)       & (0.047)   & (0.038)      & (0.035)         & (0.021)       & (0.009)             \\
            cohort 2015                     & -0.264*** & -0.121**      & -0.018    & -0.088*      & -0.056*         & -0.073        & -0.001              \\
            & (0.091)   & (0.056)       & (0.048)   & (0.047)      & (0.033)         & (0.060)       & (0.003)             \\
            cohort 2017                     & -0.109    & -0.091        & -0.115    & 0.104        & 0.011           & -0.000        & -0.006              \\
            & (0.152)   & (0.116)       & (0.112)   & (0.099)      & (0.101)         & (0.037)       & (0.007)             \\
            controls                        & Yes       & Yes           & Yes       & Yes          & Yes             & Yes           & Yes                 \\
            year FE                         & Yes       & Yes           & Yes       & Yes          & Yes             & Yes           & Yes                 \\
            facility FE                     & Yes       & Yes           & Yes       & Yes          & Yes             & Yes           & Yes                 \\
            border-county FE                & Yes       & Yes           & Yes       & Yes          & Yes             & Yes           & Yes                 \\
            toxic chemical FE               & Yes       & Yes           & Yes       & Yes          & Yes             & Yes           & Yes                 \\
            toxic chemical LTs              & Yes       & Yes           & Yes       & Yes          & Yes             & Yes           & Yes                 \\
            border-county LTs               & Yes       & Yes           & Yes       & Yes          & Yes             & Yes           & Yes                 \\ \midrule
            $R^2$                           & 0.727     & 0.745         & 0.718     & 0.669        & 0.592           & 0.507         & 0.168               \\
            Obsservations                   & 1,893,689 & 1,893,689     & 1,893,689 & 1,893,689    & 1,893,689       & 1,893,689     & 1,893,689           \\ \bottomrule\bottomrule
        \end{tabular}%
    }
    \begin{minipage}{\columnwidth}
        \vspace{0.05in}
        \tiny NOTES: These results are obtained from estimating equation~\ref{eq:heterogeneous-onsite-releases-intensity-fintech}. Three-way clustered robust standard errors are reported in parentheses, and clustered at the toxic chemical, industry and state levels. ***, **, and * denote significance levels at the less than $1\%$, $5\%$ and $10\%$, respectively.
    \end{minipage}
\end{table}
        \begin{figure}[H]
    \centering
    \includegraphics[width=1\textwidth, height=0.5\textheight,keepaspectratio]{fig_sdid_total_onsite_releases_int_tech_wages}
    \caption{Triple-Differences: Onsite Total Releases Intensities given Production Technology (Wages)}
    \label{fig:heterogeneous-onsite-releases-intensities-wages-tech}
    \begin{minipage}{\columnwidth}
        \vspace{0.05in}
        \tiny NOTES: The event study model of equation~\ref{eq:heterogeneous-onsite-releases-intensity-fintech} is $P_{f,cp,c,t}^{fintech} = \sum_{{e = -3},{e \neq -1}}^{3} \beta (Treated^{e} \cdot D)_{f,s,t} + \psi (Treated^{e})_{s,t} + \vartheta (Treated \cdot D)_{f,s,t} + \mu (Post \cdot D)_{f,s,t} + \tau Treated_{s,t} + \rho D_{f,s,t} + \alpha Post_{t} + \delta X_{v,c,t-1} + \omega F_{f,t} + \lambda_{t} + \gamma_{f} + \phi_{cp} + \zeta_{c} + \eta_{c,t} + \theta_{cp,t} + \varepsilon_{f,cp,c,t}$. Three-way clustered robust standard errors are reported in parentheses, and clustered at the toxic chemical, industry and state levels. HEIs mean highest emitting industries. HWRR means high wages to revenues ratio---labour-intensive manufacturing industries. LWRR means low wages to revenues ratio---capital-intensive manufacturing industries.
    \end{minipage}
\end{figure}
        \begin{figure}[H]
    \centering
    \includegraphics[width=1\textwidth,keepaspectratio]{C:/Users/david/OneDrive/Documents/ULMS/PhD/Thesis/chapter3/src/climate_change/latex/fig_sdid_profits}
    \caption{Manufacturing Industry Profits}
    \label{fig:baseline-manufacturing-industry-profits}
    \begin{minipage}{\columnwidth}
        \vspace{0.05in}
        \tiny NOTES: The event study model of equation~\ref{eq:baseline-wages} is $R_{i,cp,t} = \sum_{{e = -3},{e \neq -1}}^{3} \beta Treated_{s,t}^e = \textbf{1}[t - G_{s,t}] + \delta X_{v,c,t-1} + \omega F_{f,t} + \lambda_{t} + \sigma_{c} + \phi_{cp} + \zeta_{cp,t} + \epsilon_{i,cp,t}$; where $R_{i,cp,t}$ is the profit and margin vector. Standard errors are clustered at the state level. de Chaisemartin and D'Haultfoeuille Decomposition: $\sum dCDH_{ATTs}^{weights(+)} = 1$ and $\sum dCDH_{ATTs}^{weights(-)} = 0$.
    \end{minipage}
\end{figure}
        % Please add the following required packages to your document preamble:
% \usepackage{booktabs}
% \usepackage{graphicx}
\begin{table}[H]
    \centering
    \caption{Production Technology: Onsite Waste Management Activities}
    \label{tab:prod-tech-mechanisms-onsite-waste-management-activities}
    \resizebox{\columnwidth}{!}{%
        \begin{tabular}{@{}llllll@{}}
            \toprule\toprule
            Waste Management Activities & \multicolumn{3}{c}{Treatment} & \multicolumn{2}{c}{Energy Recovery \& Recycling} \\ \cmidrule(lr){3-4} \cmidrule(lr){5-6}

            & total     & biological & incineration & boiler    & recycling \\ \midrule
            $Treated^{e} \cdot D$ & -0.121    & 0.018***   & 0.013        & 0.006**   & -0.185    \\
            & (0.184)   & (0.007)    & (0.010)      & (0.003)   & (0.165)   \\
            $Treated^{e}$         & -0.217*   & -0.021***  & -0.013       & -0.002    & -0.178**  \\
            & (0.127)   & (0.007)    & (0.011)      & (0.003)   & (0.072)   \\
            controls              & Yes       & Yes        & Yes          & Yes       & Yes       \\
            year FE               & Yes       & Yes        & Yes          & Yes       & Yes       \\
            facility FE           & Yes       & Yes        & Yes          & Yes       & Yes       \\
            border-county FE      & Yes       & Yes        & Yes          & Yes       & Yes       \\
            toxic chemical FE     & Yes       & Yes        & Yes          & Yes       & Yes       \\
            toxic chemical LTs    & Yes       & Yes        & Yes          & Yes       & Yes       \\
            border-county LTs     & Yes       & Yes        & Yes          & Yes       & Yes       \\\midrule
            Observations          & 1,893,689 & 1,893,689  & 1,893,689    & 1,893,689 & 1,893,689 \\
            $R^2$                 & 0.779     & 0.615      & 0.668        & 0.683     & 0.751     \\ \bottomrule \bottomrule
        \end{tabular}%
    }
    \begin{minipage}{\columnwidth}
        \vspace{0.05in}
        \tiny \textbf{NOTES:} These results are obtained from estimating model~\ref{eq:mechanisms-waste-management}. ***, **, and * denote significance levels at the less than $1\%$, $5\%$, and $10\%$, respectively. This result is based on total production wages to revenue ratio.
    \end{minipage}
\end{table}
        % Please add the following required packages to your document preamble:
% \usepackage{booktabs}
% \usepackage{graphicx}
\begin{table}[H]
    \centering
    \caption{Production Technology: Onsite Source Reduction Activities}
    \label{tab:production-technology-onsite-source-reduction-activities-wrr}
    \resizebox{\columnwidth}{!}{%
        \begin{tabular}{@{}lllllllllll@{}}
            \toprule\toprule
            Source Reduction Activities & \multicolumn{4}{c}{Material Substitution} & \multicolumn{2}{c}{Process Modification}   & \multicolumn{3}{c}{Operations Activities}   & \multicolumn{1}{c}{Inventory Management} \\
            \cmidrule(lr){3-5} \cmidrule(lr){6-7} \cmidrule(lr){8-10} \cmidrule(lr){11-11}
            & all       & chemical purity & clean fuel & energy cost intensity & new green tech & recycling & prod quality analysis & operating trainings  & imp material handle& container size change \\ \midrule
            $Treated^{e} \cdot D$ & -0.018    & 0.003***        & 0.003      & 0.004                 & 0.007***       & 0.056***  & 0.000                 & 0.001               & 0.000    & 0.003**                            \\
            & (0.026)   & (0.001)         & (0.004)    & (0.011)               & (0.002)        & (0.012)   & (0.001)               & (0.001)             & (0.001)             & (0.001)               \\
            $Treated^{e}$         & -0.054**  & -0.001*         & -0.016***  & 0.011                 & -0.006***      & -0.036*** & 0.000                 & -0.002**              & -0.003*** & 0.001                         \\
            & (0.021)   & (0.000)         & (0.004)    & (0.009)               & (0.002)        & (0.007)   & (0.000)               & (0.001)             & (0.001)             & (0.001)               \\
            controls              & Yes       & Yes             & Yes        & Yes                   & Yes            & Yes       & Yes                   & Yes                 & Yes                 & Yes                   \\
            year FE               & Yes       & Yes             & Yes        & Yes                   & Yes            & Yes       & Yes                   & Yes                 & Yes                 & Yes                   \\
            facility FE           & Yes       & Yes             & Yes        & Yes                   & Yes            & Yes       & Yes                   & Yes                 & Yes                 & Yes                   \\
            border-county FE      & Yes       & Yes             & Yes        & Yes                   & Yes            & Yes       & Yes                   & Yes                 & Yes                 & Yes                   \\
            toxic chemical FE     & Yes       & Yes             & Yes        & Yes                   & Yes            & Yes       & Yes                   & Yes                 & Yes                 & Yes                   \\
            toxic chemical LTs    & Yes       & Yes             & Yes        & Yes                   & Yes            & Yes       & Yes                   & Yes                 & Yes                 & Yes                   \\
            border-county LTs     & Yes       & Yes             & Yes        & Yes                   & Yes            & Yes       & Yes                   & Yes                 & Yes                 & Yes                   \\\midrule
            Observations          & 1,893,689 & 1,893,689       & 1,893,689  & 1,893,689             & 1,893,689      & 1,893,689 & 1,893,689             & 1,893,689           & 1,893,689             & 1,893,689           \\
            $R^2$                 & 0.699     & 0.147           & 0.781      & 0.987                 & 0.354          & 0.503     & 0.154                 & 0.167               & 0.433               & 0.260                 \\ \bottomrule \bottomrule
        \end{tabular}%
    }
    \begin{minipage}{\columnwidth}
        \vspace{0.05in}
        \tiny NOTES: These results are obtained from estimating model~\ref{eq:mechanisms-source-reduction}. ***, **, and * denote significance levels at the less than $1\%$, $5\%$ and $10\%$, respectively. SRA means source reduction activities. This results are based on using total production wages to revenue ratio as a measure of labour v. capital intensive manufacturing industries.
    \end{minipage}
\end{table}


        \section{Distribution of Industries and Pollution Emissions Intensities}\label{sec:appendix-distribution-of-industries-and-pollution-emissions-intensities}
        \begin{figure}[H]
    \centering
    \includegraphics[width=0.85\textwidth]{C:/Users/david/OneDrive/Documents/ULMS/PhD/Thesis/chapter3/src/climate_change/latex/fig_naics_distribution}
    \caption{Distribution of Manufacturing Industries in the Sample}
    \label{fig:naics-manufacturing-industries}
\end{figure}
        \begin{figure}[H]
    \centering
    \includegraphics[width = 0.8\textwidth]{fig_releases_distribution}
    \caption{Distribution of Total Onsite Releases Intensity across Manufacturing Industries}
    \label{fig:releases-distribution}
\end{figure}
        \begin{figure}[H]
    \centering
    \includegraphics[width = 0.8\textwidth]{C:/Users/david/OneDrive/Documents/ULMS/PhD/Thesis/chapter3/src/climate_change/latex/fig_air_emissions_distribution_naics}
    \caption{Distribution of Total Onsite Air Emissions Intensity across Manufacturing Industries}
    \label{fig:air-emissions-distribution-naics}
\end{figure}
        \begin{figure}[H]
    \centering
    \includegraphics[width = 0.8\textwidth]{fig_water_distribution_naics}
    \caption{Distribution of Total Onsite Surface Water Discharge Intensity across Manufacturing Industries}
    \label{fig:water-discharge-distribution-naics}
\end{figure}
        \begin{figure}[H]
    \centering
    \includegraphics[width = 0.8\textwidth]{C:/Users/david/OneDrive/Documents/ULMS/PhD/Thesis/chapter3/src/climate_change/latex/fig_land_releases_distribution_naics}
    \caption{Distribution of Total Onsite Land Releases Intensity across Manufacturing Industries}
    \label{fig:land-releases-distribution-naics}
\end{figure}
        \begin{figure}[H]
    \centering
    \includegraphics[width = 0.8\textwidth]{fig_releases_distribution_states}
    \caption{Distribution of Total Onsite Releases Intensity between the Treated and Control States}
    \label{fig:releases-distribution}
\end{figure}
        \begin{figure}[H]
    \centering
    \includegraphics[width = 0.8\textwidth]{fig_air_emissions_distribution_state}
    \caption{Distribution of Total Air Emission Intensity between the Treated and Control States.}
    \label{fig:air-emissions-distribution}
\end{figure}
        \begin{figure}[H]
    \centering
    \includegraphics[width = 0.8\textwidth]{C:/Users/david/OneDrive/Documents/ULMS/PhD/Thesis/chapter3/src/climate_change/latex/fig_water_discharge_distribution_state}
    \caption{Distribution of Total Surface Water Discharge Intensity between the Treated and Control States}
    \label{fig:water-discharge-distribution}
\end{figure}
        \begin{figure}[H]
    \centering
    \includegraphics[width = 0.8\textwidth]{fig_land_releases_distribution_state}
    \caption{Distribution of Total Onsite Land Releases Intensity between the Treated and Control States}
    \label{fig:land-releases-distribution}
\end{figure}
        \begin{figure}[H]
    \centering
    \includegraphics[width = 0.8\textwidth]{fig_releases_distribution_carcinogenic}
    \caption{Distribution of Average Total Onsite Carcinogenic Releases Intensity between the Treated and Control States}
    \label{fig:releases-distribution-carcinogenic}
\end{figure}
        \begin{figure}[H]
    \centering
    \includegraphics[width = 0.8\textwidth]{C:/Users/david/OneDrive/Documents/ULMS/PhD/Thesis/chapter3/src/climate_change/latex/fig_releases_distribution_caa}
    \caption{Distribution of Average Total Onsite CAA Releases Intensity between the Treated and Control States}
    \label{fig:releases-distribution-caa}
\end{figure}
        \begin{figure}[H]
    \centering
    \includegraphics[width = 0.8\textwidth]{fig_releases_distribution_haps}
    \caption{Distribution of Total Onsite HAPs Releases Intensity between the Treated and Control States}
    \label{fig:releases-distribution-haps}
\end{figure}
        \begin{figure}[H]
    \centering
    \includegraphics[width = 0.8\textwidth]{fig_releases_distribution_pbts}
    \caption{Distribution of Average Total Onsite PBT Releases Intensity between the Treated and Control States}
    \label{fig:releases-distribution-pbts}
\end{figure}


        \section{Baseline Robustness Tables and Figures}\label{sec:appendix-baseline-robustness-tables-and-figures}
        \begin{figure}[H]
    \centering
    \includegraphics[width=1\textwidth,keepaspectratio]{fig_sdid_industry_costs_skilled}
    \caption{Manufacturing Industry Costs for High vs. Low-Skilled Workers}
    \label{fig:baseline-manufacturing-industry-costs-skilled}
    \begin{minipage}{\columnwidth}
        \vspace{0.05in}
        \tiny NOTES: The event study model of equation~\ref{eq:baseline-wages} is $C_{i,cp,t} = \sum_{{e = -3},{e \neq -1}}^{3} \beta (Treated^{e} \cdot D)_{i,s,t} + \psi (Treated^{e})_{s,t} + \vartheta (Treated \cdot D)_{i,s,t} + \mu (Post \cdot D)_{i,s,t} + \tau Treated_{s,t} + \rho D_{i,s,t} + \alpha Post_{t} + \delta X_{v,c,t-1} + \omega F_{f,t} + \lambda_{t} + \sigma_{c} + \phi_{cp} + \zeta_{cp,t} + \epsilon_{i,cp,t}$. Standard errors are clustered at the state level. $D_{i,s,t}$ is unity for low-skilled and zero for high-skilled workers; $Treated_{s,t}$ is unity for treated and zero for control states; and $Post_{t}$ is unity for post-treatment and zero for pre-treatment periods.
    \end{minipage}
\end{figure}
        % Please add the following required packages to your document preamble:
% \usepackage{booktabs}
% \usepackage{graphicx}
\begin{table}[H]
    \centering
    \caption{Potential Cross County/State Mobility}
    \label{tab:baseline-cross-county-state-mobility}
    \resizebox{\columnwidth}{!}{%
        \begin{tabular}{@{}lllllll@{}}
            \toprule\toprule
            & \multicolumn{2}{c}{Employment (log)} & \multicolumn{2}{c}{Production Workers (log)} & \multicolumn{2}{c}{Production Hours (log)} \\
            \cmidrule(lr){2-3} \cmidrule(lr){4-5} \cmidrule(lr){6-7}
            employment \& hours          & 1         & 2         & 3         & 4         & 5         & 6         \\ \midrule
            $Treated^{e} \cdot distance$ & -0.000    & 0.000     & -0.000    & 0.000     & -0.001    & -0.000    \\
            & (0.001)   & (0.001)   & (0.002)   & (0.001)   & (0.001)   & (0.001)   \\
            $Treated^{e}$                & -0.005    & 0.005     & -0.017    & -0.004    & -0.010    & 0.001     \\
            & (0.034)   & (0.033)   & (0.036)   & (0.030)   & (0.034)   & (0.030)   \\
            $distance \cdot$ cohort 2014 & -0.001    & -0.001*** & -0.001    & -0.002*** & -0.001    & -0.002*** \\
            & (0.001)   & (0.000)   & (0.001)   & (0.000)   & (0.001)   & (0.000)   \\
            $distance \cdot$ cohort 2015 & -0.000    & 0.003     & 0.000     & 0.004     & -0.000    & 0.004     \\
            & (0.001)   & (0.002)   & (0.002)   & (0.002)   & (0.002)   & (0.002)   \\
            $distance \cdot$ cohort 2017 & -0.003**  & -0.019*** & -0.002    & -0.019*** & -0.002    & -0.019*** \\
            & (0.001)   & (0.003)   & (0.002)   & (0.004)   & (0.002)   & (0.003)   \\
            controls                     & Yes       & Yes       & Yes       & Yes       & Yes       & Yes       \\
            year FE                      & Yes       & Yes       & Yes       & Yes       & Yes       & Yes       \\
            county FE                    & Yes       & Yes       & Yes       & Yes       & Yes       & Yes       \\
            border-county FE             & No        & Yes       & No        & Yes       & No        & Yes       \\
            border-county LTs            & No        & Yes       & No        & Yes       & No        & Yes       \\ \midrule
            Observations                 & 1,893,689 & 1,893,689 & 1,893,689 & 1,893,689 & 1,893,689 & 1,893,689 \\
            $R^2$                        & 0.354     & 0.393     & 0.336     & 0.378     & 0.341     & 0.386     \\
            Baseline Mean                & 44.99     & 44.99     & 31.42     & 31.42     & 64.82     & 64.82     \\ \bottomrule \bottomrule
        \end{tabular}%
    }
    \begin{minipage}{\columnwidth}
        \vspace{0.05in}
        \tiny NOTES: These results are obtained from estimating model $E_{i,cp,t} = \beta (Treated^{e} \cdot D)_{h,s,t} + \psi (Treated^{e})_{s,t} + \vartheta (Treated \cdot D)_{h,s,t} + \mu (Post \cdot D)_{h,s,t} + \tau Treated_{s,t} + \rho D_{h,s,t} + \alpha Post_{t} + \delta X_{v,c,t-1} + \omega F_{f,t} + \lambda_{t} + \sigma_{h} + \phi_{cp} + \zeta_{cp,t} + \epsilon_{i,cp,t}$. Robust standard errors clustered at the state level are reported in parentheses. ***, **, and * denote significance levels at the less than $1\%$, $5\%$ and $10\%$, respectively.
    \end{minipage}
\end{table}
        \begin{figure}[H]
    \centering
    \includegraphics[width=1\textwidth, keepaspectratio]{fig_sdid_emp_hours_skilled}
    \caption{Industry Employment, Production Workers and Hours for High vs. Low-Skilled Workers}
    \label{fig:baseline-employment-hours-skilled}
    \begin{minipage}{\columnwidth}
        \vspace{0.05in}
        \tiny NOTES: The event study model of equation~\ref{eq:baseline-emp-hours} is $E_{i,cp,t} = \sum_{{e = -3},{e \neq -1}}^{3} \beta (Treated^{e} \cdot D)_{i,s,t} + \psi (Treated^{e})_{s,t} + \vartheta (Treated \cdot D)_{i,s,t} + \mu (Post \cdot D)_{i,s,t} + \tau Treated_{s,t} + \rho D_{i,s,t} + \alpha Post_{t} + \delta X_{v,c,t-1} + \omega F_{f,t} + \lambda_{t} + \sigma_{c} + \phi_{cp} + \zeta_{cp,t} + \epsilon_{i,cp,t}$. Standard errors are clustered at the state level. $D_{i,s,t}$ is unity for low-skilled and zero for high-skilled workers; $Treated_{s,t}$ is unity for treated and zero for control states; and $Post_{t}$ is unity for post-treatment and zero for pre-treatment periods.
    \end{minipage}
\end{figure}

    \end{appendices}
%======================================================================================================================%
    \newpage
    \printbibliography
\end{document}
%    \section{Literature Gaps}\label{sec1:literature-gaps}
%    The minimum wage policy and its effect on employment has been contentious since history~\parencite{neumark1992employment, card1993minimum}. Following the $2007$ federal wage floor, the arguments on the subject has proliferated. Many arguments favour the negative effects on employment while other scholars found sharp null zero effects. For example,~\cite{cengiz2019effect} used event study bunching analysis to study the effect of minimum wage rise on low wages in different wage bins, and further examined its effect on employment. They document that earnings rose but overall employment remained unchanged following a minimum wage rise in the US. Moreover, they find that minimum wage reduced employment of workers in tradable sectors such as restaurant and retail. However, one major drawback of this study is the event study-bunching analysis design. It used a federal wage floor and grouped states into treated and untreated units without accounting for specific years that each state actually raised its minimum wage. This type of setting has been criticised in the literature to recover at best a volume weighted ATT on each wage bin without accounting for new entrant low-wage workers and subsequent increases in minimum wage by different states. Even worse, it ignores the fact that some untreated states may have raised their minimum wage in the future, the later treated states, as well as new entrant low-wage workers that subsequent wage increases may have induced. Because of this, it is possible that the recovered ATT may suffer from attenuation bias, and worst case, have a wrong sign in the event study analysis~\parencite{goodman2021difference}. Furthermore,~\cite{riley2017raising} examined the effect of minimum wage on labour productivity in Britain using traditional difference-in-differences method. They proxied wage with labour cost and found that minimum wage raised labour costs for low-wage workers, and persisted during and after the global financial crisis. Consequently, they show that firms tend to raise their labour productivity as a result of minimum wage. They explained that this increased productivity is driven by changes in total factor productivity rather than decline in firms labour force or capital-labour substitution. One concern here is that it is not clear how the total factor productivity was combined to drive labour productivity. Is it more energy intensive or energy efficient? How does firms adjust the energy intensity in their production process in reaction to the raised minimum wage? And is this adjustment behaviour pro-environment?
%
%    ~\cite{li2023does} attempted to investigate a related question by looking at the effect of minimum wage standard on pollutions in China's manufacturing sector. Using panel FE and hierarchical modelling they evince that increase in minimum wage increases manufacturing firm's emissions. They further argued that the increase in pollutions is driven by manufacturing firms shifting investments to cost-effective traditional energy sources such as fossil fuel energy as they adjust their production functions for the wage-induced labour costs. They also argue that firms have less inputs of pollution treatments in their production process. The shortcomings of this study is that they failed to present a causal link between minimum wage and pollution from firms, rather they evince a correlation. This kind of evidence can be biased in many ways. One of which is that the effect may be driven by other employees in the higher end of the wage distribution for whom minimum wage is not targeted. That is, it will pick up effects from high-wage employee groups other than those that minimum wage affect directly - low-wage workers. Thus, confounding the effect of raising minimum wage on emissions. Further, suppose there is indeed evidence that minimum wage raises firm's pollution through increased crude energy use or investment, as well as using less pollution treatment in their production process. It is still not clear how this new firm adjustment behaviour transmits to pollution. One possible reason would be the energy intensity of the firms. How much energy are firms using to produce one unit of good following a minimum wage raise? To answer this kind of question requires a more precise method, to investigate the effects on pollution post minimum wage raise. And it may be possible that firms invest in more energy efficient production technologies to increase their productivity given the new higher labour cost. In this setting, we should expect a reduced energy intensity following a raise in minimum wage, and an opposite effect on emissions. But this subject is yet to be explored.
%
%    \section*{Contribution}\label{sec:contribution}
%    Given the above literature gaps, I adopt a more robust approach to investigate the effects of minimum wage in the US. First, revist the effect on wages and employment - first stage. Second, examine the effect on labour productivity. Third, investigate the effect on emissions. Fourth, trace the channels of the effects, by decomposing total factor productivity and narrow in on firm's energy intensity to evidence the channel of effect of minimum wage on emissions.


%    \section{Considered identification}\label{sec:considered-identification}
%    The US federal government raised the minimum wage for low-wage workers from $\$5.15$ in $1997$ to $5.85$ in $2007$ and by $2009$ they had raised it to $\$7.25$. However, different states raised their state minimum wages in different years since $2004$. I exploit this time variations in state minimum wage to identify the effect of minimum wage on emissions. This provides for a staggered adoption of minimum wage raise across US states. Between $2004-2019$, a total of $12$ states raised their minimum wage in $2005$ and $2006$, $6$ different states first raising minimum wage each year. Another $12$ states raised theirs in $2007$; $17$ states raised theirs in $2008$; $2$ states in $2009$; and another $2$ states in $2010$ and $2019$, one in each year. A total of $5$ states has no minimum wage policy (Alabama, Louisiana, Mississippi, South Carolina, and Tennessee).
%======================================================================================================================%