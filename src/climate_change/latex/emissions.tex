\documentclass[12pt, english]{article}
%======================================================================================================================%
% Preamble
\usepackage[
    backend = biber,
    style = apa,
    citestyle = authoryear-comp,
    sorting = ydnt,
    mincitenames = 1,
    maxcitenames = 2,
    uniquelist = minyear
]{biblatex}
\addbibresource{emissions.bib}
%\AtBeginBibliography{\small}

\usepackage{hyperref}
\hypersetup{colorlinks = true, citecolor = blue, linkcolor = blue, urlcolor = blue, hypertexnames = true}
\newcommand{\shortlink}[1]{\href{https://www.#1}{\texttt{#1}}}

\DeclareCiteCommand{\cite} % Ensures author-year hyperlink applies to \cite
{\usebibmacro{prenote}}
{\usebibmacro{citeindex}%
\printtext[bibhyperref]{\usebibmacro{cite}}}
{\multicitedelim}
{\usebibmacro{postnote}}

\DeclareCiteCommand{\parencite}[\mkbibparens] % Ensures author-year hyperlink applies to \parencite
{\usebibmacro{prenote}}
{\usebibmacro{citeindex}%
\printtext[bibhyperref]{\usebibmacro{cite}}}
{\multicitedelim}
{\usebibmacro{postnote}}

\DeclareCiteCommand{\cite} % Ensures that year appears in parentheses
{\usebibmacro{prenote}}
{\usebibmacro{citeindex}%
\printtext[bibhyperref]{\printnames{labelname} \mkbibparens{\printfield{year}}}}
{\multicitedelim}
{\usebibmacro{postnote}}

\usepackage{authblk}
\usepackage{times}
\usepackage{parskip}
\usepackage{graphicx}
\usepackage{amsmath}
\usepackage{amsfonts}
\usepackage{mathrsfs}
\usepackage{float}
\usepackage{geometry}
\usepackage{scrextend}
\geometry{papersize = {9in, 11in}, left = 2.5cm, right = 2.5cm, top = 2.5cm, bottom = 2.5cm}
\usepackage[doublespacing]{setspace}
\usepackage{textcomp}
\usepackage{csquotes}
\usepackage{appendix}

% Tables Packages
\usepackage{booktabs}

%%%%%%%%%%%%%%%%%%%%%%%%%%%%%%%%%%%%%%%%%%%%%%%%%%%%%%%%%%%%%%%%%%%%%%%%%%%%%%%%%%%%%%%%%%%%%%%%%%%%%%%%%%%%%%%%%%%%%%%%
% Title Page
\title{{Unforeseen Minimum Wage Consequences}}
\author[1]{D.O. Ekeocha}
%\author[2]{D.F. Giuseppe}
%\author[3]{J. Lonsky}
\affil[1]{
    University of Liverpool Management School
    \\ \texttt{davidmac.ekeocha@liverpool.ac.uk}
%    \\ \texttt{giuseppe.de-feo@liverpool.ac.uk}
%    \\ \texttt{jakub.lonsky@liverpool.ac.uk}
}
%\affil[2]{University of Liverpool Management School}
%\affil[3]{University of Liverpool Management School}

\date{\today}

%%%%%%%%%%%%%%%%%%%%%%%%%%%%%%%%%%%%%%%%%%%%%%%%%%%%%%%%%%%%%%%%%%%%%%%%%%%%%%%%%%%%%%%%%%%%%%%%%%%%%%%%%%%%%%%%%%%%%%%%
\begin{document}
    \maketitle
%    \newpage
%    \tableofcontents
%    \newpage
%    \listoffigures
%    \newpage
%    \listoftables
%    \newpage


    \section{Introduction}\label{sec:introduction}
    The effects of raising the minimum wage (MW) have garnered interest in the literature, reaching mixed conclusions for employment given assumed labour market conditions. For perfect competition, some studies have recorded disemployment effects of an MW raise on low-wage workers~\parencite{stigler1946economics, hamermesh1982minimum, neumark1992employment, brown1999minimum, machin2004minimum, neumark2000minimum, borjas2010labor}, whereas in monopsonistic markets, limited or a sharp null employment effect is documented~\parencite{lester1960employment, card1993minimum, card2000minimum, aaronson2018industry, cengiz2019effect, wong2019minimum, dustmann2022reallocation}. Other studies have recorded heterogeneous employment effects~\parencite{okudaira2019minimum, medrano2023minimum, meer2023effects, gregory2022minimum}.

    On the other hand, the symmetry between raising MW and labour costs or wages of low-wage workers is well established~\parencite{medrano2023minimum,clemens2023important}. Career progressions and high labour demand, higher job/worker effort and labour productivity, especially for incumbent workers in the $40th$ percentile of the wage distribution relative to workers in higher percentiles, largely drive the increase in wages of low-wage workers~\parencite{riley2017raising, kim2019minimum, wong2019minimum, baek2021impact, zhao2021effects, seok2022macroeconomic, ku2022does, coviello2022minimum, alexandre2022minimum}. They argued that firms' adjustments in total factor productivity, lower hiring and layoff and monitoring incentive explain the increased labour productivity, and that the increase in worker effort can offset about $50\%$ of projected rise in labour cost. Furthermore,~\cite{harasztosi2019pays} argued that the cost-effect of raising MW is not only borne by employers and workers, but that firms tend to pass some of these burdens to consumers via higher prices.

    Interestingly, the question regarding the existence of other pathways through which firms pass-through the burdens of higher-cost-induced MW is still nascent. These studies focused on Chinese manufacturing firms to document higher emissions and pollution emissions intensity induced MW raise~\parencite{li2023does, zhang2023unintended}~\footnote{\tiny Two hypotheses present opposing views on the direction of environmental impact of MW. The labour-technology mix which argues that firms react to labour cost induced MW by adopting clean production/operation technologies and labour-savings to increase energy efficiency, labour productivity thereby decreasing their pollution intensity. Conversely, the crowding-out effect hypothesis argues in favour of increased firms' pollution intensity. Given MW, firms struggle financially and to maintain pre-reform operation/production capacity while managing cost, it crowds-out clean production technologies and green innovative practices leading to increased pollution intensity. Thus, to understand how labour-market shocks translate to environmental consequences becomes an empirical exercise.}. They argued that this effect is more prominent in financially constrained firms such that the labour cost pass-through of MW shifts firms energy usage towards cost-effective but crude energy sources, as well as reduced pollution abatement inputs and declining green innovation in their production functions. Albeit these studies used border-county design, they identified their effect using endogenous changes in MW, possibly indexed to inflation, thereby masking their causal claims of rising pollution emissions intensity. Hence, their results are indeed not causal as it is not clear whether the documented pollution responses are due to an increase in MW or inflation.

    However, this study leverages state-level exogenous changes in MW over time and administrative plant-level toxic emissions data (including air, land and water) from the environmental protection agency (EPA) to examine toxic responses of manufacturing firms in the United States (US) to the MW policy. The evidence on the environmental consequences of labour market shocks is particularly important for better health, environmental and labour market policy designs.~\cite{shapiro2018pollution} developed a model to historically explain the fall in the US pollution emissions in the manufacturing sector and revealed several findings including outsourcing the production of pollution-intensive goods to other countries like China and Mexico; environmental regulations such as pollution tax have the most significant negative impact on US pollution intensity through increased investment in effective abatement technologies; and rising labour productivity decreases pollution intensity, thus decreasing pollution emissions. Moreover, some literature has found causal evidence on the effects of air pollution $(PM_{2.5})$ and water pollution on cancer in humans and aquatic animals, respectively~\parencite{turner2020outdoor, turner2017ambient, baines2021linking}.~\cite{coneus2012pollution} found reductions in infant birth weights and increasing bronchitis and respiratory illnesses in toddlers due to increasing carbon monoxides and $O_{3}$ emissions in Germany.

    Two hypotheses exist on possible transmission mechanisms of environmental consequences of raising MW: the factor substitution and crowding out effect hypotheses~\parencite{zhang2023unintended}. The factor substitution hypothesis argues for the reductions in firms pollution emissions intensity. To maximize profits amid rising labour costs due to a raised MW floor, firms adjust their factor inputs by switching to automated production processes from labour-intensive manual processes. Thus, replacing manual labour with machines and technologies, and increasing capital per worker while prioritizing resource allocations toward improving production functions through research~\parencite{harasztosi2019pays,hau2020firm, geng2022minimum,dai2023minimum, li2020labor}. This will further increase labour productivity and total factor productivity~\parencite{riley2017raising}. Thus, raising MW will cause firms to use efficient capital intensive methods, reducing energy intensity and ultimately decreasing emissions per unit of output.~\footnote{\tiny Similarly, the possiblity exists that firms may replace expensive factor inputs with cheaper but crude energy inputs with higher pollution emission potentials, in their production process, which in turn increases energy and pollution emission intensities.} Conversely, the crowding out effect hypothesis argues in favour of increased firms' pollution intensity. These studies have evinced that this effect is attributed to the declining pressure on firms' profitability and heavier financial constraints resulting from wage hikes~\parencite{draca2011minimum, bell2018minimum, du2022minimum}. Consequently, constrained financial resources limit firms' investment in pollution abatement activities (as they focus on core production), leading to reduced pollution removal and higher emission intensity.~\footnote{\tiny In a different way, innovation is influenced by resource scarcity, leading to a preference for certain types of solutions. Technological advancement continues unabated, but resource constraints necessitate a balance between different types of solutions. Labour deficits, often caused by increases in the minimum wage, promote the development of labour-saving technologies, which can impede the progress of labour-intensive green technologies~\parencite{acemoglu2010does}. Firms often prioritize improvements in production efficiency over environmental considerations due to limited resources for innovation. Interestingly, some environmentally friendly technologies, such as end-of-pipe treatments, require significant labour. This emphasis on efficient automation could divert resources away from the development of clean energy and waste management technologies, potentially resulting in increased pollution emissions intensity.}

    To document the causal effect of raising MW on pollution emission intensity, I use a difference-in-differences framework exploiting copious exogenous changes in state-level MW ($\geq \$0.5$ per hour) with clearly defined before- and after-periods. I exploit increases in Arkansas, California, Delaware, Maine, Massachusetts, Maryland, Michigan, Minnesota, Nebraska, New Jersey, New York, South Dakota and West Virginia in $2014$, $2015$ and $2017$. Hereafter, the treated states. I match each treated state to a set of adjacent control states that never implemented any MW policy from $2012-2017$~\parencite{gopalan2021state}. I further restrict the sample to border-counties~\parencite{dube2010minimum}. The identifying assumption is that in the absence of any MW policy, economic conditions in adjacent cross-border counties would have evolved in parallel. Given this assumption, I show that the pre-treatment trends between the treated and control states are similar prior to the MW policy.


%    Third, following the results from the above exercises, I revisit the nascent literature on the impact of MW on firms output per capita and profitability~\parencite{van2006minimum, draca2011minimum, bell2018minimum} documenting declined profits and firms overall value in the UK. To date, the methods used in the literature is the traditional difference-in-differences design which under-performs especially in multiple period settings that the studies cover~\parencite{callaway2021difference, goodman2021difference}. I extend this literature by considering firms in the US and employing the advantages of a staggered design to account for the dynamic raising of MW across US states and dynamic firms adjustment as a result. This has not been accounted for before in the literature.

    The following sections are organized as follows. Section $2$ provides the policy context of the study. Section $3$ discusses the data and descriptive statistics. Section $4$ presents the effect on industry labour costs and employment, and robustness. Section $5$ discusses the effect on onsite toxic releases. Section $6$ discusses the effect on releases to offsite and publicly owned treatments works (POTWs). Section $7$ discusses robustness exercises. Section $8$ focuses on the mechanism analyses. Section $9$ investigates the heterogeneous effects. Section $10$ concludes with policy implications.


    \section{Policy Context}\label{sec:policy-context}
    This section discusses the exogenous state minimum wage changes exploited in the covered sample period $(2011-2017)$, for the causal identification.

    \subsection{Minimum Wage Changes across States}\label{subsec:minimum-wage-changes-across-states}
    About $27$ states raised MW between $2011$ and $2017$. Except for Nevada, the majority of the changes in MW between $2011-2013$ are attributed to inflation. However, following the labour union protests in $2012$ for higher MW, many states responded by instituting a one- or multiphased large MW changes~\parencite{lathrop2021raises}. Specifically, $13$ states implemented a statutory MW raise of at least $\$0.5$ in $2014$, $2015$ and $2017$. Pertinently, there was no federal MW raise in the covered sample period. Table~\ref{tab:states-mw-changes} presents the state-level changes in MW for the sample period.

    \subsection{Selecting the Treated and Control States}\label{subsec:selecting-the-treated-and-control-states}
    The identification strategy focuses on the copious state-level MW changes. Particularly, I restrict the sample to MW changes that meet the following conditions: $(i)$ states with a large MW raise of $\geq$ $\$0.5/hour$ in one year and had never raised their MW since $2012$; $(ii)$ subsequent raises must either be equal to or greater than $\$0.5/hour$ in the post year, or the sum of post-multiphase raises (in any $2$-years) must be equal or greater than the first initial raise for that state. These conditions ensure that the exploited MW changes are not indexed to inflation but statutorily driven. Thirteen $(13)$ states meet the above conditions. They include Arkansas, California, Delaware, Maine, Massachusetts, Maryland, Michigan, Minnesota, Nebraska, New Jersey, New York, South Dakota, and West Virginia.~\footnote{\tiny New Jersey and South Dakota do not meet the second condition but are included in the treated sample. Hence, I explicitly control for US city average inflation in the model to net out any inflationary effects. Controlling for the year fixed effects in the model nets out this inflationary effect.} Table~\ref{tab:states-mw-adjustments-t-and-c} provides a summary of the minimum wage adjustments in the sample. There are four increases of $\$0.75/hour$, four increases of $\$1.00/hour$, four increases of $\geq \$1.00/hour$ with a max of $\$1.85/hour$ and one increase of $\$0.5/hour$ in the sample. The average initial MW change is about $\$1.02/hour$ $(7.6\%)$. Following the first MW raise till the end of the covered period, there are seven post-initial total MW changes between $\$1.00/hour \leq \Delta MW \leq \$1.75/hour$, four total post-initial MW changes between $\$2.00/hour \leq \Delta MW \leq \$2.50/hour$ and two total post-initial MW changes between $\$3.00/hour \leq \Delta MW \leq \$3.35/hour$. These have a total of post-initial MW average of $\$1.95/hour$, about $(7.6\%)$ total average MW changes between $2011$ and $2017$.
    %\usepackage{booktabs}
\begin{table}[H]
    \centering
    \caption{Exogenous State-level MW Adjustments}
    \label{tab:states-mw-adjustments-t-and-c}
    \resizebox{\columnwidth}{!}{%
        \begin{tabular}{lrrrrrrlrr}
            \toprule\toprule
            treated states & MW $\Delta$ year & MW $\Delta$ amount & $\sum_{1}^{2}\Delta MW$ & total MW $\Delta$ amount & start MW & end MW & control states & \# of border counties (T) & \# of border counties (C) \\ \midrule\midrule
            MN             & 2014             & 1.85               & 1.50                    & 3.35                     & 6.16     & 9.51   & (IA, ND, WI)   & 44                        & 57                        \\
            MA             & 2015             & 1.00               & 2.00                    & 3.00                     & 8.00     & 11.00  & (NH)           & 9                         & 6                         \\
            CA             & 2014             & 1.00               & 1.50                    & 2.50                     & 8.00     & 10.50  & (NV)           & 14                        & 16                        \\
            NY             & 2014             & 0.75               & 1.70                    & 2.45                     & 7.25     & 9.70   & (PA)           & 25                        & 16                        \\
            AR             & 2015             & 1.25               & 1.00                    & 2.25                     & 6.25     & 8.50   & (OK, TX)       & 19                        & 18                        \\
            MD             & 2015             & 1.00               & 1.00                    & 2.00                     & 7.25     & 9.25   & (PA, VA)       & 26                        & 42                        \\
            NE             & 2015             & 0.75               & 1.00                    & 1.75                     & 7.25     & 9.00   & (IA, KS, WY)   & 32                        & 31                        \\
            ME             & 2017             & 1.50               & 0.00                    & 1.50                     & 7.50     & 9.00   & (NH)           & 5                         & 8                         \\
            MI             & 2014             & 0.75               & 0.75                    & 1.50                     & 7.40     & 8.90   & (IL, IN, WI)   & 32                        & 40                        \\
            WV             & 2015             & 0.75               & 0.75                    & 1.50                     & 7.25     & 8.75   & (KY*, PA, VA)   & 39                        & 31                        \\
            DE             & 2014             & 0.50               & 0.50                    & 1.00                     & 7.25     & 8.25   & (PA)           & 3                         & 6                         \\ \bottomrule\bottomrule
        \end{tabular}
    }
    \begin{minipage}{17.5cm}
        \vspace{0.01in}
        \tiny NOTES: This table summarizes the exogenous state-level MW changes from $2012$ to $2017$. There are eleven $(11)$ treated and $(14)$ control states. The definition of treated and control states is given in sub-section~\ref{subsec:selecting-the-treated-and-control-states}. MW $\Delta$ year represents the year in which a treated state first raised its MW. MW $\Delta$ amount corresponds to the first MW raised amount for that year. $\sum_{1}^{2}\Delta MW$ denotes the sum of any post-two-year MW raises after the first initial raise. Total MW $\Delta$ amount corresponds to the total MW raised amount till the end of the sample. Start (End) is the MW at the start (end) of the sample. Control states are the set of control states for each treated states. These states never raised MW between $2012$ and $2017$. \# of border counties (T) is the number of counties in a treated state that border at least one county in a control state. And \# of border counties (C) is the number of counties in a control state that border at least one county in a treated state. * means not used in the POTWs sample.
    \end{minipage}
\end{table}

    Furthermore, I match each treated state to adjacent control states that never raised their MW between $2012$ and $2017$, and follow~\cite{dube2010minimum} and~\cite{gopalan2021state} to limit the sample to border counties in treated and control states~\footnote{\tiny Other recent papers to use this identification strategy include~\cite{aaronson2018industry},~\cite{dube2019fairness},~\cite{jardim2018minimum}, and~\cite{zhang2019distributional}. The control states in the sample include Iowa, Illinois, Indiana, Kansas, Kentucky, North Dakota, New Hampshire, Nevada, Oklahoma, Pennsylvania, Texas, Virginia, Wisconsin, and Wyoming. Importantly, Georgia, Idaho, New Mexico, North Carolina, and Utah are removed from the list of control states as they are not adjacent to any treated states.}. Table~\ref{tab:states-mw-adjustments-t-and-c} shows that there are $13$ treated and $14$ control states in the sample. The last two columns further show that there are a total of $272$ treated and $274$ control border counties. Figures~\ref{fig:border-state-map} and~\ref{fig:border-county-map} show the geographical locations of the treated and control states and counties, respectively.
    \begin{figure}[H]
    \centering
    \includegraphics[width=0.85\textwidth, height=0.4\textheight]{C:/Users/david/OneDrive/Documents/ULMS/PhD/Thesis/chapter3/src/climate_change/latex/fig_border_state_map}
    \caption{Map of Treated and Control States}
    \label{fig:border-state-map}
\end{figure}
    \begin{figure}[H]
    \centering
    \includegraphics[width=0.85\textwidth, height=0.4\textheight]{fig_border_county_map}
    \caption{Map of Treated and Control Counties}
    \label{fig:border-county-map}
\end{figure}
    Each treated border county is paired with a cross-border control county, described as the pair of the adjacent treated and control counties. The identifying assumption in a cross-border county pair is that the evolution of economic conditions for the pairs is symmetric, but the MW levels vary discontinuously at the border. To address the concern raised in~\cite{neumark2014revisiting} on the validity of border counties as counterfactuals, Figures~\ref{fig:county-level-macroeconomic-trends-in-border-counties} and ~\ref{fig:state-level-macroeconomic-trends-in-border-states} present a comparison of economic conditions before the first initial year of the MW change. Along most of the observable pre-treatment variables, the trends appear to be statistically parallel for the treated and control border counties and states.~\footnote{\tiny Tables~\ref{tab:descriptive-statistics-control-border-counties} and~\ref{tab:descriptive-statistics-control-border-states} of Appendix I presents a comparison of the means of the pre-treatment variables for the treated and control border counties. For most of the variables, the results show no substantial differences in their means for the treated and control border counties.}


    \section{The Data}\label{sec:data}
    For the empirical analysis, the novel data used come from five different sources and combined with the administrative facility level toxic release inventory (TRI) data from the environmental protection agency (EPA) for the United States.

    \subsection{Toxic Release Inventory Data}\label{subsec:toxic-release-inventory-data}
    I collect TRI-form-R data from EPA, which contains inventory of toxic chemical releases that are either manufactured, processed, otherwise used, and/or managed at private, state and federal industrial facilities across the US States. To file a TRI reporting form R, a facility must have at least ten full-time employees, and manufactures (including import) or processes more than $25,000$ pounds or otherwise uses more than $10,000$ pounds of a TRI-listed chemical during a calendar year. TRI data reflect, among other things, quantities of chemicals managed by facilities as waste, including those quantities released into the environment (as air emissions, water and land pollution), treated, burned for energy, recycled, and transferred from one facility to another for release or further management. It provides facility-level information based on $5$-digit zip codes identifying the exact location of the facility within each city, county, and states in the US.

    I collate onsite facility panel information as well as offsite transfers and publicly owned treatment works (POTWs) toxic chemical releases and waste management practices. There are $167$ different toxic chemical releases consistently reported by the same facilities from $2011-2017$ belonging to $213$ different NAICS manufacturing industries~\footnote{\tiny These manufacturing industries are categorized into: Beverage and tobacco product, chemical, computer and electronic product, food, forging and stamping, furniture and related products, household appliances, leather and allied products, machinery, miscelleneous, non-metallic mineral product, paper, petroleum and coal products, plastics and rubber products, primary metal, printing and related support activities, textile mills, textile product mills, transportation equipment, and wood product manufacturing.}.

    \subsection{Wage Data}\label{subsec:wage-data}
    I use the industry level wage data from the national bureau of economic research-centre for economic studies (NBER-CES). The NBER-CES data contains complete information on industry-level production workers and their wages, production workers' hours, and total payroll for manufacturing industries. Further, using this information I construct the industry-level production workers' wage per hour and wages per worker.

    \subsection{Other Industry and Macroeconomic Data}\label{subsec:other-industry-and-macroeconomic-data}
    The industry level data are from the NBER-CES and contains other variables including employment, number of total revenues, material costs, energy use, value added and total factor productivity, etc., for only manufacturing industries in the US. The macroeconomic data are both at the county and state level. The county-level data are sourced from the quarterly census of employment and wages (QCEW) of the US Bureau of Labour Statistics (BLS) and include variables such as the average number of establishments and industry ownership. Other county-level macroeconomic indicators include personal income, gross domestic product, and a state-level regional price indicator, all sourced from the bureau of economic analysis (BEA). The inflation data is got from the BLS consumer price index historical dataset for all urban consumers.

    \subsection{Joining the Datasets and the Sample}\label{subsec:joining-the-datasets-and-the-sample}
    I begin by joining the US zip-code and county-level geographic shapefiles (by zip-codes) to the TRI data to get the corresponding FIPS codes. These were then used to join the BEA, BLS and QCEW data by year, FIPS and $6$-digit NAICS codes. The resulting data were further merged to the NBER-CES dataset by their NAICS codes. Finally, this merged data was then joined with the prepared US geographic adjacent county shapefile that has information on county-level population and county distance to a state border. I prepared these shapefiles in the spirit of~\cite{dube2010minimum} and~\cite{gopalan2021state}, where each treated county or state is matched to at least one adjacent cross-border county or state, yielding cross-border county pairs. This final novel dataset is used for the empirical analysis of this paper. For the state level analysis, I aggregate the merged dataset to the state level. Table~\ref{variable-definitions} describes all the variables used in the analysis.

    From the merged dataset, I get a balanced panel sample subsets of onsite facilities, as well as their transfers to offsite and POTWs locations. The onsite data sample size is $1,893,689$ and consists of $1276$ manufacturing facilities belong to $213$ NAICS codes (in $20$ manufacturing industries) and a panel of $167$ toxic chemicals. Figure~\ref{fig:naics-manufacturing-industries} of Appendix III shows the distribution of these manufacturing industries, which reports that chemical, forging and stamping, primary metals, petroleum and coal products, transportation equipment, and machinery manufacturing industries are the most common in the sample. This is an administrative facility-level panel located in the same $5$-digit zip-codes in both the treated $(13)$ and adjacent control border $(14)$ states. A total of $27$ states and the number of cross-border counties are as described in section~\ref{subsec:selecting-the-treated-and-control-states}. The offsite and POTWs samples are subsets of the onsite sample. The offsite sample size is $1,179,754$ and POTWs sample is $308,943$. The original number of the same onsite facilities reduced to $680$ and $236$ in the offsite and POTWs samples, respectively. The number of toxic chemicals in the offsite and POTWs samples are $125$ and $69$, respectively. See Table~\ref{tab:analyzed-chemicals} of Appendix II for the list of analysed chemicals. The number of NAICS industries in the offsite and POTWs samples are $171$ and $101$. Finally, there are $14$ and $9$ control states in the offsite and POTWs samples; $13$ and $12$ treated states in the offsite and POTWs samples, respectively (See Table~\ref{tab:states-mw-adjustments-t-and-c}). Importantly, the number of onsite treated and control states remained unchanged in the offsite sample.

    A total of $827$ offsite facilities are located in $1056$ zip-codes in $573$ cities of $299$ counties in $45$ US states, whereas the $114$ POTWs sites are found in $202$ zip-codes in $123$ of $63$ US cities in $24$ US states.~\footnote{\tiny The offsite states include Alabama, Arkansas, Arizona, California, Colorado, Connecticut, Delaware, Florida, Georgia, Iowa, Idaho, Illinois, Indiana, Kansas, Kentucky, Louisiana, Massachussetts, Maryland, Maine, Michigan, Minnesota, Missouri, North Carolina, North Dakota, Nebraska, New Hampshire, New Jersey, New Mexico, Nevada, New York, Ohio, Okhlahoma, Oregon, Pennsylvania, Rhode Island, South Carolina, South Dakota,Tennesse, Texas, Utah, Virginia, Vermont, Wisconsin, and West Virginia. The POTWs states include Arkansas, California, District of Columbia, Delaware, Iowa, Illinois, Indiana, Massachussetts, Maryland, Maine, Michigan, Minnesota, North Dakota, Nebraska, New Hampshire, New Jersey, New York, Pennsylvania, Rhode Island, Texas, Virginia, Wisconsin, West Virginia, and Wyoming.}

    \subsection{Descriptive Statistics}\label{subsec:descriptive-statistics}
    The summary statistics are reported in Tables~\ref{tab:sumstat-onsite},~\ref{tab:sumstat-offsite} and~\ref{tab:sumstat-potws}. They show the facility average onsite releases, as well as those transferred offsite and to POTWs locations for further waste management.
    \begin{table}[H]
    \centering
    \caption{Summary Statistics (Onsite)}
    \label{tab:sumstat-onsite}
    \scalebox{0.5}{
%        \resizebox{\textwidth}{!}{
        \begin{tabular}{lrrrrr}
            \toprule \toprule
            Variable                                                & Obs     & Mean      & StdDev     & Min     & Max        \\ \midrule
            GDP per capita $(\$1000)$                               & 1893689 & 48.88     & 11.07      & 19.60   & 163.74     \\
            industry employment (1000's)                            & 1893689 & 44.99     & 42.29      & 2.90    & 365.80     \\
            annual average establishments                           & 1893689 & 5.44      & 12.38      & 0.00    & 330.00     \\
            population (county) (1000's)                            & 1893689 & 693432.18 & 1247538.81 & 1466.00 & 5194675.00 \\
            city region average consumer price index $(\$)$         & 1893689 & 235.46    & 6.47       & 224.94  & 245.12     \\
            federal facility                                        & 1893689 & 0.00      & 0.01       & 0.00    & 1.00       \\
            chemical ancillary use                                  & 1893689 & 0.25      & 0.43       & 0.00    & 1.00       \\
            chemical formulation component                          & 1893689 & 0.32      & 0.47       & 0.00    & 1.00       \\
            chemical manufacturing aid                              & 1893689 & 0.11      & 0.31       & 0.00    & 1.00       \\
            max number of chemicals at facility                     & 1893689 & 3.89      & 1.43       & 1.0     & 19.00      \\
            imported chemicals at facility                          & 1893689 & 0.07      & 0.25       & 0.0     & 1.00       \\
            produced chemicals at facility                          & 1893689 & 0.24      & 0.42       & 0.0     & 1.00       \\
            production ratio or activity index                      & 1893689 & 1.59      & 178.58     & 0.0     & 117229.00  \\
            total releases intensity (lbs)                          & 1893689 & 87.99     & 1065.44    & 0.00    & 122005.98  \\
            total air emissions intensity (lbs)                     & 1893689 & 60.02     & 616.84     & 0.00    & 40743.89   \\
            total fugitive air emissions intensity (lbs)            & 1893689 & 10.95     & 163.84     & 0.00    & 21484.45   \\
            total point air emissions intensity (lbs)               & 1893689 & 49.07     & 537.76     & 0.00    & 31559.41   \\
            total land releases intensity (lbs)                     & 1893689 & 7.93      & 701.61     & 0.00    & 122005.98  \\
            total underground injection intensity (lbs)             & 1893689 & 4.80      & 697.74     & 0.00    & 122005.98  \\
            total landfills intensity (lbs)                         & 1893689 & 1.43      & 53.40      & 0.00    & 6892.31    \\
            total releases to-land treatment intensity (lbs)        & 1893689 & 0.66      & 34.58      & 0.00    & 6006.01    \\
            total surface impoundment intensity (lbs)               & 1893689 & 0.03      & 2.40       & 0.00    & 929.15     \\
            total land releases intensity, others (lbs)             & 1893689 & 1.01      & 30.47      & 0.00    & 2299.05    \\
            total surface water discharge intensity (lbs)           & 1893689 & 20.04     & 475.39     & 0.00    & 41422.43   \\
            total number of receiving streams, onsite (lbs)         & 1893689 & 0.39      & 0.50       & 0.00    & 4.00       \\
            total release intensity, from catastrophic events (lbs) & 1893689 & 4.36      & 249.63     & 0.00   & 42103.29  \\
            total industry payroll $(\$1m)$                         & 1893689 & 2962.68   & 2630.57    & 127.00  & 16647.90   \\
            production workers (1000's)                             & 1893689 & 31.42     & 31.71      & 1.40    & 280.60     \\
            production hours (1m)                                   & 1893689 & 64.82     & 63.90      & 3.10    & 561.50     \\
            production workers' wages per hour                      & 1893689 & 26.56     & 7.35       & 12.24   & 54.35      \\
            cost of materials $(\$1m)$                              & 1893689 & 61328.05  & 162337.03  & 271.20  & 690771.20  \\
            industry value added (output) $(\$100m)$                & 1893689 & 177.13    & 272.06     & 3.00    & 1180.37    \\
            output per hour ($\$1000$)                              & 1893689 & 2.62      & 2.82       & 0.44    & 34.28      \\
            output per worker ($\$1000$)                            & 1893689 & 3.65      & 3.99       & 0.63    & 44.31      \\ \bottomrule\bottomrule
        \end{tabular}
%        }
    }
\end{table}

    \begin{table}[H]
    \centering
    \caption{Summary Statistics (Offsite)}
    \label{tab:sumstat-offsite}
    \resizebox{\textwidth}{!}{
    \begin{tabular}{lrrrrr}
        \toprule \toprule
        Variable (lbs)                               & Obs     & Mean   & StdDev  & Min & Max       \\ \midrule
        total releases intensity                     & 1179754 & 257.85 & 2453.53 & 0   & 125639.48 \\
        total land releases intensity                & 1179754 & 196.20 & 2037.73 & 0   & 125639.48 \\
        total land releases other intensity          & 1179754 & 1.24   & 22.82   & 0   & 1637.57   \\
        total landfills intensity                    & 1179754 & 172.99 & 1958.09 & 0   & 125639.48 \\
        total surface impoundment intensity          & 1179754 & 0.63   & 58.77   & 0   & 7517.19   \\
        total underground injection intensity        & 1179754 & 18.44  & 557.48  & 0   & 59894.46  \\
        total wastewater releases intensity          & 1179754 & 6.00   & 125.39  & 0   & 15101.70  \\
        total releases (metal solidify) intensity    & 1179754 & 61.22  & 1507.92 & 0   & 84868.80  \\
        total releases (storage) intensity           & 1179754 & 0.87   & 34.06   & 0   & 6755.87   \\
        total releases (other mgt) intensity         & 1179754 & 5.18   & 99.26   & 0   & 11850.66  \\
        total releases (to-land) treatment intensity & 1179754 & 2.91   & 90.79   & 0   & 7875.12   \\
        total releases (unknown) intensity           & 1179754 & 6.30   & 59.87   & 0   & 3610.69   \\
        total releases (waste broker) intensity      & 1179754 & 8.28   & 115.33  & 0   & 6738.71   \\ \bottomrule\bottomrule
    \end{tabular}
    }
\end{table}

    \begin{table}[H]
    \centering
    \caption{Summary Statistics (POTWs)}
    \label{tab:sumstat-potws}
    \begin{tabular}{lrrrrr}
        \toprule\toprule
        Variable (lbs)                         & Obs    & Mean  & StdDev & Min & Max      \\ \midrule
        total releases intensity               & 308943 & 17.87 & 404.45 & 0   & 27648.29 \\
        underground releases intensity         & 308943 & 6.69  & 288.78 & 0   & 27648.29 \\
        underground releases intensity (other) & 308943 & 11.18 & 253.74 & 0   & 26548.32 \\ \bottomrule \bottomrule
    \end{tabular}
\end{table}

    \begin{table}[H]
    \centering
    \caption{Summary Statistics of Onsite Mechanisms}
    \label{tab:sumstat-onsite-mechanisms}
%    \scalebox{0.8}{
    \resizebox{\textwidth}{!}{
        \begin{tabular}{lrrrrr}
            \toprule\toprule
            Variable                                          & Obs     & Mean     & SD        & Min & Max      \\ \midrule
%            total waste management                            & 1893689 & 83667.25 & 894161.45 & 0   & 45000000 \\
            biological treatment                              & 1893689 & 0.06     & 0.24      & 0   & 1        \\
            physical treatment                                & 1893689 & 0.18     & 0.38      & 0   & 1        \\
            incineration or thermal treatment                 & 1893689 & 0.13     & 0.34      & 0   & 1        \\
%            industrial boiler energy recovery method          & 1893689 & 0.01     & 0.11      & 0   & 1        \\
            recycling quantity                                & 1893689 & 22386.84 & 427110.33 & 0   & 44938800 \\
            recycling to reuse in production process          & 1893689 & 0.04     & 0.19      & 0   & 1        \\
            source reduction activities                       & 1893689 & 0.25     & 0.43      & 0   & 1        \\
            chemical purity modification                      & 1893689 & 0.00     & 0.03      & 0   & 1        \\
            clean fuel substitution                           & 1893689 & 0.01     & 0.12      & 0   & 1        \\
            organic solvent substitution                      & 1893689 & 0.00     & 0.02      & 0   & 1        \\
            new technology or technique in production process & 1893689 & 0.00     & 0.06      & 0   & 1        \\
            recirculation in production process               & 1893689 & 0.00     & 0.04      & 0   & 1        \\
            recycling in production process                   & 1893689 & 0.03     & 0.16      & 0   & 1        \\
            product quality analysis                          & 1893689 & 0.00     & 0.02      & 0   & 1        \\
            operating practices training                      & 1893689 & 0.00     & 0.03      & 0   & 1        \\
            changing size of storage containers               & 1893689 & 0.00     & 0.02      & 0   & 1        \\
            improved material handling                        & 1893689 & 0.00     & 0.02      & 0   & 1        \\ \bottomrule
        \end{tabular}
    }
%    }
\end{table}


    \section{Industry Costs, Employment and Outputs}\label{sec:industry-costs-employment-and-outputs}
    The empirical analysis begins in this section by examining the effect of MW on industry costs (labour and materials), employment and outputs.~\footnote{\tiny To rule out any possible treatment selection, I estimate this equation at both the county and state level: $Treated_{cp,s,t}^e = \beta Z_{f,i,cp,s,t} + \lambda_{t} + \phi_{cp} + \delta_{s} + \zeta_{cp,t} + \epsilon_{cp,s,t}$. Where $Treated_{cp,s,t}^e = 1[t - G_{cp,s,t}]$ denotes treated states that are $e$-periods away from the initial treatment date, and $G_{cp,s,t}$ is the vector of initial treatment dates. $Z_{f,i,v,cp,s,t}$ is the vector of facility and industry, city, county-pair and state-level covariates, and $\beta$ is the vector of coefficients. Albeit, the year fixed effects, $\lambda_{t}$, nets out any inflationary effects, city-region inflation is explicitly controlled for in the model. Cross-border county pair, $\phi_{cp}$, and state, $\phi_{cp}$, fixed effects are controlled for to account for within cross-border county pair and state differences that may affect the MW policy. Finally, I control for cross-border county pair linear trends, $\zeta_{cp,t}$, to account for the evolution of the MW policy in paired cross-border counties. The result (reported in Table~\ref{tab:treatment-selection}) showed no significant treatment selection effects in the following covariates: lagged values of county-level gross domestic product (GDP), GDP per capita, annual average establishments, and population; inflation; and industry level plant and equipment capital costs. Others include industry ownership binary variables (federal, state and private).}

    \subsection{Baseline Results: Industry Costs}\label{subsec:baseline-results-industry-costs}
    In what follows, I estimate wage, employment and output responses of manufacturing industry employees in the baseline. The baseline model is given by:
    \begin{equation}
        C_{i,cp,s,t} = \beta Treated_{s,t}^e + \delta X_{v,c,t-1} + \omega P_{f,t} + \lambda_{t} + \phi_{cp} + \sigma_{s} + \zeta_{cp,t} + \epsilon_{i,cp,s,t},\label{eq:baseline-wages}
    \end{equation}
    where $C_{i,cp,s,t}$ is the vector of industry costs (hourly wages, wages per worker, total payroll, total production workers wage and material costs) of manufacturing industry, $i$ in cross-border county pairs, $cp$ for the state, $s$ in the year, $t$. $Treated_{s,t}^e = \textbf{1}[t - G_{s,t}]$ is unity for the treated states that are $e$-periods away from the vector of initial treatment dates, $G_{s,t}$ and zero for the control states. $X_{v,c,t-1}$ denotes lagged values of county-level GDP per capita, annual average establishments, population and city-region inflation~\parencite{gopalan2021state, dube2010minimum, clemens2019making}. $P_{f,t}$ contains facility-level dummies on industry ownership. I control for year fixed effects, $\lambda_{t}$ to account for time varying differences in the MW policy as well as trending inflation. Cross-border county pair, $\phi_{cp}$ and state, $\sigma_{s}$ fixed effects are controlled for to account for within county pair and state differences that may affect the MW policy such as within county/state industry compositions and political climate. Finally, $\zeta_{cp,t}$ is the cross-border county pair linear trends to control for the evolution of common shocks in cross-border county pairs. Standard errors are clustered at the state level as there are possibilities that changes in MW may be correlated within a state.
    % Please add the following required packages to your document preamble:
% \usepackage{booktabs}
\begin{table}[H]
    \centering
    \caption{Effect of the MW Policy on Industry Costs}
    \label{tab:baseline-industry-costs}
    \begin{tabular}{@{}lllllll@{}}
        \toprule\toprule
        Industry costs & \multicolumn{2}{c}{Hourly wage} & \multicolumn{2}{c}{Total payroll} & \multicolumn{2}{c}{Material cost (log)} \\
        \cmidrule(lr){2-3}\cmidrule(lr){4-5}\cmidrule(lr){6-7}
        & 1         & 2         & 3         & 4         & 5         & 6         \\ \midrule
        treated           & 0.9382**  & 0.6065*   & 193.7**   & 122.9     & 0.0790    & 0.1054**  \\
        & (0.3768)  & (0.3247)  & (80.44)   & (112.6)   & (0.0827)  & (0.0474)  \\
        controls          & Yes       & Yes       & Yes       & Yes       & Yes       & Yes       \\
        year FE           & Yes       & Yes       & Yes       & Yes       & Yes       & Yes       \\
        county FE         & Yes       & Yes       & Yes       & Yes       & Yes       & Yes       \\
        border-county FE  & No        & Yes       & No        & Yes       & No        & Yes       \\
        border-county LTs & No        & Yes       & No        & Yes       & No        & Yes       \\ \midrule
        Observations      & 1,893,689 & 1,893,689 & 1,893,689 & 1,893,689 & 1,893,689 & 1,893,689 \\
        $R^2$             & 0.5380    & 0.6242    & 0.3528    & 0.4127    & 0.5243    & 0.6187    \\
        Baseline Mean     & 26.56     & 26.56     & 2962.68   & 2962.68   & 61328.05  & 61328.05  \\ \bottomrule \bottomrule
    \end{tabular}
    \begin{minipage}{15.5cm}
        \vspace{0.05in}
        These results are obtained from estimating model~\ref{eq:baseline-wages}. Robust standard errors clustered at the state level are reported in parentheses. ***, **, and * denote significance levels at the less than $1\%$, $5\%$ and $10\%$, respectively. de Chaisemartin and D'Haultfoeuille Decomposition: $\sum dCDH_{ATTs}^{weights(+)} = 1$ and $\sum dCDH_{ATTs}^{weights(-)} = 0$.
    \end{minipage}
\end{table}

    The average treatment effect on the treated (ATT) is captured by $\beta$, which is the difference in the average effect of raising the MW floor on manufacturing industry costs in treated counties relative to adjacent control counties. The results on industry costs are reported in Table~\ref{tab:baseline-industry-costs}. I find that a higher MW policy increased industry labour costs in treated counties relative to adjacent control counties. Manufacturing industry wages per hour rose by $\$1.06$ while wages per worker are twice as large, $\$2.35$. This documented result on manufacturing industry wages per hour is larger than that documented in~\cite{gopalan2021state} for the industry-wide effect on hourly wages of $\$0.48$. This suggests that the manufacturing industry is strongly affected by the MW policy relative to the rest of the industries in the US. The effect size on wages per industry production worker suggests that some of them may receive excess amounts required to comply with the new MW floor. There are no statistically significant differences between the treated and control counties in terms of material costs, total payroll, and total production workers wage.
    \input{fig_did_wages_perhr_worker}

    Figure~\ref{fig:baseline-hourly-wages-and-wages-per-worker} reports the dynamic effects. It reports instantaneous increases in hourly wages and wages per worker and persists even three years after the treatment, in $2015-2016$. Moreover, I find no statistically significant evidence of pre-trends. The timing and size of the effect are consistent with my data and setting. Similarly, Figure~\ref{fig:baseline-total-production-worker-wages} reports evidence of an increase in industry payroll in the second year following an MW policy. Again, no effects are recorded for material cost and total production workers wages.
    \begin{figure}[H]
    \centering
    \includegraphics[width=1\textwidth, keepaspectratio]{fig_did_total_prod_wages}
    \caption{Total Production Workers' Wages, Payroll and Materials Cost}
    \label{fig:baseline-total-production-worker-wages}
    \begin{minipage}{12cm}
        \vspace{0.05in}
        NOTES: The event study model of equation~\ref{eq:baseline-wages} is $C_{i,cp,s,t} = \sum_{{e = 2011},{e \neq 2013}}^{2017} \beta Treated_{s,t}^e = \textbf{1}[t - G_{s,t}] + \delta X_{v,c,t-1} + \omega P_{f,t} + \lambda_{t} + \phi_{cp} + \sigma_{s} + \zeta_{cp,t} + \epsilon_{i,cp,s,t}$. Standard errors are clustered at the state level.
    \end{minipage}
\end{figure}

    \subsection{Baseline Results: Employment}\label{subsec:baseline-results-employment}
    This subsection estimates the effect of raising MW on employment of manufacturing industry workers and their production hours. The model is given by:
    \begin{equation}
        E_{i,cp,s,t} = \beta Treated_{s,t}^e + \delta X_{v,c,t-1} + \omega P_{f,t} + \lambda_{t} + \phi_{cp} + \sigma_{s} + \zeta_{cp,t} + \epsilon_{i,cp,s,t},\label{eq:baseline-emp-hours}
    \end{equation}
    where $E_{i,cp,s,t}$ is the vector of employment and production workers hours of manufacturing industry, $i$ in cross-border county pairs, $cp$ for the state, $s$ in the year, $t$. Standard errors are clustered at the state level.
    % Please add the following required packages to your document preamble:
% \usepackage{booktabs}
% \usepackage{graphicx}
\begin{table}[H]
    \centering
    \caption{Effect of the MW Policy on Employment and Production Workers' Hours}
    \label{tab:baseline-employment-hours}
    \resizebox{\columnwidth}{!}{%
        \begin{tabular}{@{}lll@{}}
            \toprule\toprule
            employment \& hours & employment (log) & production hours (log) \\ \midrule
            treated             & 0.0008           & -0.0111                \\
            & (0.0376)         & (0.0321)               \\
            controls            & Yes              & Yes                    \\
            year FE             & Yes              & Yes                    \\
            county-pair FE      & Yes              & Yes                    \\
            state FE            & Yes              & Yes                    \\
            county-pair LTs     & Yes              & Yes                    \\ \midrule \midrule
            Observations        & 1,893,689        & 1,893,689              \\
            $R^2$               & 0.3253           & 0.3158                 \\
            Baseline Mean       & 44.99            & 64.82                  \\ \bottomrule\bottomrule
        \end{tabular}%
    }
    \begin{minipage}{18cm}
        \vspace{0.05in}
        These results are obtained from estimating model~\ref{eq:baseline-emp-hours}. Robust standard errors clustered at the state level are reported in parentheses. ***, **, and * denote significance levels at the less than $1\%$, $5\%$ and $10\%$, respectively.
    \end{minipage}
\end{table}

    The average treatment effect on the treated (ATT) is captured by $\beta$, which is the difference in the average effect of raising the MW floor on manufacturing industry employment and hours of production workers in treated counties relative to adjacent control counties. The results are reported in Table~\ref{tab:baseline-employment-hours}. The results show no significant changes in the manufacturing industry employment and production workers hours following an MW policy in the treated counties relative to adjacent control counties. Particularly, the size of the effect suggests a sharp null effect of the MW policy on manufacturing industry employment. On the other hand, the decline in production workers hours is not statistically different between the treated and adjacent control counties. The results are consistent with the labour market literature assuming monopsonistic competition~\parencite{card2000minimum, aaronson2018industry, cengiz2019effect, wong2019minimum, dustmann2022reallocation}. Similar results are documented in the dynamic treatment effects in Figure~\ref{fig:baseline-employment-hours}. There is no evidence of pre-trends.
    \begin{figure}[H]
    \centering
    \includegraphics[width=1\textwidth, keepaspectratio]{fig_did_emp_hours}
    \caption{Industry Employment and Production Workers Hours}
    \label{fig:baseline-employment-hours}
    \begin{minipage}{18cm}
        \vspace{0.05in}
        NOTES: The event study model of equation~\ref{eq:baseline-emp-hours} is $E_{i,cp,t} = \sum_{{e = -3},{e \neq -1}}^{3} \beta Treated_{s,t}^e = \textbf{1}[t - G_{s,t}] + \delta X_{v,c,t-1} + \omega F_{f,t} + \lambda_{t} + \sigma_{c} + \phi_{cp} + \zeta_{cp,t} + \epsilon_{i,cp,t}$. Standard errors are clustered at the state level.
    \end{minipage}
\end{figure}

    \subsection{Baseline Results: Industrial Output}\label{subsec:baseline-results-industrial-output}
    This subsection estimates the effect of raising MW on outputs of the manufacturing industry. The model is given by:
    \begin{equation}
        Y_{i,cp,s,t} = \beta Treated_{s,t}^e + \delta X_{v,c,t-1} + \omega P_{f,t} + \lambda_{t} + \phi_{cp} + \sigma_{s} + \zeta_{cp,t} + \epsilon_{i,cp,s,t},\label{eq:baseline-output}
    \end{equation}
    where $Y_{i,cp,s,t}$ is the vector of manufacturing industry output, output per hour and output per worker, in manufacturing industry, $i$ in cross-border county pairs, $cp$ for the state, $s$ in the year, $t$. Standard errors are clustered at the state level.
    % Please add the following required packages to your document preamble:
% \usepackage{booktabs}
% \usepackage{graphicx}
\begin{table}[H]
    \centering
    \caption{Effect of the MW policy on Manufacturing Industry Output}
    \label{tab:baseline-industry-output}
    \resizebox{\columnwidth}{!}{%
        \begin{tabular}{@{}lllllll@{}}
            \toprule\toprule
            Industry outputs (log) & \multicolumn{2}{c}{Output} & \multicolumn{2}{c}{Output per Hour} & \multicolumn{2}{c}{Output per Worker} \\
            \cmidrule(lr){2-3} \cmidrule(lr){4-5} \cmidrule(lr){6-7}
            & 1         & 2         & 3         & 4         & 5         & 6         \\ \midrule
            treated           & 0.0362    & 0.1175*** & 0.0825    & 0.1322*** & 0.0669    & 0.1170*** \\
            & (0.0731)  & (0.0366)  & (0.0844)  & (0.0296)  & (0.0802)  & (0.0246)  \\
            controls          & Yes       & Yes       & Yes       & Yes       & Yes       & Yes       \\
            year FE           & Yes       & Yes       & Yes       & Yes       & Yes       & Yes       \\
            county FE         & Yes       & Yes       & Yes       & Yes       & Yes       & Yes       \\
            border-county FE  & No        & Yes       & No        & Yes       & No        & Yes       \\
            border-county LTs & No        & Yes       & No        & Yes       & No        & Yes       \\ \midrule
            Observations      & 1,893,689 & 1,893,689 & 1,893,689 & 1,893,689 & 1,893,689 & 1,893,689 \\
            $R^2$             & 0.4678    & 0.5485    & 0.5587    & 0.6296    & 0.5571    & 0.6481    \\
            Baseline Mean     & 177.13    & 177.13    & 2.62      & 2.62      & 3.65      & 3.65      \\ \bottomrule\bottomrule
        \end{tabular}%
    }
    \begin{minipage}{18cm}
        \vspace{0.05in}
        These results are obtained from estimating model~\ref{eq:baseline-output}. Robust standard errors clustered at the state level are reported in parentheses. ***, **, and * denote significance levels at the less than $1\%$, $5\%$ and $10\%$, respectively. de Chaisemartin and D'Haultfoeuille Decomposition: $\sum dCDH_{ATTs}^{weights(+)} = 1$ and $\sum dCDH_{ATTs}^{weights(-)} = 0$.
    \end{minipage}
\end{table}

    The average treatment effect on the treated (ATT) is captured by $\beta$, which is the difference in the average effect of raising the MW floor on manufacturing industry output, output per hour and output per worker in treated counties relative to adjacent control counties. The results are reported in Table~\ref{tab:baseline-industry-output}. Following an MW policy, I document large statistically significant increases in manufacturing industry outputs by $12$ percentage points (ppts) in treated counties relative to adjacent control counties. Specifically, the output per hour and per worker rose by $13$ and $12$ (ppts), respectively.

    Figure~\ref{fig:baseline-industry-output} records consistent results. The MW policy caused an instantaneous significant increases in manufacturing industry outputs, output per hour and output per worker in treated counties relative to adjacent control counties. The effects on output per hour and output per worker persists throughout the spectrum. Importantly, there is no evidence of significant pre-trends.
    \begin{figure}[H]
    \centering
    \includegraphics[width=1\textwidth, keepaspectratio]{fig_did_output}
    \caption{Manufacturing Industry Output: Output per Hour and Output per Worker}
    \label{fig:baseline-industry-output}
    \begin{minipage}{12cm}
        \vspace{0.05in}
        NOTES: The event study model of equation~\ref{eq:baseline-wages} is $Y_{i,cp,s,t} = \sum_{{e = 2011},{e \neq 2013}}^{2017} \beta Treated_{s,t}^e = \textbf{1}[t - G_{s,t}] + \delta X_{v,c,t-1} + \omega P_{f,t} + \lambda_{t} + \phi_{cp} + \sigma_{s} + \zeta_{cp,t} + \epsilon_{i,cp,s,t}$. Standard errors are clustered at the state level.
    \end{minipage}
\end{figure}

    \subsection{Baseline Robustness}\label{subsec:baseline-robustness}
    I conduct several robustness exercises to supplement industry results presented above.

    \paragraph{Standard Errors:} Standard errors are clustered at the facility, zipcode, industry NAICS codes, and county levels. I document that the results are not sensitive to these alternative clustering. See Tables~\ref{tab:baseline-wage-ses},~\ref{tab:baseline-employ-ses} and~\ref{tab:baseline-industry-output-ses}.

    \paragraph{Cross-County Mobility:}~\cite{neumark2019econometrics} argues that cross-border studies may be biased against detecting disemployment effects due to worker mobility spillovers. Hence, I test whether my results are driven by cross-county mobility (i.e., a violation of the stable unit treatment assumption), using the interaction of the difference-in-differences coefficient and the distance between population centres. The underlying assumption here is that worker mobility to a state with higher MW declines if the geographic distance to the state increases. Table~\ref{tab:baseline-mobility} reports that worker mobility does not drive the baseline results.

    \paragraph{State-level Results:} I repeat all the above analysis at the state-level. The results persist even at this level, and are presented in the online supplementary material.

    I have established that raising MW indeed increases industrial labour costs and output. In what follows, I examine the questions: are the burden of higher labour costs passed to the environment? And, does rising output imply higher pollution emissions per unit of output?


    \section{Onsite Toxic Releases}\label{sec:onsite-toxic-releases}
    I begin by estimating the effect of raising MW on total onsite toxic releases intensity generated at manufacturing industry facilities. The baseline model is given by:
    \begin{equation}
        P_{f,c,i,cp,s,t} = \beta Treated_{s,t}^e + \delta X_{v,c,t-1} + \omega F_{f,t} + \gamma_{f} + \phi_{cp} + \eta_{c,t} + \left[\lambda_{t} + \theta_{h} + \sigma_{s} + \zeta_{c} \right] + \varepsilon_{f,c,i,cp,s,t},\label{eq:baseline-total-onsite-releases-intensity}
    \end{equation}
    where $P_{f,c,i,cp,s,t}$ is the total onsite releases intensity at manufacturing facility, $f$ through toxic chemical use, $c$ in industry, $i$ in cross-border county pairs, $cp$ in state, $s$ in the year, $t$. Total onsite releases intensity is the sum of weights of total onsite toxic air emissions, land releases, and surface water discharges. $Treated_{s,t}^e = \textbf{1}[t - G_{s,t}]$ and $X_{v,c,t-1}$ are as defined in subsection~\ref{subsec:baseline-results-industry-costs}. $P_{f,t}$ contains facility-level variables regarding the maximum number of toxic chemicals at the facility at any point in time, production/activity ratio index of the facility, and the following dummies: industry ownership, and toxic chemical attributes, that is, whether the chemical was produced at or imported to the facility, used as a formulation or article component, used as a manufacturing aid or for ancillary purposes. I control for facility-level fixed effect, $\gamma_{f}$ to account for within facility-level differences in the management of toxic releases at the facility due to the MW policy; cross-border county pairs, $\phi_{cp}$ fixed effect as in subsection~\ref{subsec:baseline-results-industry-costs}; and $\eta_{c,t}$ toxic chemical linear trends which control for the evolution of common time varying shocks that may affect the nature of chemical usage at manufacturing facilities. In my tightest estimation, I further control for year fixed effect, $\lambda_{t}$ and state, $\sigma_{s}$ fixed effects as in subsection~\ref{subsec:baseline-results-industry-costs}. Additionally,  $\theta_{h}$ is the FIPs and county-level fixed effects to control for within county differences such as between county geographic distance, natural resources or historical pollution patterns. And $\zeta_{c}$ is the toxic chemical-fixed effect that controls for within chemical usage or mixture or compound differences in manufacturing industries production functions given the MW policy. $\varepsilon_{f,c,i,cp,s,t}$ is the idiosyncratic error term. Finally, I employ a three-way clustering of the standard errors at the chemical use, industry, and state levels, as there are possibilities that changes in MW may be correlated within a state, and that the response to this change may also be correlated within manufacturing industries in the state and the nature of toxic chemical usage in the industry.
    % Please add the following required packages to your document preamble:
% \usepackage{booktabs}
% \usepackage{graphicx}
\begin{table}[H]
    \centering
    \caption{Effect of the MW policy on Total Onsite Toxic Releases Intensity}
    \label{tab:baseline-total-onsite-releases-intensity}
    \resizebox{\columnwidth}{!}{%
        \begin{tabular}{@{}llll@{}}
            \toprule\toprule
            Total releases intensity (log) & 1         & 2         & 3         \\ \midrule
            treated                        & 0.1329*** & 0.1329*** & 0.1242**  \\
            & (0.0413)  & (0.0413)  & (0.0499)  \\
            controls                       & Yes       & Yes       & Yes       \\
            year FE                        & Yes       & Yes       & Yes       \\
            facility FE                    & Yes       & Yes       & Yes       \\
            border-county FE               & No        & Yes       & Yes       \\
            toxic chemical FE              & No        & No        & Yes       \\
            toxic chemical LTs             & No        & No        & Yes       \\\midrule
            Observations                   & 1,893,689 & 1,893,689 & 1,893,689 \\
            $R^2$                          & 0.5199    & 0.5199    & 0.7194    \\
            Baseline Mean                  & 87.99     & 87.99     & 87.99     \\ \bottomrule\bottomrule
        \end{tabular}%
    }
    \begin{minipage}{18cm}
        \vspace{0.05in}
        These results are obtained from estimating model~\ref{eq:baseline-total-onsite-releases-intensity}. Three-way clustered robust standard errors are reported in parentheses, and clustered at the toxic chemical, industry and state levels. ***, **, and * denote significance levels at the less than $1\%$, $5\%$ and $10\%$, respectively. de Chaisemartin and D'Haultfoeuille Decomposition: $\sum dCDH_{ATTs}^{weights(+)} = 1$ and $\sum dCDH_{ATTs}^{weights(-)} = 0$.
    \end{minipage}
\end{table}

    The average treatment effect on the treated (ATT) is captured by $\beta$, which is the difference in the average effect of raising the MW floor on total onsite releases intensity at manufacturing facilities in treated counties relative to adjacent control counties. The results on total onsite releases are reported in Table~\ref{tab:baseline-total-onsite-releases-intensity}. The results show that a higher MW policy increased total onsite releases per unit of manufacturing industry output by $16.9ppts$ in treated counties relative to adjacent control counties. This translates to a $67.5lbs$ additional increase in total onsite releases intensity for every $\$1.06/hr$ increase in the wages of manufacturing industry workers.~\footnote{\tiny Alternatively, this means that for every $\$2.35$ increase in the wage of a manufacturing industry worker, total onsite releases intensity increases by $67.5lbs$.}
    \begin{figure}[H]
    \centering
    \includegraphics[width=1\textwidth, height=0.5\textheight,keepaspectratio]{fig_did_total_releases_onsite_int}
    \caption{Total Onsite Releases Intensity}
    \label{fig:baseline-total-onsite-releases-intensity}
    \begin{minipage}{12cm}
        \vspace{0.05in}
        NOTES: The event study model of equation~\ref{eq:baseline-total-onsite-releases-intensity} is $P_{f,c,i,cp,s,t} = \sum_{{e = 2011},{e \neq 2013}}^{2017} \beta Treated_{s,t}^e + \delta X_{v,c,t-1} + \omega F_{f,t} + \gamma_{f} + \phi_{cp} + \eta_{c,t} + \left[\lambda_{t} + \theta_{f,h} + \sigma_{s} + \zeta_{c} \right] + \varepsilon_{f,c,i,cp,s,t}$. Three-way clustered robust standard errors are reported in parentheses, and clustered at the toxic chemical, industry and state levels. The test for the presence of pre-trends shows and F-statistic of $0.8466$ $(0.6549)$, p-value (in parentheses) using F-Statistic $(\chi^2): \sum_{2011}^{2012} \beta_{e} = 0$.
    \end{minipage}
\end{figure}

    Moreover, Figure~\ref{fig:baseline-total-onsite-releases-intensity} shows an instantaneous increase in total onsite releases, reaching $26.6ppts$ three years after the initial treatment date for treated counties relative to adjacent control counties. There is no significant evidence of pre-trends.

    \subsection{Air Emissions Intensity}\label{subsec:air-emission-intensity}
    Here, I estimate the effect of raising MW on onsite air emission intensity. The model is given by:
    \begin{equation}
        A_{f,c,i,cp,s,t} = \beta Treated_{s,t}^e + \delta X_{v,c,t-1} + \omega F_{f,t} + \gamma_{f} + \phi_{cp} + \eta_{c,t} + \left[\lambda_{t} + \theta_{h} + \sigma_{s} + \zeta_{c} \right] + \varepsilon_{f,c,i,cp,s,t},\label{eq:baseline-onsite-air-emission-intensity}
    \end{equation}
    where $A_{f,c,i,cp,s,t}$ is the total onsite air emission intensity from manufacturing facility, $f$ through toxic chemical, $c$ used in industry, $i$ in cross-border county pairs, $cp$ in state, $s$ in the year, $t$. Total onsite air emission intensity is the sum of point/stack air emissions or fugitive air emissions. Point air emissions usually involve the release of pollutants or substances into the atmosphere from industrial or commercial sources through a designated stack, chimney, or venting mechanism. These emissions occur as a result of combustion processes, industrial operations, or other activities that involve the burning, processing, or handling of materials. Stack emissions are typically regulated and monitored to ensure compliance with environmental regulations and standards. Fugitive air emissions on the other hand, refer to the release of pollutants or substances into the atmosphere from various industrial, commercial, or other sources that are not captured by a stack, duct, or other venting mechanism. These emissions typically occur during the handling, storage, processing, or transportation of materials and can originate from leaks, spills, evaporation, or other unintended releases. Examples of emitted pollutants include volatile organic compounds, hazardous air pollutants (HAPs), and particulate matter, \textit{inter alia}. Monitoring and controlling stack emissions are critical for minimizing air pollution, protecting public health, and reducing environmental impacts.
    % Please add the following required packages to your document preamble:
% \usepackage{booktabs}
% \usepackage{graphicx}
\begin{table}[H]
    \centering
    \caption{Effect of the MW policy on Air Emissions Intensity}
    \label{tab:baseline-onsite-air-emissions-intensity}
    \resizebox{\columnwidth}{!}{%
        \begin{tabular}{@{}lllllll@{}}
            \toprule\toprule
            & \multicolumn{2}{c}{Total} & \multicolumn{2}{c}{Point} & \multicolumn{2}{c}{Fugitive} \\
            \cmidrule(lr){2-3}  \cmidrule(lr){4-5} \cmidrule(lr){6-7}
            Air emissions intensity (log) & 1         & 2         & 3         & 4         & 5         & 6         \\ \midrule
            treated                       & 0.0938*   & 0.0938*   & 0.0157    & 0.0157    & 0.0852*   & 0.0852*   \\
            & (0.0533)  & (0.0533)  & (0.0348)  & (0.0348)  & (0.0510)  & (0.0510)  \\
            controls                      & Yes       & Yes       & Yes       & Yes       & Yes       & Yes       \\
            year FE                       & No        & Yes       & No        & Yes       & No        & Yes       \\
            facility FE                   & Yes       & Yes       & Yes       & Yes       & Yes       & Yes       \\
            county FE                     & No        & Yes       & No        & Yes       & No        & Yes       \\
            county-pair FE                & Yes       & Yes       & Yes       & Yes       & Yes       & Yes       \\
            facility state FE             & No        & Yes       & No        & Yes       & No        & Yes       \\
            toxic chemical FE             & No        & Yes       & No        & Yes       & No        & Yes       \\
            toxic chemical LTs            & Yes       & Yes       & Yes       & Yes       & Yes       & Yes       \\ \midrule \midrule
            Observations                  & 1,893,689 & 1,893,689 & 1,893,689 & 1,893,689 & 1,893,689 & 1,893,689 \\
            $R^2$                         & 0.7675    & 0.7675    & 0.7430    & 0.7430    & 0.6969    & 0.6969    \\
            Baseline Mean                 & 272.50    & 272.50    & 222.80    & 222.80    & 49.70     & 49.70     \\ \bottomrule\bottomrule
        \end{tabular}%
    }
    \begin{minipage}{18cm}
        \vspace{0.05in}
        These results are obtained from estimating model~\ref{eq:baseline-onsite-air-emission-intensity}. Three-way clustered robust standard errors are reported in parentheses, and clustered at the toxic chemical, industry and state levels. ***, **, and * denote significance levels at the less than $1\%$, $5\%$ and $10\%$, respectively.
    \end{minipage}
\end{table}

    The parameter of interest is $\beta$ which measures the ATT of the MW policy on air emissions intensity at manufacturing facilities in treated counties relative to adjacent control counties. The results are presented in Table~\ref{tab:baseline-onsite-air-emissions-intensity}. I find that total air emissions intensity increases by $9.4ppts$ in treated counties relative to adjacent control counties. Albeit the reported effect size is smaller than the size of the effect in~\cite{zhang2023unintended}, the result is similar to their conclusions for Chinese manufacturing industries. The focus only specific toxic chemical air emission types may provide tangible insights regarding the distance in the effect size in the Chinese paper. Further this effect is substantially driven by the fugitive air emissions and remain unchanged even in the tightest estimation. Moreover, the results of the dynamic effects in Figure~\ref{fig:baseline-onsite-air-emission-intensity} show an instantaneous increase in air emissions intensity, reaching $16.2ppts$ three years after the initial treatment date. Similar patterns are recorded for the fugitive air emissions, where significant effect is found only three years post-treatment for the point air emissions. No significant pre-trends are found.
    \input{fig_did_onsite_air_emissions_int}

    \subsection{Water Discharge Intensity}\label{subsec:water-discharge-intensity}
    Next, I estimate the effect of raising MW on total onsite surface water discharge intensity. The model is given by:
    \begin{equation}
        W_{f,c,i,cp,s,t} = \beta Treated_{s,t}^e + \delta X_{v,c,t-1} + \omega F_{f,t} + \gamma_{f} + \phi_{cp} + \eta_{c,t} + \left[\lambda_{t} + \theta_{h} + \sigma_{s} + \zeta_{c} \right] + \varepsilon_{f,c,i,cp,s,t},\label{eq:baseline-onsite-water-discharge-intensity}
    \end{equation}
    where $W_{f,c,i,cp,s,t}$ is the total onsite surface water discharge intensity from a manufacturing facility, $f$ through a toxic chemical, $c$ used in industry, $i$ in cross-border county pairs, $cp$ in state, $s$ in the year, $t$. Surface water discharge intensity is the total weight of toxic chemicals per unit of output in the form of contaminants, or wastewater from industrial or commercial facilities that are released into surface water bodies such as streams. This discharge can occur through various pathways, including direct discharge through pipes, drains, or outfalls, as well as through runoff from facility surfaces or surrounding areas. Surface water discharge at facilities is regulated by environmental agencies (the clean water act of $1972$) to protect water quality, safeguard human health, and preserve aquatic ecosystems. Facilities are often required to obtain permits, monitor their discharge, and implement pollution control measures to minimize the impact of their activities on surface water resources.
    % Please add the following required packages to your document preamble:
% \usepackage{booktabs}
% \usepackage{graphicx}
\begin{table}[H]
    \centering
    \caption{Effect of the MW policy on Onsite Surface Water Discharge Intensity}
    \label{tab:baseline-onsite-water-disc-int}
    \resizebox{\columnwidth}{!}{%
        \begin{tabular}{@{}lllll@{}}
            \toprule\toprule
            Surface water discharge intensity (log) & \multicolumn{2}{c}{Total} & \multicolumn{2}{c}{Number of Receiving Streams} \\
            \cmidrule(lr){2-3}  \cmidrule(lr){4-5}
            & 1         & 2         & 3         & 4         \\ \midrule
            treated            & 0.0330    & 0.0439    & 0.0096    & 0.0096    \\
            & (0.0233)  & (0.0340)  & (0.0181)  & (0.0181)  \\
            controls           & Yes       & Yes       & Yes       & Yes       \\
            year FE            & Yes       & Yes       & Yes       & Yes       \\
            facility FE        & Yes       & Yes       & Yes       & Yes       \\
            border-county FE   & Yes       & Yes       & Yes       & Yes       \\
            toxic chemical FE  & No        & Yes       & No        & Yes       \\
            toxic chemical LTs & No        & Yes       & No        & Yes       \\\midrule
            Observations       & 1,893,689 & 1,893,689 & 1,893,689 & 1,893,689 \\
            $R^2$              & 0.2954    & 0.5836    & 0.7914    & 0.7914    \\
            Baseline Mean      & 20.04     & 20.04     & 0.39      & 0.39      \\ \bottomrule\bottomrule
        \end{tabular}%
    }
    \begin{minipage}{18cm}
        \vspace{0.05in}
        These results are obtained from estimating model~\ref{eq:baseline-onsite-water-discharge-intensity}. Three-way clustered robust standard errors are reported in parentheses, and clustered at the toxic chemical, industry and state levels. ***, **, and * denote significance levels at the less than $1\%$, $5\%$ and $10\%$, respectively. de Chaisemartin and D'Haultfoeuille Decomposition: $\sum dCDH_{ATTs}^{weights(+)} = 1$ and $\sum dCDH_{ATTs}^{weights(-)} = 0$.
    \end{minipage}
\end{table}

    The average treatment effect on the treated (ATT) is captured by $\beta$, which is the difference in the average effect of raising the MW floor on total surface water discharge intensity at manufacturing facilities in treated counties relative to adjacent control counties. The results on total surface water discharge are reported in Table~\ref{tab:baseline-onsite-water-disc-int}. This results show that due a higher MW policy, surface water discharge increases by $7.3ppts$ in treated counties relative to adjacent control counties. Further results in Figure~\ref{fig:baseline-onsite-water-discharge-intensity} show that this effect jumped to $13ppts$ three years after the treatment. I find no evidence of significant pre-trends nor any significant effect on the number of receiving streams.
    \input{fig_did_onsite_water_discharge_int}

    \subsection{Land Releases Intensity}\label{subsec:land-releases-intensity}
    Lastly, I estimate the effect of raising MW on total onsite land releases intensity. The model is given by:
    \begin{equation}
        L_{f,c,i,cp,s,t} = \beta Treated_{s,t}^e + \delta X_{v,c,t-1} + \omega F_{f,t} + \gamma_{f} + \phi_{cp} + \eta_{c,t} + \left[\lambda_{t} + \theta_{h} + \sigma_{s} + \zeta_{c} \right] + \varepsilon_{f,c,i,cp,s,t},\label{eq:baseline-onsite-land-releases-intensity}
    \end{equation}
    where $L_{f,c,i,cp,s,t}$ is the total onsite land releases intensity from manufacturing facility, $f$ through toxic chemical, $c$ used in industry, $i$ in cross-border county pairs, $cp$ in state, $s$ in the year, $t$. Total onsite land releases intensity is the total weight of toxic chemicals per unit of output that are released to land.
    % Please add the following required packages to your document preamble:
% \usepackage{booktabs}
% \usepackage{graphicx}
\begin{table}[H]
    \centering
    \caption{Effect of the MW policy on Onsite Land Releases Intensity}
    \label{tab:baseline-onsite-land-releases-intensity}
    \resizebox{\columnwidth}{!}{%
        \begin{tabular}{@{}lllllllllllll@{}}
            \toprule\toprule
            & \multicolumn{2}{c}{Total} & \multicolumn{2}{c}{Underground Injection} & \multicolumn{2}{c}{Landfills} & \multicolumn{2}{c}{To-Land Treatment} & \multicolumn{2}{c}{Surface Impoundment} & \multicolumn{2}{c}{Land Releases (Others)} \\
            \cmidrule(lr){2-3}  \cmidrule(lr){4-5}  \cmidrule(lr){6-7}  \cmidrule(lr){8-9}  \cmidrule(lr){10-11}  \cmidrule(lr){12-13}
            Onsite land releases intensity (log) & 1         & 2         & 3         & 4         & 5         & 6         & 7         & 8         & 9         & 10        & 11        & 12        \\ \midrule
            treated                              & 0.0208    & 0.0208    & 0.0001    & 0.0001    & -0.0074   & -0.0074   & -0.0008   & -0.0008   & 0.0267**  & 0.0267**            & 0.0031                 & 0.0031                 \\
            & (0.0152)  & (0.0152)  & (0.0005)  & (0.0005)  & (0.0052)  & (0.0052)  & (0.0060)  & (0.0060)  & (0.0125)            & (0.0125)            & (0.0092)               & (0.0092)               \\
            controls                             & Yes       & Yes       & Yes       & Yes       & Yes       & Yes       & Yes       & Yes       & Yes       & Yes       & Yes       & Yes       \\
            year FE                              & No        & Yes       & No        & Yes       & No        & Yes       & No        & Yes       & No        & Yes       & No        & Yes       \\
            facility FE                          & Yes       & Yes       & Yes       & Yes       & Yes       & Yes       & Yes       & Yes       & Yes       & Yes       & Yes       & Yes       \\
            county FE                            & No        & Yes       & No        & Yes       & No        & Yes       & No        & Yes       & No        & Yes       & No        & Yes       \\
            county-pair FE                       & Yes       & Yes       & Yes       & Yes       & Yes       & Yes       & Yes       & Yes       & Yes       & Yes       & Yes       & Yes       \\
            facility state FE                    & No        & Yes       & No        & Yes       & No        & Yes       & No        & Yes       & No        & Yes       & No        & Yes       \\
            toxic chemical FE                    & No        & Yes       & No        & Yes       & No        & Yes       & No        & Yes       & No        & Yes       & No        & Yes       \\
            toxic chemical LTs                   & Yes       & Yes       & Yes       & Yes       & Yes       & Yes       & Yes       & Yes       & Yes       & Yes       & Yes       & Yes       \\\midrule\midrule
            Observations                         & 1,893,689 & 1,893,689 & 1,893,689 & 1,893,689 & 1,893,689 & 1,893,689 & 1,893,689         & 1,893,689         & 1,893,689           & 1,893,689           & 1,893,689              & 1,893,689              \\
            $R^2$                                & 0.5259    & 0.5259    & 0.3875    & 0.3875    & 0.4852    & 0.4852    & 0.3404    & 0.3404    & 0.1441    & 0.1441    & 0.6202                 & 0.6202                 \\
            Baseline Mean                        & 36.02     & 36.02     & 21.77     & 21.77     & 6.51      & 6.51      & 3.01      & 3.01      & 0.12      & 0.12      & 4.60      & 4.60      \\ \bottomrule\bottomrule
        \end{tabular}%
    }
    \begin{minipage}{18cm}
        \vspace{0.05in}
        These results are obtained from estimating model~\ref{eq:baseline-onsite-land-releases-intensity}. Three-way clustered robust standard errors are reported in parentheses, and clustered at the toxic chemical, industry and state levels. ***, **, and * denote significance levels at the less than $1\%$, $5\%$ and $10\%$, respectively.
    \end{minipage}
\end{table}

    The average treatment effect on the treated (ATT) is captured by $\beta$, which is the difference in the average effect of raising the MW floor on total surface water discharge intensity at manufacturing facilities in treated counties relative to adjacent control counties. The results on total land releases intensity are reported in Table~\ref{tab:baseline-onsite-land-releases-intensity}. Except for the total surface impoundment intensity, I find no statistically and economically significant effect on the intensities of total land releases, underground injection, landfills, to-land treatment (used for land fertilization), and other land releases. A surface impoundment is a type of containment structure used for the storage and management of liquid wastes, such as industrial wastewater, hazardous chemicals, or contaminated water. It consists of an excavated depression or basin lined with impermeable materials such as clay or synthetic liners to prevent the leakage of liquids into the surrounding soil and groundwater. The results show that a higher MW floor increases surface impoundment intensity by $2.7ppts$ in treated counties relative to adjacent control counties. The event study shows an instantaneous effect which persists up to four years after the treatment, reaching $5ppts$. I record no significant evidence of pre-trends.
    \begin{figure}[H]
    \centering
    \includegraphics[width=1\textwidth, height=0.5\textheight,keepaspectratio]{fig_did_total_land_releases_onsite}
    \caption{Total Onsite Land Releases Intensity}
    \label{fig:baseline-onsite-land-releases-intensity}
    \begin{minipage}{12cm}
        \vspace{0.05in}
        NOTES: The event study model of equation~\ref{eq:baseline-onsite-land-releases-intensity} is $L_{f,c,i,cp,s,t} = \sum_{{e = 2011},{e \neq 2013}}^{2017} \beta Treated_{s,t}^e + \delta X_{v,c,t-1} + \omega F_{f,t} + \gamma_{f} + \phi_{cp} + \eta_{c,t} + \left[\lambda_{t} + \theta_{f,h} + \sigma_{s} + \zeta_{c} \right] + \varepsilon_{f,c,i,cp,s,t}$. Three-way clustered robust standard errors are reported in parentheses, and clustered at the toxic chemical, industry and state levels.
    \end{minipage}
\end{figure}


    \section{Offsite and POTWs Toxic Releases}\label{sec:offsite-and-potws-toxic-releases}
    \begin{figure}[H]
    \centering
    \includegraphics[width=1\textwidth, height=0.5\textheight,keepaspectratio]{fig_did_total_releases_offsite}
    \caption{Total Offsite Releases Intensity}
    \label{fig:baseline-offsite-total-releases-intensity}
    \begin{minipage}{12cm}
        \vspace{0.05in}
        NOTES: The event study model of equation~\ref{eq:baseline-offsite-total-releases-intensity} is $P_{f,c,i,cp,s,t}^{offsite} = \sum_{{e = 2011},{e \neq 2013}}^{2017} \beta Treated_{s,t}^e + \delta X_{v,c,t-1} + \omega F_{f,t} + \gamma_{f} + \phi_{cp} + \eta_{c,t} + \left[\lambda_{t} + \theta_{f,h} + \sigma_{s} + \zeta_{c} \right] + \varepsilon_{f,c,i,cp,s,t}$. Three-way clustered robust standard errors are reported in parentheses, and clustered at the toxic chemical, industry and state levels.
    \end{minipage}
\end{figure}
    \input{fig_did_offsite_land_releases_int}
    \input{fig_did_potws_total_releases}


    \section{Robustness Exercises}\label{sec:robustness-exercises}
    I conduct several robustness exercises to raise the credibility of the above results. This ranges from placebo exercises, alternative clustering of the standard errors, economic growth effect on pollution emission intensity, and removal of states with the highest total emission intensity.

    \subsection{Placebo Exercise}\label{subsec:placebo-exercise}
    To increase the credibility in the reported emission intensity results; I model the placebo effect of raising the MW on total releases intensity using total onsite releases intensity from catastrophic events. Such events include chemical spills, fire, explosions and natural disasters at specific manufacturing facilities. Catastrophic events are expected to be uncorrelated with the MW policy. Hence, I do not expect to see any statistical significance in the releases per unit of output, from catastrophic events. I estimate the following model:
    \begin{equation}
        P_{f,c,i,cp,s,t}^{catastrophic} = \beta Treated_{s,t}^e + \delta X_{v,c,t-1} + \omega F_{f,t} + \gamma_{f} + \phi_{cp} + \eta_{c,t} + \left[\lambda_{t} + \theta_{h} + \sigma_{s} + \zeta_{c} \right] + \varepsilon_{f,c,i,cp,s,t},\label{eq:robustness-placebo}
    \end{equation}
    % Please add the following required packages to your document preamble:
% \usepackage{booktabs}
% \usepackage{graphicx}
\begin{table}[H]
    \centering
    \caption{Effect of the MW policy on Total Onsite Toxic Releases Intensity, from Catastrophic Events}
    \label{tab:robustness-placebo}
    \resizebox{\columnwidth}{!}{%
        \begin{tabular}{@{}lll@{}}
            \toprule\toprule
            Onsite releases intensity from catastrophic events (log) & 1         & 2         \\ \midrule
            treated                                                  & -0.0123   & -0.0123   \\
            & (0.0284)  & (0.0284)  \\
            controls                                                 & Yes       & Yes       \\
            year FE                                                  & No        & Yes       \\
            facility FE                                              & Yes       & Yes       \\
            county FE                                                & No        & Yes       \\
            county-pair FE                                           & Yes       & Yes       \\
            facility state FE                                        & No        & Yes       \\
            toxic chemical FE                                        & No        & Yes       \\
            toxic chemical LTs                                       & Yes       & Yes       \\ \midrule \midrule
            Observations                                             & 1,893,689 & 1,893,689 \\
            $R^2$                                                    & 0.4149    & 0.4149    \\
            Baseline Mean                                            & 19.77     & 19.77     \\ \bottomrule \bottomrule
        \end{tabular}%
    }
    \begin{minipage}{18cm}
        \vspace{0.05in}
        These results are obtained from estimating model~\ref{eq:robustness-placebo}. Three-way clustered robust standard errors are reported in parentheses, and clustered at the toxic chemical, industry and state levels. ***, **, and * denote significance levels at the less than $1\%$, $5\%$ and $10\%$, respectively.
    \end{minipage}
\end{table}

    where $P_{f,c,i,cp,s,t}$ is the total onsite releases intensity from catastrophic events at manufacturing facility, $f$ through toxic chemical use, $c$ in industry, $i$ in cross-border county pairs, $cp$ in state, $s$ in the year, $t$. The ATT measures the difference in the average effect of raising the MW on total onsite releases intensity from catastrophic events at manufacturing facilities in treated counties relative to adjacent control counties. The results are presented in Table~\ref{tab:robustness-placebo}. Indeed, the result is consistent with the null hypothesis of the MW policy on total releases intensity, from catastrophic events. This effect remains unchanged even in my tightest estimation in column $2$. Additionally, Figure~\ref{fig:baseline-placebo} shows the dynamic placebo effect on total releases and no evidence of significant pre-trends.
    \input{fig_did_total_releases_placebo}

    \subsection{Alternative Clustering of the SEs}\label{subsec:alternative-clustering-of-the-ses}
    % Please add the following required packages to your document preamble:
% \usepackage{booktabs}
% \usepackage{graphicx}
\begin{table}[H]
    \centering
    \caption{Total Onsite Releases Intensity: Alternative clustering of the SEs}
    \label{tab:robustness-ses-clustering-total-releases-onsite}
    \resizebox{\columnwidth}{!}{%
        \begin{tabular}{@{}lllllllllllll@{}}
            \toprule\toprule
            Total releases intensity (log) & 1         & 2         & 3         & 4         & 5         & 6         & 7                    & 8                    & 9                    & 10                & 11                & 12                \\ \midrule
            treated                        & 0.1691**  & 0.1691*** & 0.1691**  & 0.1691**  & 0.1691*** & 0.1691**  & 0.1691***            & 0.1691***            & 0.1691***            & 0.1691***         & 0.1691**          & 0.1691**          \\
            & (0.0659)  & (0.0627)  & (0.0678)  & (0.0669)  & (0.0602)  & (0.0616)  & (0.0606)             & (0.0606)             & (0.0601)             & (0.0602)          & (0.0659)          & (0.0668)          \\
            controls                       & Yes       & Yes       & Yes       & Yes       & Yes       & Yes       & Yes                  & Yes                  & Yes                  & Yes               & Yes               & Yes               \\
            year FE                        & Yes       & Yes       & Yes       & Yes       & Yes       & Yes       & Yes                  & Yes                  & Yes                  & Yes               & Yes               & Yes               \\
            facility FE                    & Yes       & Yes       & Yes       & Yes       & Yes       & Yes       & Yes                  & Yes                  & Yes                  & Yes               & Yes               & Yes               \\
            county FE                      & Yes       & Yes       & Yes       & Yes       & Yes       & Yes       & Yes                  & Yes                  & Yes                  & Yes               & Yes               & Yes               \\
            county-pair FE                 & Yes       & Yes       & Yes       & Yes       & Yes       & Yes       & Yes                  & Yes                  & Yes                  & Yes               & Yes               & Yes               \\
            facility state FE              & Yes       & Yes       & Yes       & Yes       & Yes       & Yes       & Yes                  & Yes                  & Yes                  & Yes               & Yes               & Yes               \\
            toxic chemical FE              & Yes       & Yes       & Yes       & Yes       & Yes       & Yes       & Yes                  & Yes                  & Yes                  & Yes               & Yes               & Yes               \\
            toxic chemical LTs             & Yes       & Yes       & Yes       & Yes       & Yes       & Yes       & Yes                  & Yes                  & Yes                  & Yes               & Yes               & Yes               \\ \midrule \midrule
            clustered at the:              & facility  & zipcode   & county    & industry  & chemical  & state     & facility \& chemical & facility \& industry & chemical \& industry & chemical \& state & facility \& state & industry \& state \\
            Observations                   & 1,893,689 & 1,893,689 & 1,893,689 & 1,893,689 & 1,893,689 & 1,893,689 & 1,893,689            & 1,893,689            & 1,893,689            & 1,893,689         & 1,893,689         & 1,893,689         \\
            $R^2$                          & 0.7457    & 0.7457    & 0.7457    & 0.7457    & 0.7457    & 0.7457    & 0.7457               & 0.7457               & 0.7457               & 0.7457            & 0.7457            & 0.7457            \\ \bottomrule \bottomrule
        \end{tabular}%
    }
    \begin{minipage}{18cm}
        \vspace{0.05in}
        These results are obtained from estimating model~\ref{eq:baseline-total-onsite-releases-intensity}. ***, **, and * denote significance levels at the less than $1\%$, $5\%$ and $10\%$, respectively.
    \end{minipage}
\end{table}
    % Please add the following required packages to your document preamble:
% \usepackage{booktabs}
% \usepackage{graphicx}
\begin{table}[H]
    \centering
    \caption{Total Onsite Air Emissions Intensity: Alternative clustering of the SEs}
    \label{tab:robustness-ses-clustering-total-air-releases-onsite}
    \resizebox{\columnwidth}{!}{%
        \begin{tabular}{@{}lllllllllllll@{}}
            \toprule \toprule
            Total air emissions intensity (log) & 1         & 2         & 3         & 4         & 5         & 6         & 7                    & 8                    & 9                    & 10                & 11                & 12                \\ \midrule
            treated                             & 0.0938    & 0.0938*   & 0.0938    & 0.0938*   & 0.0938*   & 0.0938    & 0.0938               & 0.0938               & 0.0938*              & 0.0938*           & 0.0938            & 0.0938*           \\
            & (0.0605)  & (0.0565)  & (0.0630)  & (0.0561)  & (0.0534)  & (0.0586)  & (0.0605)             & (0.0605)             & (0.0533)             & (0.0534)          & (0.0605)          & (0.0561)          \\
            controls                            & Yes       & Yes       & Yes       & Yes       & Yes       & Yes       & Yes                  & Yes                  & Yes                  & Yes               & Yes               & Yes               \\
            year FE                             & Yes       & Yes       & Yes       & Yes       & Yes       & Yes       & Yes                  & Yes                  & Yes                  & Yes               & Yes               & Yes               \\
            facility FE                         & Yes       & Yes       & Yes       & Yes       & Yes       & Yes       & Yes                  & Yes                  & Yes                  & Yes               & Yes               & Yes               \\
            county FE                           & Yes       & Yes       & Yes       & Yes       & Yes       & Yes       & Yes                  & Yes                  & Yes                  & Yes               & Yes               & Yes               \\
            county-pair FE                      & Yes       & Yes       & Yes       & Yes       & Yes       & Yes       & Yes                  & Yes                  & Yes                  & Yes               & Yes               & Yes               \\
            facility state FE                   & Yes       & Yes       & Yes       & Yes       & Yes       & Yes       & Yes                  & Yes                  & Yes                  & Yes               & Yes               & Yes               \\
            toxic chemical FE                   & Yes       & Yes       & Yes       & Yes       & Yes       & Yes       & Yes                  & Yes                  & Yes                  & Yes               & Yes               & Yes               \\
            toxic chemical LTs                  & Yes       & Yes       & Yes       & Yes       & Yes       & Yes       & Yes                  & Yes                  & Yes                  & Yes               & Yes               & Yes               \\\midrule\midrule
            clustered at the:                   & facility  & zipcode   & county    & industry  & chemical  & state     & facility \& chemical & facility \& industry & chemical \& industry & chemical \& state & facility \& state & industry \& state \\
            Observations                        & 1,893,689 & 1,893,689 & 1,893,689 & 1,893,689 & 1,893,689 & 1,893,689 & 1,893,689            & 1,893,689            & 1,893,689            & 1,893,689         & 1,893,689         & 1,893,689         \\
            $R^2$                               & 0.7675    & 0.7675    & 0.7675    & 0.7675    & 0.7675    & 0.7675    & 0.7675               & 0.7675               & 0.7675               & 0.7675            & 0.7675            & 0.7675            \\ \bottomrule\bottomrule
        \end{tabular}%
    }
    \begin{minipage}{18cm}
        \vspace{0.05in}
        These results are obtained from estimating model~\ref{eq:baseline-onsite-air-emission-intensity}. ***, **, and * denote significance levels at the less than $1\%$, $5\%$ and $10\%$, respectively.
    \end{minipage}
\end{table}
    % Please add the following required packages to your document preamble:
% \usepackage{booktabs}
% \usepackage{graphicx}% \usepackage{graphicx}
\begin{table}[H]
    \centering
    \caption{Total Onsite Surface Water Discharge Intensity: Alternative clustering of the SEs}
    \label{tab:robustness-ses-clustering-water-discharge-onsite-intensity}
    \resizebox{\columnwidth}{!}{%
        \begin{tabular}{@{}lllllllllllll@{}}
            \toprule\toprule
            Total surface water discharge intensity (log) & 1         & 2         & 3         & 4         & 5         & 6         & 7                    & 8                    & 9                    & 10                & 11                & 12                \\ \midrule
            treated                                       & 0.0439    & 0.0439    & 0.0439    & 0.0439    & 0.0439    & 0.0439    & 0.0439               & 0.0439               & 0.0439               & 0.0439            & 0.0439          & 0.0439       \\
            & (0.0268)  & (0.0269)  & (0.0269)  & (0.0288)  & (0.0340)  & (0.0289)  & (0.0268)             & (0.0268)             & (0.0340)             & (0.0340)          & (0.0268)          & (0.0288)          \\
            controls                                      & Yes       & Yes       & Yes       & Yes       & Yes       & Yes       & Yes                  & Yes                  & Yes                  & Yes               & Yes               & Yes               \\
            year FE                                       & Yes       & Yes       & Yes       & Yes       & Yes       & Yes       & Yes                  & Yes                  & Yes                  & Yes               & Yes               & Yes               \\
            facility FE                                   & Yes       & Yes       & Yes       & Yes       & Yes       & Yes       & Yes                  & Yes                  & Yes                  & Yes               & Yes               & Yes               \\
            border-county FE                                & Yes       & Yes       & Yes       & Yes       & Yes       & Yes       & Yes                  & Yes                  & Yes                  & Yes               & Yes               & Yes               \\
            toxic chemical FE                             & Yes       & Yes       & Yes       & Yes       & Yes       & Yes       & Yes                  & Yes                  & Yes                  & Yes               & Yes               & Yes               \\
            toxic chemical LTs                            & Yes       & Yes       & Yes       & Yes       & Yes       & Yes       & Yes                  & Yes                  & Yes                  & Yes               & Yes               & Yes               \\ \midrule
            clustered at the:                             & facility  & zipcode   & county    & industry  & chemical  & state     & facility \& chemical & facility \& industry & chemical \& industry & chemical \& state & facility \& state & industry \& state \\
            Observations                                  & 1,893,689 & 1,893,689 & 1,893,689 & 1,893,689 & 1,893,689 & 1,893,689 & 1,893,689            & 1,893,689            & 1,893,689            & 1,893,689         & 1,893,689         & 1,893,689         \\
            $R^2$                                         & 0.5836    & 0.5836    & 0.5836    & 0.5836    & 0.5836    & 0.5836    & 0.5836               & 0.5836               & 0.5836               & 0.5836            & 0.5836            & 0.5836            \\ \bottomrule \bottomrule
        \end{tabular}%
    }
    \begin{minipage}{18cm}
        \vspace{0.05in}
        These results are obtained from estimating model~\ref{eq:baseline-onsite-water-discharge-intensity}. ***, **, and * denote significance levels at the less than $1\%$, $5\%$ and $10\%$, respectively.
    \end{minipage}
\end{table}
    % Please add the following required packages to your document preamble:
% \usepackage{booktabs}
% \usepackage{graphicx}
\begin{table}[H]
    \centering
    \caption{Total Onsite Land Releases Intensity: Alternative Clustering of SEs}
    \label{tab:robustness-ses-clustering-total-land-releases-intensity}
    \resizebox{\columnwidth}{!}{%
        \begin{tabular}{@{}lllllllllllll@{}}
            \toprule\toprule
            Total onsite land releases intensity (log) & 1         & 2         & 3         & 4         & 5         & 6         & 7                    & 8                    & 9                    & 10                & 11                & 12                \\ \midrule
            treated                                    & 0.0208    & 0.0208    & 0.0208    & 0.0208    & 0.0208    & 0.0208    & 0.0208               & 0.0208               & 0.0208               & 0.0208            & 0.0208            & 0.0208            \\
            & (0.0272)  & (0.0272)  & (0.0269)  & (0.0284)  & (0.0152)  & (0.0185)  & (0.0272)             & (0.0272)             & (0.0152)             & (0.0152)          & (0.0272)          & (0.0284)          \\
            controls                                   & Yes       & Yes       & Yes       & Yes       & Yes       & Yes       & Yes                  & Yes                  & Yes                  & Yes               & Yes               & Yes               \\
            year FE                                    & Yes       & Yes       & Yes       & Yes       & Yes       & Yes       & Yes                  & Yes                  & Yes                  & Yes               & Yes               & Yes               \\
            facility FE                                & Yes       & Yes       & Yes       & Yes       & Yes       & Yes       & Yes                  & Yes                  & Yes                  & Yes               & Yes               & Yes               \\
            county FE                                  & Yes       & Yes       & Yes       & Yes       & Yes       & Yes       & Yes                  & Yes                  & Yes                  & Yes               & Yes               & Yes               \\
            county-pair FE                             & Yes       & Yes       & Yes       & Yes       & Yes       & Yes       & Yes                  & Yes                  & Yes                  & Yes               & Yes               & Yes               \\
            facility state FE                          & Yes       & Yes       & Yes       & Yes       & Yes       & Yes       & Yes                  & Yes                  & Yes                  & Yes               & Yes               & Yes               \\
            toxic chemical FE                          & Yes       & Yes       & Yes       & Yes       & Yes       & Yes       & Yes                  & Yes                  & Yes                  & Yes               & Yes               & Yes               \\
            toxic chemical LTs                         & Yes       & Yes       & Yes       & Yes       & Yes       & Yes       & Yes                  & Yes                  & Yes                  & Yes               & Yes               & Yes               \\\midrule\midrule
            clustered at the:                          & facility  & zipcode   & county    & industry  & chemical  & state     & facility \& chemical & facility \& industry & chemical \& industry & chemical \& state & facility \& state & industry \& state \\
            Observations                               & 1,893,689 & 1,893,689 & 1,893,689 & 1,893,689 & 1,893,689 & 1,893,689 & 1,893,689            & 1,893,689            & 1,893,689            & 1,893,689         & 1,893,689         & 1,893,689         \\
            $R^2$                                      & 0.5259    & 0.5259    & 0.5259    & 0.5259    & 0.5259    & 0.5259    & 0.5259               & 0.5259               & 0.5259               & 0.5259            & 0.5259            & 0.5259            \\ \bottomrule\bottomrule
        \end{tabular}%
    }
    \begin{minipage}{18cm}
        \vspace{0.05in}
        These results are obtained from estimating model~\ref{eq:baseline-onsite-land-releases-intensity}. ***, **, and * denote significance levels at the less than $1\%$, $5\%$ and $10\%$, respectively.
    \end{minipage}
\end{table}

    \subsection{Removal of States with the Highest Total Emissions Intensity}\label{subsec:removal-of-states-with-the-highest-total-emissions-intensity}

    \subsection{Alternative Specifications}\label{subsec:alternative-specifications}


    \section{Mechanism Analyses}\label{sec:mechanism-analyses}
    In this section, I investigate the transmission mechanisms of the above results. I explore the following potential mechanisms at onsite manufacturing facilities through two broad lenses: onsite waste management, and onsite source reduction activities.

    \subsection{Onsite Waste Management Activities}\label{subsec:onsite-waste-management-activities}
    Onsite waste management activities refers to the handling and management of already generated toxic wastes at manufacturing facilities. These include treatment, energy recovery, and recycling. Treatment involves processes used to change the physical, chemical, or biological composition of a waste to make it less hazardous or easier to manage. It uses the following methods: air emissions, biological, chemical, physical and incineration methods. Similarly, energy recovery involves processes that use waste materials as a source of energy through methods such as industrial kiln, furnace and boiler. Energy recovery can reduce the volume of disposed waste and offset the use of fossil fuels. Lastly, facilities may engage in recycling activities to reuse or reclaim materials from waste streams. This can involve processes such as metal and solvent recoveries, and reuse. To investigate these onsite waste management activities, I estimate the following model:
    \begin{equation}
        M_{f,c,i,cp,s,t}^{wma} = \beta Treated_{s,t}^e + \delta X_{v,c,t-1} + \omega F_{f,t} + \gamma_{f} + \phi_{cp} + \eta_{c,t} + \left[\lambda_{t} + \theta_{h} + \sigma_{s} + \zeta_{c} \right] + \varepsilon_{f,c,i,cp,s,t},\label{eq:mechanisms-waste-management}
    \end{equation}
    where $M_{f,c,i,cp,s,t}^{wma}$ is the vector of logged total onsite waste management activities (including treatment, energy recovery and recycling) of already generated toxic wastes at manufacturing facility, $f$ through toxic chemical use, $c$ in industry, $i$ in cross-border county pairs, $cp$ in state, $s$ in the year, $t$, and dummies of associated methods. The ATT is captured by $\beta$, which is the difference in the average effect of higher MW on total onsite waste management activities at manufacturing facilities in treated counties relative to adjacent control counties. The results on total onsite releases are reported in Table~\ref{tab:mechanisms-onsite-waste-management-activities}.
    % Please add the following required packages to your document preamble:
% \usepackage{booktabs}
% \usepackage{graphicx}
\begin{table}[H]
    \centering
    \caption{Onsite Waste Management Activities}
    \label{tab:mechanisms-onsite-waste-management-activities}
    \resizebox{\columnwidth}{!}{%
        \begin{tabular}{@{}llllllllllllllll@{}}
            \toprule\toprule
            \multicolumn{6}{r}{Treatment} & \multicolumn{4}{r}{Energy Recovery} & \multicolumn{4}{r}{Recycling} \\
            \cmidrule(lr){3-8}\cmidrule(lr){9-12}\cmidrule(lr){13-16}
            Onsite WMA         & total     & treatment & air       & biological & chemical  & physical  & thermal   & energy recovery & kiln      & boiler    & furnace   & recycling & reuse     & metal recovery & solvent recovery \\ \midrule
            treated            & -0.1120   & -0.1875*  & 0.0674*** & -0.0138    & -0.0077   & -0.0146   & 0.0032    & 0.0748**        & 0.0020*** & 0.0019    & 0.0012    & -0.0186   & -0.0110** & 0.0056         & 0.0106           \\
            & (0.1293)  & (0.0976)  & (0.0129)  & (0.0089)   & (0.0088)  & (0.0145)  & (0.0117)  & (0.0297)        & (0.0005)  & (0.0016)  & (0.0011)  & (0.1206)  & (0.0051)  & (0.0055)       & (0.0073)         \\
            controls           & Yes       & Yes       & Yes       & Yes        & Yes       & Yes       & Yes       & Yes             & Yes       & Yes       & Yes       & Yes       & Yes       & Yes            & Yes              \\
            year FE            & Yes       & Yes       & Yes       & Yes        & Yes       & Yes       & Yes       & Yes             & Yes       & Yes       & Yes       & Yes       & Yes       & Yes            & Yes              \\
            facility FE        & Yes       & Yes       & Yes       & Yes        & Yes       & Yes       & Yes       & Yes             & Yes       & Yes       & Yes       & Yes       & Yes       & Yes            & Yes              \\
            county FE          & Yes       & Yes       & Yes       & Yes        & Yes       & Yes       & Yes       & Yes             & Yes       & Yes       & Yes       & Yes       & Yes       & Yes            & Yes              \\
            county-pair FE     & Yes       & Yes       & Yes       & Yes        & Yes       & Yes       & Yes       & Yes             & Yes       & Yes       & Yes       & Yes       & Yes       & Yes            & Yes              \\
            facility state FE  & Yes       & Yes       & Yes       & Yes        & Yes       & Yes       & Yes       & Yes             & Yes       & Yes       & Yes       & Yes       & Yes       & Yes            & Yes              \\
            toxic chemical FE  & Yes       & Yes       & Yes       & Yes        & Yes       & Yes       & Yes       & Yes             & Yes       & Yes       & Yes       & Yes       & Yes       & Yes            & Yes              \\
            toxic chemical LTs & Yes       & Yes       & Yes       & Yes        & Yes       & Yes       & Yes       & Yes             & Yes       & Yes       & Yes       & Yes       & Yes       & Yes            & Yes              \\ \midrule \midrule
            Observations       & 1,893,689 & 1,893,689 & 1,893,689 & 1,893,689  & 1,893,689 & 1,893,689 & 1,893,689 & 1,893,689       & 1,893,689 & 1,893,689 & 1,893,689 & 1,893,689 & 1,893,689 & 1,893,689      & 1,893,689        \\
            $R^2$              & 0.7703    & 0.7777    & 0.5764    & 0.6023     & 0.6218    & 0.6237    & 0.6548    & 0.6998          & 0.4023    & 0.6804    & 0.6700    & 0.7358    & 0.7066    & 0.7998         & 0.7482           \\ \bottomrule \bottomrule
        \end{tabular}%
    }
    \begin{minipage}{18cm}
        \vspace{0.05in}
        These results are obtained from estimating model~\ref{eq:mechanisms-waste-management}. ***, **, and * denote significance levels at the less than $1\%$, $5\%$ and $10\%$, respectively.
    \end{minipage}
\end{table}

    \subsection{Onsite Source Reduction Activities}\label{subsec:onsite-source-reduction-activities}
    Onsite source reduction refers to any practice that minimizes or eliminates the generation of toxic wastes or pollutants at the very beginning, before they enter any waste stream or get released into the environment. This is also referred to as Pollution Prevention (P2). These activities aim to minimize or eliminate the creation of waste materials or the use of hazardous substances in manufacturing processes. They include: $(i)$ material substitution and modification which involves changing input purity or dimensions, or replacing a raw material, fuel, organic solvents, feedstock, reagent, ancillary chemicals, or other substance with environmentally preferable alternatives; $(ii)$ product modifications which refer to changing the end product through design, composition, formulation, or packaging changes, as well as full final product replacements that reduce the generation of waste; $(iii)$ process and equipment modifications which involve improvements to industrial processes and/or associated equipment including implementation of new processes that produce less waste, direct reuse of chemicals, or technological changes impacting synthesis, formulation, fabrication, and assembly, and surface treatment such as cleaning, degreasing, surface preparation, and finishing. All aimed to improve efficiency and reduce waste generation; $(iv)$ inventory management, refers to improvements in procurement, inventory tracking, preventative monitoring, and storage and handling of chemicals and materials as they move through a facility to optimize their use and prevent spills and leaks during operation; and $(v)$ operating practices and training, refers to improvements in maintenance, production scheduling, process monitoring, and other practices that enhance operator expertise and housekeeping measures that eliminate or minimize waste generation. To investigate onsite source reduction activities, I estimate the following model:
    \begin{equation}
        M_{f,c,i,cp,s,t}^{sra} = \beta Treated_{s,t}^e + \delta X_{v,c,t-1} + \omega F_{f,t} + \gamma_{f} + \phi_{cp} + \eta_{c,t} + \left[\lambda_{t} + \theta_{h} + \sigma_{s} + \zeta_{c} \right] + \varepsilon_{f,c,i,cp,s,t},\label{eq:mechanisms-source-reduction}
    \end{equation}
    where $M_{f,c,i,cp,s,t}^{sra}$ is the vector of onsite source reduction activities to reduce the generation of toxic wastes at manufacturing facility, $f$ through toxic chemical use, $c$ in industry, $i$ in cross-border county pairs, $cp$ in state, $s$ in the year, $t$. The ATT is captured by $\beta$, which is the difference in the average effect of higher MW on onsite source reduction activities at manufacturing facilities in treated counties relative to adjacent control counties. The results on total onsite releases are reported in Tables~\ref{tab:mechanisms-onsite-source-reduction-activities}, ~\ref{tab:mechanisms-onsite-sra-material-substitution-product-modifications}, ~\ref{tab:mechanisms-onsite-sra-process-modifications}, and ~\ref{tab:mechanisms-onsite-sra-inventory-operating-and-training-activities}.
    % Please add the following required packages to your document preamble:
% \usepackage{booktabs}
% \usepackage{graphicx}
\begin{table}[H]
    \centering
    \caption{Mechanisms of Onsite Source Reduction Activities}
    \label{tab:mechanisms-onsite-source-reduction-activities}
    \scalebox{1.6}{
%    \resizebox{\columnwidth}{!}{%
        \begin{tabular}{@{}ll@{}}
            \toprule\toprule
            Onsite source reduction & 1          \\ \midrule
            treated                 & -0.1331*** \\
            & (0.0166)   \\
            controls                & Yes        \\
            year FE                 & Yes        \\
            facility FE             & Yes        \\
            county FE               & Yes        \\
            county-pair FE          & Yes        \\
            facility state FE       & Yes        \\
            toxic chemical FE       & Yes        \\
            toxic chemical LTs      & Yes        \\ \midrule \midrule
            Observations            & 1,893,689  \\
            $R^2$                   & 0.6149     \\ \bottomrule\bottomrule
        \end{tabular}%
%    }
    }
    \begin{minipage}{10cm}
        \vspace{0.05in}
        These results are obtained from estimating model~\ref{eq:mechanisms-source-reduction}. ***, **, and * denote significance levels at the less than $1\%$, $5\%$ and $10\%$, respectively.
    \end{minipage}
\end{table}
    % Please add the following required packages to your document preamble:
% \usepackage{booktabs}
% \usepackage{graphicx}
\begin{table}[H]
    \centering
    \caption{Onsite SRA: Material Substitution and Product Modifications}
    \label{tab:mechanisms-onsite-sra-material-substitution-product-modifications}
    \resizebox{\columnwidth}{!}{%
        \begin{tabular}{@{}lllllllllllll@{}}
            \toprule \toprule
            Onsite SRA         & mat submod & organic solvent sub & feedstock and reagents & ancillary chems & purity chems & clean fuel & others    & prod mod  & new prod line & mod packaging & energy intensity (log) & others    \\ \midrule
            treated            & 0.0030     & -0.0010***          & 0.0005*                & 0.0002*         & -0.0002      & -0.0242*** & 0.0011    & -0.0011   & 0.0026***     & 0.0070**   & 0.0175**    & -0.0002   \\
            & (0.0019)   & (0.0004)            & (0.0003)               & (9.31e-5)       & (0.0007)     & (0.0051)   & (0.0010)  & (0.0011)  & (0.0009)      & (0.0031)  & (0.0087)    & (0.0007)  \\
            controls           & Yes        & Yes                 & Yes                    & Yes             & Yes          & Yes        & Yes       & Yes       & Yes           & Yes           & Yes              & Yes       \\
            year FE            & Yes        & Yes                 & Yes                    & Yes             & Yes          & Yes        & Yes       & Yes       & Yes           & Yes           & Yes              & Yes       \\
            facility FE        & Yes        & Yes                 & Yes                    & Yes             & Yes          & Yes        & Yes       & Yes       & Yes           & Yes           & Yes              & Yes       \\
            county FE          & Yes        & Yes                 & Yes                    & Yes             & Yes          & Yes        & Yes       & Yes       & Yes           & Yes           & Yes              & Yes       \\
            county-pair FE     & Yes        & Yes                 & Yes                    & Yes             & Yes          & Yes        & Yes       & Yes       & Yes           & Yes           & Yes              & Yes       \\
            facility state FE  & Yes        & Yes                 & Yes                    & Yes             & Yes          & Yes        & Yes       & Yes       & Yes           & Yes           & Yes              & Yes       \\
            toxic chemical FE  & Yes        & Yes                 & Yes                    & Yes             & Yes          & Yes        & Yes       & Yes       & Yes           & Yes           & Yes              & Yes       \\
            toxic chemical LTs & Yes        & Yes                 & Yes                    & Yes             & Yes          & Yes        & Yes       & Yes       & Yes           & Yes           & Yes              & Yes       \\\midrule\midrule
            Observations       & 1,893,689  & 1,893,689           & 1,893,689              & 1,893,689       & 1,893,689    & 1,893,689  & 1,893,689 & 1,893,689 & 1,893,689     & 1,893,689  & 1,893,689  & 1,893,689 \\
            $R^2$              & 0.1490     & 0.0708              & 0.0520                 & 0.0284          & 0.1283       & 0.7078     & 0.1649    & 0.1264    & 0.1164        & 0.5309        & 0.9825  & 0.0693    \\ \bottomrule\bottomrule
        \end{tabular}%
    }
    \begin{minipage}{18cm}
        \vspace{0.05in}
        These results are obtained from estimating model~\ref{eq:mechanisms-source-reduction}. ***, **, and * denote significance levels at the less than $1\%$, $5\%$ and $10\%$, respectively.
    \end{minipage}
\end{table}
    % Please add the following required packages to your document preamble:
% \usepackage{booktabs}
% \usepackage{graphicx}
\begin{table}[H]
    \centering
    \caption{Onsite SRA: Process Modifications}
    \label{tab:mechanisms-onsite-sra-process-modifications}
    \resizebox{\columnwidth}{!}{%
        \begin{tabular}{@{}llllllll@{}}
            \toprule\toprule
            Onsite SRA         & process mod & optimised efficiency & recirculation in process & new tech in process & recycle to reuse & r and d   & others    \\ \midrule
            treated            & 0.0007      & 0.0007**             & 0.0032                   & -0.0039             & -0.0106*         & 0.0068*** & 0.0028    \\
            & (0.0019)    & (0.0003)             & (0.0022)                 & (0.0032)            & (0.0056)         & (0.0018)  & (0.0036)  \\
            controls           & Yes         & Yes                  & Yes                      & Yes                 & Yes              & Yes       & Yes       \\
            year FE            & Yes         & Yes                  & Yes                      & Yes                 & Yes              & Yes       & Yes       \\
            facility FE        & Yes         & Yes                  & Yes                      & Yes                 & Yes              & Yes       & Yes       \\
            county FE          & Yes         & Yes                  & Yes                      & Yes                 & Yes              & Yes       & Yes       \\
            county-pair FE     & Yes         & Yes                  & Yes                      & Yes                 & Yes              & Yes       & Yes       \\
            facility state FE  & Yes         & Yes                  & Yes                      & Yes                 & Yes              & Yes       & Yes       \\
            toxic chemical FE  & Yes         & Yes                  & Yes                      & Yes                 & Yes              & Yes       & Yes       \\
            toxic chemical LTs & Yes         & Yes                  & Yes                      & Yes                 & Yes              & Yes       & Yes       \\ \midrule \midrule
            Observations       & 1,893,689   & 1,893,689            & 1,893,689                & 1,893,689           & 1,893,689        & 1,893,689 & 1,893,689 \\
            $R^2$              & 0.2141      & 0.1185               & 0.1504                   & 0.2803              & 0.4054           & 0.9434    & 0.1796    \\ \bottomrule\bottomrule
        \end{tabular}%
    }
    \begin{minipage}{18cm}
        \vspace{0.05in}
        These results are obtained from estimating model~\ref{eq:mechanisms-source-reduction}. ***, **, and * denote significance levels at the less than $1\%$, $5\%$ and $10\%$, respectively.
    \end{minipage}
\end{table}
    % Please add the following required packages to your document preamble:
% \usepackage{booktabs}
% \usepackage{graphicx}
\begin{table}[H]
    \centering
    \caption{Onsite SRA: Inventory Material Management and Operating and Training Activities}
    \label{tab:mechanisms-onsite-sra-inventory-operating-and-training-activities}
    \resizebox{\columnwidth}{!}{%
        \begin{tabular}{@{}lllllllllll@{}}
            \toprule\toprule
            Onsite SRA         & inventory mgt & better labels & container size & mat handling & imp monitoring & others    & operating pract and training & imp schedule opt & changed prod schedule & imp prod quality \\ \midrule
            treated            & 0.0057**      & -0.0002       & 0.0006         & -0.0046***   & -0.0006*       & 0.0113*** & -0.0026***                   & 0.0020           & 0.0040*               & 0.0003           \\
            & (0.0025)      & (0.0001)      & (0.0005)       & (0.0018)     & (0.0003)       & (0.0032)  & (0.0008)                     & (0.0013)         & (0.0020)              & (0.0002)         \\
            controls           & Yes           & Yes           & Yes            & Yes          & Yes            & Yes       & Yes                          & Yes              & Yes                   & Yes              \\
            year FE            & Yes           & Yes           & Yes            & Yes          & Yes            & Yes       & Yes                          & Yes              & Yes                   & Yes              \\
            facility FE        & Yes           & Yes           & Yes            & Yes          & Yes            & Yes       & Yes                          & Yes              & Yes                   & Yes              \\
            county FE          & Yes           & Yes           & Yes            & Yes          & Yes            & Yes       & Yes                          & Yes              & Yes                   & Yes              \\
            county-pair FE     & Yes           & Yes           & Yes            & Yes          & Yes            & Yes       & Yes                          & Yes              & Yes                   & Yes              \\
            facility state FE  & Yes           & Yes           & Yes            & Yes          & Yes            & Yes       & Yes                          & Yes              & Yes                   & Yes              \\
            toxic chemical FE  & Yes           & Yes           & Yes            & Yes          & Yes            & Yes       & Yes                          & Yes              & Yes                   & Yes              \\
            toxic chemical LTs & Yes           & Yes           & Yes            & Yes          & Yes            & Yes       & Yes                          & Yes              & Yes                   & Yes              \\\midrule\midrule
            Observations       & 1,893,689     & 1,893,689     & 1,893,689      & 1,893,689    & 1,893,689      & 1,893,689 & 1,893,689                    & 1,893,689        & 1,893,689             & 1,893,689        \\
            $R^2$              & 0.1423        & 0.0320        & 0.2326         & 0.3752       & 0.1609         & 0.1736    & 0.1254                       & 0.1631           & 0.3495                & 0.1405           \\ \bottomrule \bottomrule
        \end{tabular}%
    }
    \begin{minipage}{18cm}
        \vspace{0.05in}
        These results are obtained from estimating model~\ref{eq:mechanisms-source-reduction}. ***, **, and * denote significance levels at the less than $1\%$, $5\%$ and $10\%$, respectively.
    \end{minipage}
\end{table}


    \section{Heterogeneous Effects}\label{sec:heterogeneous-effects}

    \subsection{Economic Growth Patterns}\label{subsec:economic-growth-patterns}
    In this subsection, I check whether the increasing onsite releases intensity due to a higher MW floor is peculiar to treated counties with high economic growth patterns. Theory suggests that economic growth patterns are correlated with pollutant emissions with a turning point at higher economic growth~\parencite{grossman1995economic, shapiro2018pollution}. To investigate the differential effect of higher MW on onsite total releases intensity, I estimate the following model:
    \begin{align}
        G_{f,c,i,cp,s,t}^{gdp} &= \beta (Treated \cdot Post \cdot D)_{h,s,t} + \psi (Treated \cdot Post)_{s,t} + \vartheta (Treated \cdot D)_{h,s,t} \nonumber \\
        &\quad + \mu (Post \cdot D)_{h,s,t} + \tau Treated_{s,t} + \rho D_{h,s,t} + \alpha Post_{t} + \delta X_{v,c,t-1} + \omega F_{f,t} \nonumber \\
        &\quad + \gamma_{f} + \phi_{cp} + \eta_{c,t} + \left[\lambda_{t} + \theta_{h} + \sigma_{s} + \zeta_{c} \right] + \varepsilon_{f,c,i,cp,s,t},\label{eq:heterogeneous-onsite-releases-intensity-gdp}
    \end{align}
    where $G_{f,c,i,cp,s,t}^{gdp}$ is the vector of total onsite releases intensity (air e.g., point and fugitive, water and land) in a high GDP county at a manufacturing facility, $f$ through a toxic chemical, $c$ used in industry, $i$ in a cross-border county pair, $cp$ in state, $s$ in the year, $t$. $Post_{t}$ is a dummy that is equal to $1$ if the year $t$ is a post-treatment year, and $0$ otherwise. And $D_{h,s,t}$ is a dummy that is unity for a high gross domestic product (GDP) of county, $h$ in state, $s$ in the year, $t$ and $0$ otherwise (i.e., low GDP). High GDP is defined as those counties with GDP above the median quantile of the GDP distribution of all counties.
    % Please add the following required packages to your document preamble:
% \usepackage{booktabs}
% \usepackage{graphicx}
\begin{table}[H]
    \centering
    \caption{Effect of the MW policy on Onsite Releases Intensity given GDP patterns}
    \label{tab:heterogeneous-onsite-releases-int-GDP}
    \resizebox{\columnwidth}{!}{%
        \begin{tabular}{@{}lllll@{}}
            \toprule\toprule
            Onsite releases intensity (log) & total     & air emissions & water discharge & land releases \\ \midrule
            $Treated \cdot Post \cdot D$    & 0.1803*** & 0.1029**      & 0.1002**        & 8.63e-11      \\
            & (0.0599)  & (0.0512)      & (0.0502)        & (1.79e-10)    \\
            $Treated \cdot Post$            & 0.0740    & 0.0203        & 0.0447          & 0.0226        \\
            & (0.0542)  & (0.0498)      & (0.0391)        & (0.0139)      \\
            controls                        & Yes       & Yes           & Yes             & Yes           \\
            year FE                         & Yes       & Yes           & Yes             & Yes           \\
            facility FE                     & Yes       & Yes           & Yes             & Yes           \\
            county FE                       & Yes       & Yes           & Yes             & Yes           \\
            county-pair FE                  & Yes       & Yes           & Yes             & Yes           \\
            facility state FE               & Yes       & Yes           & Yes             & Yes           \\
            toxic chemical FE               & Yes       & Yes           & Yes             & Yes           \\
            toxic chemical LTs              & Yes       & Yes           & Yes             & Yes           \\ \midrule \midrule
            Observations                    & 1,893,689 & 1,893,689     & 1,893,689       & 1,893,689     \\
            $R^2$                           & 0.7457    & 0.7675        & 0.5929          & 0.5272 \\ \bottomrule\bottomrule
        \end{tabular}%
    }
    \begin{minipage}{18cm}
        \vspace{0.05in}
        These results are obtained from estimating model~\ref{eq:heterogeneous-onsite-releases-intensity-gdp}. Three-way clustered robust standard errors are reported in parentheses, and clustered at the toxic chemical, industry and state levels. ***, **, and * denote significance levels at the less than $1\%$, $5\%$ and $10\%$, respectively.
    \end{minipage}
\end{table}

    The parameter of interest here is the triple-differences parameter $\beta$ which measures the differential impact on onsite total releases intensity due to a higher MW policy in high GDP counties. $\psi$ measures the differential change in onsite total releases intensity in low GDP counties. And $\beta + \psi$ measures the overall relative change in onsite total releases intensity in high GDP counties. To have a causal interpretation, I assume a weaker parallel trends assumption and is only required to hold for one of the groups in triple differences~\parencite{olden2022triple}. The results are presented in Table~\ref{tab:heterogeneous-onsite-releases-int-gdp}. Albeit the triple-difference coefficient is positive $(7.97ppts)$ for high GDP counties. Similarly, the effect on low GDP is also positive $8.58ppts$. However, these effects are not statistically significant, suggesting limited evidence of a differential increase in total onsite releases intensity for both low and high GDP counties. This effect is lower than the baseline estimate of $16.91ppts$. The coefficient sum is significantly positive $(\beta + \psi = 16.55ppts)$ which shows the overall size of the relative change in the increase in total onsite releases intensity for high GDP counties. Further, Table~\ref{tab:heterogeneous-onsite-releases-int-gdp} also shows more limited evidence of a differential increase in total air emissions, surface water discharge and land releases intensities, respectively, for both high and low GDP counties. Lastly, I find that the differential impact on total surface impoundment intensity is significantly positive for both high and low GDP counties, with a size of $0.69ppts$ and $0.85ppts$, respectively. These effects are lower than the baseline of $2.67ppts$. The overall size of the relative change in the increase in total surface impoundment intensity is $1.54ppts$ for high GDP counties.
    \begin{figure}[H]
    \centering
    \includegraphics[width=1\textwidth, height=0.5\textheight,keepaspectratio]{fig_did_total_releases_onsite_int_GDP}
    \caption{Triple-Differences: Onsite Total Releases Intensity}
    \label{fig:heterogeneous-onsite-releases-intensity}
    \begin{minipage}{12cm}
        \vspace{0.05in}
        NOTES: The event study model of equation~\ref{eq:heterogeneous-onsite-releases-intensity-gdp} is $G_{f,c,i,cp,s,t} = \sum_{{e = 2011},{e \neq 2013}}^{2017} \beta (Treated \cdot Post \cdot D)_{h,s,t} + \psi (Treated \cdot Post)_{s,t} + \rho D_{h,s,t} + \delta X_{v,c,t-1} + \omega F_{f,t} + \gamma_{f} + \phi_{cp} + \eta_{c,t} + \left[\lambda_{t} + \theta_{f,h} + \sigma_{s} + \zeta_{c} \right] + \varepsilon_{f,c,i,cp,s,t}$. Three-way clustered robust standard errors are reported in parentheses, and clustered at the toxic chemical, industry and state levels.
    \end{minipage}
\end{figure}

    Figure~\ref{fig:heterogeneous-onsite-releases-intensity-gdp} plots their dynamic trends across counties. It shows a significant differential increase in total onsite releases intensity for high GDP counties after three years of initial treatment date, reaching a magnitude of $21.39ppts$, on average. I find limited evidence of an increase in total onsite releases in low GDP counties throughout the spectrum. Similarly, I find a statistically significant differential positive effect in total air emissions intensity for high GDP counties after the second and third year of post-treatment, reaching a maximum size of $22.34ppts$ in the fourth year of post-treatment. I record no evidence of an increase in total air emissions intensity for low GDP counties. Moreover, these differential positive effects in total air emissions intensity are predominantly driven by both point and fugitive air emissions intensities for high GDP counties after the second and third of initial treatment date. On the other hand, there is a significant differential negative impact on total point air emissions intensity of $15.40ppts$ in the second year of post-treatment, and limited evidence on fugitive air emissions intensities for low GDP counties. Additionally, there is differential negative evidence on total land releases intensities for high and low GDP counties in the second and third year of post-treatment. Lastly, I document significant differential increases in total surface impoundment intensity for high and low GDP counties in the third and fourth year of post-treatment. There is no evidence of significant pre-trends.

    \subsection{Financial Constraints}\label{subsec:financial-constraints}

    \subsection{Industry Concentration}\label{subsec:industry-concentration}

    \subsection{Labour Intensity}\label{subsec:labour-intensity}

    \subsection{Carcinogenic Chemicals}\label{subsec:carcinogenic-chemicals}
    In this subsection, I check whether the MW policy is potentially carcinogenic. To investigate the differential effect of higher MW on onsite total releases intensity of carcinogenic chemicals, I estimate the following model:
    \begin{align}
        G_{f,c,i,cp,s,t}^{carcinogen} &= \beta (Treated \cdot Post \cdot D)_{f,s,t} + \psi (Treated \cdot Post)_{s,t} + \vartheta (Treated \cdot D)_{f,s,t} \nonumber \\
        &\quad + \mu (Post \cdot D)_{f,s,t} + \tau Treated_{s,t} + \rho D_{f,s,t} + \alpha Post_{t} + \delta X_{v,c,t-1} + \omega F_{f,t} \nonumber \\
        &\quad + \gamma_{f} + \phi_{cp} + \eta_{c,t} + \left[\lambda_{t} + \theta_{h} + \sigma_{s} + \zeta_{c} \right] + \varepsilon_{f,c,i,cp,s,t},\label{eq:heterogeneous-onsite-releases-intensity-carcinogens}
    \end{align}
    where $G_{f,c,i,cp,s,t}^{carcinogen}$ is the vector of total onsite releases intensity (air, water and land) of toxic chemicals at a manufacturing facility, $f$ through a toxic chemical, $c$ used in industry, $i$ in a cross-border county pair, $cp$ in state, $s$ in the year, $t$. $Post_{t}$ is a dummy that is equal to $1$ if the year $t$ is a post-treatment year, and $0$ otherwise. And $D_{f,s,t}$ is a dummy that is unity for a carcinogenic chemical at manufacturing facility, $f$ in state, $s$ in the year, $t$ and $0$ otherwise. Carcinogenic chemicals are toxic chemicals that can cause cancer in both humans and animals alike. Examples include benzene, formaldehyde, arsenic, and vinyl chloride, etc.
    % Please add the following required packages to your document preamble:
% \usepackage{booktabs}
% \usepackage{graphicx}
\begin{table}[H]
    \centering
    \caption{Onsite Releases Intensity for Carcinogenic Chemicals}
    \label{tab:heterogeneous-onsite-releases-int-carcinogens}
    \resizebox{\columnwidth}{!}{%
        \begin{tabular}{@{}llllllll@{}}
            \toprule\toprule
            Onsite releases intensity (log) & total     & air emissions & point air & fugitive air & water discharge & land releases & surface impoundment \\ \midrule
            $Treated \cdot Post \cdot D$    & 0.1741    & 0.2555**      & 0.2937**  & 0.1266       & -0.1161         & -0.0672**     & -0.0128             \\
            & (0.1223)  & (0.1178)      & (0.1183)  & (0.1033)     & (0.0724)        & (0.0336)      & (0.0080)            \\
            $Treated \cdot Post$            & 0.0631    & -0.0194       & -0.0488   & -0.0230      & 0.0848**        & 0.0400***     & 0.0185*             \\
            & (0.0543)  & (0.0410)      & (0.0441)  & (0.0400)     & (0.0395)        & (0.0135)      & (0.0101)            \\
            controls                        & Yes       & Yes           & Yes       & Yes          & Yes             & Yes           & Yes                 \\
            year FE                         & Yes       & Yes           & Yes       & Yes          & Yes             & Yes           & Yes                 \\
            facility FE                     & Yes       & Yes           & Yes       & Yes          & Yes             & Yes           & Yes                 \\
            county FE                       & Yes       & Yes           & Yes       & Yes          & Yes             & Yes           & Yes                 \\
            county-pair FE                  & Yes       & Yes           & Yes       & Yes          & Yes             & Yes           & Yes                 \\
            facility state FE               & Yes       & Yes           & Yes       & Yes          & Yes             & Yes           & Yes                 \\
            toxic chemical FE               & Yes       & Yes           & Yes       & Yes          & Yes             & Yes           & Yes                 \\
            toxic chemical LTs              & Yes       & Yes           & Yes       & Yes          & Yes             & Yes           & Yes                 \\ \midrule \midrule
            Observations                    & 1,893,689 & 1,893,689     & 1,893,689 & 1,893,689    & 1,893,689       & 1,893,689     & 1,893,689           \\
            $R^2$                           & 0.7196    & 0.7390        & 0.7122    & 0.6597       & 0.5844          & 0.4999        & 0.1254              \\ \bottomrule\bottomrule
        \end{tabular}%
    }
    \begin{minipage}{18cm}
        \vspace{0.05in}
        These results are obtained from estimating model~\ref{eq:heterogeneous-onsite-releases-intensity-carcinogens}. Three-way clustered robust standard errors are reported in parentheses, and clustered at the toxic chemical, industry and state levels. ***, **, and * denote significance levels at the less than $1\%$, $5\%$ and $10\%$, respectively.
    \end{minipage}
\end{table}

    The parameter of interest here is the triple-differences parameter $\beta$ which measures the differential impact on onsite total releases intensity due to a higher MW policy for carcinogenic chemicals at manufacturing facilities. $\psi$ measures the relative change in onsite total releases intensity for non-carcinogenic chemicals. And $\beta + \psi$ measures the total relative change in onsite total releases intensity for carcinogenic chemicals. The results are presented in Table~\ref{tab:heterogeneous-onsite-releases-int-carcinogens}.
    \input{fig_did_heter_onsite_releases_int_CARCINOGENS}

    \subsection{Clean Air Act Chemicals}\label{subsec:clean-air-act-chemicals}
    This section investigates the question: does higher MW have any significant differential impact on total onsite releases intensities of clean air act chemicals (CAA)? CAA chemicals are toxic chemicals heavily regulated under the CAA of $1970$. The CAA is a comprehensive federal law enacted by the United States Congress to address air pollution control and improve air quality standards across the nation. The primary goals of the CAA are to protect public health and the environment by regulating the emission of harmful air pollutants. To investigate the above question, I estimate the following model:
    \begin{align}
        G_{f,c,i,cp,s,t}^{caa} &= \beta (Treated \cdot Post \cdot D)_{f,s,t} + \psi (Treated \cdot Post)_{s,t} + \vartheta (Treated \cdot D)_{f,s,t} \nonumber \\
        &\quad + \mu (Post \cdot D)_{f,s,t} + \tau Treated_{s,t} + \rho D_{f,s,t} + \alpha Post_{t} + \delta X_{v,c,t-1} + \omega F_{f,t} \nonumber \\
        &\quad + \gamma_{f} + \phi_{cp} + \eta_{c,t} + \left[\lambda_{t} + \theta_{h} + \sigma_{s} + \zeta_{c} \right] + \varepsilon_{f,c,i,cp,s,t},\label{eq:heterogeneous-onsite-releases-intensity-caa}
    \end{align}
    where $G_{f,c,i,cp,s,t}^{caa}$ is the vector of total onsite releases intensity (air, water and land) of toxic chemicals at a manufacturing facility, $f$ through a toxic chemical, $c$ used in industry, $i$ in a cross-border county pair, $cp$ in state, $s$ in the year, $t$. $Post_{t}$ is a dummy that is equal to $1$ if the year $t$ is a post-treatment year, and $0$ otherwise. And $D_{f,s,t}$ is a dummy that is unity for a CAA chemical at manufacturing facility, $f$ in state, $s$ in the year, $t$ and $0$ otherwise.
    % Please add the following required packages to your document preamble:
% \usepackage{booktabs}
% \usepackage{graphicx}
\begin{table}[H]
    \centering
    \caption{Onsite Releases Intensity for CAA Chemicals}
    \label{tab:heterogeneous-onsite-releases-int-caa}
    \resizebox{\columnwidth}{!}{%
        \begin{tabular}{@{}llllllll@{}}
            \toprule\toprule
            Onsite releases intensity (log) & total     & air emissions & point air & fugitive air & water discharge & land releases & surface impoundment \\ \midrule
            $Treated \cdot Post \cdot D$    & 0.0178    & 0.2978        & 0.2369    & -0.1187      & -0.3148***      & -0.0949       & -0.0069             \\
            & (0.2038)  & (0.2107)      & (0.2030)  & (0.1433)     & (0.1048)        & (0.0615)      & (0.0150)            \\
            $Treated \cdot Post$            & 0.1294    & -0.1545       & -0.1555   & 0.1088       & 0.3028***       & 0.1036**      & 0.0282**            \\
            & (0.1633)  & (0.1495)      & (0.1419)  & (0.1032)     & (0.0894)        & (0.0407)      & (0.0141)            \\
            controls                        & Yes       & Yes           & Yes       & Yes          & Yes             & Yes           & Yes                 \\
            year FE                         & Yes       & Yes           & Yes       & Yes          & Yes             & Yes           & Yes                 \\
            facility FE                     & Yes       & Yes           & Yes       & Yes          & Yes             & Yes           & Yes                 \\
            county FE                       & Yes       & Yes           & Yes       & Yes          & Yes             & Yes           & Yes                 \\
            county-pair FE                  & Yes       & Yes           & Yes       & Yes          & Yes             & Yes           & Yes                 \\
            facility state FE               & Yes       & Yes           & Yes       & Yes          & Yes             & Yes           & Yes                 \\
            toxic chemical FE               & Yes       & Yes           & Yes       & Yes          & Yes             & Yes           & Yes                 \\
            toxic chemical LTs              & Yes       & Yes           & Yes       & Yes          & Yes             & Yes           & Yes                 \\\midrule\midrule
            Observations                    & 1,893,689 & 1,893,689     & 1,893,689 & 1,893,689    & 1,893,689       & 1,893,689     & 1,893,689           \\
            $R^2$                           & 0.7463    & 0.7677        & 0.7431    & 0.6970       & 0.5981          & 0.5262        & 0.1438              \\ \bottomrule\bottomrule
        \end{tabular}%
    }
    \begin{minipage}{18cm}
        \vspace{0.05in}
        These results are obtained from estimating model~\ref{eq:heterogeneous-onsite-releases-intensity-caa}. Three-way clustered robust standard errors are reported in parentheses, and clustered at the toxic chemical, industry and state levels. ***, **, and * denote significance levels at the less than $1\%$, $5\%$ and $10\%$, respectively.
    \end{minipage}
\end{table}

    The parameter of interest here is the triple-differences parameter $\beta$ which measures the differential impact on onsite total releases intensity due to a higher MW policy for CAA chemicals at manufacturing facilities. $\psi$ measures the relative change in onsite total releases intensity for non-CAA chemicals. And $\beta + \psi$ measures the overall relative change in onsite total releases intensity for CAA chemicals. The results are presented in Table~\ref{tab:heterogeneous-onsite-releases-int-caa}.
    \begin{figure}[H]
    \centering
    \includegraphics[width=1\textwidth, height=0.5\textheight,keepaspectratio]{fig_did_total_onsite_releases_int_caa}
    \caption{Triple-Differences: Onsite Total Releases Intensity for CAA}
    \label{fig:heterogeneous-onsite-releases-intensity-caa}
    \begin{minipage}{18cm}
        \vspace{0.05in}
        NOTES: The event study model of equation~\ref{eq:heterogeneous-onsite-releases-intensity-caa} is $G_{f,cp,c,t}^{caa} = \sum_{{e = 2011},{e \neq 2013}}^{2017} \beta (Treated \cdot Post \cdot D)_{f,s,t} + \psi (Treated \cdot Post)_{s,t} + \vartheta (Treated \cdot D)_{f,s,t} + \mu (Post \cdot D)_{f,s,t} + \tau Treated_{s,t} + \rho D_{f,s,t} + \alpha Post_{t} + \delta X_{v,c,t-1} + \omega F_{f,t} + \lambda_{t} + \gamma_{f} + \phi_{cp} + \zeta_{c} + \eta_{c,t} + \varepsilon_{f,cp,c,t}$. Three-way clustered robust standard errors are reported in parentheses, and clustered at the toxic chemical, industry and state levels.
    \end{minipage}
\end{figure}

    \subsection{Hazardous Air Pollutants}\label{subsec:hazardous-air-pollutants}
    In this section, I answer the question: does higher MW have a significant differential effect on total releases intensities of hazardous air pollutants (HAPs)? HAPs, also known as air toxics, are pollutants that are known or suspected to cause serious health and environmental effects. These pollutants are regulated under the CAA Amendments of $1990$.~\footnote{\tiny  Examples include heavy metals (e.g., include mercury, lead, cadmium, and chromium are examples of heavy metals that can be released into the air from industrial processes, combustion of fossil fuels, and waste incineration), VOCs (e.g., include benzene, toluene, xylene, and formaldehyde, which are easily emitted into the air from industrial sources, motor vehicles, and consumer products. Exposure to VOCs can cause respiratory problems, neurological effects, and contribute to the formation of ground-level ozone and smog), polycyclic aromatic hydrocarbons (PAHs) (e.g., include organic compounds formed during the incomplete combustion of fossil fuels, wood, and other organic materials. Sources of PAHs is from industrial processes. Some PAHs are known carcinogens and can also cause developmental and reproductive effects), persistent organic compounds (organic compounds that resist degradation in the environment and can accumulate in living organisms. Examples include dioxins, and certain pesticides such as dichlorodiphenyltrichloroethane (DDT) and PCBs. These chemicals can travel long distances through air and water, posing risks to ecosystems and human health), and chlorinated compounds (such as chloroform and vinyl chloride, are byproducts of industrial processes, including chemical manufacturing and waste incineration. Exposure to these chemicals can cause liver and kidney damage, as well as neurological effects). Regulation of hazardous air pollutants under the Clean Air Act involves the development of technology-based standards for industrial sources to control emissions of these pollutants. The EPA establishes Maximum Achievable Control Technology (MACT) standards for specific source categories, such as chemical plants, petroleum refineries, and pulp and paper mills, to reduce emissions of hazardous air pollutants to the maximum extent feasible. Facilities subject to MACT standards are required to install pollution control equipment and implement management practices to minimize emissions of hazardous air pollutants. Compliance with MACT standards helps protect public health and the environment by reducing exposure to toxic air pollutants and preventing adverse health effects.} To investigate the above question, I estimate the following model:
    \begin{align}
        G_{f,c,i,cp,s,t}^{haps} &= \beta (Treated \cdot Post \cdot D)_{f,s,t} + \psi (Treated \cdot Post)_{s,t} + \vartheta (Treated \cdot D)_{f,s,t} \nonumber \\
        &\quad + \mu (Post \cdot D)_{f,s,t} + \tau Treated_{s,t} + \rho D_{f,s,t} + \alpha Post_{t} + \delta X_{v,c,t-1} + \omega F_{f,t} \nonumber \\
        &\quad + \gamma_{f} + \phi_{cp} + \eta_{c,t} + \left[\lambda_{t} + \theta_{h} + \sigma_{s} + \zeta_{c} \right] + \varepsilon_{f,c,i,cp,s,t},\label{eq:heterogeneous-onsite-releases-intensity-haps}
    \end{align}
    where $G_{f,c,i,cp,s,t}^{haps}$ is the vector of total onsite releases intensity (air, water and land) of toxic chemicals at a manufacturing facility, $f$ through a toxic chemical, $c$ used in industry, $i$ in a cross-border county pair, $cp$ in state, $s$ in the year, $t$. $Post_{t}$ is a dummy that is equal to $1$ if the year $t$ is a post-treatment year, and $0$ otherwise. And $D_{f,s,t}$ is a dummy that is unity for a HAPs chemical at manufacturing facility, $f$ in state, $s$ in the year, $t$ and $0$ otherwise.
    % Please add the following required packages to your document preamble:
% \usepackage{booktabs}
% \usepackage{graphicx}
\begin{table}[H]
    \centering
    \caption{Onsite Releases Intensity for HAPs}
    \label{tab:heterogeneous-onsite-releases-int-haps}
    \resizebox{\columnwidth}{!}{%
        \begin{tabular}{@{}llllllll@{}}
            \toprule\toprule
            Onsite releases intensity (log) & total     & air emissions & point air & fugitive air & water discharge & land releases & surface impoundment \\ \midrule
            $Treated \cdot Post \cdot D$    & 0.2598**  & 0.1536        & 0.1362    & -0.0169      & 0.1305          & 0.0500        & 0.0017              \\
            & (0.1274)  & (0.1252)      & (0.0821)  & (0.0916)     & (0.0846)        & (0.0419)      & (0.0081)            \\
            $Treated \cdot Post$            & -0.1057   & -0.0796       & -0.0849   & 0.0229       & -0.0530         & -0.0184       & 0.0139**            \\
            & (0.0939)  & (0.1046)      & (0.0821)  & (0.0725)     & (0.0523)        & (0.0389)      & (0.0060)            \\
            controls                        & Yes       & Yes           & Yes       & Yes          & Yes             & Yes           & Yes                 \\
            year FE                         & Yes       & Yes           & Yes       & Yes          & Yes             & Yes           & Yes                 \\
            facility FE                     & Yes       & Yes           & Yes       & Yes          & Yes             & Yes           & Yes                 \\
            border-county FE                  & Yes       & Yes           & Yes       & Yes          & Yes             & Yes           & Yes                 \\
            toxic chemical FE               & Yes       & Yes           & Yes       & Yes          & Yes             & Yes           & Yes                 \\
            toxic chemical LTs              & Yes       & Yes           & Yes       & Yes          & Yes             & Yes           & Yes                 \\\midrule\midrule
            Observations                    & 1,893,689 & 1,893,689     & 1,893,689 & 1,893,689    & 1,893,689       & 1,893,689     & 1,893,689           \\
            $R^2$                           & 0.7196    & 0.7388        & 0.7119    & 0.6594       & 0.5850          & 0.4999        & 0.1252              \\ \bottomrule\bottomrule
        \end{tabular}%
    }
    \begin{minipage}{18cm}
        \vspace{0.05in}
        These results are obtained from estimating model~\ref{eq:heterogeneous-onsite-releases-intensity-haps}. Three-way clustered robust standard errors are reported in parentheses, and clustered at the toxic chemical, industry and state levels. ***, **, and * denote significance levels at the less than $1\%$, $5\%$ and $10\%$, respectively.
    \end{minipage}
\end{table}

    The parameter of interest here is the triple-differences parameter $\beta$ which measures the differential impact on onsite total releases intensity due to a higher MW policy for HAPs chemicals at manufacturing facilities. $\psi$ measures the relative change in onsite total releases intensity for non-HAPs chemicals. And $\beta + \psi$ measures the overall relative change in onsite total releases intensity for HAPs chemicals. The results are presented in Table~\ref{tab:heterogeneous-onsite-releases-int-haps}.
    \input{fig_did_heter_onsite_releases_int_haps}

    \subsection{Persistent Bio-accumulative Toxic Chemicals}\label{subsec:persistent-bioaccumulative-toxic-chemicals}
    This sections asks the question: does a higher MW regime exert significant differential impact on persistent bio-accumulative toxic chemicals (PBTs)?~\footnote{\tiny PBTs are a group of chemicals characterized by their persistence in the environment, ability to accumulate in living organisms, and toxicity. These chemicals pose significant risks to human health and the environment due to their long-term effects and potential for biomagnification in food chains. Examples of PBTs include certain persistent organic pollutants (POPs), such as polychlorinated biphenyls (PCBs), dioxins, and certain pesticides like dichlorodiphenyltrichloroethane (DDT). These chemicals have been widely used in industrial processes, agriculture, and consumer products, but their harmful effects have led to regulatory action to restrict or phase out their production and use. Due to their persistence, bioaccumulation potential, and toxicity, PBTs are of particular concern to environmental regulators and policymakers. Efforts to reduce PBT emissions and exposure often involve regulatory measures, such as bans or restrictions on their use, as well as pollution prevention and cleanup programs to mitigate their impact on human health and the environment. PBTs are heavily regulated under the Toxic Substances Control Act (TSCA) of $1976$ by the US EPA. Some of these regulatory actions by the EPA on manufacturing facilities on the use of PBTs include chemical testing, reporting and recording keeping, restriction and bans, risk management assessment, lebelling and notification, etc.} To investigate the above question, I estimate the following model:
    \begin{align}
        G_{f,c,i,cp,s,t}^{pbts} &= \beta (Treated \cdot Post \cdot D)_{f,s,t} + \psi (Treated \cdot Post)_{s,t} + \vartheta (Treated \cdot D)_{f,s,t} \nonumber \\
        &\quad + \mu (Post \cdot D)_{f,s,t} + \tau Treated_{s,t} + \rho D_{f,s,t} + \alpha Post_{t} + \delta X_{v,c,t-1} + \omega F_{f,t} \nonumber \\
        &\quad + \gamma_{f} + \phi_{cp} + \eta_{c,t} + \left[\lambda_{t} + \theta_{h} + \sigma_{s} + \zeta_{c} \right] + \varepsilon_{f,c,i,cp,s,t},\label{eq:heterogeneous-onsite-releases-intensity-pbts}
    \end{align}
    where $G_{f,c,i,cp,s,t}^{pbts}$ is the vector of total onsite releases intensity (air, water and land) of toxic chemicals at a manufacturing facility, $f$ through a toxic chemical, $c$ used in industry, $i$ in a cross-border county pair, $cp$ in state, $s$ in the year, $t$. $Post_{t}$ is a dummy that is equal to $1$ if the year $t$ is a post-treatment year, and $0$ otherwise. And $D_{f,s,t}$ is a dummy that is unity for a PBT chemical at manufacturing facility, $f$ in state, $s$ in the year, $t$ and $0$ otherwise.
    % Please add the following required packages to your document preamble:
% \usepackage{booktabs}
% \usepackage{graphicx}
\begin{table}[H]
    \centering
    \caption{Onsite Releases Intensity for PBTs}
    \label{tab:heterogeneous-onsite-releases-int-pbts}
    \resizebox{\columnwidth}{!}{%
        \begin{tabular}{@{}llllllll@{}}
            \toprule\toprule
            Onsite releases intensity (log) & total     & air emissions & point air & fugitive air & water discharge & land releases & surface impoundment \\ \midrule
            $Treated \cdot Post \cdot D$    & -0.2412   & -0.1003       & -0.0530   & 0.0577       & -0.1485         & -0.0417       & -0.0025             \\
            & (0.1521)  & (0.1661)      & (0.1246)  & (0.1265)     & (0.1094)        & (0.0421)      & (0.0137)            \\
            $Treated \cdot Post$            & 0.1860*** & 0.0800        & 0.0265    & 0.0120       & 0.1018*         & 0.0420***     & 0.0237*             \\
            & (0.0704)  & (0.0527)      & (0.0541)  & (0.0481)     & (0.0518)        & (0.0151)      & (0.0129)            \\
            controls                        & Yes       & Yes           & Yes       & Yes          & Yes             & Yes           & Yes                 \\
            year FE                         & Yes       & Yes           & Yes       & Yes          & Yes             & Yes           & Yes                 \\
            facility FE                     & Yes       & Yes           & Yes       & Yes          & Yes             & Yes           & Yes                 \\
            county FE                       & Yes       & Yes           & Yes       & Yes          & Yes             & Yes           & Yes                 \\
            county-pair FE                  & Yes       & Yes           & Yes       & Yes          & Yes             & Yes           & Yes                 \\
            facility state FE               & Yes       & Yes           & Yes       & Yes          & Yes             & Yes           & Yes                 \\
            toxic chemical FE               & Yes       & Yes           & Yes       & Yes          & Yes             & Yes           & Yes                 \\
            toxic chemical LTs              & Yes       & Yes           & Yes       & Yes          & Yes             & Yes           & Yes                 \\\midrule\midrule
            Observations                    & 1,893,689 & 1,893,689     & 1,893,689 & 1,893,689    & 1,893,689       & 1,893,689     & 1,893,689           \\
            $R^2$                           & 0.7458    & 0.7675        & 0.7430    & 0.6970       & 0.5938          & 0.5260        & 0.1438              \\ \bottomrule\bottomrule
        \end{tabular}%
    }
    \begin{minipage}{18cm}
        \vspace{0.05in}
        These results are obtained from estimating model~\ref{eq:heterogeneous-onsite-releases-intensity-pbts}. Three-way clustered robust standard errors are reported in parentheses, and clustered at the toxic chemical, industry and state levels. ***, **, and * denote significance levels at the less than $1\%$, $5\%$ and $10\%$, respectively.
    \end{minipage}
\end{table}

    The parameter of interest here is the triple-differences parameter $\beta$ which measures the differential impact on onsite total releases intensity due to a higher MW policy for PBTs chemicals at manufacturing facilities. $\psi$ measures the relative change in onsite total releases intensity for non-PBTs chemicals. And $\beta + \psi$ measures the overall relative change in onsite total releases intensity for PBTs chemicals. The results are presented in Table~\ref{tab:heterogeneous-onsite-releases-int-pbts}.
    \begin{figure}[H]
    \centering
    \includegraphics[width=1\textwidth, height=0.5\textheight,keepaspectratio]{fig_did_total_onsite_releases_int_pbts}
    \caption{Triple-Differences: Onsite Total Releases Intensity for PBTs}
    \label{fig:heterogeneous-onsite-releases-intensity-pbts}
    \begin{minipage}{18cm}
        \vspace{0.05in}
        NOTES: The event study model of equation~\ref{eq:heterogeneous-onsite-releases-intensity-pbts} is $G_{f,cp,c,t}^{pbts} = \sum_{{e = 2011},{e \neq 2013}}^{2017} \beta (Treated \cdot Post \cdot D)_{f,s,t} + \psi (Treated \cdot Post)_{s,t} + \vartheta (Treated \cdot D)_{f,s,t} + \mu (Post \cdot D)_{f,s,t} + \tau Treated_{s,t} + \rho D_{f,s,t} + \alpha Post_{t} + \delta X_{v,c,t-1} + \omega F_{f,t} + \lambda_{t} + \gamma_{f} + \phi_{cp} + \zeta_{c} + \eta_{c,t} + \varepsilon_{f,cp,c,t}$. Three-way clustered robust standard errors are reported in parentheses, and clustered at the toxic chemical, industry and state levels.
    \end{minipage}
\end{figure}


    \section{Conclusions}\label{sec:conclusions}

    \newpage
    \section*{Appendices}\label{sec:appendices}
    \begin{appendices}
        \renewcommand\thesection{\Roman{section}} % Use Roman numerals for section numbers in appendices


        \section{State Minimum Wage and Balance Tests}\label{sec:state-minimum-wage-and-balance-tests}
        % Please add the following required packages to your document preamble:
% \usepackage{booktabs}
% \usepackage{graphicx}
\begin{table}[H]
    \centering
    \caption{Minimum Wage Changes in US States from $2011-2017$}
    \label{tab:states-mw-changes}
    \scalebox{0.65}{
%        \resizebox{\columnwidth}{!}{%
        \begin{tabular}{@{}llllllllll@{}}
            \toprule \toprule
            states         & 2011 & 2012 & 2013  & 2014 & 2015 & 2016 & 2017 & start MW & end MW \\ \midrule
            Alaska         & 0    & 0    & 0     & 0    & 1    & 1    & 0.05 & 7.75     & 9.8    \\
            Arkansas       & 0    & 0    & 0     & 0    & 1.25 & 0.5  & 0.5  & 6.25     & 8.5    \\
            Arizona        & 0.1  & 0.3  & 0.15  & 0.1  & 0.15 & 0    & 1.95 & 7.35     & 10     \\
            California     & 0    & 0    & 0     & 1    & 0    & 1    & 0.5  & 8        & 10.5   \\
            Colorado       & 0.12 & 0.28 & 0.14  & 0.22 & 0.23 & 0.08 & 0.99 & 7.36     & 9.3    \\
            Connecticut    & 0    & 0    & 0     & 0.45 & 0.45 & 0.45 & 0.5  & 8.25     & 10.1   \\
            Delaware       & 0    & 0    & 0     & 0.5  & 0.5  & 0    & 0    & 7.25     & 8.25   \\
            Florida        & 0    & 0.42 & 0.12  & 0.14 & 0.12 & 0    & 0.05 & 7.21     & 8.1    \\
            Georgia        & 0    & 0    & 0     & 0    & 0    & 0    & 0    & 5.15     & 5.15   \\
            Hawaii         & 0    & 0    & 0     & 0    & 0.5  & 0.75 & 0.75 & 7.25     & 9.25   \\
            Iowa           & 0    & 0    & 0     & 0    & 0    & 0    & 0    & 7.25     & 7.25   \\
            Idaho          & 0    & 0    & 0     & 0    & 0    & 0    & 0    & 7.25     & 7.25   \\
            Illinois       & 0    & 0    & 0     & 0    & 0    & 0    & 0    & 8.25     & 8.25   \\
            Indiana        & 0    & 0    & 0     & 0    & 0    & 0    & 0    & 7.25     & 7.25   \\
            Kansas         & 0    & 0    & 0     & 0    & 0    & 0    & 0    & 7.25     & 7.25   \\
            Kentucky       & 0    & 0    & 0     & 0    & 0    & 0    & 0    & 7.25     & 7.25   \\
            Massachusetts  & 0    & 0    & 0     & 0    & 1    & 1    & 1    & 8        & 11     \\
            Maryland       & 0    & 0    & 0     & 0    & 1    & 0.5  & 0.5  & 7.25     & 9.25   \\
            Maine          & 0    & 0    & 0     & 0    & 0    & 0    & 1.5  & 7.5      & 9      \\
            Michigan       & 0    & 0    & 0     & 0.75 & 0    & 0.35 & 0.4  & 7.4      & 8.9    \\
            Minnesota      & 0    & 0    & -0.01 & 1.85 & 1    & 0.5  & 0    & 6.16     & 9.5    \\
            Missouri       & 0    & 0    & 0.1   & 0.15 & 0.15 & 0    & 0.05 & 7.25     & 7.7    \\
            Montana        & 0.1  & 0.3  & 0.15  & 0.1  & 0.15 & 0    & 0.1  & 7.35     & 8.15   \\
            North Carolina & 0    & 0    & 0     & 0    & 0    & 0    & 0    & 7.25     & 7.25   \\
            North Dakota   & 0    & 0    & 0     & 0    & 0    & 0    & 0    & 7.25     & 7.25   \\
            Nebraska       & 0    & 0    & 0     & 0    & 0.75 & 1    & 0    & 7.25     & 9      \\
            New Hampshire  & 0    & 0    & 0     & 0    & 0    & 0    & 0    & 7.25     & 7.25   \\
            New Jersey     & 0    & 0    & 0     & 1    & 0.13 & 0    & 0.06 & 7.25     & 8.44   \\
            New Mexico     & 0    & 0    & 0     & 0    & 0    & 0    & 0    & 7.5      & 7.5    \\
            Nevada         & 0.7  & 0    & 0     & 0    & 0    & 0    & 0    & 8.25     & 8.25   \\
            New York       & 0    & 0    & 0     & 0.75 & 0.75 & 0.25 & 0.7  & 7.25     & 9.7    \\
            Ohio           & 0.1  & 0.3  & 0.15  & 0.1  & 0.15 & 0    & 0.05 & 7.4      & 8.1    \\
            Oklahoma       & 0    & 0    & 0     & 0    & 0    & 0    & 0    & 7.25     & 7.25   \\
            Oregon         & 0.1  & 0.3  & 0.15  & 0.15 & 0.15 & 0.5  & 0.5  & 8.5      & 10.25  \\
            Pennsylvania   & 0    & 0    & 0     & 0    & 0    & 0    & 0    & 7.25     & 7.25   \\
            Rhode Island   & 0    & 0    & 0.35  & 0.25 & 1    & 0.6  & 0    & 7.4      & 9.6    \\
            South Dakota   & 0    & 0    & 0     & 0    & 1.25 & 0.05 & 0.1  & 7.25     & 8.65   \\
            Texas          & 0    & 0    & 0     & 0    & 0    & 0    & 0    & 7.25     & 7.25   \\
            Utah           & 0    & 0    & 0     & 0    & 0    & 0    & 0    & 7.25     & 7.25   \\
            Virgina        & 0    & 0    & 0     & 0    & 0    & 0    & 0    & 7.25     & 7.25   \\
            Vermont        & 0.09 & 0.31 & 0.14  & 0.13 & 0.42 & 0.45 & 0.4  & 8.15     & 10     \\
            Washington     & 0.12 & 0.37 & 0.15  & 0.13 & 0.15 & 0    & 1.53 & 8.67     & 11     \\
            Wisconsin      & 0    & 0    & 0     & 0    & 0    & 0    & 0    & 7.25     & 7.25   \\
            West Virginia  & 0    & 0    & 0     & 0    & 0.75 & 0.75 & 0    & 7.25     & 8.75   \\
            Wyoming        & 0    & 0    & 0     & 0    & 0    & 0    & 0    & 5.15     & 5.15   \\ \bottomrule\bottomrule
        \end{tabular}%
%        }
    }

\end{table}
        \begin{table}[H]
    \centering
    \caption{Descriptive Statistics: Treated v. Control Border Counties}
    \label{tab:descriptive-statistics-control-border-counties}
    \begin{tabular}{lrrrr}
        \toprule \toprule
        Variable                                     & Mean  & SD     & T     & C     \\ \midrule
        GDP per capita (1000's)                      & 44.92 & 8.56   & 45.07 & 44.89 \\
        industry employment (1000's)                 & 43.18 & 39.26  & 46.88 & 42.48 \\
        annual average establishments                & 4.88  & 12.57  & 3.19  & 5.20  \\
        chemical ancillary use (onsite)              & 0.21  & 0.41   & 0.28  & 0.20  \\
        chemical formulation component (onsite)      & 0.32  & 0.47   & 0.31  & 0.33  \\
        chemical manufacturing aid (onsite)          & 0.10  & 0.30   & 0.15  & 0.09  \\
        max number of chemicals at facility (onsite) & 3.86  & 1.52   & 3.86  & 3.87  \\
        entire facility (onsite)                     & 1.00  & 0.02   & 1.00  & 1.00  \\
        private facility (onsite)                    & 1.00  & 0.01   & 1.00  & 1.00  \\
        imported chemicals at facility (onsite)      & 0.06  & 0.23   & 0.11  & 0.04  \\
        produced chemicals at facility (onsite)      & 0.19  & 0.39   & 0.33  & 0.16  \\
        production ratio or activity index (onsite)  & 3.07  & 485.56 & 13.70 & 1.06  \\ \bottomrule\bottomrule
    \end{tabular}
    \begin{minipage}{13.5cm}
        \vspace{0.05in}
        \tiny NOTES: The table contains county-level descriptive statistics as of the year immediately before the first initial MW change. The sample is restricted to border counties in treated and control states (See Table~\ref{tab:states-mw-adjustments-t-and-c}).
    \end{minipage}
\end{table}

        \begin{table}[H]
    \centering
    \caption{Descriptive Statistics: Treated v. Control Border States}
    \label{tab:descriptive-statistics-control-border-states}
    \begin{tabular}{lrrrr}
        \toprule \toprule
        Variable                                     & Mean  & SD      & T     & C     \\ \midrule\midrule
        GDP per capita (1000's)                      & 45.67 & 8.92    & 43.95 & 46.14 \\
        industry employment (1000's)                 & 46.25 & 47.71   & 48.87 & 45.52 \\
        annual average establishments                & 9.14  & 23.47   & 3.92  & 10.59 \\
        chemical ancillary use (onsite)              & 0.19  & 0.39    & 0.23  & 0.18  \\
        chemical formulation component (onsite)      & 0.24  & 0.43    & 0.24  & 0.24  \\
        chemical manufacturing aid (onsite)          & 0.10  & 0.30    & 0.12  & 0.09  \\
        max number of chemicals at facility (onsite) & 3.78  & 1.61    & 3.69  & 3.80  \\
        entire facility (onsite)                     & 1.00  & 0.04    & 1.00  & 1.00  \\
        private facility (onsite)                    & 1.00  & 0.01    & 1.00  & 1.00  \\
        imported chemicals at facility (onsite)      & 0.06  & 0.25    & 0.08  & 0.06  \\
        produced chemicals at facility (onsite)      & 0.27  & 0.44    & 0.33  & 0.25  \\
        production ratio or activity index (onsite)  & 12.07 & 1130.57 & 51.29 & 1.19  \\ \bottomrule\bottomrule
    \end{tabular}
    \begin{minipage}{13.5cm}
        \vspace{0.05in}
        \tiny NOTES: The table contains state-level descriptive statistics as of the year immediately before the first initial MW change. The sample is restricted to border counties in treated and control states (See Table~\ref{tab:states-mw-adjustments-t-and-c}).
    \end{minipage}
\end{table}

        \begin{figure}[H]
    \centering
    \includegraphics[width=1\textwidth, height=0.45\textheight]{fig_pre_evolution}
    \caption{County-level Macroeconomic Trends in Border Counties}
    \label{fig:county-level-macroeconomic-trends-in-border-counties}
    \begin{minipage}{18cm}
        \vspace{0.05in}
        {NOTES: This figure is obtained from estimating this equation $y_{cp,t} = \sum_{t = 2011}^{2013} \beta (Treated \cdot B)_{s,t} + \lambda_{t} + \Phi_{cp} + \zeta_{cp,t} + \epsilon_{cp,t}$. Where $y_{cp,t}$ is the vector of observables. Treated is the grouping variable that is unity for the treated states and zero for the control states. $B_{t}$ is a dummy variable with three levels of time, $2011$, $2012$, and $2013$. $\beta$ is the parameter vector of coefficients. $\lambda_{t}$ is the year fixed effects; $\Phi_{cp}$ is the border-county pair fixed effects; and $\zeta_{cp,t}$ is the border-county-pair-year fixed effects. $\epsilon_{cp,t}$ is the error term. Robust standard errors are clustered at the state level. \par}
    \end{minipage}
\end{figure}
        \begin{figure}[H]
    \centering
    \includegraphics[width=1\textwidth, height=0.45\textheight]{fig_pre_evolution_state}
    \caption{State-level Macroeconomic Trends in Border States}
    \label{fig:state-level-macroeconomic-trends-in-border-states}
    \begin{minipage}{18cm}
        \vspace{0.05in}
        {NOTES: This figure is obtained from estimating this equation $y_{s,t} = \sum_{t = 2011}^{2013} \beta (Treated \cdot B)_{s,t} + \lambda_{t} + \Phi_{sp} + \zeta_{sp,t} + \epsilon_{s,t}$. Where $y_{s,t}$ is the vector of observables. Treated is the grouping variable that is unity for the treated states and zero for the control states. $B_{t}$ is a dummy variable with three levels of time, $2011$, $2012$, and $2013$. $\beta$ is the state-level parameter vector of coefficients. $\lambda_{t}$ is the year fixed effects; $\Phi_{sp}$ is the border-state pair fixed effects; and $\zeta_{sp,t}$ is the border-state-pair-year fixed effects. $\epsilon_{s,t}$ is the error term. Robust standard errors are clustered at the state level. \par}
    \end{minipage}
\end{figure}


        \section{List of Chemicals}\label{sec:list-of-chemicals}
        \begin{table}[H]
    \centering
    \caption{Analyzed Chemicals}
    \label{tab:analyzed-chemicals}
%    \scalebox{0.35}{
    \resizebox{\textwidth}{!}{
        \begin{tabular}{llllllllllll}
            \toprule\toprule
            chemical name                                                              & classification & attribute             & onsite & offsite & potw & chemical name                                                                                                      & classification & attribute & onsite & offsite & potw\\
            \midrule
            1-Chloro-1,1-difluoroethane (HCFC-142b)                                    & TRI            & ancillary use         & yes    & yes     & NA   & Ethylene                                                                                                           & TRI & formulation component & yes & yes & yes\\
            1,1,1-Trichloroethane                                                      & TRI            & clean air act         & yes    & yes     & NA   & Ethylene glycol                                                                                                    & TRI            & clean air act & yes & yes & yes\\
            1,2-Butylene oxide                                                         & TRI            & carcinogenic          & yes    & yes     & NA   & Ethylene oxide                                                                                                     & TRI            & carcinogenic          & yes & yes & NA\\
            1,2-Dibromoethane                                                          & TRI            & carcinogenic          & yes    & yes     & NA   & Fluorine                                                                                                           & TRI            & article component     & yes & NA & NA\\
            1,2-Dichlorobenzene                                                        & TRI            & formulation component & yes    & yes     & yes  & Folpet                                                                                                             & TRI            & formulation component & yes & yes & NA\\
            1,2-Dichloroethane                                                         & TRI            & carcinogenic          & yes    & yes     & yes  & Fomesafen                                                                                                          & TRI            & formulation component & yes & yes & yes\\
            1,2-Dichloropropane                                                        & TRI            & carcinogenic          & yes    & NA      & NA   & Formaldehyde                                                                                                       & TRI            & carcinogenic          & yes & yes & yes\\
            1,2,4-Trichlorobenzene                                                     & TRI            & clean air act         & yes    & yes     & NA   & Formic acid                                                                                                        & TRI            & formulation component & yes & yes & yes\\
            1,2,4-Trimethylbenzene                                                     & TRI            & formulation component & yes    & yes     & yes  & Hexachlorobenzene                                                                                                  & PBT & carcinogenic & yes & yes & yes\\
            1,3-Butadiene                                                              & TRI            & carcinogenic          & yes    & yes     & yes  & Hydrazine                                                                                                          & TRI            & carcinogenic          & yes    & yes     & NA   \\
            1,3-Phenylenediamine                                                       & TRI            & formulation component & yes    & NA      & yes  & Hydrochloric acid (acid aerosols including mists, vapors, gas, fog, and other airborne forms of any particle size) & TRI & clean air act & yes & yes & NA\\
            1,4-Dioxane                                                                & TRI            & carcinogenic          & yes    & yes     & yes  & Hydrogen cyanide                                                                                                   & TRI            & article component     & yes & yes & yes\\
            2-Ethoxyethanol                                                            & TRI            & formulation component & yes    & yes     & NA   & Hydrogen fluoride                                                                                                  & TRI            & clean air act & yes & yes & yes\\
            2-Methoxyethanol                                                           & TRI            & article component     & yes    & yes     & yes  & Hydroquinone                                                                                                       & TRI            & clean air act & yes & yes & yes\\
            2-Phenylphenol                                                             & TRI            & ancillary use         & yes    & yes     & NA   & Isobutyraldehyde                                                                                                   & TRI            & article component & yes & yes & NA\\
            2,2-Bis(bromomethyl)-1,3-propanediol                                       & TRI            & carcinogenic          & yes    & NA      & NA   & Lead                                                                                                               & PBT            & carcinogenic & yes & yes & yes\\
            2,2-Dichloro-1,1,1-trifluoroethane (HCFC-123)                              & TRI            & manufacturing aid     & yes    & yes     & yes  & Lead compounds & PBT & clean air act & yes & yes & yes\\
            2,4-D                                                                      & TRI            & carcinogenic          & yes    & yes     & NA   & Lithium carbonate                                                                                                  & TRI            & metal restricted      & yes    & yes & yes\\
            2,4-D 2-butoxyethyl ester                                                  & TRI            & carcinogenic          & yes    & yes     & NA   & m-Cresol                                                                                                           & TRI            & clean air act & yes & yes & NA\\
            2,4-D 2-ethylhexyl ester                                                   & TRI            & carcinogenic          & yes    & yes     & NA   & m-Xylene                                                                                                           & TRI            & clean air act         & yes & yes & NA\\
            2,4-Dimethylphenol                                                         & TRI            & article component     & yes    & yes     & NA   & Maleic anhydride                                                                                                   & TRI            & clean air act & yes & yes & yes\\
            2,4-Dinitrophenol                                                          & TRI            & clean air act         & yes    & NA      & NA   & Manganese                                                                                                          & TRI            & clean air act         & yes    & yes & yes\\
            2,4-Dinitrotoluene                                                         & TRI            & carcinogenic          & yes    & yes     & NA   & Manganese compounds                                                                                                & TRI            & clean air act & yes & yes & yes\\
            2,6-Dinitrotoluene                                                         & TRI            & carcinogenic          & yes    & NA      & NA   & Mercury                                                                                                            & PBT            & clean air act         & yes    & yes     & yes  \\
            3-Iodo-2-propynyl butylcarbamate                                           & TRI            & formulation component & yes    & yes     & yes  & Mercury compounds & PBT & clean air act & yes & yes & yes\\
            3,3'-Dichlorobenzidine dihydrochloride                                     & TRI            & carcinogenic          & yes    & yes     & yes  & Methanol                                                                                                           & TRI & clean air act & yes & yes & yes\\
            4,4'-Isopropylidenediphenol                                                & TRI            & formulation component & yes    & yes     & yes  & Methoxone                                                                                                          & TRI            & carcinogenic & yes & yes & NA\\
            4,4'-Methylenebis(2-chloroaniline)                                         & TRI            & carcinogenic          & yes    & NA      & NA   & Methyl acrylate                                                                                                    & TRI & formulation component & yes & yes & yes\\
            4,4'-Methylenedianiline                                                    & TRI            & carcinogenic          & yes    & yes     & NA   & Methyl isobutyl ketone                                                                                             & TRI            & carcinogenic & yes & yes & yes\\
            Acetaldehyde                                                               & TRI            & carcinogenic          & yes    & yes     & yes  & Methyl methacrylate                                                                                                & TRI            & clean air act         & yes & yes & yes\\
            Acetamide                                                                  & TRI            & carcinogenic          & yes    & NA      & NA   & Methyl tert-butyl ether                                                                                            & TRI            & clean air act         & yes & yes & yes\\
            Acetonitrile                                                               & TRI            & clean air act         & yes    & yes     & yes  & Mixture                                                                                                            & TRI            & formulation component & yes & yes & yes\\
            Acetophenone                                                               & TRI            & clean air act         & yes    & yes     & yes  & Molybdenum trioxide                                                                                                & TRI            & metal restricted & yes & yes & yes\\
            Acrolein                                                                   & TRI            & clean air act         & yes    & yes     & yes  & n-Butyl alcohol                                                                                                    & TRI            & formulation component & yes & yes & yes\\
            Acrylamide                                                                 & TRI            & carcinogenic          & yes    & yes     & yes  & n-Hexane                                                                                                           & TRI            & clean air act         & yes    & yes     & yes  \\
            Acrylic acid                                                               & TRI            & clean air act         & yes    & yes     & yes  & N-Methyl-2-pyrrolidone                                                                                             & TRI            & formulation component & yes & yes & yes\\
            Acrylonitrile                                                              & TRI            & carcinogenic          & yes    & yes     & yes  & N-Methylolacrylamide                                                                                               & TRI            & others                & yes    & yes & yes\\
            Allyl alcohol                                                              & TRI            & formulation component & yes    & yes     & yes  & N-Nitrosodi-n-propylamine                                                                                          & TRI & carcinogenic & yes & yes & NA\\
            Allyl chloride                                                             & TRI            & clean air act         & yes    & yes     & NA   & N,N-Dimethylaniline                                                                                                & TRI            & clean air act & yes & yes & yes\\
            Aluminum (fume or dust)                                                    & TRI            & metal restricted      & yes    & yes     & NA   & N,N-Dimethylformamide                                                                                              & TRI & clean air act & yes & yes & yes\\
            Aluminum oxide (fibrous forms)                                             & TRI            & metal restricted      & yes    & yes     & yes  & Naphthalene                                                                                                        & TRI            & carcinogenic & yes & yes & yes\\
            Ammonia                                                                    & TRI            & formulation component & yes    & yes     & yes  & Nickel                                                                                                             & TRI            & carcinogenic          & yes    & yes     & yes  \\
            Aniline                                                                    & TRI            & clean air act         & yes    & yes     & yes  & Nickel compounds                                                                                                   & TRI            & carcinogenic          & yes    & yes     & yes  \\
            Anthracene                                                                 & TRI            & formulation component & yes    & yes     & yes  & Nicotine and salts                                                                                                 & TRI            & formulation component & yes & yes & yes\\
            Antimony                                                                   & TRI            & clean air act         & yes    & yes     & yes  & Nitrate compounds (water dissociable; reportable only when in aqueous solution) & TRI & formulation component & yes & yes & yes\\
            Antimony compounds                                                         & TRI            & clean air act         & yes    & yes     & yes  & Nitric acid                                                                                                        & TRI            & formulation component & yes & yes & yes\\
            Arsenic                                                                    & TRI            & carcinogenic          & yes    & yes     & yes  & Nitrobenzene                                                                                                       & TRI            & carcinogenic          & yes    & NA      & NA   \\
            Arsenic compounds                                                          & TRI            & clean air act         & yes    & yes     & yes  & o-Cresol                                                                                                           & TRI            & clean air act         & yes    & yes & NA\\
            Asbestos (friable)                                                         & TRI            & carcinogenic          & yes    & yes     & NA   & o-Toluidine                                                                                                        & TRI            & carcinogenic          & yes    & yes & yes\\
            Atrazine                                                                   & TRI            & formulation component & yes    & yes     & yes  & o-Xylene                                                                                                           & TRI            & clean air act         & yes    & yes & yes\\
            Barium                                                                     & TRI            & metal restricted      & yes    & NA      & NA   & Octachlorostyrene                                                                                                  & PBT            & others                & yes    & yes     & NA   \\
            Barium compounds (except for barium sulfate (CAS No. 7727-43-7))           & TRI            & metal restricted & yes & yes & yes & Oxadiazon & TRI & article component & yes & yes & yes\\
            Benzene                                                                    & TRI            & carcinogenic          & yes    & yes     & yes  & Ozone                                                                                                              & TRI            & ancillary use         & yes    & NA      & NA   \\
            Benzo[g,h,i]perylene                                                       & PBT            & clean air act         & yes    & yes     & yes  & p-Chloroaniline                                                                                                    & TRI            & carcinogenic & yes & yes & yes\\
            Benzoyl peroxide                                                           & TRI            & formulation component & yes    & yes     & yes  & p-Cresol                                                                                                           & TRI            & clean air act & yes & yes & NA\\
            Beryllium                                                                  & TRI            & carcinogenic          & yes    & yes     & NA   & p-Xylene                                                                                                           & TRI            & clean air act         & yes    & yes     & NA   \\
            Biphenyl                                                                   & TRI            & clean air act         & yes    & yes     & yes  & Pentachlorobenzene                                                                                                 & PBT            & others                & yes    & yes     & NA   \\
            Boron trichloride                                                          & TRI            & metal restricted      & yes    & NA      & NA   & Pentachlorophenol                                                                                                  & TRI            & carcinogenic & yes & yes & NA\\
            Boron trifluoride                                                          & TRI            & metal restricted      & yes    & yes     & NA   & Peracetic acid                                                                                                     & TRI            & formulation component & yes & yes & yes\\
            Bromomethane                                                               & TRI            & clean air act         & yes    & NA      & NA   & Permethrin                                                                                                         & TRI            & formulation component & yes & yes & NA\\
            Bromoxynil octanoate                                                       & TRI            & formulation component & yes    & yes     & NA   & Phenanthrene                                                                                                       & TRI            & clean air act & yes & yes & yes\\
            Butyl acrylate                                                             & TRI            & formulation component & yes    & yes     & yes  & Phenol                                                                                                             & TRI            & clean air act         & yes & yes & yes\\
            Cadmium                                                                    & TRI            & carcinogenic          & yes    & yes     & yes  & Phosgene                                                                                                           & TRI            & clean air act         & yes    & NA      & NA   \\
            Cadmium compounds                                                          & TRI            & carcinogenic          & yes    & yes     & yes  & Phosphorus (yellow or white)                                                                                       & TRI            & clean air act & yes & NA & yes\\
            Carbon disulfide                                                           & TRI            & clean air act         & yes    & yes     & yes  & Phthalic anhydride                                                                                                 & TRI            & clean air act & yes & yes & yes\\
            Carbonyl sulfide                                                           & TRI            & clean air act         & yes    & NA      & yes  & Picloram                                                                                                           & TRI            & formulation component & yes & yes & NA\\
            Catechol                                                                   & TRI            & carcinogenic          & yes    & NA      & yes  & Polychlorinated biphenyls                                                                                          & PBT            & carcinogenic          & yes & yes & yes\\
            Certain glycol ethers                                                      & TRI            & clean air act         & yes    & yes     & yes  & Polycyclic aromatic compounds & PBT & carcinogenic & yes & yes & yes\\
            Chlorine                                                                   & TRI            & clean air act         & yes    & yes     & yes  & Propiconazole                                                                                                      & TRI            & formulation component & yes & yes & yes\\
            Chlorine dioxide                                                           & TRI            & article component     & yes    & NA      & yes  & Propionaldehyde                                                                                                    & TRI            & clean air act & yes & yes & NA\\
            Chlorobenzene                                                              & TRI            & clean air act         & yes    & yes     & yes  & Propylene                                                                                                          & TRI            & formulation component & yes & yes & yes\\
            Chlorodifluoromethane (HCFC-22)                                            & TRI            & formulation component & yes    & NA      & NA   & Propylene oxide & TRI & carcinogenic & yes & NA & yes\\
            Chloroethane                                                               & TRI            & clean air act         & yes    & NA      & NA   & Propyleneimine                                                                                                     & TRI            & carcinogenic          & yes    & NA      & NA   \\
            Chloroform                                                                 & TRI            & carcinogenic          & yes    & yes     & yes  & Pyridine                                                                                                           & TRI            & formulation component & yes    & yes & yes\\
            Chloromethane                                                              & TRI            & clean air act         & yes    & NA      & yes  & Quinoline                                                                                                          & TRI            & clean air act         & yes    & yes     & yes  \\
            Chlorophenols                                                              & TRI            & carcinogenic          & yes    & yes     & NA   & sec-Butyl alcohol                                                                                                  & TRI            & formulation component & yes & yes & yes\\
            Chlorothalonil                                                             & TRI            & carcinogenic          & yes    & yes     & yes  & Selenium                                                                                                           & TRI            & clean air act         & yes    & yes     & yes  \\
            Chromium                                                                   & TRI            & clean air act         & yes    & yes     & yes  & Selenium compounds                                                                                                 & TRI            & clean air act         & yes    & yes & NA\\
            Chromium compounds (except for chromite ore mined in the Transvaal Region) & TRI            & clean air act & yes & yes & yes & Silver & TRI & metal restricted & yes & yes & yes\\
            Cobalt                                                                     & TRI            & carcinogenic          & yes    & yes     & yes  & Silver compounds                                                                                                   & TRI            & metal restricted      & yes    & yes & yes\\
            Cobalt compounds                                                           & TRI            & clean air act         & yes    & yes     & yes  & Sodium dimethyldithiocarbamate                                                                                     & TRI & formulation component & yes & NA & yes\\
            Copper                                                                     & TRI            & metal restricted      & yes    & yes     & yes  & Sodium nitrite                                                                                                     & TRI            & metal restricted      & yes    & yes & yes\\
            Copper compounds                                                           & TRI            & metal restricted      & yes    & yes     & yes  & Styrene                                                                                                            & TRI            & carcinogenic          & yes    & yes & yes\\
            Creosote                                                                   & TRI            & carcinogenic          & yes    & yes     & NA   & Sulfuric acid (acid aerosols including mists, vapors, gas, fog, and other airborne forms of any particle size) & TRI & formulation component & yes & yes & NA\\
            Cresol (mixed isomers)                                                     & TRI            & clean air act         & yes    & yes     & yes  & Sulfuryl fluoride                                                                                                  & TRI            & ancillary use & yes & NA & NA\\
            Cumene                                                                     & TRI            & carcinogenic          & yes    & yes     & yes  & tert-Butyl alcohol                                                                                                 & TRI            & formulation component & yes & yes & yes\\
            Cumene hydroperoxide                                                       & TRI            & manufacturing aid     & yes    & yes     & NA   & Tetrabromobisphenol A                                                                                              & PBT            & formulation component & yes & yes & yes\\
            Cyanide compounds                                                          & TRI            & clean air act         & yes    & yes     & yes  & Tetrachloroethylene                                                                                                & TRI            & carcinogenic & yes & yes & yes\\
            Cyclohexane                                                                & TRI            & formulation component & yes    & yes     & yes  & Thiabendazole                                                                                                      & TRI            & formulation component & yes & yes & yes\\
            Decabromodiphenyl oxide                                                    & TRI            & formulation component & yes    & yes     & yes  & Thiophanate-methyl & TRI & formulation component & yes & yes & NA\\
            Di(2-ethylhexyl) phthalate                                                 & TRI            & carcinogenic          & yes    & yes     & yes  & Thiourea                                                                                                           & TRI            & carcinogenic & yes & yes & yes\\
            Diaminotoluene (mixed isomers)                                             & TRI            & carcinogenic          & yes    & yes     & NA   & Thiram                                                                                                             & TRI            & article component & yes & yes & yes\\
            Dibenzofuran                                                               & TRI            & clean air act         & yes    & yes     & yes  & Titanium tetrachloride                                                                                             & TRI            & clean air act & yes & yes & NA\\
            Dibutyl phthalate                                                          & TRI            & clean air act         & yes    & yes     & yes  & Toluene                                                                                                            & TRI            & clean air act         & yes    & yes & yes\\
            Dicamba                                                                    & TRI            & formulation component & yes    & yes     & NA   & Toluene-2,4-diisocyanate                                                                                           & TRI            & carcinogenic & yes & yes & NA\\
            Dichlorobenzene (mixed isomers)                                            & TRI            & carcinogenic          & yes    & NA      & NA   & Toluene-2,6-diisocyanate & TRI & carcinogenic & yes & yes & NA\\
            Dichloromethane                                                            & TRI            & carcinogenic          & yes    & yes     & yes  & Toluene diisocyanate (mixed isomers) & TRI & carcinogenic & yes & yes & NA\\
            Dicyclopentadiene                                                          & TRI            & formulation component & yes    & yes     & yes  & Trade Secret                                                                                                       & TRI            & manufacturing aid & yes & NA & NA\\
            Diethanolamine                                                             & TRI            & clean air act         & yes    & yes     & yes  & Trichloroethylene                                                                                                  & TRI            & carcinogenic          & yes & yes & yes\\
            Diisocyanates                                                              & TRI            & clean air act         & yes    & yes     & NA   & Triclopyr-triethylammonium salt                                                                                    & TRI            & formulation component & yes & yes & NA\\
            Dimethyl phthalate                                                         & TRI            & clean air act         & yes    & yes     & NA   & Triethylamine                                                                                                      & TRI            & clean air act         & yes & yes & yes\\
            Dimethylamine                                                              & TRI            & formulation component & yes    & yes     & yes  & Trifluralin                                                                                                        & PBT            & clean air act & yes & yes & NA\\
            Dimethylamine dicamba                                                      & TRI            & others                & yes    & yes     & NA   & Vanadium compounds                                                                                                 & TRI            & metal restricted & yes & yes & yes\\
            Dioxin and dioxin-like compounds                                           & TRI            & carcinogenic          & yes    & yes     & yes  & Vinyl acetate                                                                                                      & TRI            & carcinogenic & yes & yes & yes\\
            Diphenylamine                                                              & TRI            & article component     & yes    & yes     & yes  & Vinyl chloride                                                                                                     & TRI            & carcinogenic          & yes & yes & NA\\
            Epichlorohydrin                                                            & TRI            & carcinogenic          & yes    & yes     & yes  & Xylene (mixed isomers)                                                                                             & TRI            & clean air act & yes & yes & yes\\
            Ethyl acrylate                                                             & TRI            & carcinogenic          & yes    & yes     & yes  & Zinc (fume or dust)                                                                                                & TRI            & metal restricted & yes & yes & NA\\
            Ethylbenzene                                                               & TRI            & carcinogenic          & yes    & yes     & yes  & Zinc compounds                                                                                                     & TRI            & metal restricted      & yes & yes & yes\\
            \bottomrule\bottomrule
        \end{tabular}
    }
%    }
    \begin{minipage}\linewidth
        \vspace{0.01in}
        \tiny NOTES: NA means absent; POTW means publicly owned treatment works.
    \end{minipage}
\end{table}



        \section{Distribution of Industries and Pollution Emissions Intensities}\label{sec:distribution-of-industries-and-pollution-emissions-intensities}
        \begin{figure}[H]
    \centering
    \includegraphics[width=0.85\textwidth]{fig_naics_distribution}
    \caption{Distribution of Manufacturing Industries in the Sample}
    \label{fig:naics-manufacturing-industries}
\end{figure}
        \begin{figure}[H]
    \centering
    \includegraphics[width = 0.8\textwidth]{fig_releases_distribution_naics}
    \caption{Distribution of Total Onsite Releases Intensity across Manufacturing Industries}
    \label{fig:releases-distribution-naics}
\end{figure}
        \begin{figure}[H]
    \centering
    \includegraphics[width = 0.8\textwidth]{fig_releases_distribution_states}
    \caption{Distribution of Total Onsite Releases Intensity between the Treated and Control States}
    \label{fig:releases-distribution}
\end{figure}
        \begin{figure}[H]
    \centering
    \includegraphics[width = 0.8\textwidth]{C:/Users/david/OneDrive/Documents/ULMS/PhD/Thesis/chapter3/src/climate_change/latex/fig_air_emissions_distribution_state}
    \caption{Distribution of Total Air Emission Intensity between the Treated and Control States.}
    \label{fig:air-emissions-distribution}
\end{figure}
        \begin{figure}[H]
    \centering
    \includegraphics[width = 0.8\textwidth]{fig_water_discharge_distribution}
    \caption{Distribution of Total Surface Water Discharge Intensity between the Treated and Control States}
    \label{fig:water-discharge-distribution}
\end{figure}
        \input{fig_land_releases_distribution}
        \begin{figure}[H]
    \centering
    \includegraphics[width = 0.8\textwidth]{C:/Users/david/OneDrive/Documents/ULMS/PhD/Thesis/chapter3/src/climate_change/latex/fig_releases_distribution_carcinogenic}
    \caption{Distribution of Average Total Onsite Carcinogenic Releases Intensity between the Treated and Control States}
    \label{fig:releases-distribution-carcinogenic}
\end{figure}
        \begin{figure}[H]
    \centering
    \includegraphics[width = 0.8\textwidth]{fig_releases_distribution_caa}
    \caption{Distribution of Average Total Onsite CAA Releases Intensity between the Treated and Control States}
    \label{fig:releases-distribution-caa}
\end{figure}
        \input{fig_releases_distribution_haps}
        \begin{figure}[H]
    \centering
    \includegraphics[width = 0.8\textwidth]{fig_releases_distribution_pbts}
    \caption{Distribution of Total Onsite PBT Releases Intensity between the Treated and Control States}
    \label{fig:releases-distribution-pbts}
\end{figure}

    \end{appendices}
%%%%%%%%%%%%%%%%%%%%%%%%%%%%%%%%%%%%%%%%%%%%%%%%%%%%%%%%%%%%%%%%%%%%%%%%%%%%%%%%%%%%%%%%%%%%%%%%%%%%%%%%%%%%%%%%%%%%%%%%

%    \section{Literature Gaps}\label{sec1:literature-gaps}
%    The minimum wage policy and its effect on employment has been contentious since history~\parencite{neumark1992employment, card1993minimum}. Following the $2007$ federal wage floor, the arguments on the subject has proliferated. Many arguments favour the negative effects on employment while other scholars found sharp null zero effects. For example,~\cite{cengiz2019effect} used event study bunching analysis to study the effect of minimum wage rise on low wages in different wage bins, and further examined its effect on employment. They document that earnings rose but overall employment remained unchanged following a minimum wage rise in the US. Moreover, they find that minimum wage reduced employment of workers in tradable sectors such as restaurant and retail. However, one major drawback of this study is the event study-bunching analysis design. It used a federal wage floor and grouped states into treated and untreated units without accounting for specific years that each state actually raised its minimum wage. This type of setting has been criticised in the literature to recover at best a volume weighted ATT on each wage bin without accounting for new entrant low-wage workers and subsequent increases in minimum wage by different states. Even worse, it ignores the fact that some untreated states may have raised their minimum wage in the future, the later treated states, as well as new entrant low-wage workers that subsequent wage increases may have induced. Because of this, it is possible that the recovered ATT may suffer from attenuation bias, and worst case, have a wrong sign in the event study analysis~\parencite{goodman2021difference}. Furthermore,~\cite{riley2017raising} examined the effect of minimum wage on labour productivity in Britain using traditional difference-in-differences method. They proxied wage with labour cost and found that minimum wage raised labour costs for low-wage workers, and persisted during and after the global financial crisis. Consequently, they show that firms tend to raise their labour productivity as a result of minimum wage. They explained that this increased productivity is driven by changes in total factor productivity rather than decline in firms labour force or capital-labour substitution. One concern here is that it is not clear how the total factor productivity was combined to drive labour productivity. Is it more energy intensive or energy efficient? How does firms adjust the energy intensity in their production process in reaction to the raised minimum wage? And is this adjustment behaviour pro-environment?
%
%    ~\cite{li2023does} attempted to investigate a related question by looking at the effect of minimum wage standard on pollutions in China's manufacturing sector. Using panel FE and hierarchical modelling they evince that increase in minimum wage increases manufacturing firm's emissions. They further argued that the increase in pollutions is driven by manufacturing firms shifting investments to cost-effective traditional energy sources such as fossil fuel energy as they adjust their production functions for the wage-induced labour costs. They also argue that firms have less inputs of pollution treatments in their production process. The shortcomings of this study is that they failed to present a causal link between minimum wage and pollution from firms, rather they evince a correlation. This kind of evidence can be biased in many ways. One of which is that the effect may be driven by other employees in the higher end of the wage distribution for whom minimum wage is not targeted. That is, it will pick up effects from high-wage employee groups other than those that minimum wage affect directly - low-wage workers. Thus, confounding the effect of raising minimum wage on emissions. Further, suppose there is indeed evidence that minimum wage raises firm's pollution through increased crude energy use or investment, as well as using less pollution treatment in their production process. It is still not clear how this new firm adjustment behaviour transmits to pollution. One possible reason would be the energy intensity of the firms. How much energy are firms using to produce one unit of good following a minimum wage raise? To answer this kind of question requires a more precise method, to investigate the effects on pollution post minimum wage raise. And it may be possible that firms invest in more energy efficient production technologies to increase their productivity given the new higher labour cost. In this setting, we should expect a reduced energy intensity following a raise in minimum wage, and an opposite effect on emissions. But this subject is yet to be explored.
%
%    \section*{Contribution}\label{sec:contribution}
%    Given the above literature gaps, I adopt a more robust approach to investigate the effects of minimum wage in the US. First, revist the effect on wages and employment - first stage. Second, examine the effect on labour productivity. Third, investigate the effect on emissions. Fourth, trace the channels of the effects, by decomposing total factor productivity and narrow in on firm's energy intensity to evidence the channel of effect of minimum wage on emissions.


%    \section{Considered identification}\label{sec:considered-identification}
%    The US federal government raised the minimum wage for low-wage workers from $\$5.15$ in $1997$ to $5.85$ in $2007$ and by $2009$ they had raised it to $\$7.25$. However, different states raised their state minimum wages in different years since $2004$. I exploit this time variations in state minimum wage to identify the effect of minimum wage on emissions. This provides for a staggered adoption of minimum wage raise across US states. Between $2004-2019$, a total of $12$ states raised their minimum wage in $2005$ and $2006$, $6$ different states first raising minimum wage each year. Another $12$ states raised theirs in $2007$; $17$ states raised theirs in $2008$; $2$ states in $2009$; and another $2$ states in $2010$ and $2019$, one in each year. A total of $5$ states has no minimum wage policy (Alabama, Louisiana, Mississippi, South Carolina, and Tennessee).




    \newpage
    \printbibliography
\end{document}