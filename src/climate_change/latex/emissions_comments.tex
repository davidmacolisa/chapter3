\documentclass[12pt, english]{article}
%======================================================================================================================%
% Preamble
\usepackage[
    backend = biber,
    style = apa,
    citestyle = authoryear-comp,
    sorting = ydnt,
    mincitenames = 1,
    maxcitenames = 2,
    uniquelist = minyear
]{biblatex}
\addbibresource{emissions.bib}
%\AtBeginBibliography{\small}

\usepackage{hyperref}
\hypersetup{colorlinks = true, citecolor = blue, linkcolor = blue, urlcolor = blue, hypertexnames = true}
\newcommand{\shortlink}[1]{\href{https://www.#1}{\texttt{#1}}}

\DeclareCiteCommand{\cite} % Ensures author-year hyperlink applies to \cite
{\usebibmacro{prenote}}
{\usebibmacro{citeindex}%
\printtext[bibhyperref]{\usebibmacro{cite}}}
{\multicitedelim}
{\usebibmacro{postnote}}

\DeclareCiteCommand{\parencite}[\mkbibparens] % Ensures author-year hyperlink applies to \parencite
{\usebibmacro{prenote}}
{\usebibmacro{citeindex}%
\printtext[bibhyperref]{\usebibmacro{cite}}}
{\multicitedelim}
{\usebibmacro{postnote}}

\DeclareCiteCommand{\cite} % Ensures that year appears in parentheses
{\usebibmacro{prenote}}
{\usebibmacro{citeindex}%
\printtext[bibhyperref]{\printnames{labelname} \mkbibparens{\printfield{year}}}}
{\multicitedelim}
{\usebibmacro{postnote}}

\usepackage{authblk}
\usepackage{times}
\usepackage{parskip}
\usepackage{graphicx}
\usepackage{amsmath}
\usepackage{amsfonts}
\usepackage{mathrsfs}
\usepackage{float}
\usepackage{geometry}
\geometry{papersize = {9in, 11in}, left = 2.5cm, right = 2.5cm, top = 2.5cm, bottom = 2.5cm}
\usepackage{scrextend}
\usepackage[doublespacing]{setspace}
\usepackage{textcomp}
\usepackage{csquotes}

% Tables Packages
\usepackage{booktabs}
\usepackage{titlesec} %for modified section numbering
\setcounter{secnumdepth}{4}  % How deep you want to number
\titleformat{\paragraph}
{\normalfont\normalsize\bfseries}{\theparagraph}{1em}{}
\titlespacing*{\paragraph}
{0pt}{3.25ex plus 1ex minus 0.2ex}{1.5ex plus 0.2ex}

%Appendix
\usepackage[toc, page]{appendix}
\usepackage{titleps}
\usepackage{caption}
\usepackage{supertabular}

%======================================================================================================================%
% Title Page
\title{{Progress Report: Unforeseen Minimum Wage Consequences}}
\author[1]{Davidmac O. Ekeocha}
\affil[1]{
    University of Liverpool Management School \\
    \texttt{davidmac.ekeocha@liverpool.ac.uk}
}
%\affil[2]{University of Liverpool Management School}
%\affil[3]{University of Liverpool Management School}

\date{\today}

%======================================================================================================================%
\begin{document}
    \maketitle
%----------------------------------------------------------------------------------------------------------------------%
    \section*{List of comments and responses}
    This document highlights the most recent changes to the original submission.

    \date{\texttt{12 August 2024}}
    \begin{itemize}
        \item \textit{Comments:} Move the baseline results to the appendix.
        \item \textit{Response:} I think that presenting the baseline results in the main text is important to aid a coherent discussion. However, I have moved the corresponding baseline heterogeneous effects and robustness results to the appendix.
        \item \textit{Comments:} Classify the heterogeneous dimensions conditional on the first pre-treatment period.
        \item \textit{Response:} All constructed heterogeneous dimensions have been re-constructed and reanalyzed accordingly.
        \item \textit{Comments:} Address one heterogeneous dimension at a time and call it what it is.
        \item \textit{Response:} I have shown the correlation matrix in Figure~\ref{fig:correlation-het} which shows no correlations across the main heterogeneous dimensions. Based on this, I have constructed dummies of high-profit-labour-intensive v. high-profit-capital-intensive manufacturing industries, and low-profit-labour-intensive v. low-profit-capital-intensive manufacturing industries. \begin{figure}[H]
    \centering
    \includegraphics[width=0.85\textwidth]{C:/Users/david/OneDrive/Documents/ULMS/PhD/Thesis/chapter3/src/climate_change/latex/fig_correlation_het}
    \caption{Correlation matrix of heterogeneous dimensions}
    \label{fig:correlation-het}
\end{figure}
        \item \textit{Comments:} Conduct similar heterogeneous analysis by industry structure for the baseline.
        \item \textit{Response:} I show in Figures~\ref{fig:baseline-manufacturing-industry-cost-heter},~\ref{fig:baseline-manufacturing-industry-employment-heter} and~\ref{fig:baseline-manufacturing-industry-output-heter} the results of this heterogeneous analysis based on these dimensions low- v. high-skilled workers, high- v. low-profit manufacturing industries, and labour- v. capital-intensive manufacturing industries. Albeit, I cannot claim causality in Figures~\ref{fig:baseline-manufacturing-industry-cost-heter-hpli-lpli},~\ref{fig:baseline-manufacturing-industry-employment-heter-hpli-lpli} and~\ref{fig:baseline-manufacturing-industry-output-heter-hpli-lpli}, given the significant pre-trends, I have only shown the results of these sub-dimensions of high-profit-labour-intensive v. high-profit-capital-intensive manufacturing industries (HPLI) and low-profit-labour-intensive v. low-profit-capital-intensive manufacturing industries (LPLI), for completeness sake. However, most reviewed famous literatures limited theirs to the type of workers in the industry as a sufficient baseline evidence~\parencite{cengiz2019effect, gopalan2021state, clemens2019making, dube2010minimum, dustmann2022reallocation}. I believe that the intuition for this choice is that the target of the minimum wage (MW) policy is on industry workforce regardless of the industry structure.
        \input{fig_sdid_industry_cost_heter}\input{fig_sdid_industry_emp_hours_heter}\begin{figure}[H]
    \centering
    \includegraphics[width=1\textwidth,keepaspectratio]{C:/Users/david/OneDrive/Documents/ULMS/PhD/Thesis/chapter3/src/climate_change/latex/fig_sdid_output_heter}
    \caption{Heterogeneous Effects of the MW Policy on Outputs}
    \label{fig:baseline-manufacturing-industry-output-heter}
    \begin{minipage}{\columnwidth}
        \vspace{0.05in}
        \tiny NOTES: The event study model of equation~\ref{eq:baseline-wages} is $Y_{i,cp,t} = \sum_{{e = -3},{e \neq -1}}^{3} \beta (Treated^{e} \cdot D)_{i,s,t} + \psi (Treated^{e})_{s,t} + \vartheta (Treated \cdot D)_{i,s,t} + \mu (Post \cdot D)_{i,s,t} + \tau Treated_{s,t} + \rho D_{i,s,t} + \alpha Post_{t} + \delta X_{v,c,t-1} + \omega F_{f,t} + \lambda_{t} + \sigma_{c} + \phi_{cp} + \zeta_{cp,t} + \epsilon_{i,cp,t}$. Standard errors are clustered at the state level. $D_{i,s,t}$ is unity for low-skilled and zero for high-skilled workers; $Treated_{s,t}$ is unity for treated and zero for control states; and $Post_{t}$ is unity for post-treatment and zero for pre-treatment periods.
    \end{minipage}
\end{figure}
        \item \textit{Comments:} Mechanism analysis.
        \item \textit{Response:} Based on the sub-dimensions of HPLI and LPLI groupings, I conduct differential effects to explore the transmission mechanisms. Results are presented and interpreted in the main text.
    \end{itemize}
%======================================================================================================================%
    \newpage
    \printbibliography
\end{document}
%======================================================================================================================%