%======================================================================================================================%
\documentclass{C:/Users/david/OneDrive/Documents/ULMS/PhD/Thesis/chapter3/src/climate_change/latex/Economic_Journal/OUP-EJ}
%\documentclass{oup-authoring-template}
%======================================================================================================================%
% Preamble
%\usepackage{times}
\usepackage{float}
\usepackage[a4paper, left=2.5cm, right=2.5cm]{geometry}
\usepackage{hyperref}
\hypersetup{colorlinks = true, citecolor = blue, linkcolor = blue, urlcolor = blue, hypertexnames = true}
\newcommand{\shortlink}[1]{\href{https://www.#1}{\texttt{#1}}}
%======================================================================================================================%
\begin{document}
%======================================================================================================================%
% Title Page
    \title[Minimum Wage and Toxic Releases]{Unforeseen Minimum Wage Consequences}
    \author[1,*]{Davidmac O. Ekeocha}
    \affil[1]{University of Liverpool Management School, Chatham Street, Liverpool, L69 7ZH, United Kingdom}
    \corres[*]{Correspondence e-mail: davidmac.ekeocha@liverpool.ac.uk}
    \aunotes[\dagger]{University of Liverpool Management School, Chatham Street, Liverpool, L69 7ZH, United Kingdom}
%======================================================================================================================%
    \begin{abstract}
        \noindent This study investigates the unintended environmental impacts of higher minimum wage (MW) policies, using the staggered difference-in-differences estimator, on detailed facility-level toxic release inventory and manufacturing payroll data. The results uncover complex environmental outcomes, potentially carcinogenic to both humans and aquatic lives. High-profit-capital-intensive manufacturing industries experienced increased toxic release intensities, driven by higher intensities in air emissions, surface water discharges, and land releases, even within heavily regulated toxic chemical domains. This pollution effect is evident in both the highest- and lowest-emitting industries in more competitive markets. In contrast, both low- and high-profit-labour-intensive manufacturing industries showed a reduction in toxic release intensities. Mechanism analyses suggest that these shifts in toxic release intensities are directly linked to corresponding inverse changes in onsite source reduction and waste management activities, particularly in green raw material substitutions, process and equipment modifications, operating practices, and inventory management. Policymakers should consider incentivizing the reduction and management of toxic chemical releases in high-profit-capital-intensive industries, alongside statutory MW policies, to mitigate the adverse environmental impacts.
    \end{abstract}
    \keywords{Minimum Wage, Toxic Releases, Staggered Difference-in-Differences, United States}
%======================================================================================================================%
    \maketitle
%======================================================================================================================%


    \section{Introduction}\label{sec:introduction}
    The effects of raising the minimum wage (MW) have garnered interest in the literature, reaching mixed conclusions for employment given assumed labour market conditions. For perfect competition, some studies have recorded disemployment effects of an MW raise on low-wage workers~\citep{stigler1946economics, hamermesh1982minimum, neumark1992employment, brown1999minimum, machin2004minimum, neumark2000minimum, borjas2010labor}, whereas in monopsonistic markets, limited or a sharp null employment effect is documented~\citep{lester1960employment, card1993minimum, card2000minimum, aaronson2018industry, cengiz2019effect, wong2019minimum, dustmann2022reallocation}. Other studies have recorded heterogeneous employment effects~\citep{okudaira2019minimum, medrano2023minimum, meer2023effects, gregory2022minimum}.

    On the other hand, the symmetry between raising MW and labour costs or wages of low-wage workers is well established~\citep{medrano2023minimum,clemens2023important}. Career progressions and high labour demand, higher job/worker effort and labour productivity, especially for incumbent workers in the $40th$ percentile of the wage distribution relative to workers in higher percentiles, largely drive the increase in wages of low-wage workers~\citep{riley2017raising, kim2019minimum, wong2019minimum, baek2021impact, zhao2021effects, seok2022macroeconomic, ku2022does, coviello2022minimum, alexandre2022minimum}. They argued that firms' adjustments in total factor productivity, lower hiring and layoff and monitoring incentive explain the increased labour productivity, and that the increase in worker effort can offset about $50\%$ of projected rise in labour cost. Furthermore,~\citet{harasztosi2019pays} argued that the cost-effect of raising MW is not only borne by employers and workers, but that firms tend to pass some of these burdens to consumers via higher prices.

    Interestingly, the question regarding the existence of other pathways through which firms pass-through the burdens of higher-cost-induced MW is still nascent. These studies focused on Chinese manufacturing firms to document higher emissions and pollution emissions intensity induced MW raise~\citep{li2023does, zhang2023unintended}.~\footnote{\tiny Two hypotheses present opposing views on the direction of environmental impact of MW. The labour-technology mix which argues that firms react to labour cost induced MW by adopting clean production/operation technologies and labour-savings to increase energy efficiency, labour productivity thereby decreasing their pollution intensity. Conversely, the crowding-out effect hypothesis argues in favour of increased firms' pollution intensity. Given MW, firms struggle financially and to maintain pre-reform operation/production capacity while managing cost, it crowds-out clean production technologies and green innovative practices leading to increased pollution intensity. Thus, to understand how labour-market shocks translate to environmental consequences becomes an empirical exercise.} They argued that this effect is more prominent in financially constrained firms such that the labour cost pass-through of MW shifts firms energy usage towards cost-effective but crude energy sources, as well as reduced pollution abatement inputs and declining green innovation in their production functions. Albeit these studies used border-county design, they identified their effect using endogenous changes in MW, possibly indexed to inflation, thereby masking their causal claims of rising pollution emissions intensity. Hence, their results are indeed not causal as it is not clear whether the documented pollution responses are due to an increase in MW or inflation.

    However, this study leverages state-level exogenous changes in MW over time and administrative plant-level toxic emissions data (including air, land and water) from the environmental protection agency (EPA) to examine toxic responses of manufacturing firms in the United States (US) to the MW policy. The evidence on the environmental consequences of labour market shocks is particularly important for better health, environmental and labour market policy designs.~\citet{shapiro2018pollution} developed a model to historically explain the fall in the US pollution emissions in the manufacturing sector and revealed several findings including outsourcing production of pollution-intensive goods to other countries like China and Mexico; environmental regulations such as pollution tax have the most significant negative impact on US pollution intensity through increased investment in effective abatement technologies; and rising labour productivity decreases pollution intensity, thus decreasing pollution emissions. Moreover, some literature found causal evidence on the effects of air pollution $(PM_{2.5})$ and water pollution on cancer in humans and aquatic animals, respectively~\citep{turner2020outdoor, turner2017ambient, baines2021linking}.~\citet{coneus2012pollution} found reductions in infant birth weights and increasing bronchitis and respiratory illnesses in toddlers due to increasing carbon monoxides and $O_{3}$ emissions in Germany.

    Two hypotheses exist on possible transmission mechanisms of environmental consequences of raising MW: the factor substitution and crowding out effect hypotheses~\citep{zhang2023unintended}. The factor substitution hypothesis argues for the reductions in firms pollution emissions intensity. To maximize profits amid rising labour costs due to a raised MW floor, firms adjust their factor inputs by switching to automated production processes from labour-intensive manual processes. Thus, replacing manual labour with machines and technologies, and increasing capital per worker while prioritizing resource allocations toward improving production functions through research~\citep{harasztosi2019pays,hau2020firm, geng2022minimum,dai2023minimum, li2020labor}. This will further increase labour productivity and total factor productivity~\citep{riley2017raising}. Thus, raising MW will cause firms to use efficient capital intensive methods, reducing energy intensity and ultimately decreasing emissions per unit of output.~\footnote{\tiny Similarly, the possiblity exists that firms may replace expensive factor inputs with cheaper but crude energy inputs with higher pollution emission potentials, in their production process, which in turn increases energy and pollution emission intensities.} Conversely, the crowding out effect hypothesis argues in favour of increased firms' pollution intensity. These studies have evinced that this effect is attributed to the declining pressure on firms' profitability and heavier financial constraints resulting from wage hikes~\citep{draca2011minimum, bell2018minimum, du2022minimum}. Consequently, constrained financial resources limit firms' investment in pollution abatement activities (as they focus on core production), leading to reduced pollution removal and higher emission intensity.~\footnote{\tiny In a different way, innovation is influenced by resource scarcity, leading to a preference for certain types of solutions. Technological advancement continues unabated, but resource constraints necessitate a balance between different types of solutions. Labour deficits, often caused by increases in the minimum wage, promote the development of labour-saving technologies, which can impede the progress of labour-intensive green technologies~\citep{acemoglu2010does}. Firms often prioritize improvements in production efficiency over environmental considerations due to limited resources for innovation. Interestingly, some environmentally friendly technologies, such as end-of-pipe treatments, require significant labour. This emphasis on efficient automation could divert resources away from the development of clean energy and waste management technologies, potentially resulting in increased pollution emissions intensity.}

    To document the causal effect of raising MW on pollution emission intensity, I use the~\citet{sun2021estimating} staggered difference-in-differences estimator, exploiting copious exogenous changes in state-level MW ($\geq \$0.5$ per hour) with clearly defined before- and after-periods. I exploit increases in Arkansas, California, Delaware, Maine, Massachusetts, Maryland, Michigan, Minnesota, Nebraska, New Jersey, New York, South Dakota and West Virginia in $2014$, $2015$ and $2017$. Hereafter, the treated states. I match each treated state to a set of adjacent control states that never implemented any MW policy from $2012-2017$, similar to \citet{gopalan2021state}. I further restrict the sample to border-counties~\citep{dube2010minimum}. The identifying assumption is that in the absence of any MW policy, economic conditions in adjacent cross-border counties would have evolved in parallel. Given this assumption, I show that the pre-treatment trends between the treated and control states are similar prior to the MW policy.

    The preliminary findings provide causal evidence of increased cost burdens in both low- and high-skilled/wage workers, as well as in production materials, due to a higher MW floor. This is followed by rising outputs, labour productivity, and profits. Although overall effects on employment and production workers' hours are null, cohort-specific effects are more nuanced. Disemployment is more pronounced among low-skilled workers, while positive employment effects dominate in high-skilled workers. The disemployment effect for low-skilled workers is primarily driven by their reluctance to commute to distant higher MW counties/states, whereas the increasing demand for high-skilled/wage workers is independent of cross-county/state worker mobility. However, there is notable decline in material cost and output in the $2017$ cohort from a corresponding disemployment effect. These preliminary findings have significant implications for the environmental consequences of the higher MW policy.

    The main results reveal that the environmental consequences of higher MW are nuanced as manufacturing industries adjust either towards capital- or labour-intensive technologies based on their profit maximization objectives. The increasing toxic release intensities in high-profit-capital-intensive manufacturing industries is dominated in the $2014$ and $2015$ cohorts. This increase is driven by rises in total air emissions intensities, from fugitive sources, surface water discharge, and land releases intensities. These increases are present in both highest- and lowest-emitting industries in high competitive environments, and even in highly regulated domains by the Clean Air Act of $1970$ and as amended in $1990$, and the Toxic Substance Control Act of $1976$, potentially resulting to carcinogenic emissions. There is limited evidence of increased onsite releases intensity in low-profit-capital-intensive manufacturing industries. In contrast, onsite toxic release intensity in low- and high-profit-labour-intensive manufacturing industries decreased in the $2015$ cohort. This reduction is driven by corresponding decreases in point and fugitive air emissions intensities. Increases in surface impoundment intensity, particularly in $2015$ and $2017$ cohorts, also contribute to the differential decline.

    The mechanism analyses reveal that the observed effects are transmitted through the reductions in both onsite source reduction and waste management activities. The analysis indicates that increased toxic release intensities in high-profit-capital-intensive manufacturing industries are primarily due to declines in green raw material substitutions, including the use of organic solvents and cleaner energy sources; process upgrades and equipment modifications, such as the adoption of new green technologies, recycling enhancements, and spray equipment optimization; inventory management improvements, including better packaging and storage practices; and better operating activities, such as leak inspections and the decommissioning of inefficient equipment. Additionally, significant reductions in waste management practices, including biological and physical waste treatment and recycling, further exacerbate the rise in toxic release intensities. Conversely, the decline in total release intensities in both low- and high-profit-labour-intensive manufacturing industries is attributable to improvements in the same areas. Specifically, the transition to clean energy sources, green organic solvents, advanced green technologies, process recirculation, spray equipment upgrades, enhanced packaging and storage, leak inspections, and the shutdown of inefficient equipment, alongside the increases in chemical and physical waste treatments and recycling, contribute to the reduced toxic release intensities.

    The rest of the article is organized as follows: Section~\ref{sec:policy-context} provides the policy context of the study. Section~\ref{sec:data} discusses the data and descriptive statistics. Section~\ref{sec:industry-costs-employment-and-outputs} presents the effect on industry labour costs and employment, and robustness. Section~\ref{sec:onsite-toxic-releases} discusses the effects on onsite toxic releases intensities. Section~\ref{sec:robustness-exercises} discusses robustness exercises. Section~\ref{sec:heterogeneous-effects} focuses on the heterogeneous effects. Section~\ref{sec:mechanism-analyses} investigates the potential mechanisms. Section~\ref{sec:concluding-remarks} concludes with policy implications.


    \section{Policy Context}\label{sec:policy-context}
    This section discusses the exogenous state minimum wage changes exploited in the covered sample period $(2011-2017)$, for the causal identification.

    \subsection{Minimum Wage Changes across States}\label{subsec:minimum-wage-changes-across-states}
    About $27$ states raised MW between $2011$ and $2017$. Except for Nevada, the majority of the changes in MW between $2011-2013$ are attributed to inflation. However, following the labour union protests in $2012$ for higher MW, many states responded by instituting a one- or multiphased large MW changes~\citep{lathrop2021raises}. Specifically, $13$ states implemented a statutory MW raise of at least $\$0.5$ in $2014$, $2015$ and $2017$. Pertinently, there was no federal MW raise in the covered sample period. Table~\ref{tab:states-mw-changes} of Appendix~\ref{sec:appendix-state-minimum-wage-and-balance-tests} presents the state-level changes in MW for the sample period.

    \subsection{Selecting the Treated and Control States}\label{subsec:selecting-the-treated-and-control-states}
    The identification strategy focuses on the copious state-level MW changes. Particularly, I restrict the sample to MW changes that meet the following conditions: $(i)$ states with a large MW raise of $\geq$ $\$0.5/hour$ in one year and had never raised their MW since $2012$; $(ii)$ subsequent raises must either be equal to or greater than $\$0.5/hour$ in the post year, or the sum of post-multiphase raises (in any $2$-years) must be equal or greater than the first initial raise for that state. These conditions ensure that the exploited MW changes capture the statutory MW policies, and not indexed to inflation. Thirteen $(13)$ states meet the above conditions. They include Arkansas, California, Delaware, Maine, Massachusetts, Maryland, Michigan, Minnesota, Nebraska, New Jersey, New York, South Dakota, and West Virginia.~\footnote{\tiny New Jersey and South Dakota do not meet the second condition but are included in the treated sample. Hence, I explicitly control for US city average inflation in the model to net out any inflationary effects. Controlling for the year fixed effects in the model nets out this inflationary effect.} Table~\ref{tab:states-mw-adjustments-t-and-c} provides a summary of the minimum wage adjustments in the sample.

    There are four increases of $\$0.75/hour$, four increases of $\$1.00/hour$, four increases of $\gg\$1.00/hour$ with a max of $\$1.85/hour$ and one increase of $\$0.5/hour$ in the sample. The average initial MW change is about $\$1.02/hour$ $(7.6\%)$. Following the first MW raise till the end of the covered period, there are seven post-initial total MW changes between $\$1.00/hour \leq \Delta MW \leq \$1.75/hour$, four total post-initial MW changes between $\$2.00/hour \leq \Delta MW \leq \$2.50/hour$ and two total post-initial MW changes between $\$3.00/hour \leq \Delta MW \leq \$3.35/hour$. These have a total of post-initial MW average of $\$1.95/hour$, about $(7.6\%)$ total average MW changes between $2011$ and $2017$. Further, the total percentage change in MW from $2011$ to $2017$ for each treated state range between $13.79$ and $54.38\%$.
    %\usepackage{booktabs}
\begin{table}[H]
    \centering
    \caption{Exogenous State-level MW Adjustments}
    \label{tab:states-mw-adjustments-t-and-c}
    \resizebox{\columnwidth}{!}{%
        \begin{tabular}{lrrrrrrlrr}
            \toprule\toprule
            treated states & MW $\Delta$ year & MW $\Delta$ amount & $\sum_{1}^{2}\Delta MW$ & total MW $\Delta$ amount & start MW & end MW & control states & \# of border counties (T) & \# of border counties (C) \\ \midrule\midrule
            MN             & 2014             & 1.85               & 1.50                    & 3.35                     & 6.16     & 9.51   & (IA, ND, WI)   & 42                        & 54                        \\
            MA             & 2015             & 1.00               & 2.00                    & 3.00                     & 8.00     & 11.00  & (NH)           & 8                         & 6                         \\
            CA             & 2014             & 1.00               & 1.50                    & 2.50                     & 8.00     & 10.50  & (NV)           & 10                        & 11                        \\
            NY             & 2014             & 0.75               & 1.70                    & 2.45                     & 7.25     & 9.70   & (PA)           & 27                        & 16                        \\
            AR             & 2015             & 1.25               & 1.00                    & 2.25                     & 6.25     & 8.50   & (OK, TX)       & 19                        & 12                        \\
            MD             & 2015             & 1.00               & 1.00                    & 2.00                     & 7.25     & 9.25   & (PA, VA)       & 21                        & 39                        \\
            NE             & 2015             & 0.75               & 1.00                    & 1.75                     & 7.25     & 9.00   & (IA, KS, WY)   & 28                        & 24                        \\
            ME             & 2017             & 1.50               & 0.00                    & 1.50                     & 7.50     & 9.00   & (NH)           & 5                         & 7                         \\
            MI             & 2014             & 0.75               & 0.75                    & 1.50                     & 7.40     & 8.90   & (IL, IN, WI)   & 32                        & 39                        \\
            WV             & 2015             & 0.75               & 0.75                    & 1.50                     & 7.25     & 8.75   & (KY, PA, VA)   & 32                        & 28                        \\
            SD             & 2015             & 1.25               & 0.15                    & 1.40                     & 7.25     & 8.65   & (IA, ND, WY)   & 27                        & 19                        \\
            NJ             & 2014             & 1.00               & 0.19                    & 1.19                     & 7.25     & 8.44   & (PA)           & 16                        & 13                        \\
            DE             & 2014             & 0.50               & 0.50                    & 1.00                     & 7.25     & 8.25   & (PA)           & 5                         & 6                         \\ \bottomrule\bottomrule
        \end{tabular}
    }
    \begin{minipage}{17.5cm}
        \vspace{0.01in}
        \tiny NOTES: This table summarizes the exogenous state-level MW changes from $2012$ to $2017$. There are thirteen $(13)$ treated and $(14)$ control states. The definition of treated and control states is given in sub-section~\ref{subsec:selecting-the-treated-and-control-states}. MW $\Delta$ year represents the year in which a treated state first raised its MW. MW $\Delta$ amount corresponds to the first MW raised amount for that year. $\sum_{1}^{2}\Delta MW$ denotes the sum of any post-two-year MW raises after the first initial raise. Total MW $\Delta$ amount corresponds to the total MW raised amount till the end of the sample. Start (End) is the MW at the start (end) of the sample. Control states are the set of control states for each treated states. These states never raised MW between $2012$ and $2017$. \# of border counties (T) is the number of counties in a treated state that border at least one county in a control state. And \# of border counties (C) is the number of counties in a control state that border at least one county in a treated state. * means not used in the POTWs sample.
    \end{minipage}
\end{table}


    Furthermore, I match each treated state to adjacent control states that never raised their MW between $2012$ and $2017$, and follow~\citet{dube2010minimum} and~\citet{gopalan2021state} to limit the sample to border counties in treated and control states.~\footnote{\tiny Other recent papers to use this identification strategy include~\citet{aaronson2018industry},~\citet{dube2019fairness},~\citet{jardim2018minimum}, and~\citet{zhang2019distributional}. The control states in the sample include Iowa, Illinois, Indiana, Kansas, Kentucky, North Dakota, New Hampshire, Nevada, Oklahoma, Pennsylvania, Texas, Virginia, Wisconsin, and Wyoming. Importantly, Georgia, Idaho, New Mexico, North Carolina, and Utah are removed from the list of control states as they are not adjacent to any treated states.} Table~\ref{tab:states-mw-adjustments-t-and-c} shows that there are $13$ treated and $14$ control states in the sample. The last two columns further show that there are a total of $272$ treated and $274$ control border counties. Figures~\ref{fig:border-state-map} and~\ref{fig:border-county-map} show the geographical locations of the treated and control states and counties, respectively.
    \begin{figure}[H]
    \centering
    \includegraphics[width=0.85\textwidth, height=0.4\textheight]{border_state_map}
    \caption{Map of Treated and Control States}
    \label{fig:border-state-map}
\end{figure}
    \begin{figure}[H]
    \centering
    \includegraphics[width=0.85\textwidth, height=0.4\textheight]{border_county_map}
    \caption{Map of Treated and Control Counties}
    \label{fig:border-county-map}
\end{figure}

    Each treated border county is paired with a cross-border control county, described as the pair of the adjacent treated and control counties. The identifying assumption in a cross-border county pair is that the evolution of economic conditions for the pairs is symmetric and parallel, but the MW levels vary discontinuously at the border. To address the concern raised in~\citet{neumark2014revisiting} on the validity of border counties as counterfactuals, Figures~\ref{fig:county-level-macroeconomic-trends-in-border-counties} and ~\ref{fig:state-level-macroeconomic-trends-in-border-states} present a comparison of economic conditions before the first initial year of the MW change. Along most of the observable pre-treatment variables, the trends appear to be statistically parallel for the treated and control border counties and states.~\footnote{\tiny Tables~\ref{tab:descriptive-statistics-control-border-counties} and~\ref{tab:descriptive-statistics-control-border-states} of Appendix~\ref{sec:appendix-state-minimum-wage-and-balance-tests} presents a comparison of the means of the pre-treatment variables for the treated and control border counties. For most of the variables, the results show no substantial differences in their means for the treated and control border counties. The state-level descriptive statistics uses state-level aggregated dataset.}


    \section{Theoretical Framework}\label{sec:theoretical-framework}


    \section{The Data}\label{sec:data}
    For the empirical analysis, the novel data used come from five different sources and combined with the administrative facility level Toxic Release Inventory (TRI) data from the Environmental Protection Agency (EPA) for the United States.

    \subsection{Toxic Release Inventory Data}\label{subsec:toxic-release-inventory-data}
    I collect TRI-form-R data from EPA, which contains inventory of toxic chemical releases that are either manufactured, processed, otherwise used, and/or managed at private, state and federal industrial facilities across the US States. To file a TRI reporting form R, a facility must have at least ten full-time employees, and manufactures (including import) or processes more than $25,000$ pounds or otherwise uses more than $10,000$ pounds of a TRI-listed chemical during a calendar year. TRI data reflect, among other things, quantities of chemicals managed by facilities as waste, including those quantities released into the environment (as air emissions, water and land pollution), treated, burned for energy, recycled, and transferred from one facility to another for release or further management. It provides facility-level information based on $5$-digit zip codes identifying the exact location of the facility within each city, county, and states in the US.

    I collate onsite facility panel information as well as offsite transfers and Publicly Owned Treatment Works (POTWs) toxic chemical releases and waste management practices. There are $167$ different toxic chemical releases consistently reported by the same facilities from $2011-2017$ belonging to $213$ different NAICS manufacturing industries.~\footnote{\tiny These manufacturing industries are categorized into: Beverage and tobacco product, chemical, computer and electronic product, food, forging and stamping, furniture and related products, household appliances, leather and allied products, machinery, miscelleneous, non-metallic mineral product, paper, petroleum and coal products, plastics and rubber products, primary metal, printing and related support activities, textile mills, textile product mills, transportation equipment, and wood product manufacturing.}

    \subsection{Wage Data}\label{subsec:wage-data}
    I use the industry level wage data from the National Bureau of Economic Research-Centre for Economic Studies (NBER-CES). The NBER-CES data contains complete information on industry-level production workers and their wages, production workers' hours, and total payroll for manufacturing industries. Using this information I construct the industry-level production workers' wage per hour and wages per worker.

    \subsection{Other Industry and Macroeconomic Data}\label{subsec:other-industry-and-macroeconomic-data}
    The industry level data are from the NBER-CES and contains other variables including employment, number of total revenues, material costs, energy use, value added and total factor productivity, etc., for only manufacturing industries in the US. The macroeconomic data are at the county sourced from the Quarterly Census of Employment and Wages (QCEW) of the US Bureau of Labour Statistics (BLS) and include variables such as the average number of establishments and industry ownership. The county-level gross domestic product is sourced from the Bureau of Economic Analysis (BEA). The inflation data is got from the BLS consumer price index historical dataset for all urban consumers.

    \subsection{Joining the Datasets and the Sample}\label{subsec:joining-the-datasets-and-the-sample}
    I begin by joining the US zip-code and county-level geographic shapefiles (by zip-codes) to the TRI data to get the corresponding FIPS codes. These were then used to join the BEA, BLS and QCEW data by year, FIPS and $6$-digit NAICS codes. The resulting data were further merged to the NBER-CES dataset by their NAICS codes. Finally, this merged data was then joined with the prepared US geographic adjacent county shapefile that has information on county-level population and county distance to a state border. I prepared these shapefiles in the spirit of~\citet{dube2010minimum} and~\citet{gopalan2021state}, where each treated county or state is matched to at least one adjacent cross-border county or state, yielding cross-border county/state pairs. This final novel dataset is used for the empirical analysis of this paper.

    From the merged dataset, I subset an unbalanced panel sample of onsite facilities, as well as their transfers to offsite and POTWs locations. The onsite data sample size is $1,893,689$ and consists of $1276$ manufacturing facilities belonging to $213$ NAICS codes (in $20$ manufacturing industries) and a panel of $167$ toxic chemicals. Figure~\ref{fig:naics-manufacturing-industries} of Appendix~\ref{sec:appendix-distribution-of-industries-and-pollution-emissions-intensities} shows the distribution of these manufacturing industries, which reports that chemical, forging and stamping, primary metals, petroleum and coal products, transportation equipment, and machinery manufacturing industries are the most common in the sample. This is an administrative facility-level panel located in a $5$-digit zip-codes in both the treated $(13)$ and adjacent control $(14)$ border states. A total of $27$ states and the number of cross-border counties are as described in section~\ref{subsec:selecting-the-treated-and-control-states}.

    The offsite and POTWs samples are subsets of the onsite sample. The unbalanced offsite panel sample size is $1,179,754$ and POTWs sample is $308,943$. The original panel of onsite facilities across time reduced to $680$ and $236$ in the offsite and POTWs samples, respectively. The number of toxic chemicals in the offsite and POTWs samples are $125$ and $69$, respectively. See Table~\ref{tab:analyzed-chemicals} of Appendix~\ref{sec:appendix-descriptive-stat-list-of-toxic-chemicals-trends-mechanisms-and-correlations} for the list of analysed chemicals. The number of NAICS industries in the offsite and POTWs samples are $171$ and $101$. Finally, there are $14$ and $9$ control states in the offsite and POTWs samples; $13$ and $12$ treated states in the offsite and POTWs samples, respectively (see Table~\ref{tab:states-mw-adjustments-t-and-c}). Importantly, the number of onsite treated and control states remained unchanged in the offsite sample. A total of $827$ offsite facilities are located in $1056$ zip-codes in $573$ cities of $299$ counties in $45$ US states, whereas the $114$ POTWs sites are found in $202$ zip-codes in $123$ cities of $63$ US counties in $24$ US states.~\footnote{\tiny The offsite states include Alabama, Arkansas, Arizona, California, Colorado, Connecticut, Delaware, Florida, Georgia, Iowa, Idaho, Illinois, Indiana, Kansas, Kentucky, Louisiana, Massachussetts, Maryland, Maine, Michigan, Minnesota, Missouri, North Carolina, North Dakota, Nebraska, New Hampshire, New Jersey, New Mexico, Nevada, New York, Ohio, Okhlahoma, Oregon, Pennsylvania, Rhode Island, South Carolina, South Dakota,Tennesse, Texas, Utah, Virginia, Vermont, Wisconsin, and West Virginia. The POTWs states include Arkansas, California, District of Columbia, Delaware, Iowa, Illinois, Indiana, Massachussetts, Maryland, Maine, Michigan, Minnesota, North Dakota, Nebraska, New Hampshire, New Jersey, New York, Pennsylvania, Rhode Island, Texas, Virginia, Wisconsin, West Virginia, and Wyoming.}


    \section{Industry Costs, Employment and Outputs}\label{sec:industry-costs-employment-and-outputs}
    The empirical analyses begin in this section by examining the effect of raising MW on manufacturing industry costs (labour and materials), employment, production workers and hours, and outputs.~\footnote{\tiny To rule out any possible treatment selection, I estimate this equation at both the county and state level: $Treated_{s,t}^e = \beta Z_{f,cp,t} + \lambda_{t} + \phi_{cp} + \delta_{h} + \zeta_{cp,t} + \epsilon_{f,cp,t}$. Where $Treated_{s,t}^e = 1[t - G_{s,t}]$ denotes treated states that are $e$-periods away from the initial treatment date, and $G_{s,t}$ is the vector of initial treatment dates. $Z_{f,cp,t}$ is the vector of facilities by county-pair covariates, and $\beta$ is the vector of coefficients. Albeit, the year fixed effects, $\lambda_{t}$, nets out any inflationary effects, city-region inflation is explicitly controlled for in the model. Cross-border county pair, $\phi_{cp}$, and county, $\delta_{h}$, fixed effects are controlled for to account for within cross-border county pair and county differences that may affect the MW policy. Finally, I control for cross-border county pair linear trends, $\zeta_{cp,t}$, to account for time-varying common shocks affecting the evolution of the MW policy in paired cross-border counties. The result (reported in Table~\ref{tab:treatment-selection}) showed no significant treatment selection effects in the following covariates: lagged values of county-level gross domestic product (GDP), GDP per capita, annual average establishments; inflation; average number of establishments, industry ownership and coverage.}

    \subsection{Baseline Results: Industry Costs}\label{subsec:baseline-results-industry-costs}
    In what follows, I estimate wage responses of manufacturing industry employees in the baseline. The baseline model is given by:
    \begin{equation}
        C_{i,cp,t} = \beta Treated_{s,t}^e + \delta X_{v,c,t-1} + \omega F_{f,t} + \lambda_{t} + \sigma_{c} + \phi_{cp} + \zeta_{cp,t} + \epsilon_{i,cp,t},\label{eq:baseline-wages}
    \end{equation}

    where $C_{i,cp,t}$ is the vector of industry costs (hourly wages, total payroll and material costs) of manufacturing industry, $i$ in cross-border county pairs, $cp$ in the year, $t$. $Treated_{s,t}^e = \textbf{1}[t - G_{s,t}]$ is unity for the treated states that are $e$-periods away from the vector of initial treatment dates, $G_{s,t}$ and zero for the control states. $X_{v,c,t-1}$ denotes lagged values of county-level GDP per capita, annual average establishments, and city-region inflation~\citep{gopalan2021state, dube2010minimum, clemens2019making}. $F_{f,t}$ contains facility-level dummies on industry ownership.

    I control for year fixed effects, $\lambda_{t}$ to account for time varying differences in the MW policy as well as trending inflation. Cross-border county pair, $\phi_{cp}$ and county, $\sigma_{c}$ fixed effects are controlled for to account for within county pair and county differences that may affect the MW policy such as within county industry compositions and political climate. Finally, $\zeta_{cp,t}$ is the cross-border county pair linear trends to control for the evolution of common shocks in cross-border county pairs. Standard errors are clustered at the state level as there are possibilities that changes in MW may be correlated within a state.
    % Please add the following required packages to your document preamble:
% \usepackage{booktabs}
\begin{table}[H]
    \centering
    \caption{Effect of the MW Policy on Industry Costs}
    \label{tab:baseline-industry-costs}
    \begin{tabular}{@{}lllllll@{}}
        \toprule\toprule
        Industry costs & \multicolumn{2}{c}{Hourly wage} & \multicolumn{2}{c}{Total payroll (log)} & \multicolumn{2}{c}{Material cost (log)} \\
        \cmidrule(lr){2-3}\cmidrule(lr){4-5}\cmidrule(lr){6-7}
        & 1         & 2         & 3         & 4         & 5         & 6         \\ \midrule
        $Treated^{e}$     & 0.397     & 0.889*    & -0.015    & 0.043*    & -0.031    & 0.129*    \\
        & (0.358)   & (0.452)   & (0.031)   & (0.025)   & (0.113)   & (0.069)   \\
        cohort 2014       & 0.582     & 1.197     & -0.030    & 0.004     & -0.081    & 0.187*    \\
        & (0.523)   & (0.707)   & (0.032)   & (0.039)   & (0.174)   & (0.047)   \\
        cohort 2015       & 0.091     & 0.382*    & 0.010     & 0.115***  & 0.051     & 0.035     \\
        & (0.260)   & (0.217)   & (0.062)   & (0.015)   & (0.094)   & (0.077)   \\
        cohort 2017       & -0.265    & -0.208    & -0.113*** & -0.613*** & 0.024     & -0.343*** \\
        & (0.226)   & (0.323)   & (0.019)   & (0.126)   & (0.031)   & (0.100)   \\
        controls          & Yes       & Yes       & Yes       & Yes       & Yes       & Yes       \\
        year FE           & Yes       & Yes       & Yes       & Yes       & Yes       & Yes       \\
        county FE         & Yes       & Yes       & Yes       & Yes       & Yes       & Yes       \\
        border-county FE  & No        & Yes       & No        & Yes       & No        & Yes       \\
        border-county LTs & No        & Yes       & No        & Yes       & No        & Yes       \\ \midrule
        Observations      & 1,893,689 & 1,893,689 & 1,893,689 & 1,893,689 & 1,893,689 & 1,893,689 \\
        $R^2$             & 0.581     & 0.624     & 0.404     & 0.440     & 0.577     & 0.619     \\
        Baseline Mean     & 26.56     & 26.56     & 2962.68   & 2962.68   & 61328.05  & 61328.05  \\ \bottomrule \bottomrule
    \end{tabular}
    \begin{minipage}{\columnwidth}
        \vspace{0.05in}
        \tiny NOTES: These results are obtained from estimating model~\ref{eq:baseline-wages}. Robust standard errors clustered at the state level are reported in parentheses. ***, **, and * denote significance levels at the less than $1\%$, $5\%$ and $10\%$, respectively.
    \end{minipage}
\end{table}

    The average treatment effect on the treated (ATT) is captured by $\beta$, which is the difference in the average effect of raising the MW floor on manufacturing industry costs in treated counties relative to adjacent control counties. The effects measured by $\beta$ are recovered using the~\citet{sun2021estimating} staggered difference-in-differences estimator, an improvement over the TWFE estimator. I report the decomposition results from~\citet{de2020two} investigating possible negative weights in the TWFE estimator when the treatment assignment is staggered. Albeit the results from both estimators are largely similar, the analyses in this paper are guided by the coefficients of the staggered difference-in-differences estimator.

    The results on industry costs are reported in Table~\ref{tab:baseline-industry-costs}. I find that a higher MW policy increases manufacturing industry wages (labour costs) in treated counties relative to adjacent control counties. Manufacturing industry wages per hour rose by $\$0.89$. This result is almost twice as strong than that documented in~\citet{gopalan2021state} for the industry-wide effect on hourly wages of $\$0.48$. This suggests that the manufacturing industry is strongly affected by the MW policy relative to other industries in the US. Hence, the bite in industry costs due to a higher MW floor is higher in the manufacturing industry compared to others.~\footnote{\tiny Figure~\ref{fig:baseline-manufacturing-industry-cost-heter} of Appendix~\ref{sec:appendix-baseline-robustness-tables-and-figures} illustrates that the impact on manufacturing industry labour costs is more pronounced in high-skilled workers compared to low-skilled workers. The figure shows an immediate increase in hourly wages. While this increase is short-lived for low-skilled workers, it persists for up to three years in post-treatment for high-skilled workers. Similarly, total payroll for high-skilled workers increases, whereas it substantially declines for low-skilled workers in the second year of post-treatment. These findings indicate that the labour cost effect of raising the minimum wage is stronger for high-skilled workers in manufacturing industries, a spillover effect. Notably, low-skilled workers, defined as those below the 30th percentile of the first-year of the pre-treatment wage distribution and comprising $18.2\%$ of workers, further demonstrate that the manufacturing industry workforce is skewed towards high-skilled workers. Furthermore, the labour cost burden is dominated in high-profit and capital-intensive industries relative to low-profit and labour-intensive industries, where there is declining labour cost throughout the spectrum.} I further document an increase in total payroll in treated counties relative to adjacent control counties by $4.4$ percentage points (ppts), as well as an increase in cost of materials in manufacturing industries by $12.9ppts$.

    The cohorts specific effects reveal that except for the $2015$ cohort where labour cost rose by $\$0.38$, it is mute for both the $2014$ and $2017$ cohorts. This suggests that the bite of raising MW is strongest for states that first raised their MW floor in $2015$. Similarly, I find an increase in total payroll for the $2015$ cohort, and an increase in material cost for the $2014$ cohort. However, I find a declining effect in total payroll and material cost for the $2017$ cohort---Maine. The disemployment effect in~\ref{subsec:baseline-results-employment-and-hours} for the $2017$ cohort explains this declining labour cost.
    \begin{figure}[H]
    \centering
    \includegraphics[width=1\textwidth,keepaspectratio]{fig_sdid_industry_costs}
    \caption{Manufacturing Industry Costs}
    \label{fig:baseline-manufacturing-industry-costs}
    \begin{minipage}{14cm}
        \vspace{0.05in}
        NOTES: The event study model of equation~\ref{eq:baseline-wages} is $C_{i,cp,t} = \sum_{{e = -3},{e \neq -1}}^{3} \beta Treated_{s,t}^e = \textbf{1}[t - G_{s,t}] + \delta X_{v,c,t-1} + \omega P_{f,t} + \lambda_{t} + \sigma_{c} + \phi_{cp} + \zeta_{cp,t} + \epsilon_{i,cp,t}$. Standard errors are clustered at the state level. de Chaisemartin and D'Haultfoeuille Decomposition: $\sum dCDH_{ATTs}^{weights(+)} = 1$ and $\sum dCDH_{ATTs}^{weights(-)} = 0$.
    \end{minipage}
\end{figure}

    Figure~\ref{fig:baseline-manufacturing-industry-costs} reports the dynamic effects. It shows increases in hourly wages in the first and second year following an MW policy. Moreover, I find instantaneous increases in the manufacturing industry's total payrolls and cost of materials, persisting up to three years after initially raising MW. There is no evidence of significant pre-trends. The timing and size of the effect are consistent with my data and setting.

    \subsection{Baseline Results: Employment and Hours}\label{subsec:baseline-results-employment-and-hours}
    This subsection estimates the effect of raising MW on employment of manufacturing industry workers and their production hours. The model is given by:
    \begin{equation}
        E_{i,cp,t} = \beta Treated_{s,t}^e + \delta X_{v,c,t-1} + \omega F_{f,t} + \lambda_{t} + \sigma_{c} + \phi_{cp} + \zeta_{cp,t} + \epsilon_{i,cp,t},\label{eq:baseline-emp-hours}
    \end{equation}
    where $E_{i,cp,t}$ is the vector of employment, total production workers and hours in manufacturing industry, $i$ in cross-border county pairs, $cp$ in the year, $t$. Standard errors are clustered at the state level.
    % Please add the following required packages to your document preamble:
% \usepackage{booktabs}
% \usepackage{graphicx}
\begin{table}[H]
    \centering
    \caption{Effect of the MW Policy on Employment and Production Workers' Hours}
    \label{tab:baseline-employment-hours}
    \resizebox{\columnwidth}{!}{%
        \begin{tabular}{@{}lllllll@{}}
            \toprule\toprule
            & \multicolumn{2}{c}{Employment (log)} & \multicolumn{2}{c}{Production Workers (log)} & \multicolumn{2}{c}{Production Hours (log)} \\
            \cmidrule(lr){2-3} \cmidrule(lr){4-5} \cmidrule(lr){6-7}
            employment \& hours & 1         & 2         & 3         & 4         & 5         & 6         \\ \midrule
            treated             & -0.043    & -0.002    & -0.066*   & -0.023    & -0.064*   & -0.019    \\
            & (0.031)   & (0.025)   & (0.035)   & (0.033)   & (0.035)   & (0.034)   \\
            cohort 2014         & -0.063**  & -0.058    & -0.097*** & -0.097*   & -0.096*** & -0.094*   \\
            & (0.027)   & (0.039)   & (0.032)   & (0.050)   & (0.033)   & (0.053)   \\
            cohort 2015         & -0.010    & 0.095***  & -0.014    & 0.104***  & -0.012    & 0.111***  \\
            & (0.034)   & (0.018)   & (0.070)   & (0.024)   & (0.070)   & (0.024)   \\
            cohort 2017         & -0.117*** & -0.552*** & -0.088*** & -0.568*** & -0.068*** & -0.556*** \\
            & (0.019)   & (0.136)   & (0.022)   & (0.156)   & (0.023)   & (0.144)   \\
            controls            & Yes       & Yes       & Yes       & Yes       & Yes       & Yes       \\
            year FE             & Yes       & Yes       & Yes       & Yes       & Yes       & Yes       \\
            county FE           & Yes       & Yes       & Yes       & Yes       & Yes       & Yes       \\
            border-county FE    & No        & Yes       & No        & Yes       & No        & Yes       \\
            border-county LTs   & No        & Yes       & No        & Yes       & No        & Yes       \\ \midrule
            Observations        & 1,893,689 & 1,893,689 & 1,893,689 & 1,893,689 & 1,893,689 & 1,893,689 \\
            $R^2$               & 0.358     & 0.393     & 0.342     & 0.378     & 0.350     & 0.385     \\
            Baseline Mean       & 44.99     & 44.99     & 31.42     & 31.42     & 64.82     & 64.82     \\ \bottomrule \bottomrule
        \end{tabular}%
    }
    \begin{minipage}{18cm}
        \vspace{0.05in}
        These results are obtained from estimating model~\ref{eq:baseline-emp-hours}. Robust standard errors clustered at the state level are reported in parentheses. ***, **, and * denote significance levels at the less than $1\%$, $5\%$ and $10\%$, respectively.
    \end{minipage}
\end{table}

    The average treatment effect on the treated (ATT) is captured by $\beta$, which is the difference in the average effect of raising the MW floor on manufacturing industry employment, production workers and hours in treated counties relative to adjacent control counties.

    The results are reported in Table~\ref{tab:baseline-employment-hours}. It shows no significant changes in overall manufacturing industry employment, total production workers and hours following an MW policy in the treated counties relative to adjacent control counties. Particularly, the size of the effect suggests a sharp null effect of the MW policy on manufacturing industry employment including production workers and hours, after accounting for time-varying common shocks to border counties. These results are consistent with the labour market literature assuming monopsonistic competition~\citep{card2000minimum, aaronson2018industry, cengiz2019effect, wong2019minimum, dustmann2022reallocation}.

    However, I find heterogeneity in the cohort-specific effects. While there are disemployment effects in the $2014$ and $2017$ cohorts, there is a significant positive effect on employment in the $2015$ cohort. These disemployment effects as shown in Figure~\ref{fig:baseline-manufacturing-industry-employment-heter} in Appendix~\ref{sec:appendix-baseline-robustness-tables-and-figures} are dominated in low-skilled workers, low-profit and labour-intensive industries, and driven by the declining number of production workers and hours due to higher labour costs. These disemployment effects explain the decline in total payroll and material cost in the $2017$ cohort in subsection~\ref{subsec:baseline-results-industry-costs}. On the other hand, the positive employment effect is dominated in high-skilled workers, high-profit and capital-intensive industries. Similarly, the dynamic treatment effects in Figure~\ref{fig:baseline-employment-hours} show an instantaneous increase in manufacturing industry employment (including production workers and hours) followed by a decline in the third year after the initial raise in MW.\footnote{\tiny~\citet{neumark2019econometrics} argues that cross-border studies may be biased against detecting disemployment effects due to worker mobility spillovers. To test if my results are influenced by cross-county mobility, potentially violating the stable unit treatment assumption, I use the interaction of the staggered difference-in-differences coefficient with the distance between population centers. The principle here is that worker mobility to counties/states with higher MW decreases as geographic distance to that county/state increases. Table~\ref{tab:baseline-cross-county-state-mobility} in Appendix~\ref{sec:appendix-baseline-robustness-tables-and-figures} indicates that worker mobility does not affect the overall baseline results, including the positive employment effect specific to the $2015$ cohort. However, the disemployment effects observed for the $2014$ and $2017$ cohorts are explained by workers' unwillingness to commute to distant counties/states with higher MW. This supports the hypothesis that worker mobility decreases with increasing distance to higher MW regions. This indicates that the higher MW policy heightens the reluctance of low-skilled manufacturing industry workers to commute to regions with higher MW, as the demand for high-skilled manufacturing industry workers increase.} There is no evidence of pre-trends.
    \begin{figure}[H]
    \centering
    \includegraphics[width=1\textwidth, keepaspectratio]{fig_sdid_emp_hours}
    \caption{Industry Employment and Production Workers Hours}
    \label{fig:baseline-employment-hours}
    \begin{minipage}{14cm}
        \vspace{0.05in}
        NOTES: The event study model of equation~\ref{eq:baseline-emp-hours} is $E_{i,cp,t} = \sum_{{e = -3},{e \neq -1}}^{3} \beta Treated_{s,t}^e = \textbf{1}[t - G_{s,t}] + \delta X_{v,c,t-1} + \omega F_{f,t} + \lambda_{t} + \sigma_{c} + \phi_{cp} + \zeta_{cp,t} + \epsilon_{i,cp,t}$. Standard errors are clustered at the state level. de Chaisemartin and D'Haultfoeuille Decomposition: $\sum dCDH_{ATTs}^{weights(+)} = 1$ and $\sum dCDH_{ATTs}^{weights(-)} = 0$.
    \end{minipage}
\end{figure}

    \subsection{Baseline Results: Industrial Output}\label{subsec:baseline-results-industrial-output}
    This subsection estimates the effect of raising MW on outputs of the manufacturing industry. The model is given by:
    \begin{equation}
        Y_{i,cp,t} = \beta Treated_{s,t}^e + \delta X_{v,c,t-1} + \omega F_{f,t} + \lambda_{t} + \sigma_{c} + \phi_{cp} + \zeta_{cp,t} + \epsilon_{i,cp,t},\label{eq:baseline-output}
    \end{equation}
    where $Y_{i,cp,t}$ is the vector of manufacturing industry output, output per hour and output per worker (labour productivity), in manufacturing industry, $i$ in cross-border county pairs, $cp$ in the year, $t$. Standard errors are clustered at the state level.
    % Please add the following required packages to your document preamble:
% \usepackage{booktabs}
% \usepackage{graphicx}
\begin{table}[H]
    \centering
    \caption{Effect of the MW policy on Manufacturing Industry Output}
    \label{tab:baseline-industry-output}
    \resizebox{\columnwidth}{!}{%
        \begin{tabular}{@{}lllllll@{}}
            \toprule\toprule
            Industry outputs (log) & \multicolumn{2}{c}{Output} & \multicolumn{2}{c}{Output per Hour} & \multicolumn{2}{c}{Output per Worker} \\
            \cmidrule(lr){2-3} \cmidrule(lr){4-5} \cmidrule(lr){6-7}
            & 1         & 2         & 3         & 4         & 5         & 6         \\ \midrule
            treated           & -0.020    & 0.125***  & 0.045     & 0.144***  & 0.024     & 0.127***  \\
            & (0.080)   & (0.032)   & (0.083)   & (0.038)   & (0.082)   & (0.032)   \\
            cohort 2014       & -0.078    & 0.122**   & 0.018     & 0.216***  & -0.015    & 0.180***  \\
            & (0.123)   & (0.050)   & (0.131)   & (0.059)   & (0.129)   & (0.049)   \\
            cohort 2015       & 0.079     & 0.135***  & 0.090***  & 0.024     & 0.089***  & 0.039*    \\
            & (0.063)   & (0.016)   & (0.026)   & (0.028)   & (0.020)   & (0.020)   \\
            cohort 2017       & -0.086*** & 0.447***  & -0.018    & 0.108**   & 0.032     & 0.105***  \\
            & (0.025)   & (0.101)   & (0.026)   & (0.045)   & (0.025)   & (0.037)   \\
            controls          & Yes       & Yes       & Yes       & Yes       & Yes       & Yes       \\
            year FE           & Yes       & Yes       & Yes       & Yes       & Yes       & Yes       \\
            county FE         & Yes       & Yes       & Yes       & Yes       & Yes       & Yes       \\
            border-county FE  & No        & Yes       & No        & Yes       & No        & Yes       \\
            border-county LTs & No        & Yes       & No        & Yes       & No        & Yes       \\ \midrule
            Observations      & 1,893,689 & 1,893,689 & 1,893,689 & 1,893,689 & 1,893,689 & 1,893,689 \\
            $R^2$             & 0.504     & 0.548     & 0.590     & 0.630     & 0.602     & 0.648     \\
            Baseline Mean     & 177.13    & 177.13    & 2.62      & 2.62      & 3.65      & 3.65      \\ \bottomrule\bottomrule
        \end{tabular}%
    }
    \begin{minipage}{18cm}
        \vspace{0.05in}
        These results are obtained from estimating model~\ref{eq:baseline-output}. Robust standard errors clustered at the state level are reported in parentheses. ***, **, and * denote significance levels at the less than $1\%$, $5\%$ and $10\%$, respectively.
    \end{minipage}
\end{table}

    The average treatment effect on the treated (ATT) is captured by $\beta$, which is the difference in the average effect of raising the MW floor on manufacturing industry output, output per hour and output per worker in treated counties relative to adjacent control counties.

    The results are reported in Table~\ref{tab:baseline-industry-output}. Following an MW policy, I document large statistically significant increases in manufacturing industry outputs by $12.50ppts$ in treated counties for the $2014$ and $2015$ cohorts. Additionally, the output per hour and labour productivity rose by $14.4$ and $12.7$ (ppts), respectively. Thus, suggesting that for every $\$0.89/hr$ increase in wages, output per hour and labour productivity rise by those margins.~\footnote{\tiny Back of the envelop calculation shows that manufacturing industry output per hour and labour productivity increases by an additional $0.144 \cdot 2.62 \cdot \left(\frac{\$100m}{\$1m}\right) = \$37.73$ units per hour and $0.117 \cdot 3.65 \cdot \left(\frac{\$100m}{\$1000}\right) = \$42,705$ units per worker, respectively.} The cohort-specific effects reveal significant and substantially strong increases in total output, output per hour and output per worker across all cohorts, except for the $2017$ cohort. The decline in overall output for the $2017$ cohort is explained by the decline in employment and number of production hours in labour-intensive industries since low-skilled workers are less likely to commute to distant higher MW counties/states.

    Figure~\ref{fig:baseline-industry-output} records consistent results. The MW policy caused instantaneous significant increases in manufacturing industry outputs, output per hour and output per worker in treated counties relative to adjacent control counties. The effects on output per hour and output per worker persists throughout the spectrum. Importantly, there is no evidence of significant pre-trends.
    \begin{figure}[H]
    \centering
    \includegraphics[width=1\textwidth, keepaspectratio]{C:/Users/david/OneDrive/Documents/ULMS/PhD/Thesis/chapter3/src/climate_change/latex/fig_sdid_output}
    \caption{Manufacturing Industry Output: Output per Hour and Output per Worker}
    \label{fig:baseline-industry-output}
    \begin{minipage}{\columnwidth}
        \vspace{0.05in}
        \tiny NOTES: The event study model of equation~\ref{eq:baseline-wages} is $Y_{i,cp,t} = \sum_{{e = -3},{e \neq -1}}^{3} \beta Treated_{s,t}^e = \textbf{1}[t - G_{s,t}] + \delta X_{v,c,t-1} + \omega F_{f,t} + \lambda_{t} + \sigma_{c} + \phi_{cp} + \zeta_{cp,t} + \epsilon_{i,cp,t}$. Standard errors are clustered at the state level. de Chaisemartin and D'Haultfoeuille Decomposition: $\sum dCDH_{ATTs}^{weights(+)} = 1$ and $\sum dCDH_{ATTs}^{weights(-)} = 0$.
    \end{minipage}
\end{figure}

    \subsection{Baseline Robustness}\label{subsec:baseline-robustness}
    I conduct several robustness exercises to confirm the robustness of the baseline industry results presented above.

    \subsubsection{Standard Errors} Standard errors are clustered at the facility, zipcode, industry NAICS codes, and county levels. I document that the results are not sensitive to these alternative clustering. See Tables~\ref{tab:baseline-cost-robustness},~\ref{tab:baseline-employ-robustness} and~\ref{tab:baseline-output-robustness} in Appendix~\ref{sec:appendix-baseline-robustness-tables-and-figures}.

    \subsubsection{State-level Results} I repeat all the above analysis at the state-level. The results persist even at this level, and are presented in the online supplementary material.

    I have established that raising MW indeed increases manufacturing industry labour costs, cost of materials and outputs, especially in the $2014$ and $2015$ cohorts. I also find increases in output per hour and labour productivity, across all cohorts. Moreover, the hypothesis of null overall employment effect subsists with heterogeneity in specific cohorts. While cross-county worker mobility is responsible for the disemployment effects of low-skilled workers in low-profit and labour-intensive industries, the positive employment effect in the $2015$ cohort is entirely due to the hiring of more high-skilled workers in high-profit and capital-intensive industries caused by the higher MW policy. However, there is notable decline in material cost and output in the $2017$ cohort caused by a corresponding disemployment effect.

    In what follows, I examine the following topical questions: $(i)$ is the increased cost-burden due to higher MW passed onto the environment, in form of higher pollution emissions per $\$100m$ units of output? $(ii)$ is the environmental impact heterogeneous? $(iii)$ what are the transmission mechanisms of the effects?


    \section{Onsite Toxic Releases}\label{sec:onsite-toxic-releases}
    I begin by estimating the effect of raising MW on total onsite toxic releases intensity generated at manufacturing industry facilities. The baseline model is given by:
    \begin{equation}
        P_{f,cp,c,t} = \beta Treated_{s,t}^e + \delta X_{v,c,t-1} + \omega F_{f,t} + \lambda_{t} + \gamma_{f} + \phi_{cp} + \zeta_{c} + \eta_{c,t} + \theta_{cp,t} + \varepsilon_{f,cp,c,t},\label{eq:baseline-total-onsite-releases-intensity}
    \end{equation}
    where $P_{f,cp,c,t}$ is the total onsite releases intensity at manufacturing industry facility, $f$ in cross-border county pairs, $cp$ through toxic chemical use, $c$ in the year, $t$. Total onsite releases intensity is the sum of pounds weights (lbs) of total onsite toxic air emissions, land releases, and surface water discharge intensities. $Treated_{s,t}^e = \textbf{1}[t - G_{s,t}]$, $X_{v,c,t-1}$ and $\lambda_{t}$ are as defined in subsection~\ref{subsec:baseline-results-industry-costs}. $F_{f,t}$ contains facility-level variables and includes the maximum number of toxic chemicals at the facility at any point in time, production/activity ratio index of the facility, and the following dummies: industry ownership, and toxic chemical attributes, that is, whether the chemical was produced at or imported to the facility, used as a formulation or article component, used as a manufacturing aid or for ancillary purposes. Further I control for environmental policy variables such as the Clean Air Act (CAA) of $1970$ and as amended in $1990$ to capture Hazardous Air Pollutants (HAPs), proxied by the dummy of CAA and HAPs regulated chemicals, and the Toxic Substances Control Act (TSCA) of $1976$ proxied by the dummy of persistent bio-accumulative chemicals.

    I control for facility-level fixed effect, $\gamma_{f}$ to account for within facility-level differences in the management of toxic releases intensities at the facility; cross-border county pairs, $\phi_{cp}$ fixed effect as defined in subsection~\ref{subsec:baseline-results-industry-costs}. $\zeta_{c}$ is the toxic chemical-fixed effect that controls for within chemical usage or mixture or compound differences in the production functions of manufacturing facilities; and $\eta_{c,t}$ toxic chemical linear trends which controls for the evolution of common time-varying shocks that may affect the nature of chemical usage at manufacturing industry facilities. $\theta_{cp,t}$ is the border-county linear trends to control for common time-varying shocks that affect border-counties such as change in county governments and national economic conditions. Finally, $\varepsilon_{f,cp,c,t}$ is the idiosyncratic error term. I employ a three-way clustering of the standard errors at the chemical use, industry, and state levels, because changes in MW may be correlated within a state, within manufacturing industries in the state, and in the nature of toxic chemical usage in the industry.
    % Please add the following required packages to your document preamble:
% \usepackage{booktabs}
% \usepackage{graphicx}
\begin{table}[H]
    \centering
    \caption{Effect of the MW policy on Total Onsite Toxic Releases Intensity}
    \label{tab:baseline-total-onsite-releases-intensity}
    \resizebox{\columnwidth}{!}{%
        \begin{tabular}{@{}llll@{}}
            \toprule\toprule
            Total releases intensity (log) & 1         & 2         & 3         \\ \midrule
            $Treated^{e}$                  & 0.120**   & 0.120**   & 0.109**   \\
            & (0.052)   & (0.052)   & (0.049)   \\
            cohort 2014                    & 0.077     & 0.077     & 0.090**   \\
            & (0.055)   & (0.055)   & (0.044)   \\
            cohort 2015                    & 0.191***  & 0.191***  & 0.139*    \\
            & (0.073)   & (0.073)   & (0.077)   \\
            cohort 2017                    & 0.030     & 0.030     & 0.223**   \\
            & (0.061)   & (0.061)   & (0.090)   \\
            controls                       & Yes       & Yes       & Yes       \\
            year FE                        & Yes       & Yes       & Yes       \\
            facility FE                    & Yes       & Yes       & Yes       \\
            border-county FE               & No        & Yes       & Yes       \\
            toxic chemical FE              & No        & No        & Yes       \\
            toxic chemical LTs             & No        & No        & Yes       \\\midrule
            Observations                   & 1,893,689 & 1,893,689 & 1,893,689 \\
            $R^2$                          & 0.520     & 0.520     & 0.720     \\
            Baseline Mean                  & 87.99     & 87.99     & 87.99     \\ \bottomrule\bottomrule
        \end{tabular}%
    }
    \begin{minipage}{18cm}
        \vspace{0.05in}
        These results are obtained from estimating model~\ref{eq:baseline-total-onsite-releases-intensity}. Three-way clustered robust standard errors are reported in parentheses, and clustered at the toxic chemical, industry and state levels. ***, **, and * denote significance levels at the less than $1\%$, $5\%$ and $10\%$, respectively.
    \end{minipage}
\end{table}

    The overall and cohort-specific average treatment effect on the treated (ATT) is captured by $\beta$, which is the difference in the average effect of raising the MW floor on total onsite releases intensity at manufacturing industry facilities in treated counties relative to adjacent control counties. The results on total onsite releases intensity are reported in Table~\ref{tab:baseline-total-onsite-releases-intensity}, and dominated in the $2014$ and $2015$ cohorts. It shows that a higher MW policy increases total onsite releases per $\$100m$ units of manufacturing industry output by $11.9ppts$ in treated counties relative to adjacent control counties. This translates to a $10.47lbs$ additional increase in total onsite releases intensity for every $\$0.89/hr$ increase in the wages of manufacturing industry workers.~\footnote{\tiny Alternatively, this means that for every $\$12.5$ units increase in manufacturing industry output, due to a higher MW policy, total onsite releases intensity increases by an average of $10.10lbs$. Similarly, for every $14.4ppts$ and $12.7ppts$ respective increases in output per hour and labour productivity, due to higher MW, total onsite releases intensity also increases by an average of $12.67lbs$ and $11.17lbs$.} This effect persists even after controlling for time-varying common shocks that may affect border-counties and the toxic chemical usage in manufacturing industries. There is limited evidence of an increase in the total onsite releases intensity in the $2017$ cohort.
    \begin{figure}[H]
    \centering
    \includegraphics[width=1\textwidth, height=0.5\textheight,keepaspectratio]{fig_sdid_total_releases_onsite_int}
    \caption{Total Onsite Releases Intensity}
    \label{fig:baseline-total-onsite-releases-intensity}
    \begin{minipage}{18cm}
        \vspace{0.05in}
        NOTES: The event study model of equation~\ref{eq:baseline-total-onsite-releases-intensity} is $P_{f,cp,c,t} = \sum_{{e = -3},{e \neq -1}}^{3} \beta Treated_{s,t}^e + \delta X_{v,c,t-1} + \omega F_{f,t} + \lambda_{t} + \gamma_{f} + \phi_{cp} + \zeta_{c} + \eta_{c,t} + \varepsilon_{f,cp,c,t}$. Three-way clustered robust standard errors are reported in parentheses, and clustered at the toxic chemical, industry and state levels. The test for the presence of pre-trends shows and F-statistic of $0.1461$ $(0.9296)$, p-value (in parentheses) using F-Statistic $(\chi^2): \sum_{-3}^{-2} \beta_{e} = 0$. de Chaisemartin and D'Haultfoeuille Decomposition: $\sum dCDH_{ATTs}^{weights(+)} = 1$ and $\sum dCDH_{ATTs}^{weights(-)} = 0$.
    \end{minipage}
\end{figure}

    Moreover, Figure~\ref{fig:baseline-total-onsite-releases-intensity} shows an instantaneous increase of $7.7ppts$ in total onsite releases intensity, reaching $23.5ppts$ two years later and then receded to $13.2ppts$ in the third year, for treated counties relative to adjacent control counties. There is no significant evidence of pre-trends. As a robustness check, the ATTs from the alternative TWFE estimator show consistent and similar effects.

    \subsection{Air Emissions Intensity}\label{subsec:air-emission-intensity}
    Here, I estimate the effect of raising MW on onsite air emission intensity. The model is given by:
    \begin{equation}
        A_{f,cp,c,t} = \beta Treated_{s,t}^e + \delta X_{v,c,t-1} + \omega F_{f,t} + \lambda_{t} + \gamma_{f} + \phi_{cp} + \zeta_{c} + \eta_{c,t} + \theta_{cp,t} + \varepsilon_{f,cp,c,t},\label{eq:baseline-onsite-air-emission-intensity}
    \end{equation}
    where $A_{f,cp,c,t}$ is the total onsite air emission intensity from manufacturing industry facility, $f$ in cross-border county pairs, $cp$ through toxic chemical, $c$ in the year, $t$.

    Total onsite air emission intensity is the sum of point/stack air emissions and fugitive air emissions intensities. Point air emissions usually involve the release of pollutants or substances into the atmosphere from industrial or commercial sources through a designated stack, chimney, or venting mechanism. These emissions occur as a result of combustion processes, industrial operations, or other activities that involve the burning, processing, or handling of materials. Stack emissions are typically regulated and monitored to ensure compliance with environmental regulations and standards. Fugitive air emissions on the other hand, refer to the release of pollutants or substances into the atmosphere from various industrial, commercial, or other sources that are not captured by a stack/point, duct, or other venting mechanism. These emissions typically occur during the handling, storage, processing, or transportation of materials and can originate from leaks, spills, evaporation, or other unintended releases from wear and tear of equipments. Examples of emitted pollutants include volatile organic compounds, hazardous air pollutants, and particulate matter, \textit{inter alia}. Monitoring and controlling stack emissions are critical for minimizing air pollution, protecting public health, and reducing environmental impacts.
    % Please add the following required packages to your document preamble:
% \usepackage{booktabs}
% \usepackage{graphicx}
\begin{table}[H]
    \centering
    \caption{Effect of the MW policy on Air Emissions Intensity}
    \label{tab:baseline-onsite-air-emissions-intensity}
    \resizebox{\columnwidth}{!}{%
        \begin{tabular}{@{}lllllll@{}}
            \toprule\toprule
            & \multicolumn{2}{c}{Total} & \multicolumn{2}{c}{Point} & \multicolumn{2}{c}{Fugitive} \\
            \cmidrule(lr){2-3}  \cmidrule(lr){4-5} \cmidrule(lr){6-7}
            Air emissions intensity (log) & 1         & 2         & 3         & 4         & 5         & 6         \\ \midrule
            $Treated^{e}$                 & 0.101**   & 0.088**   & 0.061*    & 0.035     & 0.076     & 0.072*    \\
            & (0.047)   & (0.039)   & (0.036)   & (0.026)   & (0.050)   & (0.041)   \\
            cohort 2014                   & 0.100*    & 0.097**   & 0.107***  & 0.104***  & 0.018     & -0.000    \\
            & (0.053)   & (0.042)   & (0.047)   & (0.038)   & (0.051)   & (0.040)   \\
            cohort 2015                   & 0.103     & 0.122     & -0.014    & -0.081    & 0.175**   & 0.192***  \\
            & (0.067)   & (0.076)   & (0.072)   & (0.050)   & (0.042)   & (0.062)   \\
            cohort 2017                   & -0.045    & 0.076*    & 0.001     & 0.154**   & -0.087**  & -0.009    \\
            & (0.064)   & (0.040)   & (0.062)   & (0.073)   & (0.042)   & (0.051)   \\
            controls                      & Yes       & Yes       & Yes       & Yes       & Yes       & Yes       \\
            year FE                       & Yes       & Yes       & Yes       & Yes       & Yes       & Yes       \\
            facility FE                   & Yes       & Yes       & Yes       & Yes       & Yes       & Yes       \\
            border-county FE              & Yes       & Yes       & Yes       & Yes       & Yes       & Yes       \\
            toxic chemical FE             & No        & Yes       & No        & Yes       & No        & Yes       \\
            toxic chemical LTs            & No        & Yes       & No        & Yes       & No        & Yes       \\ \midrule
            Observations                  & 1,893,689 & 1,893,689 & 1,893,689 & 1,893,689 & 1,893,689 & 1,893,689 \\
            $R^2$                         & 0.522     & 0.739     & 0.516     & 0.712     & 0.472     & 0.660     \\
            Baseline Mean                 & 60.02     & 60.02     & 49.07     & 49.07     & 10.95     & 10.95     \\ \bottomrule\bottomrule
        \end{tabular}%
    }
    \begin{minipage}{\columnwidth}
        \vspace{0.05in}
        NOTES: These results are obtained from estimating model~\ref{eq:baseline-onsite-air-emission-intensity}. Three-way clustered robust standard errors are reported in parentheses, and clustered at the toxic chemical, industry and state levels. ***, **, and * denote significance levels at the less than $1\%$, $5\%$ and $10\%$, respectively.
    \end{minipage}
\end{table}

    The parameter of interest is $\beta$ which measures the overall and cohort-specific ATT of the MW policy on air emissions intensity at manufacturing industry facilities in treated counties relative to adjacent control counties. The results are presented in Table~\ref{tab:baseline-onsite-air-emissions-intensity}. I find an increase of $10.2ppts$ in total air emissions intensity in treated counties relative to adjacent control counties. Albeit the reported effect size is smaller than the size of the effect in~\citet{zhang2023unintended}, the result is similar to their conclusions for Chinese manufacturing industries. The focus on only specific air emitted toxic chemicals may explain the distance in the effect size here relative to the Chinese paper. Back of the envelop calculation shows that for every $\$0.89$ increase in wages per hour, air emissions intensity increases by $6.12lbs$ per $\$100m$ units in manufacturing industry outputs. Furthermore, the reported effect is predominantly driven by both point and fugitive air emissions intensities, strongest for the $2014$ and $2015$ cohorts, respectively. I find limited evidence of an increase in the $2017$ cohort.

    Moreover, the results of the dynamic effects in Figure~\ref{fig:baseline-onsite-air-emission-intensity} show an instantaneous increase of $5.7ppts$ in total air emissions intensity, reaching $16.6ppts$ two years after the initial MW raise and then dropping to $15.8ppts$ in the third year. Similarly I find significant increases in point air emission intensity between the second and third year of post treatment, while that of fugitive air emission intensity is significant only in the first and second year after the initial MW raise. No significant pre-trends is recorded.
    \begin{figure}[H]
    \centering
    \includegraphics[width=1\textwidth, height=0.5\textheight,keepaspectratio]{C:/Users/david/OneDrive/Documents/ULMS/PhD/Thesis/chapter3/src/climate_change/latex/fig_sdid_onsite_air_emissions_int}
    \caption{Total Onsite Releases Intensity}
    \label{fig:baseline-onsite-air-emission-intensity}
    \begin{minipage}{\columnwidth}
        \vspace{0.05in}
        \tiny NOTES: The event study model of equation~\ref{eq:baseline-onsite-air-emission-intensity} is $A_{f,cp,c,t} = \sum_{{e = -3},{e \neq -1}}^{3} \beta Treated_{s,t}^e + \omega F_{f,t} + \lambda_{t} + \gamma_{f} + \phi_{cp} + \zeta_{c} + \eta_{c,t} + \theta_{cp,t} + \varepsilon_{f,cp,c,t}$. Three-way clustered robust standard errors are reported in parentheses, and clustered at the toxic chemical, industry and state levels. de Chaisemartin and D'Haultfoeuille Decomposition: $\sum dCDH_{ATTs}^{weights(+)} = 1$ and $\sum dCDH_{ATTs}^{weights(-)} = 0$.
    \end{minipage}
\end{figure}

    \subsection{Water Discharge Intensity}\label{subsec:water-discharge-intensity}
    Next, I estimate the effect of raising MW on total onsite surface water discharge intensity. The model is given by:
    \begin{equation}
        W_{f,cp,c,t} = \beta Treated_{s,t}^e + \delta X_{v,c,t-1} + \omega F_{f,t} + \lambda_{t} + \gamma_{f} + \phi_{cp} + \zeta_{c} + \eta_{c,t} + \theta_{cp,t} + \varepsilon_{f,cp,c,t},\label{eq:baseline-onsite-water-discharge-intensity}
    \end{equation}
    where $W_{f,cp,c,t}$ is the total onsite surface water discharge intensity from a manufacturing industry facility, $f$ in cross-border county pairs, $cp$ through a toxic chemical, $c$ in the year, $t$.

    Surface water discharge intensity is the total weight of toxic chemicals per $\$100m$ units of output in the form of contaminants, or wastewater from industrial or commercial facilities that are released into surface water bodies such as streams. This discharge can occur through various pathways, including direct discharge through pipes, drains, or outfalls, as well as through runoff from facility surfaces or surrounding areas. Surface water discharge at facilities is regulated by the Clean Water Act of $1972$ to protect water quality, safeguard human health, and preserve aquatic ecosystems. Facilities are often required to obtain permits, monitor their discharge, and implement pollution control measures to minimize the impact of their activities on surface water resources.
    % Please add the following required packages to your document preamble:
% \usepackage{booktabs}
% \usepackage{graphicx}
\begin{table}[H]
    \centering
    \caption{Effect of the MW policy on Onsite Surface Water Discharge Intensity}
    \label{tab:baseline-onsite-water-disc-int}
    \resizebox{\columnwidth}{!}{%
        \begin{tabular}{@{}lllll@{}}
            \toprule\toprule
            Surface water discharge intensity (log) & \multicolumn{2}{c}{Total} & \multicolumn{2}{c}{Number of Receiving Streams} \\
            \cmidrule(lr){2-3}  \cmidrule(lr){4-5}
            & 1         & 2         & 3         & 4         \\ \midrule
            $Treated^{e}$      & 0.028     & 0.026     & -0.007    & -0.013    \\
            & (0.032)   & (0.033)   & (0.010)   & (0.011)   \\
            cohort 2014        & -0.037    & -0.028    & -0.019    & 0.029*    \\
            & (0.024)   & (0.025)   & (0.014)   & (0.015)   \\
            cohort 2015        & 0.136**   & 0.117**   & 0.012     & 0.014     \\
            & (0.064)   & (0.059)   & (0.011)   & (0.013)   \\
            cohort 2017        & 0.069**   & 0.073**   & 0.006     & 0.003     \\
            & (0.032)   & (0.033)   & (0.034)   & (0.035)   \\
            controls           & Yes       & Yes       & Yes       & Yes       \\
            year FE            & Yes       & Yes       & Yes       & Yes       \\
            facility FE        & Yes       & Yes       & Yes       & Yes       \\
            border-county FE   & Yes       & Yes       & Yes       & Yes       \\
            toxic chemical FE  & No        & Yes       & No        & Yes       \\
            toxic chemical LTs & No        & Yes       & No        & Yes       \\\midrule
            Observations       & 1,893,689 & 1,893,689 & 1,893,689 & 1,893,689 \\
            $R^2$              & 0.297     & 0.585     & 0.695     & 0.792     \\
            Baseline Mean      & 20.04     & 20.04     & 0.39      & 0.39      \\ \bottomrule\bottomrule
        \end{tabular}%
    }
    \begin{minipage}{18cm}
        \vspace{0.05in}
        NOTES: These results are obtained from estimating model~\ref{eq:baseline-onsite-water-discharge-intensity}. Three-way clustered robust standard errors are reported in parentheses, and clustered at the toxic chemical, industry and state levels. ***, **, and * denote significance levels at the less than $1\%$, $5\%$ and $10\%$, respectively.
    \end{minipage}
\end{table}

    The average treatment effect on the treated (ATT) is captured by $\beta$, which is the difference in the average effect of raising the MW floor on total surface water discharge intensity at manufacturing industry facilities in treated counties relative to adjacent control counties. The results on total surface water discharge are reported in Table~\ref{tab:baseline-onsite-water-disc-int}. I find no statistically significant differences in the overall positive effect of raising MW on surface water discharge intensity. However, I find heterogeneous effect between the $2014$ and $2015$ cohorts. Whereas raising MW reduces surface water discharge in the $2014$ cohort, and the opposite is documented in the $2015$ cohort. The positive effect on the $2017$ cohort is generally muted. Additonally, overall number of receiving streams decreases, although rising for the $2014$ cohort.

    Similarly, the results in Figure~\ref{fig:baseline-onsite-water-discharge-intensity} shows that surface water discharge intensity jumped to $8.4ppts$ two years later. I find no evidence of significant pre-trends. However, the declining number of receiving streams are not necessarily causal as there are significant pre-trends.
    \begin{figure}[H]
    \centering
    \includegraphics[width=1\textwidth, height=0.5\textheight,keepaspectratio]{C:/Users/david/OneDrive/Documents/ULMS/PhD/Thesis/chapter3/src/climate_change/latex/fig_sdid_onsite_water_discharge_int}
    \caption{Total Onsite Releases Intensity}
    \label{fig:baseline-onsite-water-discharge-intensity}
    \begin{minipage}{18cm}
        \vspace{0.05in}
        \tiny NOTES: The event study model of equation~\ref{eq:baseline-onsite-water-discharge-intensity} is $W_{f,cp,c,t} = \sum_{{e = -3},{e \neq -1}}^{3} \beta Treated_{s,t}^e + \delta X_{v,c,t-1} + \omega F_{f,t} + \lambda_{t} + \gamma_{f} + \phi_{cp} + \zeta_{c} + \eta_{c,t} + \theta_{cp,t} + \varepsilon_{f,cp,c,t}$. Three-way clustered robust standard errors are reported in parentheses, and clustered at the toxic chemical, industry and state levels. de Chaisemartin and D'Haultfoeuille Decomposition: $\sum dCDH_{ATTs}^{weights(+)} = 1$ and $\sum dCDH_{ATTs}^{weights(-)} = 0$.
    \end{minipage}
\end{figure}

    \subsection{Land Releases Intensity}\label{subsec:land-releases-intensity}
    Lastly, I estimate the effect of raising MW on total onsite land releases intensity. The model is given by:
    \begin{equation}
        L_{f,cp,c,t} = \beta Treated_{s,t}^e + \delta X_{v,c,t-1} + \omega F_{f,t} + \lambda_{t} + \gamma_{f} + \phi_{cp} + \zeta_{c} + \eta_{c,t} + \theta_{cp,t} + \varepsilon_{f,cp,c,t},\label{eq:baseline-onsite-land-releases-intensity}
    \end{equation}
    where $L_{f,cp,c,t}$ is the total onsite land releases intensity from manufacturing industry facility, $f$ in cross-border county pairs, $cp$ through toxic chemical, $c$ in the year, $t$. Total onsite land releases intensity is the total weight of toxic chemicals per $\$100m$ units of output that are released to land.
    % Please add the following required packages to your document preamble:
% \usepackage{booktabs}
% \usepackage{graphicx}
\begin{table}[H]
    \centering
    \caption{Effect of the MW policy on Onsite Land Releases Intensity}
    \label{tab:baseline-onsite-land-releases-intensity}
    \resizebox{\columnwidth}{!}{%
        \begin{tabular}{@{}lllllllllllll@{}}
            \toprule\toprule
            Land releases intensity (log) & \multicolumn{2}{c}{Total} & \multicolumn{2}{c}{Underground Injection} & \multicolumn{2}{c}{Landfills} & \multicolumn{2}{c}{To-Land Treatment} & \multicolumn{2}{c}{Surface Impoundment} & \multicolumn{2}{c}{Land Releases (Others)} \\
            \cmidrule(lr){2-3}  \cmidrule(lr){4-5}  \cmidrule(lr){6-7}  \cmidrule(lr){8-9}  \cmidrule(lr){10-11}  \cmidrule(lr){12-13}
            & 1         & 2         & 3         & 4         & 5         & 6         & 7         & 8         & 9         & 10        & 11        & 12        \\ \midrule
            $Treated^{e}$      & -0.025*   & -0.018    & 0.000     & 0.001     & -0.005    & -0.003    & 0.002     & 0.004     & 0.009*    & 0.015* & -0.030** & -0.034** \\
            & (0.015)   & (0.017)   & (0.000)   & (0.001)   & (0.003)   & (0.004)   & (0.007)   & (0.008)   & (0.005)   & (0.008) & (0.014) & (0.015)               \\
            cohort 2014        & -0.014    & -0.003    & -0.001    & 0.000     & -0.001    & -0.002    & 0.003     & 0.005     & 0.011*    & 0.020*    & -0.027* & -0.030* \\
            & (0.017)   & (0.021)   & (0.001)   & (0.000)   & (0.002)   & (0.003)   & (0.010)   & (0.012)   & (0.006)   & (0.009) & (0.015) & (0.016)               \\
            cohort 2015        & -0.042    & -0.044    & -0.001    & 0.000     & -0.012*   & -0.011    & 0.001     & 0.002     & 0.005*    & 0.007*    & -0.036 & -0.040 \\
            & (0.029)   & (0.033)   & (0.001)   & (0.000)   & (0.003)   & (0.009)   & (0.002)   & (0.005)   & (0.003)   & (0.004) & (0.030) & (0.034)               \\
            cohort 2017        & 0.025**   & 0.061**   & 0.002     & 0.002     & -0.001    & 0.002     & 0.000     & 0.000     & 0.014*    & 0.025*    & 0.008 & 0.032 \\
            & (0.011)   & (0.031)   & (0.001)   & (0.001)   & (0.002)   & (0.004)   & (0.002)   & (0.002)   & (0.008)   & (0.014) & (0.005) & (0.028)               \\
            controls           & Yes       & Yes       & Yes       & Yes       & Yes       & Yes       & Yes       & Yes       & Yes       & Yes       & Yes       & Yes       \\
            year FE            & Yes       & Yes       & Yes       & Yes       & Yes       & Yes       & Yes       & Yes       & Yes       & Yes       & Yes       & Yes       \\
            facility FE        & Yes       & Yes       & Yes       & Yes       & Yes       & Yes       & Yes       & Yes       & Yes       & Yes       & Yes       & Yes       \\
            border-county FE   & Yes       & Yes       & Yes       & Yes       & Yes       & Yes       & Yes       & Yes       & Yes       & Yes       & Yes       & Yes       \\
            toxic chemical FE  & No        & Yes       & No        & Yes       & No        & Yes       & No        & Yes       & No        & Yes       & No        & Yes       \\
            toxic chemical LTs & No        & Yes       & No        & Yes       & No        & Yes       & No        & Yes       & No        & Yes       & No        & Yes       \\\midrule
            Observations       & 1,893,689 & 1,893,689 & 1,893,689 & 1,893,689 & 1,893,689 & 1,893,689 & 1,893,689 & 1,893,689 & 1,893,689 & 1,893,689 & 1,893,689 & 1,893,689 \\
            $R^2$              & 0.345     & 0.500     & 0.380     & 0.382     & 0.438     & 0.466     & 0.312     & 0.323     & 0.080     & 0.126     & 0.234     & 0.592     \\
            Baseline Mean      & 7.93      & 7.93      & 4.80      & 4.80      & 1.43      & 1.43      & 0.66      & 0.66      & 0.03      & 0.03      & 1.01      & 1.01      \\ \bottomrule\bottomrule
        \end{tabular}%
    }
    \begin{minipage}{\columnwidth}
        \vspace{0.05in}
        These results are obtained from estimating model~\ref{eq:baseline-onsite-land-releases-intensity}. Three-way clustered robust standard errors are reported in parentheses, and clustered at the toxic chemical, industry and state levels. ***, **, and * denote significance levels at the less than $1\%$, $5\%$ and $10\%$, respectively.
    \end{minipage}
\end{table}

    The average treatment effect on the treated (ATT) is captured by $\beta$, which is the difference in the average effect of raising the MW floor on total surface water discharge intensity at manufacturing industry facilities in treated counties relative to adjacent control counties. The results on total land releases intensity are reported in Table~\ref{tab:baseline-onsite-land-releases-intensity}. Except for the total surface impoundment intensity, I find no statistically or economically significant effect on the intensities of total land releases, underground injection, landfills and to-land treatment (releases used for land fertilization).

    A surface impoundment is a type of containment structure used for the storage and management of liquid wastes, such as industrial wastewater, hazardous chemicals, or contaminated water. It consists of an excavated depression or basin lined with impermeable materials such as clay or synthetic liners to prevent the leakage of liquids into the surrounding soil and groundwater. The results show that a higher MW floor increases total surface impoundment intensity by $1.1ppts$ in treated counties relative to adjacent control counties.

    Similarly, the cohort-specific effect shows an increase in total land releases and landfill intensities in the $2017$ and $2014$ cohorts by $5.1ppts$ and $0.8ppts$, respectively. I find an increase in total surface impoundment intensity by$1.3ppts$ for the $2014$ cohort only. The positive effect on the $2015$ cohort is insignificant. Additionally, the event study shows an instantaneous increase in surface impoundment intensity persisting throughout the post-treatment, reaching $3.5ppts$. I record no significant evidence of pre-trends. However, I found no causal evidence on the increases in other total land releases intensity given significant pre-trends.
    \begin{figure}[H]
    \centering
    \includegraphics[width=1\textwidth, height=0.5\textheight,keepaspectratio]{fig_sdid_total_land_releases_onsite_int}
    \caption{Total Onsite Land Releases Intensity}
    \label{fig:baseline-onsite-land-releases-intensity}
    \begin{minipage}{\columnwidth}
        \vspace{0.05in}
        \tiny NOTES: The event study model of equation~\ref{eq:baseline-onsite-land-releases-intensity} is $L_{f,cp,c,t} = \sum_{{e = -3},{e \neq -1}}^{3} \beta Treated_{s,t}^e + \delta X_{v,c,t-1} + \omega F_{f,t} + \lambda_{t} + \gamma_{f} + \phi_{cp} + \zeta_{c} + \eta_{c,t} + \theta_{cp,t} + \varepsilon_{f,cp,c,t}$. Three-way clustered robust standard errors are reported in parentheses, and clustered at the toxic chemical, industry and state levels. de Chaisemartin and D'Haultfoeuille Decomposition: $\sum dCDH_{ATTs}^{weights(+)} = 1$ and $\sum dCDH_{ATTs}^{weights(-)} = 0$.
    \end{minipage}
\end{figure}


    \section{Robustness Exercises}\label{sec:robustness-exercises}
    I conduct several robustness exercises to raise the credibility of the above results. This ranges from placebo exercise, and alternative clustering of the standard errors.

    \subsection{Placebo Exercise}\label{subsec:placebo-exercise}
    To increase the credibility in the reported emission intensity results, I model the placebo effect of raising the MW on total releases intensity, from catastrophic events. Such events include accidental chemical spills, fire, explosions and natural disasters in specific manufacturing facilities. These catastrophic events are expected to be uncorrelated with the MW policy. Hence, I do not expect to see any statistically significant effect on such releases intensity. I estimate the following model:
    \begin{equation}
        P_{f,cp,c,t}^{catastrophic} = \beta Treated_{s,t}^e + \delta X_{v,c,t-1} + \omega F_{f,t} + \lambda_{t} + \gamma_{f} + \phi_{cp} + \zeta_{c} + \eta_{c,t} + \theta_{cp,t} + \varepsilon_{f,cp,c,t},\label{eq:robustness-placebo}
    \end{equation}
    % Please add the following required packages to your document preamble:
% \usepackage{booktabs}
% \usepackage{graphicx}
\begin{table}[H]
    \centering
    \caption{Effect of the MW policy on Total Onsite Toxic Releases Intensity, from Catastrophic Events}
    \label{tab:robustness-placebo}
    \resizebox{\columnwidth}{!}{%
        \begin{tabular}{@{}llll@{}}
            \toprule\toprule
            Total releases intensity, from catastrophic events (log) & 1         & 2         & 3         \\ \midrule
            $Treated^{e}$                                            & 0.008     & 0.004     & -0.010    \\
            & (0.049)   & (0.050)   & (0.058)   \\
            cohort 2014                                              & 0.068     & 0.086     & 0.024     \\
            & (0.075)   & (0.074)   & (0.082)   \\
            cohort 2015                                              & -0.092*** & -0.132*** & -0.066    \\
            & (0.034)   & (0.038)   & (0.053)   \\
            cohort 2017                                              & -0.135    & -0.120    & -0.328    \\
            & (0.239)   & (0.001)   & (0.338)   \\
            controls                                                 & Yes       & Yes       & Yes       \\
            year FE                                                  & Yes       & Yes       & Yes       \\
            facility FE                                              & Yes       & Yes       & Yes       \\
            border-county FE                                         & Yes       & Yes       & Yes       \\
            toxic chemical FE                                        & No        & Yes       & Yes       \\
            toxic chemical LTs                                       & No        & Yes       & Yes       \\
            border-county LTs                                        & No        & No        & Yes       \\ \midrule
            Observations                                             & 1,893,689 & 1,893,689 & 1,893,689 \\
            $R^2$                                                    & 0.774     & 0.803     & 0.819     \\
            Baseline Mean                                            & 4.36      & 4.36      & 4.36      \\ \bottomrule \bottomrule
        \end{tabular}%
    }
    \begin{minipage}{\columnwidth}
        \vspace{0.05in}
        \tiny NOTES: These results are obtained from estimating model~\ref{eq:robustness-placebo}. Three-way clustered robust standard errors are reported in parentheses, and clustered at the toxic chemical, industry and state levels. ***, **, and * denote significance levels at the less than $1\%$, $5\%$ and $10\%$, respectively.
    \end{minipage}
\end{table}

    where $P_{f,cp,c,t}$ is the total onsite releases intensity from catastrophic events at manufacturing industry facility, $f$ in cross-border county pairs, $cp$ through toxic chemical use, $c$ in the year, $t$.

    The ATT measures the difference in the average effect of raising the MW on total onsite releases intensity from catastrophic events at manufacturing industry facilities in treated counties relative to adjacent control counties. The results are presented in Table~\ref{tab:robustness-placebo}. Indeed, the result is consistent with the null hypothesis of the effect of raising MW on total releases intensity, from catastrophic events. This effect remains unchanged even after controlling for the nature of toxic chemical usage and time-varying common shocks affecting toxic chemical use and border-counties in columns $2$ and $3$, respectively. Additionally, Figure~\ref{fig:baseline-placebo} shows the dynamic placebo effect, revealing a null response in total releases intensity from catastrophic events and no evidence of significant pre-trends.
    \begin{figure}[H]
    \centering
    \includegraphics[width=1\textwidth, height=0.5\textheight,keepaspectratio]{fig_sdid_total_releases_onsite_catastrophicevents_int}
    \caption{Total Onsite Releases Intensity, from Catastrophic Events}
    \label{fig:baseline-placebo}
    \begin{minipage}{\columnwidth}
        \vspace{0.05in}
        \tiny NOTES: The event study model of equation~\ref{eq:robustness-placebo} is $P_{f,cp,c,t}^{catastrophic} = \sum_{{e = -5},{e \neq -1}}^{3}\beta Treated_{s,t}^e + \delta X_{v,c,t-1} + \omega F_{f,t} + \lambda_{t} + \gamma_{f} + \phi_{cp} + \zeta_{c} + \eta_{c,t} + \theta_{cp,t} + \varepsilon_{f,cp,c,t}$. Three-way clustered robust standard errors are reported in parentheses, and clustered at the toxic chemical, industry and state levels. The test for the presence of pre-trends shows and F-statistic of $0.6159$ $(0.7349)$, p-value (in parentheses) using F-Statistic $(\chi^2): \sum_{-3}^{-2} \beta_{e} = 0$. de Chaisemartin and D'Haultfoeuille Decomposition: $\sum dCDH_{ATTs}^{weights(+)} = 1$ and $\sum dCDH_{ATTs}^{weights(-)} = 0$.
    \end{minipage}
\end{figure}

    \subsection{Alternative Clustering of the SEs}\label{subsec:alternative-clustering-of-the-ses}
    The results in Tables~\ref{tab:robustness-ses-clustering-total-releases-onsite},~\ref{tab:robustness-ses-clustering-total-air-releases-onsite},~\ref{tab:robustness-ses-clustering-water-discharge-onsite-intensity}, and~\ref{tab:robustness-ses-clustering-total-land-releases-intensity} remain insensitive to alternative clustering of the standard errors.
    % Please add the following required packages to your document preamble:
% \usepackage{booktabs}
% \usepackage{graphicx}
\begin{table}[H]
    \centering
    \caption{Total Onsite Releases Intensity: Alternative clustering of the SEs}
    \label{tab:robustness-ses-clustering-total-releases-onsite}
    \resizebox{\columnwidth}{!}{%
        \begin{tabular}{@{}lllllllllllll@{}}
            \toprule\toprule
            Total releases intensity (log) & 1         & 2         & 3         & 4         & 5         & 6         & 7                    & 8                    & 9                    & 10                & 11                & 12                \\ \midrule
            $Treated^{e}$                        & 0.108**   & 0.108**   & 0.108**   & 0.108**   & 0.108**   & 0.108*    & 0.108*               & 0.108*               & 0.108**              & 0.108**         & 0.108**          & 0.108**          \\
            & (0.052)   & (0.050)   & (0.054)   & (0.055)   & (0.049)   & (0.063)   & (0.062)              & (0.062)              & (0.049)              & (0.049)          & (0.052)          & (0.055)          \\
            cohort 2014                    & 0.090     & 0.090*    & 0.090     & 0.0897    & 0.090**   & 0.090     & 0.090                & 0.090                & 0.090**              & 0.090**           & 0.090          & 0.090          \\
            & (0.056)   & (0.054)   & (0.062)   & (0.087)   & (0.044)   & (0.081)   & (0.080)              & (0.080)              & (0.044)              & (0.044)          & (0.081)          & (0.080)          \\
            cohort 2015                    & 0.139     & 0.139     & 0.139     & 0.139     & 0.139*    & 0.139     & 0.139                & 0.139                & 0.139*               & 0.139*            & 0.139          & 0.139          \\
            & (0.092)   & (0.092)   & (0.086)   & (0.055)   & (0.077)   & (0.097)   & (0.096)              & (0.096)              & (0.077)              & (0.077)          & (0.097)          & (0.096)          \\
            cohort 2017                    & 0.223     & 0.223     & 0.223     & 0.223     & 0.223**   & 0.223***  & 0.223***             & 0.223***             & 0.223**              & 0.223**         & 0.223***          & 0.223***          \\
            & (0.147)   & (0.151)   & (0.174)   & (0.141)   & (0.090)   & (0.060)   & (0.060)              & (0.060)              & (0.090)              & (0.090)          & (0.061)          & (0.060)          \\
            controls                       & Yes       & Yes       & Yes       & Yes       & Yes       & Yes       & Yes                  & Yes                  & Yes                  & Yes               & Yes               & Yes               \\
            year FE                        & Yes       & Yes       & Yes       & Yes       & Yes       & Yes       & Yes                  & Yes                  & Yes                  & Yes               & Yes               & Yes               \\
            facility FE                    & Yes       & Yes       & Yes       & Yes       & Yes       & Yes       & Yes                  & Yes                  & Yes                  & Yes               & Yes               & Yes               \\
            border-county FE               & Yes       & Yes       & Yes       & Yes       & Yes       & Yes       & Yes                  & Yes                  & Yes                  & Yes               & Yes               & Yes               \\
            toxic chemical FE              & Yes       & Yes       & Yes       & Yes       & Yes       & Yes       & Yes                  & Yes                  & Yes                  & Yes               & Yes               & Yes               \\
            toxic chemical LTs             & Yes       & Yes       & Yes       & Yes       & Yes       & Yes       & Yes                  & Yes                  & Yes                  & Yes               & Yes               & Yes               \\ \midrule
            clustered at the:              & facility  & zipcode   & county    & industry  & chemical  & state     & facility \& chemical & facility \& industry & chemical \& industry & chemical \& state & facility \& state & industry \& state \\
            Observations                   & 1,893,689 & 1,893,689 & 1,893,689 & 1,893,689 & 1,893,689 & 1,893,689 & 1,893,689            & 1,893,689            & 1,893,689            & 1,893,689         & 1,893,689         & 1,893,689         \\
            $R^2$                          & 0.720     & 0.720     & 0.720     & 0.720     & 0.720     & 0.720     & 0.720                & 0.720                & 0.720                & 0.720             & 0.720             & 0.720             \\ \bottomrule \bottomrule
        \end{tabular}%
    }
    \begin{minipage}{18cm}
        \vspace{0.05in}
        These results are obtained from estimating model~\ref{eq:baseline-total-onsite-releases-intensity}. ***, **, and * denote significance levels at the less than $1\%$, $5\%$ and $10\%$, respectively.
    \end{minipage}
\end{table}
    % Please add the following required packages to your document preamble:
% \usepackage{booktabs}
% \usepackage{graphicx}
\begin{table}[H]
    \centering
    \caption{Total Onsite Air Emissions Intensity: Alternative clustering of the SEs}
    \label{tab:robustness-ses-clustering-total-air-releases-onsite}
    \resizebox{\columnwidth}{!}{%
        \begin{tabular}{@{}lllllllllllll@{}}
            \toprule \toprule
            Total air emissions intensity (log) & 1         & 2         & 3         & 4         & 5         & 6         & 7                    & 8                    & 9                    & 10                & 11                & 12                \\ \midrule
            $Treated^{e}$                       & 0.088**   & 0.088**   & 0.088**   & 0.088*    & 0.088**   & 0.088*    & 0.088**              & 0.088**              & 0.088**              & 0.088**           & 0.088**            & 0.088*           \\
            & (0.041)   & (0.039)   & (0.042)   & (0.046)   & (0.039)   & (0.048)   & (0.041)              & (0.041)              & (0.039)              & (0.039)          & (0.041)          & (0.046)          \\
            cohort 2014                         & 0.097*    & 0.097**   & 0.097*    & 0.097*    & 0.097**   & 0.097     & 0.097*               & 0.097*               & 0.097**              & 0.097**           & 0.097*            & 0.097*           \\
            & (0.051)   & (0.049)   & (0.056)   & (0.058)   & (0.042)   & (0.069)   & (0.051)              & (0.051)              & (0.042)              & (0.042)          & (0.051)          & (0.058)          \\
            cohort 2015                         & 0.071     & 0.071     & 0.071     & 0.071*    & 0.071*    & 0.071     & 0.071                & 0.071                & 0.071                & 0.071             & 0.071             & 0.071             \\
            & (0.056)   & (0.056)   & (0.051)   & (0.055)   & (0.052)   & (0.048)   & (0.056)              & (0.056)              & (0.051)              & (0.052)          & (0.056)          & (0.055)          \\
            cohort 2017                         & 0.122     & 0.122     & 0.122     & 0.122*    & 0.122*    & 0.122***  & 0.122                & 0.122                & 0.122                & 0.122             & 0.122            & 0.122           \\
            & (0.111)   & (0.115)   & (0.131)   & (0.116)   & (0.076)   & (0.035)   & (0.111)              & (0.111)              & (0.076)              & (0.076)          & (0.111)          & (0.116)          \\
            controls                            & Yes       & Yes       & Yes       & Yes       & Yes       & Yes       & Yes                  & Yes                  & Yes                  & Yes               & Yes               & Yes               \\
            year FE                             & Yes       & Yes       & Yes       & Yes       & Yes       & Yes       & Yes                  & Yes                  & Yes                  & Yes               & Yes               & Yes               \\
            facility FE                         & Yes       & Yes       & Yes       & Yes       & Yes       & Yes       & Yes                  & Yes                  & Yes                  & Yes               & Yes               & Yes               \\
            border-county FE                    & Yes       & Yes       & Yes       & Yes       & Yes       & Yes       & Yes                  & Yes                  & Yes                  & Yes               & Yes               & Yes               \\
            toxic chemical FE                   & Yes       & Yes       & Yes       & Yes       & Yes       & Yes       & Yes                  & Yes                  & Yes                  & Yes               & Yes               & Yes               \\
            toxic chemical LTs                  & Yes       & Yes       & Yes       & Yes       & Yes       & Yes       & Yes                  & Yes                  & Yes                  & Yes               & Yes               & Yes               \\\midrule
            clustered at the:                   & facility  & zipcode   & county    & industry  & chemical  & state     & facility \& chemical & facility \& industry & chemical \& industry & chemical \& state & facility \& state & industry \& state \\
            Observations                        & 1,893,689 & 1,893,689 & 1,893,689 & 1,893,689 & 1,893,689 & 1,893,689 & 1,893,689            & 1,893,689            & 1,893,689            & 1,893,689         & 1,893,689         & 1,893,689         \\
            $R^2$                               & 0.739     & 0.739     & 0.739     & 0.739     & 0.739     & 0.739     & 0.739                & 0.739                & 0.739                & 0.739             & 0.739             & 0.739             \\ \bottomrule\bottomrule
        \end{tabular}%
    }
    \begin{minipage}{\columnwidth}
        \vspace{0.05in}
        \tiny NOTES: These results are obtained from estimating model~\ref{eq:baseline-onsite-air-emission-intensity}. ***, **, and * denote significance levels at the less than $1\%$, $5\%$ and $10\%$, respectively.
    \end{minipage}
\end{table}
    % Please add the following required packages to your document preamble:
% \usepackage{booktabs}
% \usepackage{graphicx}% \usepackage{graphicx}
\begin{table}[H]
    \centering
    \caption{Total Onsite Surface Water Discharge Intensity: Alternative clustering of the SEs}
    \label{tab:robustness-ses-clustering-water-discharge-onsite-intensity}
    \resizebox{\columnwidth}{!}{%
        \begin{tabular}{@{}lllllllllllll@{}}
            \toprule\toprule
            Total surface water discharge intensity (log) & 1         & 2         & 3         & 4         & 5         & 6         & 7                    & 8                    & 9                    & 10                & 11                & 12                \\ \midrule
            $Treated^{e}$                                 & 0.014     & 0.014     & 0.014     & 0.014     & 0.014     & 0.014     & 0.014                & 0.014                & 0.014                & 0.014             & 0.014             & 0.014             \\
            & (0.020)   & (0.020)   & (0.020)   & (0.022)   & (0.030)   & (0.018)   & (0.020)              & (0.020)              & (0.030)              & (0.030)          & (0.020)          & (0.022)          \\
            cohort 2014                                   & -0.047**  & -0.047**  & -0.047**  & -0.047*   & -0.047    & -0.047**  & -0.047**             & -0.047**               & -0.047*               & -0.047           & -0.047**         & -0.047*      \\
            & (0.024)   & (0.024)   & (0.023)   & (0.028)   & (0.028)   & (0.021)   & (0.024)              & (0.024)              & (0.028)              & (0.028)          & (0.024)          & (0.028)          \\
            cohort 2015                                   & 0.116***  & 0.116***  & 0.116***  & 0.116***  & 0.116***  & 0.116***  & 0.116***             & 0.116***               & 0.116***              & 0.116***             & 0.116***         & 0.116***      \\
            & (0.030)   & (0.030)   & (0.032)   & (0.028)   & (0.041)   & (0.032)   & (0.030)              & (0.030)              & (0.041)              & (0.041)          & (0.030)          & (0.028)          \\
            cohort 2017                                   & 0.060     & 0.060     & 0.060     & 0.060     & 0.060     & 0.060     & 0.060                & 0.060                & 0.060                & 0.060             & 0.060             & 0.060             \\
            & (0.055)   & (0.055)   & (0.046)   & (0.044)   & (0.043)   & (0.036)   & (0.055)              & (0.055)              & (0.043)              & (0.043)          & (0.055)          & (0.044)          \\
            controls                                      & Yes       & Yes       & Yes       & Yes       & Yes       & Yes       & Yes                  & Yes                  & Yes                  & Yes               & Yes               & Yes               \\
            year FE                                       & Yes       & Yes       & Yes       & Yes       & Yes       & Yes       & Yes                  & Yes                  & Yes                  & Yes               & Yes               & Yes               \\
            facility FE                                   & Yes       & Yes       & Yes       & Yes       & Yes       & Yes       & Yes                  & Yes                  & Yes                  & Yes               & Yes               & Yes               \\
            border-county FE                              & Yes       & Yes       & Yes       & Yes       & Yes       & Yes       & Yes                  & Yes                  & Yes                  & Yes               & Yes               & Yes               \\
            toxic chemical FE                             & Yes       & Yes       & Yes       & Yes       & Yes       & Yes       & Yes                  & Yes                  & Yes                  & Yes               & Yes               & Yes               \\
            toxic chemical LTs                            & Yes       & Yes       & Yes       & Yes       & Yes       & Yes       & Yes                  & Yes                  & Yes                  & Yes               & Yes               & Yes               \\
            border-county LTs                             & Yes       & Yes       & Yes       & Yes       & Yes       & Yes       & Yes                  & Yes                  & Yes                  & Yes               & Yes               & Yes               \\ \midrule
            clustered at the:                             & facility  & zipcode   & county    & industry  & chemical  & state     & facility \& chemical & facility \& industry & chemical \& industry & chemical \& state & facility \& state & industry \& state \\
            Observations                                  & 1,893,689 & 1,893,689 & 1,893,689 & 1,893,689 & 1,893,689 & 1,893,689 & 1,893,689            & 1,893,689            & 1,893,689            & 1,893,689         & 1,893,689         & 1,893,689         \\
            $R^2$                                         & 0.592     & 0.592     & 0.592     & 0.592     & 0.592     & 0.592     & 0.592                & 0.592                & 0.592                & 0.592             & 0.592             & 0.592             \\ \bottomrule \bottomrule
        \end{tabular}%
    }
    \begin{minipage}{\columnwidth}
        \vspace{0.05in}
        \tiny NOTES: These results are obtained from estimating model~\ref{eq:baseline-onsite-water-discharge-intensity}. ***, **, and * denote significance levels at the less than $1\%$, $5\%$ and $10\%$, respectively.
    \end{minipage}
\end{table}
    % Please add the following required packages to your document preamble:
% \usepackage{booktabs}
% \usepackage{graphicx}
\begin{table}[H]
    \centering
    \caption{Total Onsite Land Releases Intensity: Alternative Clustering of SEs}
    \label{tab:robustness-ses-clustering-total-land-releases-intensity}
    \resizebox{\columnwidth}{!}{%
        \begin{tabular}{@{}lllllllllllll@{}}
            \toprule\toprule
            Total onsite land releases intensity (log) & 1         & 2         & 3         & 4         & 5         & 6         & 7                    & 8                    & 9                    & 10                & 11                & 12                \\ \midrule
            treated                                    & 0.0095    & 0.0095    & 0.0095    & 0.0095    & 0.0095    & 0.0095    & 0.0095               & 0.0095               & 0.0095               & 0.0095            & 0.0095            & 0.0095            \\
            & (0.0179)  & (0.0179)  & (0.0174)  & (0.0187)  & (0.0112)  & (0.0119)  & (0.0179)             & (0.0179)             & (0.0111)             & (0.0111)          & (0.0179)          & (0.0187)          \\
            controls                                   & Yes       & Yes       & Yes       & Yes       & Yes       & Yes       & Yes                  & Yes                  & Yes                  & Yes               & Yes               & Yes               \\
            year FE                                    & Yes       & Yes       & Yes       & Yes       & Yes       & Yes       & Yes                  & Yes                  & Yes                  & Yes               & Yes               & Yes               \\
            facility FE                                & Yes       & Yes       & Yes       & Yes       & Yes       & Yes       & Yes                  & Yes                  & Yes                  & Yes               & Yes               & Yes               \\
            border-county FE                             & Yes       & Yes       & Yes       & Yes       & Yes       & Yes       & Yes                  & Yes                  & Yes                  & Yes               & Yes               & Yes               \\
            toxic chemical FE                          & Yes       & Yes       & Yes       & Yes       & Yes       & Yes       & Yes                  & Yes                  & Yes                  & Yes               & Yes               & Yes               \\
            toxic chemical LTs                         & Yes       & Yes       & Yes       & Yes       & Yes       & Yes       & Yes                  & Yes                  & Yes                  & Yes               & Yes               & Yes               \\\midrule
            clustered at the:                          & facility  & zipcode   & county    & industry  & chemical  & state     & facility \& chemical & facility \& industry & chemical \& industry & chemical \& state & facility \& state & industry \& state \\
            Observations                               & 1,893,689 & 1,893,689 & 1,893,689 & 1,893,689 & 1,893,689 & 1,893,689 & 1,893,689            & 1,893,689            & 1,893,689            & 1,893,689         & 1,893,689         & 1,893,689         \\
            $R^2$                                      & 0.4997    & 0.4997    & 0.4997    & 0.4997    & 0.4997    & 0.4997    & 0.4997               & 0.4997               & 0.4997               & 0.4997            & 0.4997            & 0.4997            \\ \bottomrule\bottomrule
        \end{tabular}%
    }
    \begin{minipage}{18cm}
        \vspace{0.05in}
        These results are obtained from estimating model~\ref{eq:baseline-onsite-land-releases-intensity}. ***, **, and * denote significance levels at the less than $1\%$, $5\%$ and $10\%$, respectively.
    \end{minipage}
\end{table}

    I have shown that a higher MW policy led to an increased total releases intensity, especially in the $2014$ and $2015$ cohorts. This effect is driven by corresponding increases in the intensities of total air emissions (both from point and fugitive sources), surface water discharge, and land releases (particularly surface impoundment). Additionally, this higher emission intensities is because of higher outputs and labour productivity induced by the MW policy, especially in capital intensive manufacturing industries.


    \section{Heterogeneous Effects}\label{sec:heterogeneous-effects}

    \subsection{Profitability and Production Technology}\label{subsec:profitability-and-production-technology}
    According to theory, financial standings and production technology are crucial in modeling industrial pollution responses. The dummy of labour-intensive manufacturing industries is proxied by the first-year pre-treatment ratios of high-payroll- and high-production-workers-wages-to-revenue. Capital-intensive manufacturing industries is proxied by the low ratios.~\footnote{High denotes above-the-median quantile, and low denotes below-the-median quantile.} I construct industry profits based on the difference between industry revenue and total cost (the sum of material cost, production workers wages, and energy cost). Thus, the dummy of high-profit industries is constructed based on the first-year pre-treatment industry profits above-the-median quantile; and below-the-median quantile denotes low-profits. Therefore, I classify manufacturing industries based on their profit levels and type of production technology. That is, I consider high-profit-labour-intensive v. high-profit-capital-intensive manufacturing industries; and low-profit-labour-intensive v. low-profit-capital-intensive manufacturing industries, to investigate the differential impacts of the above effects. Thus, I estimate the following model:
    \begin{align}
        P_{f,cp,c,t}^{protech} &= \beta (Treated^{e} \cdot D)_{f,s,t} + \psi (Treated^{e})_{s,t} + \vartheta (Treated \cdot D)_{f,s,t} + \mu (Post \cdot D)_{f,s,t} \nonumber \\
        &\quad + \tau Treated_{s,t} + \rho D_{f,s,t} + \alpha Post_{t} + \delta X_{v,c,t-1} + \omega F_{f,t} + \lambda_{t} + \gamma_{f} + \phi_{cp} \nonumber \\
        &\quad + \zeta_{c} + \eta_{c,t} + \theta_{cp,t} + \varepsilon_{f,cp,c,t},\label{eq:heterogeneous-onsite-releases-intensity-protech}
    \end{align}
    where $P_{f,cp,c,t}^{protech}$ is the vector of total onsite releases intensities (including air, water and land) of toxic chemicals at a manufacturing industry facility, $f$ in a cross-border county pair, $cp$ through a toxic chemical, $c$ in the year, $t$. $Treated_{s,t}$ is a dummy that is equal to $1$ for the treated states, and $0$ the control states. $Post_{t}$ is a dummy that is equal to $1$ if the year $t$ is a post-treatment year, and $0$ otherwise. And $D_{f,s,t}$ is a vector of each dummy of high-profit-labour-intensive v. high-profit-capital-intensive, and then low-profit-labour-intensive v. low-profit-capital-intensive manufacturing facilities $f$ in state, $s$ in the year, $t$. It is unity for high-profit-labour-intensive, and $0$ for high-profit-capital-intensive manufacturing industries. Also, it is unity for low-profit-labour-intensive and $0$ for low-profit-capital-intensive manufacturing industries. Other variables are as defined in Section~\ref{sec:onsite-toxic-releases}.

    The parameters of interest here is the triple-differences parameter $\beta$ which measures the differential average effects of higher MW on onsite total releases intensities at manufacturing facilities in treated counties relative to adjacent control counties given either their profit levels or production technology classification. That is, the separate differential impacts on either high- or low-profit-labour-intensive industries. $\psi$ captures the relative differential impact on high- or low-profit-capital-intensive industries. $\beta + \psi$ captures the total relative change on either the high- or low-profit-labour-intensive industries. To have a causal interpretation, the weaker parallel trends assumption is only required to hold for one of the groups in triple differences~\citep{olden2022triple}.~\citet{zhang2023unintended} hypothesised that more financially constrained firms and labour-intensive industries are more responsive to higher MW relative capital-intensive industries. Hence, I test whether the effects of raising MW on total onsite releases intensities are dominated in either the high- or low-profit-labour-intensive industries. The results are presented in Tables~\ref{tab:heterogeneous-onsite-releases-int-hpli} and~\ref{tab:heterogeneous-onsite-releases-int-lpli}.

    \paragraph{High-Profit-Labour-Intensive Manufacturing Industries:}
    For the high-profit-labour-intensive manufacturing industries, the triple-differences parameter $\beta$, as shown in Table~\ref{tab:heterogeneous-onsite-releases-int-hpli}, is negative and insignificant. The observed decline in toxic release intensity is largely attributable to the significant reduction in the $2015$ cohort. The overall relative change, $\beta_{2015} + \psi$, for the $2015$ cohort indicates a substantial decrease of $13.9ppts$ per $\$100$ million units of output. This suggests increased efficiency due to higher MW in high-profit-labour-intensive manufacturing industries. This reduction is primarily driven by the declining total air emission intensities (both from point and fugitive sources) in the $2015$ cohort, and the decline in land releases intensity, especially in the $2014$ and $2017$ cohorts. Additionally, the increase in surface impoundment intensity, especially in the $2015$ and $2017$ cohorts all drive this differential decline. Conversely, the increase in fugitive air emission intensity for the $2014$ is linked to the corresponding decline in surface impoundment intensity. The effect on surface water discharge intensity remains generally muted.
    % Please add the following required packages to your document preamble:
% \usepackage{booktabs}
% \usepackage{graphicx}

\begin{table}[H]
    \centering
    \caption{Onsite Releases Intensity for High-Profit-Labour- v. High-Profit-Capital-Intensive Manufacturing Industries}
    \label{tab:heterogeneous-onsite-releases-int-hpli}
    \resizebox{\columnwidth}{!}{%
        \begin{tabular}{@{}llllllll@{}}
            \toprule\toprule
            Onsite releases intensity (log) & total     & air emissions & point air & fugitive air & water discharge & land releases & surface impoundment \\ \midrule
            $Treated^{e}$                   & -0.046    & 0.009         & 0.009     & 0.038        & -0.042          & -0.062***     & 0.033**             \\
            & (0.088)   & (0.085)       & (0.084)   & (0.054)      & (0.034)         & (0.021)       & (0.013)             \\
            $Treated$                       & 0.199***  & 0.118**       & 0.017     & 0.114**      & 0.083**         & 0.046**       & 0.027**             \\
            & (0.061)   & (0.048)       & (0.051)   & (0.054)      & (0.040)         & (0.018)       & (0.011)             \\
            cohort 2014                     & 0.068     & 0.140         & 0.070     & 0.156**      & -0.058          & -0.080***     & -0.038**            \\
            & (0.109)   & (0.106)       & (0.103)   & (0.073)      & (0.041)         & (0.023)       & (0.015)             \\
            cohort 2015                     & -0.338*** & -0.328***     & -0.148*** & -0.264***    & -0.002          & -0.017        & 0.019**             \\
            & (0.108)   & (0.089)       & (0.072)   & (0.074)      & (0.062)         & (0.032)       & (0.008)             \\
            cohort 2017                     & 0.226     & -0.220        & -0.093    & -0.183       & 0.124           & -0.091**      & 0.071**             \\
            & (0.150)   & (0.150)       & (0.135)   & (0.119)      & (0.084)         & (0.044)       & (0.031)             \\
            controls                        & Yes       & Yes           & Yes       & Yes          & Yes             & Yes           & Yes                 \\
            year FE                         & Yes       & Yes           & Yes       & Yes          & Yes             & Yes           & Yes                 \\
            facility FE                     & Yes       & Yes           & Yes       & Yes          & Yes             & Yes           & Yes                 \\
            border-county FE                & Yes       & Yes           & Yes       & Yes          & Yes             & Yes           & Yes                 \\
            toxic chemical FE               & Yes       & Yes           & Yes       & Yes          & Yes             & Yes           & Yes                 \\
            toxic chemical LTs              & Yes       & Yes           & Yes       & Yes          & Yes             & Yes           & Yes                 \\
            border-county LTs               & Yes       & Yes           & Yes       & Yes          & Yes             & Yes           & Yes                 \\\midrule
            Observations                    & 785,762   & 785,762       & 785,762   & 785,762      & 785,762         & 785,762       & 785,762             \\
            $R^2$                           & 0.757     & 0.780         & 0.743     & 0.739        & 0.689           & 0.612         & 0.376               \\ \bottomrule \bottomrule
        \end{tabular}%
    }
    \begin{minipage}{\columnwidth}
        \vspace{0.05in}
        \tiny NOTES: These results are obtained from estimating equation~\ref{eq:heterogeneous-onsite-releases-intensity-fintech}. Three-way clustered robust standard errors are reported in parentheses, and clustered at the toxic chemical, industry and state levels. ***, **, and * denote significance levels at the less than $1\%$, $5\%$ and $10\%$, respectively.
    \end{minipage}
\end{table}

    In contrast, for the high-profit-capital-intensive manufacturing industries, the results show a significant positive effect of $19.9ppts$. This positive effect surpasses the baseline estimate of $11.9ppts$, indicating that toxic release intensity is highest among high-profitable-capital-intensive manufacturing industries due to a higher MW floor. This increase is driven by rises in total air emissions intensities, from fugitive sources, surface water discharge, and land releases intensities including surface impoundment. There is limited evidence of increases in point air emission intensity.

    Figure~\ref{fig:heterogeneous-onsite-releases-intensities-hpli} illustrates the dynamic effects of raising the MW on various pollution intensities. For high-profit-labour-intensive manufacturing industries, I find an immediate decline in total toxic releases intensity, primarily due to the decline in total air emission, surface water discharge, land releases intensities including surface impoundment. These negative effects persist up to three years after the initial MW increase. However, the instantaneous increase in total air emissions intensity receded, followed by another increase three years later. This is driven by the corresponding increase in fugitive air emission intensity.
    \begin{figure}[H]
    \centering
    \includegraphics[width=1\textwidth, height=0.5\textheight,keepaspectratio]{fig_sdid_total_onsite_releases_int_highprofitlab}
    \caption{Triple-Differences: Onsite Total Releases Intensities for High-Profit-Labour- v. High-Profit-Capital-Intensive Manufacturing Industries}
    \label{fig:heterogeneous-onsite-releases-intensities-hpli}
    \begin{minipage}{\columnwidth}
        \vspace{0.05in}
        \tiny NOTES: The event study model of equation~\ref{eq:heterogeneous-onsite-releases-intensity-protech} is $P_{f,cp,c,t}^{protech} = \sum_{{e = -3},{e \neq -1}}^{3} \beta (Treated^{e} \cdot D)_{f,s,t} + \psi (Treated^{e})_{s,t} + \vartheta (Treated \cdot D)_{f,s,t} + \mu (Post \cdot D)_{f,s,t} + \tau Treated_{s,t} + \rho D_{f,s,t} + \alpha Post_{t} + \delta X_{v,c,t-1} + \omega F_{f,t} + \lambda_{t} + \gamma_{f} + \phi_{cp} + \zeta_{c} + \eta_{c,t} + \theta_{cp,t} + \varepsilon_{f,cp,c,t}$. Three-way clustered robust standard errors are reported in parentheses, and clustered at the toxic chemical, industry and state levels. HRPR means high revenue to profit ratio or high profit margin---less financially constrained. LRPR means low revenue to profit ratio or low profit margin---more financially constrained.
    \end{minipage}
\end{figure}

    In contrast, significant increases in onsite toxic releases intensity driven by air emissions from fugitive sources, and surface water discharge, and land releases intensities, especially surface impoundment is prominent in high-profit-capital-intensive manufacturing industries. Whereas, the effects on land releases and surface impoundment intensities are immediate and persist throughout the post-treatment period, the effects on fugitive air emissions and surface water discharge intensities became significant in the second and third year post-treatment. There are no notable pre-trends.

    \paragraph{Low-Profit-Labour-Intensive Manufacturing Industries:}
    Table~\ref{tab:heterogeneous-onsite-releases-int-lpli} reveals a decreasing differential effect of $16.3ppts$ on toxic release intensity due to a higher MW in low-profit-labour-intensive manufacturing industries, especially in $2015$ cohorts. This effect is driven by the corresponding decline in total air emission intensities, particularly from stack/point sources. Conversely, I find a significant differential increases in fugitive air emission intensity in the $2014$ and $2017$ cohorts. The effect on the intensities of surface water discharge, land releases, and surface impoundment remains generally muted. Furthermore, except for significant declining effect on fugitive air emission intensity in low-profit-capital-intensive manufacturing industries, there is limited evidence of increases in the intensities of total toxic releases, point air emissions, land releases, and surface impoundment.
    % Please add the following required packages to your document preamble:
% \usepackage{booktabs}
% \usepackage{graphicx}
\begin{table}[H]
    \centering
    \caption{Onsite Releases Intensity for Low-Profit-Labour- v. Low-Profit-Capital-Intensive Manufacturing Industries}
    \label{tab:heterogeneous-onsite-releases-int-lpli}
    \resizebox{\columnwidth}{!}{%
        \begin{tabular}{@{}llllllll@{}}
            \toprule\toprule
            Onsite releases intensity (log) & total     & air emissions & point air & fugitive air & water discharge & land releases & surface impoundment \\ \midrule
            $Treated^{e}$                   & -0.163*   & -0.091        & -0.124    & 0.109*       & 0.002           & -0.038        & -0.004              \\
            & (0.088)   & (0.076)       & (0.091)   & (0.056)      & (0.062)         & (0.042)       & (0.004)             \\
            $Treated$                       & 0.080     & 0.021         & 0.044     & -0.094*      & -0.019          & 0.028         & 0.004               \\
            & (0.091)   & (0.076)       & (0.094)   & (0.055)      & (0.070)         & (0.039)       & (0.005)             \\
            cohort 2014                     & -0.094    & -0.031        & -0.094    & 0.146**      & -0.009          & -0.013        & -0.006              \\
            & (0.095)   & (0.080)       & (0.096)   & (0.058)      & (0.070)         & (0.040)       & (0.005)             \\
            cohort 2015                     & -0.255**  & -0.173*       & -0.164*   & 0.057        & 0.017           & -0.072        & -0.002              \\
            & (0.101)   & (0.090)       & (0.096)   & (0.075)      & (0.057)         & (0.054)       & (0.005)             \\
            cohort 2017                     & 0.052     & 0.050         & -0.045    & 0.465*       & -0.041          & 0.075         & -0.005              \\
            & (0.272)   & (0.242)       & (0.192)   & (0.282)      & (0.087)         & (0.056)       & (0.007)             \\
            controls                        & Yes       & Yes           & Yes       & Yes          & Yes             & Yes           & Yes                 \\
            year FE                         & Yes       & Yes           & Yes       & Yes          & Yes             & Yes           & Yes                 \\
            facility FE                     & Yes       & Yes           & Yes       & Yes          & Yes             & Yes           & Yes                 \\
            border-county FE                & Yes       & Yes           & Yes       & Yes          & Yes             & Yes           & Yes                 \\
            toxic chemical FE               & Yes       & Yes           & Yes       & Yes          & Yes             & Yes           & Yes                 \\
            toxic chemical LTs              & Yes       & Yes           & Yes       & Yes          & Yes             & Yes           & Yes                 \\
            border-county LTs               & Yes       & Yes           & Yes       & Yes          & Yes             & Yes           & Yes                 \\ \midrule
            Observations                    & 1,107,927 & 1,107,927     & 1,107,927 & 1,107,927    & 1,107,927       & 1,107,927     & 1,107,927           \\
            $R^2$                           & 0.788     & 0.798         & 0.791     & 0.684        & 0.638           & 0.519         & 0.225               \\\bottomrule \bottomrule
        \end{tabular}%
    }
    \begin{minipage}{\columnwidth}
        \vspace{0.05in}
        \tiny NOTES: These results are obtained from estimating equation~\ref{eq:heterogeneous-onsite-releases-intensity-protech}. Three-way clustered robust standard errors are reported in parentheses, and clustered at the toxic chemical, industry and state levels. ***, **, and * denote significance levels at the less than $1\%$, $5\%$ and $10\%$, respectively.
    \end{minipage}
\end{table}

    The dynamic effects depicted in Figure~\ref{fig:heterogeneous-onsite-releases-intensities-lpli} further illustrate that the negative effect on toxic release intensity is most significant in low-profit-labour-intensive manufacturing industries, emerging first and second year later. The increase fugitive emissions intensities is immediate and persist throughout the post-treatment period. For low-profit-capital-intensive manufacturing industries, the decreasing fugitive air emission intensities is immediate and persist throughout the post-treatment period. The dynamic effect on the intensities of total toxic releases, point air emissions, surface water discharge, land releases, and surface impoundment are generally muted. As expected, there are no significant pre-trends.
    \begin{figure}[H]
    \centering
    \includegraphics[width=1\textwidth, height=0.5\textheight,keepaspectratio]{C:/Users/david/OneDrive/Documents/ULMS/PhD/Thesis/chapter3/src/climate_change/latex/fig_sdid_total_onsite_releases_int_lowprofitlab}
    \caption{Triple-Differences: Onsite Total Releases Intensities for Low-Profit-Labour- v. Low-Profit-Capital-Intensive Manufacturing Industries}
    \label{fig:heterogeneous-onsite-releases-intensities-lpli}
    \begin{minipage}{\columnwidth}
        \vspace{0.05in}
        \tiny NOTES: The event study model of equation~\ref{eq:heterogeneous-onsite-releases-intensity-protech} is $P_{f,cp,c,t}^{protech} = \sum_{{e = -3},{e \neq -1}}^{3} \beta (Treated^{e} \cdot D)_{f,s,t} + \psi (Treated^{e})_{s,t} + \vartheta (Treated \cdot D)_{f,s,t} + \mu (Post \cdot D)_{f,s,t} + \tau Treated_{s,t} + \rho D_{f,s,t} + \alpha Post_{t} + \delta X_{v,c,t-1} + \omega F_{f,t} + \lambda_{t} + \gamma_{f} + \phi_{cp} + \zeta_{c} + \eta_{c,t} + \theta_{cp,t} + \varepsilon_{f,cp,c,t}$. Three-way clustered robust standard errors are reported in parentheses, and clustered at the toxic chemical, industry and state levels. HRPR means high revenue to profit ratio or high profit margin---less financially constrained. LRPR means low revenue to profit ratio or low profit margin---more financially constrained.
    \end{minipage}
\end{figure}

    \subsection{Highest Emitting Industries}\label{subsec:highest-emitting-industries}
    There are manufacturing industries that release and emit more toxic chemicals than others. This subsection investigates if the documented effect is heterogeneous and entirely driven by such highest emitting manufacturing industries. They include the chemical, food, leather and allied products and wood manufacturing industries (see their distributions in Figures~\ref{fig:releases-distribution-naics},~\ref{fig:air-emissions-distribution-naics},~\ref{fig:water-discharge-distribution-naics} and~\ref{fig:land-releases-distribution-naics} of Appendix~\ref{sec:appendix-distribution-of-industries-and-pollution-emissions-intensities}). To investigate the differential effect of higher MW on onsite total releases intensity of highest emitting manufacturing industries (HEIs), I estimate the following model:
    \begin{align}
        P_{f,cp,c,t}^{heis} &= \beta (Treated^{e} \cdot D)_{f,s,t} + \psi (Treated^{e})_{s,t} + \vartheta (Treated \cdot D)_{f,s,t} + \mu (Post \cdot D)_{f,s,t} \nonumber \\
        &\quad + \tau Treated_{s,t} + \rho D_{f,s,t} + \alpha Post_{t} + \delta X_{v,c,t-1} + \omega F_{f,t} + \lambda_{t} + \gamma_{f} + \phi_{cp} \nonumber \\
        &\quad + \zeta_{c} + \eta_{c,t} + \theta_{cp,t} + \varepsilon_{f,cp,c,t},\label{eq:heterogeneous-onsite-releases-intensity-heis}
    \end{align}
    where $P_{f,cp,c,t}^{heis}$ is the vector of total onsite releases intensity (air, water and land) of toxic chemicals at a manufacturing industry facility, $f$ in a cross-border county pair, $cp$ through a toxic chemical, $c$ in the year, $t$. $Treated_{s,t}$ is a dummy that is equal to $1$ for the treated states, and $0$ the control states. $Post_{t}$ is a dummy that is equal to $1$ if the year $t$ is a post-treatment year, and $0$ otherwise. And $D_{f,s,t}$ is a dummy that is unity for the set of HEIs, $f$ in state, $s$ in the year, $t$ and $0$ low and lowest emitting manufacturing industries (LEIs).
    % Please add the following required packages to your document preamble:
% \usepackage{booktabs}
% \usepackage{graphicx}
\begin{table}[H]
    \centering
    \caption{Onsite Releases Intensity given Highest Emitting Manufacturing Industries}
    \label{tab:heterogeneous-onsite-releases-int-heis}
    \resizebox{\columnwidth}{!}{%
        \begin{tabular}{@{}llllllll@{}}
            \toprule\toprule
            Onsite releases intensity (log) & total     & air emissions & point air & fugitive air & water discharge & land releases & surface impoundment \\ \midrule
            $Treated^{e} \cdot D$           & 0.360***  & 0.224***      & 0.032     & 0.235***     & 0.094           & 0.002         & -0.011**            \\
            & (0.116)   & (0.077)       & (0.066)   & (0.067)      & (0.087)         & (0.019)       & (0.005)             \\
            $Treated^{e}$                   & 0.020     & 0.023         & 0.048     & -0.011       & 0.007           & 0.007         & 0.021**             \\
            & (0.039)   & (0.039)       & (0.036)   & (0.030)      & (0.011)         & (0.013)       & (0.010)             \\
            cohort 2014 $\cdot D$           & 0.350***  & 0.305***      & 0.259**   & 0.077        & -0.079          & 0.041         & -0.011**            \\
            & (0.113)   & (0.114)       & (0.107)   & (0.077)      & (0.067)         & (0.044)       & (0.005)             \\
            cohort 2015 $\cdot D$           & 0.369***  & 0.149**       & -0.178*   & 0.381***     & 0.254**         & -0.034        & -0.012              \\
            & (0.141)   & (0.069)       & (0.091)   & (0.090)      & (0.122)         & (0.021)       & (0.007)             \\
            cohort 2017 $\cdot D$           & 0.052     & 0.032         & 0.166     & 0.145        & 0.033           & 0.025         & -0.002              \\
            & (0.174)   & (0.179)       & (0.141)   & (0.142)      & (0.091)         & (0.025)       & (0.009)             \\
            controls                        & Yes       & Yes           & Yes       & Yes          & Yes             & Yes           & Yes                 \\
            year FE                         & Yes       & Yes           & Yes       & Yes          & Yes             & Yes           & Yes                 \\
            facility FE                     & Yes       & Yes           & Yes       & Yes          & Yes             & Yes           & Yes                 \\
            border-county FE                & Yes       & Yes           & Yes       & Yes          & Yes             & Yes           & Yes                 \\
            toxic chemical FE               & Yes       & Yes           & Yes       & Yes          & Yes             & Yes           & Yes                 \\
            toxic chemical LTs              & Yes       & Yes           & Yes       & Yes          & Yes             & Yes           & Yes                 \\\midrule
            Observations                    & 1,893,689 & 1,893,689     & 1,893,689 & 1,893,689    & 1,893,689       & 1,893,689     & 1,893,689           \\
            $R^2$                           & 0.720     & 0.739         & 0.712     & 0.661        & 0.585           & 0.501         & 0.127               \\ \bottomrule \bottomrule
        \end{tabular}%
    }
    \begin{minipage}{\columnwidth}
        \vspace{0.05in}
        \tiny NOTES: These results are obtained from estimating this equation: $P_{f,cp,c,t}^{heis} = \beta (Treated^{e} \cdot D)_{f,s,t} + \psi (Treated^{e})_{s,t} + \vartheta (Treated \cdot D)_{f,s,t} + \mu (Post \cdot D)_{f,s,t} + \tau Treated_{s,t} + \rho D_{f,s,t} + \alpha Post_{t} + \delta X_{v,c,t-1} + \omega F_{f,t} + \lambda_{t} + \gamma_{f} + \phi_{cp} + \zeta_{c} + \eta_{c,t} + \varepsilon_{f,cp,c,t}$. Three-way clustered robust standard errors are reported in parentheses, and clustered at the toxic chemical, industry and state levels. ***, **, and * denote significance levels at the less than $1\%$, $5\%$ and $10\%$, respectively.
    \end{minipage}
\end{table}

    The parameter of interest here is the triple-differences parameter $\beta$ which measures the differential impact on onsite total releases intensities due to a higher MW policy for HEIs. $\psi$ measures the relative change in onsite total releases intensity for non-carcinogenic chemicals. And $\beta + \psi$ measures the total relative change in onsite total releases intensities for HEIs. The results are presented in Table~\ref{tab:heterogeneous-onsite-releases-int-heis}. I find that the increasing toxic release intensity is predominantly observed in higher-emitting manufacturing industries, particularly within the $2014$ and $2015$ cohorts. This increase is primarily driven by rises in total air emissions (both point and fugitive) and surface water discharge intensities, especially notable in these cohorts. However, I find a decline effect in point air emissions intensity in $2015$ cohort. No significant effect is recorded on surface impoundment intensity. Conversely, in lower-emitting manufacturing industries, except for a notable positive impact on surface impoundment intensities, the overall impact on total toxic releases, including all air emissions and surface water discharge intensities, remains relatively muted. Additionally, there is limited evidence of significant changes in land release intensity across the industries.
    \begin{figure}[H]
    \centering
    \includegraphics[width=1\textwidth, height=0.5\textheight,keepaspectratio]{fig_sdid_total_onsite_releases_int_EMITT}
    \caption{Triple-Differences: Onsite Total Releases Intensity given Highest Emitting Manufacturing Industries}
    \label{fig:heterogeneous-onsite-releases-intensity-emitt}
    \begin{minipage}{18cm}
        \vspace{0.05in}
        NOTES: The event study model of equation~\ref{eq:heterogeneous-onsite-releases-intensity-heis} is $G_{f,cp,c,t}^{gdp} = \sum_{{e = -3},{e \neq -1}}^{3} \beta (Treated^{e} \cdot D)_{f,s,t} + \psi (Treated^{e})_{s,t} + \vartheta (Treated \cdot D)_{f,s,t} + \mu (Post \cdot D)_{f,s,t} + \tau Treated_{s,t} + \rho D_{f,s,t} + \alpha Post_{t} + \delta X_{v,c,t-1} + \omega F_{f,t} + \lambda_{t} + \gamma_{f} + \phi_{cp} + \zeta_{c} + \eta_{c,t} + \varepsilon_{f,cp,c,t}$. Three-way clustered robust standard errors are reported in parentheses, and clustered at the toxic chemical, industry and state levels. HEIs mean highest emitting industries.
    \end{minipage}
\end{figure}

    Figure~\ref{fig:heterogeneous-onsite-releases-intensity-emitt} illustrates the dynamic effects of higher MW policies. The results indicate that the positive effects on toxic release intensities are present in both the highest and lowest emitting manufacturing industries. These effects are especially evident in air emissions (both point and fugitive), surface water discharge, and land release intensities. The impact emerges immediately and persists for up to three years later. For HEIs, the effect on surface impoundment intensity declines, whereas, for LEIs, both land releases and surface impoundment intensities increase instantaneously and persist for up to three years later.

    \subsection{Economic Growth Patterns}\label{subsec:economic-growth-patterns}
    In this subsection, I check whether the increasing onsite releases intensity due to a higher MW floor is peculiar to treated counties with high economic growth patterns. Theory suggests that economic growth patterns are correlated with pollutant emissions with a turning point at higher economic growth~\citep{grossman1995economic, shapiro2018pollution}. To investigate the differential effect of higher MW on onsite total releases intensity, I estimate the following model:
    \begin{align}
        P_{f,cp,c,t}^{gdp} &= \beta (Treated^{e} \cdot D)_{h,s,t} + \psi (Treated^{e})_{s,t} + \vartheta (Treated \cdot D)_{h,s,t} + \mu (Post \cdot D)_{h,s,t} \nonumber \\
        &\quad + \tau Treated_{s,t} + \rho D_{h,s,t} + \alpha Post_{t} + \delta X_{v,c,t-1} + \omega F_{f,t} + \lambda_{t} + \gamma_{f} + \phi_{cp} \nonumber \\
        &\quad+ \zeta_{c} + \eta_{c,t} + \theta_{cp,t} + \varepsilon_{f,cp,c,t},\label{eq:heterogeneous-onsite-releases-intensity-gdp}
    \end{align}
    where $P_{f,cp,c,t}^{gdp}$ is the vector of total onsite releases intensity (air e.g., point and fugitive, water and land) in a high GDP county at a manufacturing industry facility, $f$  in a cross-border county pair, $cp$ through a toxic chemical, $c$ in the year, $t$. $Treated_{s,t}$ is a dummy that is equal to $1$ for the treated states, and $0$ the control states. $Post_{t}$ is a dummy that is equal to $1$ if the year $t$ is a post-treatment year, and $0$ otherwise. And $D_{h,s,t}$ is a dummy that is unity for a high gross domestic product (GDP) of county, $h$ in state, $s$ in the year, $t$ and $0$ otherwise (i.e., low GDP). High GDP is defined as those counties with GDP above the median quantile of the GDP distribution of all counties in $2013$.

    The parameter of interest here is the triple-differences parameter $\beta$ which measures the differential impact on onsite total releases intensity due to a higher MW policy in high GDP counties. $\psi$ measures the differential change in onsite total releases intensity in low GDP counties. And $\beta + \psi$ measures the overall relative change in onsite total releases intensity in high GDP counties. The results are presented in Table~\ref{tab:heterogeneous-onsite-releases-int-gdp}. Notice that the $2017$ cohort is contained in the $\psi$ parameter, as their $2013$ GDP is less than the median. The triple-differences coefficient shows a negligible and insignificant effect on onsite total toxic releases intensity for high GDP counties. Similarly, I find a differential increase in the intensities of fugitive air emissions and surface water discharge in the $2017$ cohort. The effect on total air emissions intensity from point sources, and land releases intensity including surface impoundment are muted. In contrast, there is significant differential increase in total toxic release and surface impoundment intensities for low GDP counties, and limited evidence on air emissions intensity from point and fugitive sources, surface water discharge, and land releases intensities.
    % Please add the following required packages to your document preamble:
% \usepackage{booktabs}
% \usepackage{graphicx}
\begin{table}[H]
    \centering
    \caption{Onsite Releases Intensity given GDP Patterns}
    \label{tab:heterogeneous-onsite-releases-int-gdp}
    \resizebox{\columnwidth}{!}{%
        \begin{tabular}{@{}llllllll@{}}
            \toprule\toprule
            Onsite releases intensity (log) & total     & air emissions & point air & fugitive air & water discharge & land releases & surface impoundment \\ \midrule
            $Treated^{e} \cdot D$           & 0.037     & 0.094         & 0.087     & 0.036        & -0.036          & -0.009        & 0.005               \\
            & (0.082)   & (0.061)       & (0.059)   & (0.047)      & (0.042)         & (0.030)       & (0.004)             \\
            $Treated^{e}$                   & 0.050     & 0.024         & -0.027    & 0.034        & 0.012           & 0.003         & 0.012*              \\
            & (0.063)   & (0.057)       & (0.042)   & (0.056)      & (0.036)         & (0.010)       & (0.007)             \\
            cohort 2014 $\cdot D$           & -0.056    & 0.094         & 0.089     & 0.011        & -0.152          & 0.016         & 0.012**             \\
            & (0.144)   & (0.078)       & (0.064)   & (0.074)      & (0.099)         & (0.016)       & (0.006)             \\
            cohort 2015 $\cdot D$           & 0.187*    & 0.094         & 0.083     & 0.111*       & 0.148**         & -0.050        & -0.008              \\
            & (0.110)   & (0.067)       & (0.092)   & (0.062)      & (0.072)         & (0.061)       & (0.006)             \\
            controls                        & Yes       & Yes           & Yes       & Yes          & Yes             & Yes           & Yes                 \\
            year FE                         & Yes       & Yes           & Yes       & Yes          & Yes             & Yes           & Yes                 \\
            facility FE                     & Yes       & Yes           & Yes       & Yes          & Yes             & Yes           & Yes                 \\
            border-county FE                & Yes       & Yes           & Yes       & Yes          & Yes             & Yes           & Yes                 \\
            toxic chemical FE               & Yes       & Yes           & Yes       & Yes          & Yes             & Yes           & Yes                 \\
            toxic chemical LTs              & Yes       & Yes           & Yes       & Yes          & Yes             & Yes           & Yes                 \\\midrule
            Observations                    & 1,893,689 & 1,893,689     & 1,893,689 & 1,893,689    & 1,893,689       & 1,893,689     & 1,893,689           \\
            $R^2$                           & 0.720     & 0.739         & 0.712     & 0.660        & 0.586           & 0.500         & 0.127               \\ \bottomrule \bottomrule
        \end{tabular}%
    }
    \begin{minipage}{\columnwidth}
        \vspace{0.05in}
        \tiny NOTES: These results are obtained from estimating model~\ref{eq:heterogeneous-onsite-releases-intensity-gdp}. Three-way clustered robust standard errors are reported in parentheses, and clustered at the toxic chemical, industry and state levels. ***, **, and * denote significance levels at the less than $1\%$, $5\%$ and $10\%$, respectively.
    \end{minipage}
\end{table}
    \begin{figure}[H]
    \centering
    \includegraphics[width=1\textwidth, height=0.5\textheight,keepaspectratio]{fig_sdid_total_onsite_releases_int_GDP}
    \caption{Triple-Differences: Onsite Total Releases Intensity given Growth Patterns}
    \label{fig:heterogeneous-onsite-releases-intensity-gdp}
    \begin{minipage}{\columnwidth}
        \vspace{0.05in}
        \tiny NOTES: The event study model of equation~\ref{eq:heterogeneous-onsite-releases-intensity-gdp} is $G_{f,cp,c,t}^{gdp} = \sum_{{e = -3},{e \neq -1}}^{3} \beta (Treated^{e} \cdot D)_{f,s,t} + \psi (Treated^{e})_{s,t} + \vartheta (Treated \cdot D)_{f,s,t} + \mu (Post \cdot D)_{f,s,t} + \tau Treated_{s,t} + \rho D_{f,s,t} + \alpha Post_{t} + \delta X_{v,c,t-1} + \omega F_{f,t} + \lambda_{t} + \gamma_{f} + \phi_{cp} + \zeta_{c} + \eta_{c,t} + \theta_{cp,t} + \varepsilon_{f,cp,c,t}$. Three-way clustered robust standard errors are reported in parentheses, and clustered at the toxic chemical, industry and state levels.
    \end{minipage}
\end{figure}

    Figure~\ref{fig:heterogeneous-onsite-releases-intensity-gdp} illustrates dynamic trends across counties, showing limited evidence of a differential increase in total onsite releases intensity including both point and fugitive air emissions, surface water discharge and land releases and surface impoundment intensities, for high GDP counties. Conversely, low GDP counties witnessed a significant increase in total toxic releases intensity primarily driven by fugitive air emissions intensity. This effect became significant from the second year post-treatment. However, the effects on point air emissions intensity, surface water discharge, land releases and surface impoundment intensities are muted for low GDP counties. There is no evidence of significant pre-trends.

    \subsection{Industry Concentration}\label{subsec:industry-concentration}
    In theory, companies operating in highly concentrated industries (HCIs), which face less competition, benefit from economies of scale and maintain dominant market positions~\citep{baumol1982contestable}. These firms can absorb the costs associated with minimum wage increases due to their large production scales. Moreover, they can transfer the increased labour costs to consumers through higher sales prices or to suppliers by negotiating lower purchase prices. Consequently,~\citet{zhang2023unintended} argued that HCI industries will exhibit a smaller increase in total releases intensities in response to MW hikes. I test whether high concentration (less competitiveness) implies smaller increases in total releases intensities in response to MW increases. To gauge market concentration, I employ the Herfindahl-Hirschman Index (HHI)~\citep{zhang2023unintended, weinstock1982using}. This index is derived each year by summing the squared market shares of all firms within a six-digit NAICS industry codes. Utilizing revenue data to compute the HHI, I then average these values over the entire sample period and classify industries based on the median HHI value. A lower Herfindahl-Hirschman Index (HHI) indicates that the industry is less concentrated and exhibits greater competitiveness. These are industries below the median HHI value. Those above the median HHI value are classified as high-concentrated industries and exhibit less competition. To investigate the differential effect of higher MW on onsite total releases intensity given industry concentration, I estimate the following model:
    \begin{align}
        P_{f,cp,c,t}^{ind-conc} &= \beta (Treated^{e} \cdot D)_{f,s,t} + \psi (Treated^{e})_{s,t} + \vartheta (Treated \cdot D)_{f,s,t} + \mu (Post \cdot D)_{f,s,t} \nonumber \\
        &\quad + \tau Treated_{s,t} + \rho D_{f,s,t} + \alpha Post_{t} + \delta X_{v,c,t-1} + \omega F_{f,t} + \lambda_{t} + \gamma_{f} + \phi_{cp} \nonumber \\
        &\quad + \zeta_{c} + \eta_{c,t} + \theta_{cp,t} + \varepsilon_{f,cp,c,t},\label{eq:heterogeneous-onsite-releases-intensity-lcis}
    \end{align}
    where $P_{f,cp,c,t}^{ind-conc}$ is the vector of total onsite releases intensity (air, water and land) of toxic chemicals at a low-concentrated manufacturing industry facility, $f$ in a cross-border county pair, $cp$ through a toxic chemical, $c$ in the year, $t$. $Treated_{s,t}$ is a dummy that is equal to $1$ for the treated states, and $0$ the control states. $Post_{t}$ is a dummy that is equal to $1$ if the year $t$ is a post-treatment year, and $0$ otherwise. And $D_{f,s,t}$ is a dummy that is unity for a low-concentrated manufacturing industry facility, $f$ in state, $s$ in the year, $t$ and $0$ for facilities in high-concentrated industries.

    The parameter of interest here is the triple-differences parameter $\beta$ which measures the differential impact on onsite total releases intensity due to a higher MW policy for manufacturing facilities in low-concentrated industries. $\psi$ measures the relative change in onsite total releases intensity for manufacturing facilities in high-concentrated industries. And $\beta + \psi$ measures the total relative change in onsite total releases intensity for manufacturing facilities in low-concentrated industries. The results are presented in Table~\ref{tab:heterogeneous-onsite-releases-int-lcis}. Except for the significant differential increases in fugitive air emissions and surface water discharge intensities in the $2015$ cohort for low-concentrated or high competitive manufacturing facilities, the effect on the intensities of total toxic releases, air emissions from point sources, land releases and surface impoundment are muted. I also find evidence of a declining surface water discharge intensity in the $2014$ cohort. Conversely, except for the significant increases in surface impoundment intensity for the high-concentrated or less-competitive manufacturing industries, there is limited evidence of a significant effect on the intensities of total toxic releases, air emissions from point and fugitive sources, surface water discharge and land releases.
    % Please add the following required packages to your document preamble:
% \usepackage{booktabs}
% \usepackage{graphicx}
\begin{table}[H]
    \centering
    \caption{Onsite Releases Intensity given Industry Concentration}
    \label{tab:heterogeneous-onsite-releases-int-lcis}
    \resizebox{\columnwidth}{!}{%
        \begin{tabular}{@{}llllllll@{}}
            \toprule \toprule
            Onsite total releases intensity (log) & total     & air emissions & point air & fugitive air & water discharge & land releases & surface impoundment \\ \midrule
            $Treated^{e} \cdot D$                 & -0.040    & -0.025        & -0.082    & 0.054        & 0.017           & 0.028         & 0.000               \\
            & (0.068)   & (0.061)       & (0.062)   & (0.054)      & (0.027)         & (0.028)       & (0.004)             \\
            $Treated^{e}$                         & 0.085     & 0.054         & 0.045     & 0.031        & -0.001          & 0.016         & 0.009***            \\
            & (0.064)   & (0.060)       & (0.057)   & (0.050)      & (0.034)         & (0.011)       & (0.004)             \\
            cohort 2014 $\cdot D$                 & -0.103    & -0.044        & -0.092    & 0.015        & -0.073**        & -0.020        & 0.003               \\
            & (0.083)   & (0.082)       & (0.082)   & (0.067)      & (0.034)         & (0.037)       & (0.007)             \\
            cohort 2015 $\cdot D$                 & 0.030     & -0.003        & -0.072    & 0.097*       & 0.120**         & -0.037        & -0.003              \\
            & (0.088)   & (0.076)       & (0.073)   & (0.056)      & (0.050)         & (0.035)       & (0.003)             \\
            cohort 2017 $\cdot D$                 & 0.062     & 0.018         & -0.042    & 0.098        & 0.050           & 0.036         & -0.004              \\
            & (0.137)   & (0.108)       & (0.099)   & (0.104)      & (0.065)         & (0.034)       & (0.005)             \\
            controls                              & Yes       & Yes           & Yes       & Yes          & Yes             & Yes           & Yes                 \\
            year FE                               & Yes       & Yes           & Yes       & Yes          & Yes             & Yes           & Yes                 \\
            facility FE                           & Yes       & Yes           & Yes       & Yes          & Yes             & Yes           & Yes                 \\
            border-county FE                      & Yes       & Yes           & Yes       & Yes          & Yes             & Yes           & Yes                 \\
            toxic chemical FE                     & Yes       & Yes           & Yes       & Yes          & Yes             & Yes           & Yes                 \\
            toxic chemical LTs                    & Yes       & Yes           & Yes       & Yes          & Yes             & Yes           & Yes                 \\
            border-county LTs                     & Yes       & Yes           & Yes       & Yes          & Yes             & Yes           & Yes                 \\\midrule
            Observations                          & 1,893,689 & 1,893,689     & 1,893,689 & 1,893,689    & 1,893,689       & 1,893,689     & 1,893,689           \\
            $R^2$                                 & 0.728     & 0.746         & 0.719     & 0.670        & 0.595           & 0.507         & 0.159               \\ \bottomrule\bottomrule
        \end{tabular}%
    }
    \begin{minipage}{\columnwidth}
        \vspace{0.05in}
        \tiny NOTES: These results are obtained from estimating model~\ref{eq:heterogeneous-onsite-releases-intensity-lcis}. Three-way clustered robust standard errors are reported in parentheses, and clustered at the toxic chemical, industry and state levels. ***, **, and * denote significance levels at the less than $1\%$, $5\%$ and $10\%$, respectively.
    \end{minipage}
\end{table}
    \begin{figure}[H]
    \centering
    \includegraphics[width=1\textwidth, height=0.5\textheight,keepaspectratio]{fig_sdid_total_onsite_releases_int_lowindconc}
    \caption{Triple-Differences: Onsite Total Releases Intensity Industry Concentration}
    \label{fig:heterogeneous-onsite-releases-intensity-lcis}
    \begin{minipage}{18cm}
        \vspace{0.05in}
        \tiny NOTES: The event study model of equation~\ref{eq:heterogeneous-onsite-releases-intensity-lcis} is $G_{f,c,i,cp,s,t}^{ind-conc} = \sum_{{e = -3},{e \neq -1}}^{3} \beta (Treated^{e} \cdot D)_{f,s,t} + \psi (Treated^{e})_{s,t} + \vartheta (Treated \cdot D)_{f,s,t} + \mu (Post \cdot D)_{f,s,t} + \tau Treated_{s,t} + \rho D_{f,s,t} + \alpha Post_{t} + \delta X_{v,c,t-1} + \omega F_{f,t} + \lambda_{t} + \gamma_{f} + \phi_{cp} + \zeta_{c} + \eta_{c,t} + \theta_{cp,t} + \varepsilon_{f,cp,c,t}$. Three-way clustered robust standard errors are reported in parentheses, and clustered at the toxic chemical, industry and state levels.
    \end{minipage}
\end{figure}

    Similar patterns are observed in the dynamic effects presented in Figure~\ref{fig:heterogeneous-onsite-releases-intensity-lcis}. In high competitive industries, while point air emissions intensity is declining, fugitive air emission intensity and surface water discharge intensity is rising, usually between the first and second year of post-treatment. However, the effects on the intensities of total toxic releases, total air emissions, land releases and surface impoundment are muted. In contrast, in less-competitive industries, the effects on the intensities of total toxic releases, air emissions from point sources, surface water discharge and land releases are also muted, whereas, I document increases in fugitive air emission intensity and surface impoundment intensity. This effect is observed immediately and persists throughout the post-treatment years. There is no evidence of significant pre-trends.

    \subsection{Carcinogenic Chemicals}\label{subsec:carcinogenic-chemicals}
    In this subsection, I check whether the MW policy is potentially carcinogenic. To investigate the differential effect of higher MW on onsite total releases intensity of carcinogenic chemicals, I estimate the following model:
    \begin{align}
        P_{f,cp,c,t}^{carcinogen} &= \beta (Treated^{e} \cdot D)_{f,s,t} + \psi (Treated^{e})_{s,t} + \vartheta (Treated \cdot D)_{f,s,t} + \mu (Post \cdot D)_{f,s,t} \nonumber \\
        &\quad + \tau Treated_{s,t} + \rho D_{f,s,t} + \alpha Post_{t} + \delta X_{v,c,t-1} + \omega F_{f,t} + \lambda_{t} + \gamma_{f} + \phi_{cp} \nonumber \\
        &\quad + \zeta_{c} + \eta_{c,t} + \theta_{cp,t} + \varepsilon_{f,cp,c,t},\label{eq:heterogeneous-onsite-releases-intensity-carcinogens}
    \end{align}
    where $P_{f,cp,c,t}^{carcinogen}$ is the vector of total onsite releases intensity (air, water and land) of toxic chemicals at a manufacturing industry facility, $f$ in a cross-border county pair, $cp$ through a toxic chemical, $c$ in the year, $t$. $Treated_{s,t}$ is a dummy that is equal to $1$ for the treated states, and $0$ the control states. $Post_{t}$ is a dummy that is equal to $1$ if the year $t$ is a post-treatment year, and $0$ otherwise. And $D_{f,s,t}$ is a dummy that is unity for a carcinogenic chemical at manufacturing facility, $f$ in state, $s$ in the year, $t$ and $0$ otherwise. Carcinogenic chemicals are toxic chemicals that can cause cancer in both humans and animals alike. Examples include benzene, formaldehyde, arsenic, and vinyl chloride, etc.
    % Please add the following required packages to your document preamble:
% \usepackage{booktabs}
% \usepackage{graphicx}
\begin{table}[H]
    \centering
    \caption{Onsite Releases Intensity for Carcinogenic Chemicals}
    \label{tab:heterogeneous-onsite-releases-int-carcinogens}
    \resizebox{\columnwidth}{!}{%
        \begin{tabular}{@{}llllllll@{}}
            \toprule\toprule
            Onsite releases intensity (log) & total     & air emissions & point air & fugitive air & water discharge & land releases & surface impoundment \\ \midrule
            $Treated^{e} \cdot D$           & 0.012     & 0.063         & 0.106     & -0.006       & -0.121*         & -0.040        & -0.015**            \\
            & (0.095)   & (0.090)       & (0.085)   & (0.070)      & (0.065)         & (0.040)       & (0.007)             \\
            $Treated^{e}$                   & 0.106*    & 0.037         & -0.030    & 0.043        & 0.069*          & 0.022         & 0.014**             \\
            & (0.056)   & (0.045)       & (0.040)   & (0.041)      & (0.038)         & (0.014)       & (0.007)             \\
            cohort 2014 $\cdot D$           & -0.037    & -0.025        & 0.024     & -0.087       & -0.089          & -0.053        & -0.013*             \\
            & (0.108)   & (0.107)       & (0.098)   & (0.072)      & (0.062)         & (0.065)       & (0.007)             \\
            cohort 2015 $\cdot D$           & 0.105     & 0.231*        & 0.264**   & 0.146        & -0.180**        & -0.014        & -0.019**            \\
            & (0.127)   & (0.123)       & (0.113)   & (0.117)      & (0.090)         & (0.030)       & (0.008)             \\
            cohort 2017 $\cdot D$           & 0.132     & 0.267         & 0.174     & 0.345**      & -0.168          & 0.074         & -0.011              \\
            & (0.298)   & (0.259)       & (0.239)   & (0.170)      & (0.151)         & (0.047)       & (0.013)             \\
            controls                        & Yes       & Yes           & Yes       & Yes          & Yes             & Yes           & Yes                 \\
            year FE                         & Yes       & Yes           & Yes       & Yes          & Yes             & Yes           & Yes                 \\
            facility FE                     & Yes       & Yes           & Yes       & Yes          & Yes             & Yes           & Yes                 \\
            border-county FE                & Yes       & Yes           & Yes       & Yes          & Yes             & Yes           & Yes                 \\
            toxic chemical FE               & Yes       & Yes           & Yes       & Yes          & Yes             & Yes           & Yes                 \\
            toxic chemical LTs              & Yes       & Yes           & Yes       & Yes          & Yes             & Yes           & Yes                 \\
            border-county LTs               & Yes       & Yes           & Yes       & Yes          & Yes             & Yes           & Yes                 \\\midrule
            Observations                    & 1,893,689 & 1,893,689     & 1,893,689 & 1,893,689    & 1,893,689       & 1,893,689     & 1,893,689           \\
            $R^2$                           & 0.727     & 0.746         & 0.719     & 0.670        & 0.594           & 0.507         & 0.159               \\ \bottomrule\bottomrule
        \end{tabular}%
    }
    \begin{minipage}{\columnwidth}
        \vspace{0.05in}
        \tiny NOTES: These results are obtained from estimating model~\ref{eq:heterogeneous-onsite-releases-intensity-carcinogens}. Three-way clustered robust standard errors are reported in parentheses, and clustered at the toxic chemical, industry and state levels. ***, **, and * denote significance levels at the less than $1\%$, $5\%$ and $10\%$, respectively.
    \end{minipage}
\end{table}

    The parameter of interest here is the triple-differences parameter $\beta$ which measures the differential impact on onsite total releases intensity due to a higher MW policy for carcinogenic chemicals at manufacturing facilities. $\psi$ measures the relative change in onsite total releases intensity for non-carcinogenic chemicals. And $\beta + \psi$ measures the total relative change in onsite total releases intensity for carcinogenic chemicals. The results are presented in Table~\ref{tab:heterogeneous-onsite-releases-int-carcinogens}. There is limited evidence on the differential impact of higher MW on total carcinogenic release intensity, particularly in air emissions, including point and fugitive emissions. However, cohort-specific effects suggest that high MW may potentially increase carcinogenic releases in the $2015$ and $2017$ cohorts, particularly from both point and fugitive air emission sources. Additionally, there is a significant differential decline in surface water discharge and surface impoundment intensities of carcinogenic chemicals, notably in the $2014$ and $2015$ cohorts. In contrast, the intensity of non-carcinogenic chemical releases increases, particularly for surface water discharge and impoundment intensities. I find limited evidence of an increase in air emissions (both point and fugitive) and land release intensities of non-carcinogenic chemicals.
    \begin{figure}[H]
    \centering
    \includegraphics[width=1\textwidth, height=0.5\textheight,keepaspectratio]{fig_sdid_total_onsite_releases_int_carcinogens}
    \caption{Triple-Differences: Onsite Total Releases Intensity for Carcinogens}
    \label{fig:heterogeneous-onsite-releases-intensity-carcinogens}
    \begin{minipage}{18cm}
        \vspace{0.05in}
        NOTES: The event study model of equation~\ref{eq:heterogeneous-onsite-releases-intensity-carcinogens} is $G_{f,cp,c,t}^{carcinogen} = \sum_{{e = -3},{e \neq -1}}^{3} \beta (Treated^{e} \cdot D)_{h,s,t} + \psi (Treated^{e})_{s,t} + \vartheta (Treated \cdot D)_{h,s,t} + \mu (Post \cdot D)_{h,s,t} + \tau Treated_{s,t} + \rho D_{h,s,t} + \alpha Post_{t} + \delta X_{v,c,t-1} + \omega F_{f,t} + \lambda_{t} + \gamma_{f} + \phi_{cp} + \eta_{c,t} + \zeta_{c} + \varepsilon_{f,cp,c,t}$. Three-way clustered robust standard errors are reported in parentheses, and clustered at the toxic chemical, industry and state levels.
    \end{minipage}
\end{figure}

    The dynamic effects shown in Figure~\ref{fig:heterogeneous-onsite-releases-intensity-carcinogens} indicate a similar pattern. There is limited evidence of differential total toxic release intensity for carcinogenic chemicals, particularly in fugitive air emissions and land releases. Point air emissions intensity increases instantaneously but becomes negligible within one to three years after a statutory MW raise. Additionally, there is an immediate and consistent significant decline in carcinogenic surface water discharge and surface impoundment intensities. Conversely, non-carcinogenic total onsite toxic release intensity significantly increases, especially for fugitive air emissions, surface water discharge, and surface impoundment intensities, predominantly observed two to three years post-treatment. No significant pre-trends are evident.

    \subsection{Clean Air Act Chemicals and Hazardous Air Pollutants}\label{subsec:clean-air-act-chemicals-haps}
    This section investigates the question: does higher MW have any significant differential impact on total onsite releases intensities of clean air act chemicals (CAA) and hazardous air pollutants (HAPs)? CAA chemicals are toxic chemicals heavily regulated under the CAA of $1970$. The CAA is a comprehensive federal law enacted by the United States Congress to address air pollution control and improve air quality standards across the nation. The primary goals of the CAA are to protect public health and the environment by regulating the emission of harmful air pollutants. HAPs, also known as air toxics, are pollutants that are known or suspected to cause serious health and environmental effects. These pollutants are regulated under the CAA Amendments of $1990$.~\footnote{\tiny  Examples include heavy metals (mercury, lead, cadmium, and chromium are examples of heavy metals that can be released into the air from industrial processes, combustion of fossil fuels, and waste incineration), VOCs (e.g., include benzene, toluene, xylene, and formaldehyde, which are easily emitted into the air from industrial sources and consumer products. Exposure to VOCs can cause respiratory problems, neurological effects, and contribute to the formation of ground-level ozone and smog), polycyclic aromatic hydrocarbons (PAHs) from industrial processes (e.g., include organic compounds formed during the incomplete combustion of fossil fuels, wood, and other organic materials. Some PAHs are known carcinogens and can also cause developmental and reproductive effects), persistent organic compounds (organic compounds that resist degradation in the environment and can accumulate in living organisms. Examples include dioxins, and certain pesticides such as dichlorodiphenyltrichloroethane (DDT) and PCBs. These chemicals can travel long distances through air and water, posing risks to ecosystems and human health), and chlorinated compounds (such as chloroform and vinyl chloride, are byproducts of industrial processes, including chemical manufacturing and waste incineration. Exposure to these chemicals can cause liver and kidney damage, as well as neurological effects). Regulation of hazardous air pollutants under the Clean Air Act involves the development of technology-based standards for industrial sources to control emissions of these pollutants. The EPA establishes Maximum Achievable Control Technology (MACT) standards for specific source categories, such as chemical plants, petroleum refineries, and pulp and paper mills, to reduce emissions of hazardous air pollutants to the maximum extent feasible. Facilities subject to MACT standards are required to install pollution control equipment and implement management practices to minimize emissions of hazardous air pollutants. Compliance with MACT standards helps protect public health and the environment by reducing exposure to toxic air pollutants and preventing adverse health effects.} To investigate the above question, I estimate the following model:
    \begin{align}
        P_{f,cp,c,t}^{caa-haps} &= \beta (Treated^{e} \cdot D)_{f,s,t} + \psi (Treated^{e})_{s,t} + \vartheta (Treated \cdot D)_{f,s,t} + \mu (Post \cdot D)_{f,s,t} \nonumber \\
        &\quad + \tau Treated_{s,t} + \rho D_{f,s,t} + \alpha Post_{t} + \delta X_{v,c,t-1} + \omega F_{f,t} + \lambda_{t} + \gamma_{f} + \phi_{cp} \nonumber \\
        &\quad + \zeta_{c} + \eta_{c,t} + \theta_{cp,t} + \varepsilon_{f,cp,c,t},\label{eq:heterogeneous-onsite-releases-intensity-caahaps}
    \end{align}
    where $P_{f,cp,c,t}^{caa-haps}$ is the vector of total onsite releases intensity (air, water and land) of toxic chemicals at a manufacturing industry facility, $f$ in a cross-border county pair, $cp$ through a toxic chemical, $c$ in the year, $t$. $Treated_{s,t}$ is a dummy that is equal to $1$ for the treated states, and $0$ the control states. $Post_{t}$ is a dummy that is equal to $1$ if the year $t$ is a post-treatment year, and $0$ otherwise. And $D_{f,s,t}$ is a dummy that is unity for toxic chemicals at a manufacturing industry facility, $f$ in state, $s$ in the year, $t$ commonly regulated by CAA of $1970$ and $1990$ (common-CAA-HAPs), and $0$ are those toxic chemicals not common in both acts (uncommon-CAA-HAPs).
    % Please add the following required packages to your document preamble:
% \usepackage{booktabs}
% \usepackage{graphicx}
\begin{table}[H]
    \centering
    \caption{Onsite Releases Intensity for CAA and HAPs Chemicals}
    \label{tab:heterogeneous-onsite-releases-int-caa-haps}
    \resizebox{\columnwidth}{!}{%
        \begin{tabular}{@{}llllllll@{}}
            \toprule\toprule
            Onsite releases intensity (log) & total     & air emissions & point air & fugitive air & water discharge & land releases & surface impoundment \\ \midrule
            $Treated^{e} \cdot D$           & 0.076     & 0.212***      & 0.110     & 0.125**      & -0.174***       & -0.064        & -0.002              \\
            & (0.094)   & (0.082)       & (0.074)   & (0.059)      & (0.060)         & (0.049)       & (0.011)             \\
            $Treated^{e}$                   & 0.069     & -0.039        & -0.050    & 0.053        & 0.107**         & 0.035*        & 0.013**             \\
            & (0.067)   & (0.059)       & (0.055)   & (0.041)      & (0.044)         & (0.018)       & (0.005)             \\
            cohort 2014 $\cdot D$           & 0.039     & 0.173         & 0.116     & 0.094        & -0.203***       & -0.098        & -0.002              \\
            & (0.117)   & (0.107)       & (0.097)   & (0.065)      & (0.063)         & (0.073)       & (0.011)             \\
            cohort 2015 $\cdot D$           & 0.134     & 0.275**       & 0.092     & 0.180        & -0.126          & -0.006        & -0.003              \\
            & (0.129)   & (0.130)       & (0.121)   & (0.129)      & (0.087)         & (0.039)       & (0.010)             \\
            cohort 2017 $\cdot D$           & 0.875***  & 0.839***      & 0.922***  & 0.175        & 0.155           & 0.018         & -0.006              \\
            & (0.280)   & (0.267)       & (0.285)   & (0.206)      & (0.281)         & (0.043)       & (0.012)             \\
            controls                        & Yes       & Yes           & Yes       & Yes          & Yes             & Yes           & Yes                 \\
            year FE                         & Yes       & Yes           & Yes       & Yes          & Yes             & Yes           & Yes                 \\
            facility FE                     & Yes       & Yes           & Yes       & Yes          & Yes             & Yes           & Yes                 \\
            border-county FE                & Yes       & Yes           & Yes       & Yes          & Yes             & Yes           & Yes                 \\
            toxic chemical FE               & Yes       & Yes           & Yes       & Yes          & Yes             & Yes           & Yes                 \\
            toxic chemical LTs              & Yes       & Yes           & Yes       & Yes          & Yes             & Yes           & Yes                 \\
            border-county LTs               & Yes       & Yes           & Yes       & Yes          & Yes             & Yes           & Yes                 \\ \midrule
            Observations                    & 1,893,689 & 1,893,689     & 1,893,689 & 1,893,689    & 1,893,689       & 1,893,689     & 1,893,689           \\
            $R^2$                           & 0.727     & 0.746         & 0.718     & 0.671        & 0.595           & 0.507         & 0.159               \\ \bottomrule \bottomrule
        \end{tabular}%
    }
    \begin{minipage}{\columnwidth}
        \vspace{0.05in}
        \tiny NOTES: These results are obtained from estimating model~\ref{eq:heterogeneous-onsite-releases-intensity-caahaps}. Three-way clustered robust standard errors are reported in parentheses, and clustered at the toxic chemical, industry and state levels. ***, **, and * denote significance levels at the less than $1\%$, $5\%$ and $10\%$, respectively.
    \end{minipage}
\end{table}

    The parameter of interest here is the triple-differences parameter $\beta$ which measures the differential impact on onsite total releases intensity due to a higher MW policy for common-CAA-HAPs chemicals at manufacturing facilities. $\psi$ measures the relative change in onsite total releases intensity for uncommon-CAA-HAPs chemicals. And $\beta + \psi$ measures the overall relative change in onsite total releases intensity for common-CAA-HAPs. The results are presented in Table~\ref{tab:heterogeneous-onsite-releases-int-caa-haps}. The overall effect on total toxic release intensities is negligible for common-CAA-HAPs. However, there is a substantial increase in toxic chemical release intensity in the $2017$ cohorts, driven primarily by a rise in total air emission intensities (both point and fugitive sources), especially in the $2015$ and $2017$ cohorts. Conversely, there is limited evidence of a significant effect on uncommon-CAA-HAPs regarding overall releases and air emission intensities. Additionally, surface water discharge intensity declines for common-CAA-HAPs, particularly in the $2014$ cohorts, but increases for uncommon-CAA-HAPs. Similarly, land releases and surface impoundment intensities rise for uncommon-CAA-HAPs but remain generally unchanged for common-CAA-HAPs. These findings indicate that the impact of a higher MW policy is significant, even in highly regulated toxic chemical domains, with polarised implications for air emissions and aquatic habitats.
    \begin{figure}[H]
    \centering
    \includegraphics[width=1\textwidth, height=0.5\textheight,keepaspectratio]{fig_sdid_total_onsite_releases_int_caahaps}
    \caption{Triple-Differences: Onsite Total Releases Intensity for CAA and HAPs Chemicals}
    \label{fig:heterogeneous-onsite-releases-intensity-caa-haps}
    \begin{minipage}{\columnwidth}
        \vspace{0.05in}
        \tiny NOTES: The event study model of equation~\ref{eq:heterogeneous-onsite-releases-intensity-caahaps} is $G_{f,cp,c,t}^{caa} = \sum_{{e = -3},{e \neq -1}}^{3} \beta (Treated^{e} \cdot D)_{f,s,t} + \psi (Treated^{e})_{s,t} + \vartheta (Treated \cdot D)_{f,s,t} + \mu (Post \cdot D)_{f,s,t} + \tau Treated_{s,t} + \rho D_{f,s,t} + \alpha Post_{t} + \delta X_{v,c,t-1} + \omega F_{f,t} + \lambda_{t} + \gamma_{f} + \phi_{cp} + \zeta_{c} + \eta_{c,t} + \theta_{cp,t} + \varepsilon_{f,cp,c,t}$. Three-way clustered robust standard errors are reported in parentheses, and clustered at the toxic chemical, industry and state levels.
    \end{minipage}
\end{figure}

    The dynamic results presented in Figure~\ref{fig:heterogeneous-onsite-releases-intensity-caa-haps} show largely consistent effect patterns. There is an immediate differential increase in total air emission intensity for common-CAA-HAPs, persisting up to two years post-treatment. This increase is driven by the rise in fugitive air emissions for both common- and uncommon-CAA-HAPs chemicals. Similarly, surface impoundment intensity increases for uncommon-CAA-HAPs while remaining unaffected for common-CAA-HAPs. Additionally, surface water discharge intensity decreases for common-CAA-HAPs but increases for uncommon-CAA-HAPs. These changes are typically immediate and persist throughout the post-treatment period. The effects on total release, point air emission, and land release intensities are insignificant for both common- and uncommon-CAA-HAPs chemicals throughout the post-treatment period. No significant evidence of pre-trends.

    \subsection{Persistent Bio-accumulative Toxic Chemicals}\label{subsec:persistent-bioaccumulative-toxic-chemicals}
    This sections asks the question: does a higher MW regime exert significant differential impact on persistent bio-accumulative toxic chemicals (PBTs)?~\footnote{\tiny PBTs are a group of chemicals characterized by their persistence in the environment, ability to accumulate in living organisms, and toxicity. These chemicals pose significant risks to human health and the environment due to their long-term effects and potential for biomagnification in food chains. Examples of PBTs include certain persistent organic pollutants (POPs), such as polychlorinated biphenyls (PCBs), dioxins, and certain pesticides like dichlorodiphenyltrichloroethane (DDT). These chemicals have been widely used in industrial processes, agriculture, and consumer products. Due to their persistence, bioaccumulation potential, and toxicity, PBTs are of particular concern to environmental regulators and policymakers. Efforts to reduce PBT emissions and exposure often involve regulatory measures, such as bans or restrictions on their use, as well as pollution prevention and cleanup programs to mitigate their impact on human health and the environment. PBTs are heavily regulated under the Toxic Substances Control Act (TSCA) of $1976$ by the US EPA. Some of these regulatory actions by the EPA on manufacturing facilities on the use of PBTs include chemical testing, reporting and recording keeping, restriction and bans, risk management assessment, labelling and notification, etc.} To investigate the above question, I estimate the following model:
    \begin{align}
        P_{f,cp,c,t}^{pbts} &= \beta (Treated^{e} \cdot D)_{f,s,t} + \psi (Treated^{e})_{s,t} + \vartheta (Treated \cdot D)_{f,s,t} + \mu (Post \cdot D)_{f,s,t} \nonumber \\
        &\quad + \tau Treated_{s,t} + \rho D_{f,s,t} + \alpha Post_{t} + \delta X_{v,c,t-1} + \omega F_{f,t} + \lambda_{t} + \gamma_{f} + \phi_{cp} \nonumber \\
        &\quad + \zeta_{c} + \eta_{c,t} + \theta_{cp,t} + \varepsilon_{f,cp,c,t},\label{eq:heterogeneous-onsite-releases-intensity-pbts}
    \end{align}
    where $P_{f,cp,c,t}^{pbts}$ is the vector of total onsite releases intensity (air, water and land) of toxic chemicals at a manufacturing facility, $f$ in a cross-border county pair, $cp$ through a toxic chemical, $c$ in the year, $t$. $Treated_{s,t}$ is a dummy that is equal to $1$ for the treated states, and $0$ the control states. $Post_{t}$ is a dummy that is equal to $1$ if the year $t$ is a post-treatment year, and $0$ otherwise. And $D_{f,s,t}$ is a dummy that is unity for a PBT chemical at manufacturing facility, $f$ in state, $s$ in the year, $t$ and $0$ for both common- and uncommon-CAA-HAPs chemicals including dioxin compounds.
    % Please add the following required packages to your document preamble:
% \usepackage{booktabs}
% \usepackage{graphicx}
\begin{table}[H]
    \centering
    \caption{Onsite Releases Intensity for PBTs}
    \label{tab:heterogeneous-onsite-releases-int-pbts}
    \resizebox{\columnwidth}{!}{%
        \begin{tabular}{@{}llllllll@{}}
            \toprule\toprule
            Onsite releases intensity (log) & total     & air emissions & point air & fugitive air & water discharge & land releases & surface impoundment \\ \midrule
            $Treated^{e} \cdot D$           & -0.032    & -0.122        & -0.176*   & 0.037        & 0.138           & -0.076**      & -0.007              \\
            & (0.104)   & (0.105)       & (0.092)   & (0.104)      & (0.092)         & (0.035)       & (0.009)             \\
            $Treated^{e}$                   & 0.147***  & 0.101**       & 0.061*    & 0.042        & 0.032           & 0.019         & 0.012*              \\
            & (0.048)   & (0.040)       & (0.034)   & (0.033)      & (0.032)         & (0.014)       & (0.006)             \\
            cohort 2014 $\cdot D$           & 0.020     & -0.153*       & -0.183*   & 0.074        & 0.254**         & -0.095*       & -0.000              \\
            & (0.139)   & (0.090)       & (0.095)   & (0.100)      & (0.120)         & (0.054)       & (0.009)             \\
            cohort 2015 $\cdot D$           & -0.117    & -0.063        & -0.159    & -0.030       & -0.067          & -0.056        & -0.019              \\
            & (0.142)   & (0.186)       & (0.135)   & (0.189)      & (0.094)         & (0.048)       & (0.012)             \\
            cohort 2017 $\cdot D$           & -1.220*** & -0.839***     & -1.090*** & 0.074        & -0.625**        & -0.070        & -0.011              \\
            & (0.318)   & (0.286)       & (0.324)   & (0.196)      & (0.269)         & (0.053)       & (0.015)             \\
            controls                        & Yes       & Yes           & Yes       & Yes          & Yes             & Yes           & Yes                 \\
            year FE                         & Yes       & Yes           & Yes       & Yes          & Yes             & Yes           & Yes                 \\
            facility FE                     & Yes       & Yes           & Yes       & Yes          & Yes             & Yes           & Yes                 \\
            border-county FE                & Yes       & Yes           & Yes       & Yes          & Yes             & Yes           & Yes                 \\
            toxic chemical FE               & Yes       & Yes           & Yes       & Yes          & Yes             & Yes           & Yes                 \\
            toxic chemical LTs              & Yes       & Yes           & Yes       & Yes          & Yes             & Yes           & Yes                 \\
            border-county LTs               & Yes       & Yes           & Yes       & Yes          & Yes             & Yes           & Yes                 \\\midrule
            Observations                    & 1,893,689 & 1,893,689     & 1,893,689 & 1,893,689    & 1,893,689       & 1,893,689     & 1,893,689           \\
            $R^2$                           & 0.727     & 0.746         & 0.719     & 0.670        & 0.594           & 0.507         & 0.159               \\ \bottomrule\bottomrule
        \end{tabular}%
    }
    \begin{minipage}{\columnwidth}
        \vspace{0.05in}
        \tiny NOTES: These results are obtained from estimating model~\ref{eq:heterogeneous-onsite-releases-intensity-pbts}. Three-way clustered robust standard errors are reported in parentheses, and clustered at the toxic chemical, industry and state levels. ***, **, and * denote significance levels at the less than $1\%$, $5\%$ and $10\%$, respectively.
    \end{minipage}
\end{table}

    The parameter of interest here is the triple-differences parameter $\beta$ which measures the differential impact on onsite total releases intensity due to a higher MW policy for PBTs chemicals at manufacturing facilities. $\psi$ measures the relative change in onsite total releases intensity for non-PBTs chemicals. And $\beta + \psi$ measures the overall relative change in onsite total releases intensity for PBTs chemicals. The results are presented in Table~\ref{tab:heterogeneous-onsite-releases-int-pbts}. For PBT chemicals, there is limited evidence of changes in total releases intensity, including total air (both fugitive and point source) emissions, surface water discharge, and surface impoundment intensities. However, cohort-specific effects for $2014$ and $2017$ indicate significant differential declines in the intensities of total releases and air emissions, including point sources and land releases. Overall, point air emissions and land releases intensities for PBT chemicals show a significant decline. The effects on surface water discharge intensity for PBT chemicals are vary by cohorts. Whereas it declines for the $2014$ cohorts, the opposite is true for the $2015$ cohort. In contrast, for non-PBT chemicals, significant differential increases are observed in total releases intensity, driven by increases in total air emissions (point sources) and surface impoundment intensities, except for fugitive emissions, surface water discharge, and land releases intensities. Consequently, the adverse environmental impacts of higher MW are more pronounced in non-PBT chemicals.
    \begin{figure}[H]
    \centering
    \includegraphics[width=1\textwidth, height=0.5\textheight,keepaspectratio]{fig_sdid_total_onsite_releases_int_pbts}
    \caption{Triple-Differences: Onsite Total Releases Intensity for PBTs}
    \label{fig:heterogeneous-onsite-releases-intensity-pbts}
    \begin{minipage}{\columnwidth}
        \vspace{0.05in}
        \tiny NOTES: The event study model of equation~\ref{eq:heterogeneous-onsite-releases-intensity-pbts} is $G_{f,cp,c,t}^{pbts} = \sum_{{e = -3},{e \neq -1}}^{3} \beta (Treated^{e} \cdot D)_{f,s,t} + \psi (Treated^{e})_{s,t} + \vartheta (Treated \cdot D)_{f,s,t} + \mu (Post \cdot D)_{f,s,t} + \tau Treated_{s,t} + \rho D_{f,s,t} + \alpha Post_{t} + \delta X_{v,c,t-1} + \omega F_{f,t} + \lambda_{t} + \gamma_{f} + \phi_{cp} + \zeta_{c} + \eta_{c,t} + \theta_{cp,t} + \varepsilon_{f,cp,c,t}$. Three-way clustered robust standard errors are reported in parentheses, and clustered at the toxic chemical, industry and state levels.
    \end{minipage}
\end{figure}

    The dynamic effects illustrated in Figure~\ref{fig:heterogeneous-onsite-releases-intensity-pbts} reveal similar patterns. For PBT chemicals, the intensities of total air emissions from point sources and land releases (including surface impoundment) decline immediately and persist for up to three years. Conversely, surface water discharge intensities increase instantly and remain elevated for up to three years, suggesting potential adverse effects on the aquatic ecosystem due to higher MW policy. Limited evidence is found for changes in fugitive air emissions and overall air emissions intensities. In contrast, the dynamic effects indicate that the increasing adverse environmental impact is more pronounced for non-PBT chemicals, as evidenced by rises in overall toxic releases, air emissions (both point source and fugitive), surface water discharge, and surface impoundment intensities. These effects are immediate and consistent throughout the post-treatment periods. Limited evidence is found for total land releases intensity, and no significant pre-trends are observed.


    \section{Mechanism Analyses}\label{sec:mechanism-analyses}
    In this section, I investigate the potential transmission mechanisms of the above results through two broad lenses: onsite source reduction and waste management activities, given the profitability levels and type of production technology, $(D^{protech}_{f,s,t})$ at manufacturing industries, as defined in subsection~\ref{subsec:profitability-and-production-technology}.

    \subsection{Onsite Source Reduction Activities}\label{subsec:onsite-source-reduction-activities}
    Onsite source reduction refers to any practice that minimizes or eliminates the generation of toxic wastes or pollutants in production processes, before they enter any waste stream or get released into the environment. This is also referred to as Pollution Prevention (P2). These activities aim to minimize or eliminate the creation of toxic waste materials or the use of hazardous substances in manufacturing processes. They include: $(i)$ material substitution and modification which involves replacement of raw materials including changing chemical input purity, replacing crude fuel types, and organic solvents with environmentally preferable alternatives; $(ii)$ process and equipment modifications which involve improvements to industrial processes and/or associated equipment including implementation of new processes that produce less waste, direct reuse of chemicals, or technological changes impacting synthesis, formulation, fabrication, and assembly, and surface treatment such as cleaning, degreasing, surface preparation, and finishing. Recycling refers to the process where waste materials are collected, processed, and converted into new products or raw materials at the facility where they were generated. $(iv)$ inventory management includes improvements in procurement in terms of storage container sizes, and handling of chemicals and materials as they move through a facility to optimize their use and prevent spills and leaks during operation; and $(v)$ operating practices in terms of personnel training and product quality analysis to eliminate or minimize waste generation. To investigate onsite source reduction activities, I estimate the following model:
    \begin{align}
        M_{f,cp,c,t}^{sra} &= \beta (Treated^{e} \cdot D^{protech})_{f,s,t} + \psi (Treated^{e})_{s,t} + \vartheta (Treated \cdot D^{protech})_{f,s,t} + \mu (Post \cdot D^{protech})_{f,s,t} \nonumber \\
        &\quad + \tau Treated_{s,t} + \rho D_{f,s,t}^{protech} + \alpha Post_{t} + \delta X_{v,c,t-1} + \omega F_{f,t} + \lambda_{t} + \gamma_{f} + \phi_{cp} \nonumber \\
        &\quad + \zeta_{c} + \eta_{c,t} + \theta_{cp,t} + \varepsilon_{f,cp,c,t},\label{eq:mechanisms-source-reduction}
    \end{align}
    where $M_{f,cp,c,t}^{sra}$ is the vector of onsite source reduction activities to reduce the generation of toxic wastes at manufacturing industry facility, $f$ in cross-border county pairs, $cp$ through toxic chemical use, $c$ in industry, in the year, $t$. The ATT is captured by $\beta$, which is the difference in the average differential effect of higher MW on onsite source reduction activities at manufacturing facilities in treated counties relative to adjacent control counties given either their profitability levels and production technology classification. That is, the separate differential impacts on either high- or low-profit-labour-intensive industries. $\psi$ captures the relative differential impact on either high- or low-profit-capital-intensive industries. $\beta + \psi$ captures the total differential impact on either high- or low-profit-labour-intensive industries. The results on total onsite source reduction activities are reported in Tables~\ref{tab:onsite-source-reduction-activities-hpli} and~\ref{tab:onsite-source-reduction-activities-lpli}.

    The declining effect on onsite toxic release intensities in high-profit-labour-intensive manufacturing industries, due to higher MW, is primarily driven by the increase in green raw material substitutions, process and equipment upgrades and modifications, and operating activities. These industries tend to balance production efficiency and profit maximisation with environmental compliance. To maximise profit, they switch and increase the substitution of green raw materials in their production functions in terms of organic solvents use and cleaner fuels, as reflected in their high energy cost intensity. Additionally, they also improve their core production process by using new green technologies, and spray equipment upgrades in terms of optimizing spray nozzles, adjusting spray pressure, or implementing more efficient application techniques. These activities require specialized skilled workers such as chemists, lab technicians, chemical engineers, and environmental scientists. Consequently, better production scheduling combined with the increase in high-skilled workers to safely execute an efficient production process, while labour productivity and outputs remain fixed as shown in Figures~\ref{fig:baseline-manufacturing-industry-employment-heter}, and~\ref{fig:baseline-manufacturing-industry-output-heter} of Appendix~\ref{sec:appendix-baseline-robustness-tables-and-figures}, lead to a decline in the generation of toxic release intensity across all cohorts. However, a decline in recycling in $2014$ cohorts among low-skilled workers led to increased fugitive air emission intensity, as it potentially increases the risk of spills or leaks during the transportation of generated toxic chemical wastes from production points to waste management sites. Moreover, the rising improvement in inventory control in terms of better packaging and storage, results in the decline of corresponding in the fugitive air emissions intensity in the $2015$ and $2017$ cohorts.
    % Please add the following required packages to your document preamble:
% \usepackage{booktabs}
% \usepackage{graphicx}
\begin{table}[H]
    \centering
    \caption{Onsite SRA: High-Profit-Labour Intensive Manufacturing Industries}
    \label{tab:onsite-source-reduction-activities-hpli}
    \resizebox{\columnwidth}{!}{%
        \begin{tabular}{@{}llllllllllllllll@{}}
            \toprule \toprule
            & \multicolumn{1}{c}{All} & \multicolumn{5}{c}{Material Substitution} & \multicolumn{3}{c}{Process Modification} & \multicolumn{6}{c}{Inventory Management \& Operations Activities} \\
            \cmidrule(lr){2-2} \cmidrule(lr){3-7} \cmidrule(lr){8-10} \cmidrule(lr){11-16}
            Source Reduction Activities & sra      & organic solvent & chemical purity & clean fuel & energy cost intensity & others matsub & new tech & recycling & modified spray equip & other inventory mgt & training & changed prod schedule & changed inventory control & implemented leak inspection & changed operating practices \\ \midrule
            $Treated^{e} \cdot D$       & 0.195*** & 0.002**         & -0.001          & 0.025***   & 0.034***              & 0.005**       & 0.004*   & -0.016   & 0.001 & 0.003                      & 0.004    & 0.020***     & 0.003* & 0.000 & 0.004      \\
            & (0.027)  & (0.001)         & (0.000)         & (0.008)    & (0.013)               & (0.002)       & (0.002)  & (0.014)   & (0.001)              & (0.003)             & (0.004)  & (0.005)        & (0.002) & (0.000) & (0.004)    \\
            $Treated^{e}$               & -0.041** & -0.002***       & -0.001*         & -0.033***  & 0.030**               & -0.000        & -0.002*  & -0.040*** & -0.001* & -0.009***                  & -0.004*  & 0.011***    & -0.002** & -0.001** & -0.004*       \\
            & (0.018)  & (0.001)         & (0.000)         & (0.005)    & (0.014)               & (0.001)       & (0.001)  & (0.008)   & (0.000)              & (0.003)             & (0.002)  & (0.003)           & (0.001) & (0.000) & (0.002) \\
            cohort 2014                 & 0.250*** & 0.004***        & -0.0004*        & 0.020**    & 0.042***              & 0.005**       & -0.001   & -0.020**  & 0.000 & 0.002                      & 0.007    & 0.029***     & 0.002 & 0.000 & 0.007      \\
            & (0.035)  & (0.001)         & (0.000)         & (0.008)    & (0.015)               & (0.003)       & (0.002)  & (0.010)   & (0.000)              & (0.003) & (0.005)  & (0.006)         & (0.002) & (0.001) & (0.005)  \\
            cohort 2015                 & 0.046    & -0.001          & -0.000          & 0.034***   & 0.013                 & 0.003         & 0.016*** & -0.006    & 0.002                & 0.007*              & -0.004   & -0.005     & 0.005* & 0.000 & -0.004       \\
            & (0.032)  & (0.001)         & (0.000)         & (0.009)    & (0.017)               & (0.002)       & (0.006)  & (0.026)   & (0.002)              & (0.004)             & (0.002)  & (0.004)          & (0.003) & (0.001)  & (0.002)   \\
            cohort 2017                 & 0.495*** & 0.005*          & 0.003           & 0.157***   & 0.156                 & 0.064***      & -0.003   & -0.002    & -0.000               & 0.024**             & -0.005*  & 0.025***       & 0.018*** & -0.001 & -0.005*     \\
            & (0.066)  & (0.003)         & (0.002)         & (0.021)    & (0.161)               & (0.015)       & (0.005)  & (0.011)   & (0.001)              & (0.009)             & (0.003)  & (0.006)   & (0.004) & (0.001)  & (0.002)       \\
            controls                    & Yes      & Yes             & Yes             & Yes        & Yes                   & Yes           & Yes      & Yes       & Yes                  & Yes                 & Yes      & Yes                   & Yes                       & Yes                         & Yes                         \\
            year FE                     & Yes      & Yes             & Yes             & Yes        & Yes                   & Yes           & Yes      & Yes       & Yes                  & Yes                 & Yes      & Yes                   & Yes                       & Yes                         & Yes                         \\
            facility FE                 & Yes      & Yes             & Yes             & Yes        & Yes                   & Yes           & Yes      & Yes       & Yes                  & Yes                 & Yes      & Yes                   & Yes                       & Yes                         & Yes                         \\
            border-county FE            & Yes      & Yes             & Yes             & Yes        & Yes                   & Yes           & Yes      & Yes       & Yes                  & Yes                 & Yes      & Yes                   & Yes                       & Yes                         & Yes                         \\
            toxic chemical FE           & Yes      & Yes             & Yes             & Yes        & Yes                   & Yes           & Yes      & Yes       & Yes                  & Yes                 & Yes      & Yes                   & Yes                       & Yes                         & Yes                         \\
            toxic chemical LTs          & Yes      & Yes             & Yes             & Yes        & Yes                   & Yes           & Yes      & Yes       & Yes                  & Yes                 & Yes      & Yes                   & Yes                       & Yes                         & Yes                         \\
            border-county LTs           & Yes      & Yes             & Yes             & Yes        & Yes                   & Yes           & Yes      & Yes       & Yes                  & Yes                 & Yes      & Yes                   & Yes                       & Yes                         & Yes                         \\ \midrule
            Observations                & 785,762  & 785,762         & 785,762         & 785,762    & 785,762               & 785,762       & 785,762  & 785,762   & 785,762                 & 785,762                    & 785,762  & 785,762    & 785,762 & 785,762 & 785,762         \\
            $R^2$                       & 0.762    & 0.108           & 0.214           & 0.933      & 0.992                 & 0.254         & 0.250    & 0.517     & 0.122                & 0.291               & 0.269    & 0.394                 & 0.188                    & 0.251 & 0.269      \\ \bottomrule\bottomrule
        \end{tabular}%
    }
    \begin{minipage}{\columnwidth}
        \vspace{0.05in}
        \tiny NOTES: These results are obtained from estimating model~\ref{eq:mechanisms-source-reduction}. ***, **, and * denote significance levels at the less than $1\%$, $5\%$ and $10\%$, respectively. SRA means source reduction activities.
    \end{minipage}
\end{table}

    Conversely, the potential transmission mechanisms of increasing total toxic release intensities in high-profit-capital-intensive industries, include decreases in equivalent source reduction activities at manufacturing industry facilities. To maximize profits amid higher MW, as shown in Figure~\ref{fig:baseline-manufacturing-industry-profits} of Appendix~\ref{sec:appendix-descriptive-stat-list-of-toxic-chemicals-trends-mechanisms-and-correlations}, these industries prioritize economies of scale through automation often disregarding environmental impacts. Consequently, high-skilled employment and hours, labour productivity and outputs increase as shown in Figure~\ref{fig:baseline-manufacturing-industry-employment-heter} of Appendix~\ref{sec:appendix-baseline-robustness-tables-and-figures}, while green raw material substitutions, process upgrades and equipment modifications, operating activities and inventory management decline. Given the higher MW-induced labour costs, these industries reduce investments in green organic solvents use, cleaner energy sources, new green technology and spray equipment upgrades or modifications, leading to increases in their toxic release intensities, even in highly regulated common CAA-HAPs and uncommon CAA-HAPs chemical domains. Moreover, the decline in the following: recycling, inventory control in terms of better packaging and storage, leak inspections, and better operating practices in terms of shutting down inefficient equipments, further results in higher fugitive air emissions intensity.
    % Please add the following required packages to your document preamble:
% \usepackage{booktabs}
% \usepackage{graphicx}
\begin{table}[H]
    \centering
    \caption{Onsite SRA: Low-Profit-Labour Intensive Manufacturing Industries}
    \label{tab:onsite-source-reduction-activities-lpli}
    \resizebox{\columnwidth}{!}{%
        \begin{tabular}{@{}lllllll@{}}
            \toprule\toprule
            & \multicolumn{3}{c}{Material Substitution \& Process Modification} & \multicolumn{3}{c}{Inventory Management \& Operations Activities} \\
            \cmidrule(lr){2-4} \cmidrule(lr){5-7}
            Source Reduction Activities & clean fuel substitution & recirculation in process & tfp       & modified spray equipment & changed inventory control & improved loading procedures \\ \midrule
            $Treated^{e} \cdot D$       & 0.025***                & -0.014**                 & 0.013*    & -0.002*                  & -0.009***                 & -0.002                      \\
            & (0.007)                 & (0.006)                  & (0.007)   & (0.001)                  & (0.003)                   & (0.002)                     \\
            $Treated^{e}$               & -0.030***               & -0.000                   & -0.007    & 0.002***                 & 0.012***                  & 0.003*                      \\
            & (0.008)                 & (0.001)                  & (0.007)   & (0.001)                  & (0.004)                   & (0.002)                     \\
            cohort 2014                 & 0.018**                 & -0.022**                 & 0.008     & -0.002*                  & -0.012***                 & -0.004**                    \\
            & (0.008)                 & (0.010)                  & (0.007)   & (0.001)                  & (0.004)                   & (0.002)                     \\
            cohort 2015                 & 0.035***                & -0.004*                  & 0.021***  & -0.002*                  & -0.004                    & 0.001                       \\
            & (0.007)                 & (0.002)                  & (0.007)   & (0.001)                  & (0.002)                   & (0.002)                     \\
            cohort 2017                 & -0.003                  & -0.005*                  & -0.064**  & -0.009*                  & -0.012                    & -0.072                      \\
            & (0.015)                 & (0.003)                  & (0.028)   & (0.006)                  & (0.022)                   & (0.046)                     \\
            controls                    & Yes                     & Yes                      & Yes       & Yes                      & Yes                       & Yes                         \\
            year FE                     & Yes                     & Yes                      & Yes       & Yes                      & Yes                       & Yes                         \\
            facility FE                 & Yes                     & Yes                      & Yes       & Yes                      & Yes                       & Yes                         \\
            border-county FE            & Yes                     & Yes                      & Yes       & Yes                      & Yes                       & Yes                         \\
            toxic chemical FE           & Yes                     & Yes                      & Yes       & Yes                      & Yes                       & Yes                         \\
            toxic chemical LTs          & Yes                     & Yes                      & Yes       & Yes                      & Yes                       & Yes                         \\
            border-county LTs           & Yes                     & Yes                      & Yes       & Yes                      & Yes                       & Yes                         \\\midrule
            Observations                & 1,107,927               & 1,107,927                & 1,107,927 & 1,107,927                & 1,107,927                 & 1,107,927                   \\
            $R^2$                       & 0.359                   & 0.244                    & 0.708     & 0.343                    & 0.240                     & 0.076                       \\ \bottomrule\bottomrule
        \end{tabular}%
    }
    \begin{minipage}{\columnwidth}
        \vspace{0.05in}
        \tiny NOTES: These results are obtained from estimating model~\ref{eq:mechanisms-source-reduction}. ***, **, and * denote significance levels at the less than $1\%$, $5\%$ and $10\%$, respectively. SRA means source reduction activities.
    \end{minipage}
\end{table}

    Furthermore, in low-profit-labour-intensive industries, the switch to cleaner energy sources due to higher-labour-cost-induced MW results in reductions in their total release intensities. Essentially, given the higher MW floor, low-skilled employment and hours decline, labour productivity increases while output remained unchanged as shown in Figures~\ref{fig:baseline-manufacturing-industry-employment-heter} and~\ref{fig:baseline-manufacturing-industry-output-heter} of Appendix~\ref{sec:appendix-baseline-robustness-tables-and-figures}. Hence, to maximise profit, they shifted towards cleaner production, increasing labour productivity to produce the same level of output as before the MW increase. Consequently, their total release intensity declined driven by the decline in point air emission intensity. However, their fugitive air emission intensity rise due to the fall in recirculation of materials in the production process, decline in spray equipment upgrades, poor inventory control in terms of poor packaging and storage, and poor loading and unloading procedures of toxic chemicals. Given higher labour costs, investment in raw material recirculation in the production process decline, thus increasing the potential risks of spills and leaks involved in the transportation of the toxic chemical waste from production points to waste management sites. Additionally, the decline in the spray equipment upgrades and poor loading and unloading procedures further exacerbate these risks, thereby increasing fugitive air emission intensity.

    In contrast, albeit there is a switch to cheaper but crude energy sources in low-profit-capital-intensive industries, the increases in modified spray equipments and upgrades, improved inventory control in terms of better packaging and storage etc, improved loading and unloading procedures, all contributed towards reducing the risks of leaks or spills in the transportation of toxic chemical wastes from production points to waste management sites. Thus, resulting in the decline in their fugitive air emission intensity. Importantly, although labour productivity increased in low-profit-capital-intensive industries, employment and overall outputs remained unchanged as in Figures~\ref{fig:baseline-manufacturing-industry-employment-heter} and~\ref{fig:baseline-manufacturing-industry-output-heter} of Appendix~\ref{sec:appendix-baseline-robustness-tables-and-figures}.

    \subsection{Onsite Waste Management Activities}\label{subsec:onsite-waste-management-activities}
    Onsite waste management activities refers to the handling and management of already generated toxic wastes at onsite manufacturing facilities, including treatment and recycling activities. Treatment involves processes used to change either the physical, chemical or biological composition of a waste to make it less hazardous. It uses the following methods: biological, physical, chemical and thermal/incineration treatment methods.~\footnote{\tiny Biological treatment method requires highly skilled microbiologists and environmental engineers to monitor and maintain optimal conditions for the biodegradation of hazardous wastes. Thermal treatment method demands expert personnel to guarantee complete combustion, manage residual ash safely, and ensure environmental compliance. Similarly, physical treatment requires proficient labour to execute filtration, sedimentation, and adsorption processes for contaminant removal from waste. Chemical treatment method refers to processes that use chemical reactions to transform hazardous substances into less harmful or more manageable forms. These treatments can involve neutralization, oxidation, reduction, precipitation, or other chemical processes to detoxify waste or reduce its volume before release or further treatment} Recycling are activities to reuse or reclaim materials from toxic chemical waste streams such as depolymerization of plastics, requiring specialised knowledge in chemistry, material science, and engineering. To investigate these onsite waste management activities, I estimate the following model:
    \begin{align}
        M_{f,cp,c,t}^{wma} &= \beta (Treated^{e} \cdot D^{protech})_{f,s,t} + \psi (Treated^{e})_{s,t} + \vartheta (Treated \cdot D^{protech})_{f,s,t} + \mu (Post \cdot D^{protech})_{f,s,t} \nonumber \\
        &\quad + \tau Treated_{s,t} + \rho D_{f,s,t}^{protech} + \alpha Post_{t} + \delta X_{v,c,t-1} + \omega F_{f,t} + \lambda_{t} + \gamma_{f} + \phi_{cp} \nonumber \\
        &\quad + \zeta_{c} + \eta_{c,t} + \theta_{cp,t} + \varepsilon_{f,cp,c,t},\label{eq:mechanisms-waste-management}
    \end{align}
    where $M_{f,cp,c,t}^{wma}$ is the vector of logged treatment methods (biological, chemical, physical and thermal) and recycling activities of already generated toxic wastes at manufacturing industry facility, $f$ in cross-border county-pairs, $cp$ through toxic chemical use, $c$ in the year, $t$, and dummies of associated methods. The ATT is captured by $\beta$, which is the differential average effects of higher MW on treatment methods and recycling activities at manufacturing facilities in treated counties relative to adjacent control counties given either their profitability levels and production technology classification. That is, the separate differential impacts on either high- or low-profit-labour-intensive industries. $\psi$ captures the relative differential impact on either high- or low-profit-capital-intensive industries. $\beta + \psi$ captures the total differential impact on either high- or low-profit-labour-intensive industries. The results are reported in Table~\ref{tab:mechanisms-onsite-waste-management-activities}.
    % Please add the following required packages to your document preamble:
% \usepackage{booktabs}
% \usepackage{graphicx}
\begin{table}[H]
    \centering
    \caption{Onsite Waste Management Activities}
    \label{tab:mechanisms-onsite-waste-management-activities}
    \resizebox{\columnwidth}{!}{%
        \begin{tabular}{@{}lllllll@{}}
            \toprule\toprule
            & \multicolumn{5}{c}{High-Profit-Labour-Intensive Manufacturing Industries} & \multicolumn{1}{c}{Low-Profit-Labour-Intensive Manufacturing Industries} \\
            \cmidrule(lr){2-6} \cmidrule(lr){7-7}
            & \multicolumn{4}{c}{Treatment} & \multicolumn{1}{c}{Recycling} & \multicolumn{1}{c}{Treatment} \\
            \cmidrule(lr){2-5}  \cmidrule(lr){6-6} \cmidrule(lr){7-7}
            Waste Management Activities (log) & chemical & biological & thermal  & physical & recycling & physical  \\ \midrule
            $Treated^{e} \cdot D$             & 0.007    & 0.005      & -0.023   & 0.061**  & 0.083     & -0.035*   \\
            & (0.020)  & (0.021)    & (0.020)  & (0.028)  & (0.157)   & (0.019)   \\
            $Treated^{e}$                     & 0.000    & -0.025*    & 0.001    & -0.042*  & -0.224**  & 0.041**   \\
            & (0.015)  & (0.014)    & (0.012)  & (0.025)  & (0.090)   & (0.016)   \\
            cohort 2014                       & -0.004   & -0.001     & -0.047   & 0.059*   & 0.031     & -0.057*** \\
            & (0.025)  & (0.027)    & (0.029)  & (0.035)  & (0.192)   & (0.021)   \\
            cohort 2015                       & 0.036**  & 0.023      & 0.040*** & 0.068*** & 0.207     & -0.004    \\
            & (0.016)  & (0.015)    & (0.014)  & (0.023)  & (0.195)   & (0.023)   \\
            cohort 2017                       & -0.045   & -0.045     & -0.009   & -0.031   & 0.563*    & -0.218**  \\
            & (0.049)  & (0.070)    & (0.028)  & (0.044)  & (0.305)   & (0.085)   \\
            controls                          & Yes      & Yes        & Yes      & Yes      & Yes       & Yes       \\
            year FE                           & Yes      & Yes        & Yes      & Yes      & Yes       & Yes       \\
            facility FE                       & Yes      & Yes        & Yes      & Yes      & Yes       & Yes       \\
            border-county FE                  & Yes      & Yes        & Yes      & Yes      & Yes       & Yes       \\
            toxic chemical FE                 & Yes      & Yes        & Yes      & Yes      & Yes       & Yes       \\
            toxic chemical LTs                & Yes      & Yes        & Yes      & Yes      & Yes       & Yes       \\
            border-county LTs                 & Yes      & Yes        & Yes      & Yes      & Yes       & Yes       \\ \midrule
            Observations                      & 785,762  & 785,762    & 785,762  & 785,762  & 785,762   & 1,107,927 \\
            $R^2$                             & 0.674    & 0.660      & 0.633    & 0.641    & 0.776     & 0.740     \\ \bottomrule \bottomrule
        \end{tabular}%
    }
    \begin{minipage}{\columnwidth}
        \vspace{0.05in}
        \tiny NOTES: These results are obtained from estimating model~\ref{eq:mechanisms-waste-management}. ***, **, and * denote significance levels at the less than $1\%$, $5\%$, and $10\%$, respectively. The result for the production technology column is based on total payroll to revenue ratio.
    \end{minipage}
\end{table}

    The results indicate that the rising total toxic release intensities in high-profit-capital-intensive manufacturing industries, due to higher MW, is partly transmitted through substantial reductions in their biological and physical waste treatment methods, and recycling activities. To maximize profits amid higher MW, as shown in Figure~\ref{fig:baseline-manufacturing-industry-profits} of Appendix~\ref{sec:appendix-descriptive-stat-list-of-toxic-chemicals-trends-mechanisms-and-correlations}, these industries prioritize economies of scale through automation, often disregarding environmental impacts. Consequently, high-skilled employment and hours, labour productivity and outputs increase as shown in Figures~\ref{fig:baseline-manufacturing-industry-employment-heter} and~\ref{fig:baseline-manufacturing-industry-employment-heter} of Appendix~\ref{sec:appendix-baseline-robustness-tables-and-figures}, while the aforementioned waste management activities decline. These waste management activities and methods as defined above require specialized knowledge for safe execution and environmental compliance. Although employment and hours for high-skilled workers have increased, economies of scale is favored over managing toxic waste release intensities to maximize profits given the MW-induced cost burden of these high-skilled workers (see Figures~\ref{fig:baseline-manufacturing-industry-employment-heter} and~\ref{fig:baseline-manufacturing-industry-cost-heter} of Appendix~\ref{sec:appendix-baseline-robustness-tables-and-figures}). Thus, their toxic release intensities increase, stronger for fugitive than point air emissions, surface water discharge, and land releases intensities, even in highly regulated common CAA-HAPs and uncommon CAA-HAPs chemical domains  (see Figures~\ref{fig:heterogeneous-onsite-releases-intensity-caa-haps}). Additionally, for the low-profit-capital-intensive manufacturing industries, the decline in fugitive air emissions intensity is partly as a result of the increase in the physical treatment method. Given higher labour cost, more high-skilled workers are incorporated into their production functions. Hence, labour productivity and outputs increase to maximise profit, while safely applying the physical treatment method to prevent the adsorption and sedimentation of toxic wastes.

    Furthermore, the results show that declining total releases intensities in high-profit-labour-intensive manufacturing industries, driven by the reductions in air emissions from point and fugitive sources, surface water discharge, and land releases intensities in common-CAA-HAPs and PBT regulated domains are partly transmitted by the corresponding and significant increases in onsite chemical, thermal and physical treatment methods, and recycling activities for already generated toxic wastes, as indicated in Table~\ref{tab:mechanisms-onsite-waste-management-activities}. To maximise profit given MW-induced labour costs, these high-profit-labour-intensive industries balance environmental compliance with production efficiency. While their labour productivity and outputs remain fixed, they increase their employment of high-skilled workers to perform specialised onsite waste management activities in the chemical, thermal and physical treatment methods, and recycling activities that require specialized knowledge. Thus, causing reductions in their environmental footprints from air emissions intensity (from both point and fugitive sources) in the $2015$ and $2017$ cohorts, surface water discharge and land release intensities in the $2014$ cohort.


    \section{Concluding Remarks}\label{sec:concluding-remarks}
    This paper utilizes precise administrative toxic release inventory and manufacturing industry payroll data to document the unintended environmental consequences of a higher minimum wage (MW) policy. The preliminary findings provide causal evidence of increased cost burdens in both low- and high-skilled/wage workers, as well as in production materials, due to a higher MW floor. This is followed by rising outputs, labour productivity, and profits. Although overall effects on employment and production workers' hours are null, cohort-specific effects are more nuanced. Disemployment is more pronounced among low-skilled workers, while positive employment effects dominate in high-skilled workers. The disemployment effect for low-skilled workers is primarily driven by their reluctance to commute to distant higher MW counties/states, whereas the increasing demand for high-skilled/wage workers is independent of cross-county/state worker mobility. However, there is notable decline in material cost and output in the $2017$ cohort from a corresponding disemployment effect. These preliminary findings have significant implications for the environmental consequences of the higher MW policy.

    The overall results show that higher MW policies increased total onsite release intensity in treated counties, particularly in the $2014$ and $2015$ cohorts, driven by higher air emissions from point and fugitive sources. There is limited evidence of this increase in the $2017$ cohort. No significant differences were found in surface water discharge intensity overall, though the $2014$ cohort saw reductions, while the $2015$ cohort experienced increases. Land release intensities, including surface impoundment, also increased in the $2014$ and $2017$ cohorts, with no significant impact in $2015$ cohort. Further results reveal that these environmental consequences are nuanced as manufacturing industries adjust towards either a capital- or labour-intensive technology based on their profit maximisation objectives.

    The increasing toxic release intensities are dominated in high-profit-capital-intensive manufacturing industries. This increase is driven by rises in total air emissions intensities, from fugitive sources, surface water discharge, and land releases intensities including surface impoundment. These increases are present in both highest- and lowest-emitting industries in high competitive environments, and even highly regulated domains of common- and uncommon-CAA-HAPs, and PBT toxic chemicals, potentially resulting to carcinogenic emissions. There is limited evidence of increases in point air emission intensity.~\footnote{Furthermore, except for significant declining effect on fugitive air emission intensity in low-profit-capital-intensive manufacturing industries, there is limited evidence of increases in the intensities of total toxic releases, point air emissions, land releases, and surface impoundment.} In contrast, onsite toxic release intensity in high-profit-labour-intensive manufacturing industries decreased in the $2015$ cohort. This reduction is driven by corresponding decreases in point and fugitive air emissions intensities and reduced land releases in the $2014$ and $2017$ cohorts. Increases in surface impoundment intensity, particularly in $2015$ and $2017$ cohorts, also contribute to the differential decline. Conversely, the rise in fugitive air emissions in $2014$ is linked to a decreased surface impoundment intensity, while surface water discharge intensity remains largely unaffected.~\footnote{Additional findings reveal a reduction in toxic release intensity within low-profit-labor-intensive manufacturing industries, particularly in the $2015$ cohort, driven primarily by decreased air stack/point emissions. However, significant increases in fugitive air emission intensity were observed in the $2014$ and $2017$ cohorts.} Therefore, policymakers should incentivize the abatement and management of toxic chemical releases in these industries alongside a higher MW policy.

    The observed effects are driven by reductions in both onsite source reduction and waste management activities. The analysis indicates that increased toxic release intensities in high-profit-capital-intensive manufacturing industries are primarily due to declines in: $(i)$ green raw material substitutions, including the use of organic solvents and cleaner energy sources; $(ii)$ process upgrades and equipment modifications, such as the adoption of new green technologies, recycling enhancements, and spray equipment optimization; $(iii)$ inventory management improvements, including better packaging and storage practices; and $(iv)$ operating activities, such as leak inspections and the decommissioning of inefficient equipment. Additionally, significant reductions in waste management practices, including biological and physical waste treatment and recycling, further exacerbate the rise in toxic release intensities.

    Conversely, the decline in total release intensities in both low- and high-profit-labour-intensive manufacturing industries is attributable to improvements in the same areas. Specifically, the transition to clean energy sources, green organic solvents, advanced technologies, process recirculation, spray equipment upgrades, enhanced packaging and storage, leak inspections, and the shutdown of inefficient equipment, alongside increases in chemical and physical waste treatment and recycling, contribute to the reduced toxic release intensities.
%======================================================================================================================%
    \appendix


    \section{State Minimum Wage and Balance Tests}\label{sec:appendix-state-minimum-wage-and-balance-tests}
    % Please add the following required packages to your document preamble:
% \usepackage{booktabs}
% \usepackage{graphicx}
\begin{table}[H]
    \centering
    \caption{Minimum Wage Changes in US States from $2011-2017$}
    \label{tab:states-mw-changes}
    \scalebox{0.7}{
        \resizebox{\columnwidth}{!}{%
            \begin{tabular}{@{}llllllllll@{}}
                \toprule \toprule
                states         & 2011 & 2012 & 2013  & 2014 & 2015 & 2016 & 2017 & start MW & end MW \\ \midrule
                Alaska         & 0    & 0    & 0     & 0    & 1    & 1    & 0.05 & 7.75     & 9.8    \\
                Arkansas       & 0    & 0    & 0     & 0    & 1.25 & 0.5  & 0.5  & 6.25     & 8.5    \\
                Arizona        & 0.1  & 0.3  & 0.15  & 0.1  & 0.15 & 0    & 1.95 & 7.35     & 10     \\
                California     & 0    & 0    & 0     & 1    & 0    & 1    & 0.5  & 8        & 10.5   \\
                Colorado       & 0.12 & 0.28 & 0.14  & 0.22 & 0.23 & 0.08 & 0.99 & 7.36     & 9.3    \\
                Connecticut    & 0    & 0    & 0     & 0.45 & 0.45 & 0.45 & 0.5  & 8.25     & 10.1   \\
                Delaware       & 0    & 0    & 0     & 0.5  & 0.5  & 0    & 0    & 7.25     & 8.25   \\
                Florida        & 0    & 0.42 & 0.12  & 0.14 & 0.12 & 0    & 0.05 & 7.21     & 8.1    \\
                Georgia        & 0    & 0    & 0     & 0    & 0    & 0    & 0    & 5.15     & 5.15   \\
                Hawaii         & 0    & 0    & 0     & 0    & 0.5  & 0.75 & 0.75 & 7.25     & 9.25   \\
                Iowa           & 0    & 0    & 0     & 0    & 0    & 0    & 0    & 7.25     & 7.25   \\
                Idaho          & 0    & 0    & 0     & 0    & 0    & 0    & 0    & 7.25     & 7.25   \\
                Illinois       & 0    & 0    & 0     & 0    & 0    & 0    & 0    & 8.25     & 8.25   \\
                Indiana        & 0    & 0    & 0     & 0    & 0    & 0    & 0    & 7.25     & 7.25   \\
                Kansas         & 0    & 0    & 0     & 0    & 0    & 0    & 0    & 7.25     & 7.25   \\
                Kentucky       & 0    & 0    & 0     & 0    & 0    & 0    & 0    & 7.25     & 7.25   \\
                Massachusetts  & 0    & 0    & 0     & 0    & 1    & 1    & 1    & 8        & 11     \\
                Maryland       & 0    & 0    & 0     & 0    & 1    & 0.5  & 0.5  & 7.25     & 9.25   \\
                Maine          & 0    & 0    & 0     & 0    & 0    & 0    & 1.5  & 7.5      & 9      \\
                Michigan       & 0    & 0    & 0     & 0.75 & 0    & 0.35 & 0.4  & 7.4      & 8.9    \\
                Minnesota      & 0    & 0    & -0.01 & 1.85 & 1    & 0.5  & 0    & 6.16     & 9.5    \\
                Missouri       & 0    & 0    & 0.1   & 0.15 & 0.15 & 0    & 0.05 & 7.25     & 7.7    \\
                Montana        & 0.1  & 0.3  & 0.15  & 0.1  & 0.15 & 0    & 0.1  & 7.35     & 8.15   \\
                North Carolina & 0    & 0    & 0     & 0    & 0    & 0    & 0    & 7.25     & 7.25   \\
                North Dakota   & 0    & 0    & 0     & 0    & 0    & 0    & 0    & 7.25     & 7.25   \\
                Nebraska       & 0    & 0    & 0     & 0    & 0.75 & 1    & 0    & 7.25     & 9      \\
                New Hampshire  & 0    & 0    & 0     & 0    & 0    & 0    & 0    & 7.25     & 7.25   \\
                New Jersey     & 0    & 0    & 0     & 1    & 0.13 & 0    & 0.06 & 7.25     & 8.44   \\
                New Mexico     & 0    & 0    & 0     & 0    & 0    & 0    & 0    & 7.5      & 7.5    \\
                Nevada         & 0.7  & 0    & 0     & 0    & 0    & 0    & 0    & 8.25     & 8.25   \\
                New York       & 0    & 0    & 0     & 0.75 & 0.75 & 0.25 & 0.7  & 7.25     & 9.7    \\
                Ohio           & 0.1  & 0.3  & 0.15  & 0.1  & 0.15 & 0    & 0.05 & 7.4      & 8.1    \\
                Oklahoma       & 0    & 0    & 0     & 0    & 0    & 0    & 0    & 7.25     & 7.25   \\
                Oregon         & 0.1  & 0.3  & 0.15  & 0.15 & 0.15 & 0.5  & 0.5  & 8.5      & 10.25  \\
                Pennsylvania   & 0    & 0    & 0     & 0    & 0    & 0    & 0    & 7.25     & 7.25   \\
                Rhode Island   & 0    & 0    & 0.35  & 0.25 & 1    & 0.6  & 0    & 7.4      & 9.6    \\
                South Dakota   & 0    & 0    & 0     & 0    & 1.25 & 0.05 & 0.1  & 7.25     & 8.65   \\
                Texas          & 0    & 0    & 0     & 0    & 0    & 0    & 0    & 7.25     & 7.25   \\
                Utah           & 0    & 0    & 0     & 0    & 0    & 0    & 0    & 7.25     & 7.25   \\
                Virgina        & 0    & 0    & 0     & 0    & 0    & 0    & 0    & 7.25     & 7.25   \\
                Vermont        & 0.09 & 0.31 & 0.14  & 0.13 & 0.42 & 0.45 & 0.4  & 8.15     & 10     \\
                Washington     & 0.12 & 0.37 & 0.15  & 0.13 & 0.15 & 0    & 1.53 & 8.67     & 11     \\
                Wisconsin      & 0    & 0    & 0     & 0    & 0    & 0    & 0    & 7.25     & 7.25   \\
                West Virginia  & 0    & 0    & 0     & 0    & 0.75 & 0.75 & 0    & 7.25     & 8.75   \\
                Wyoming        & 0    & 0    & 0     & 0    & 0    & 0    & 0    & 5.15     & 5.15   \\ \bottomrule\bottomrule
            \end{tabular}%
        }
    }

\end{table}
    \begin{table}[H]
    \centering
    \caption{Descriptive Statistics: Treated v. Control Border Counties}
    \label{tab:descriptive-statistics-control-border-counties}
    \begin{tabular}{lrrrr}
        \toprule \toprule
        Variable                                     & Mean  & SD     & T     & C     \\ \midrule
        GDP per capita (1000's)                      & 44.92 & 8.56   & 45.07 & 44.89 \\
        industry employment (1000's)                 & 43.18 & 39.26  & 46.88 & 42.48 \\
        annual average establishments                & 4.88  & 12.57  & 3.19  & 5.20  \\
        chemical ancillary use (onsite)              & 0.21  & 0.41   & 0.28  & 0.20  \\
        chemical formulation component (onsite)      & 0.32  & 0.47   & 0.31  & 0.33  \\
        chemical manufacturing aid (onsite)          & 0.10  & 0.30   & 0.15  & 0.09  \\
        max number of chemicals at facility (onsite) & 3.86  & 1.52   & 3.86  & 3.87  \\
        entire facility (onsite)                     & 1.00  & 0.02   & 1.00  & 1.00  \\
        private facility (onsite)                    & 1.00  & 0.01   & 1.00  & 1.00  \\
        imported chemicals at facility (onsite)      & 0.06  & 0.23   & 0.11  & 0.04  \\
        produced chemicals at facility (onsite)      & 0.19  & 0.39   & 0.33  & 0.16  \\
        production ratio or activity index (onsite)  & 3.07  & 485.56 & 13.70 & 1.06  \\ \bottomrule\bottomrule
    \end{tabular}
    \begin{minipage}{13.5cm}
        \vspace{0.05in}
        \tiny NOTES: The table contains county-level descriptive statistics as of the year immediately before the first initial MW change. The sample is restricted to border counties in treated and control states (See Table~\ref{tab:states-mw-adjustments-t-and-c}).
    \end{minipage}
\end{table}

    \begin{table}[H]
    \centering
    \caption{Descriptive Statistics: Treated v. Control Border States}
    \label{tab:descriptive-statistics-control-border-states}
    \begin{tabular}{lrrrr}
        \toprule \toprule
        Variable                                     & Mean  & SD      & T     & C     \\ \midrule
        GDP per capita (1000's)                      & 45.67 & 8.92    & 43.95 & 46.14 \\
        industry employment (1000's)                 & 46.25 & 47.71   & 48.87 & 45.52 \\
        annual average establishments                & 9.14  & 23.47   & 3.92  & 10.59 \\
        chemical ancillary use (onsite)              & 0.19  & 0.39    & 0.23  & 0.18  \\
        chemical formulation component (onsite)      & 0.24  & 0.43    & 0.24  & 0.24  \\
        chemical manufacturing aid (onsite)          & 0.10  & 0.30    & 0.12  & 0.09  \\
        max number of chemicals at facility (onsite) & 3.78  & 1.61    & 3.69  & 3.80  \\
        entire facility (onsite)                     & 1.00  & 0.04    & 1.00  & 1.00  \\
        private facility (onsite)                    & 1.00  & 0.01    & 1.00  & 1.00  \\
        imported chemicals at facility (onsite)      & 0.06  & 0.25    & 0.08  & 0.06  \\
        produced chemicals at facility (onsite)      & 0.27  & 0.44    & 0.33  & 0.25  \\
        production ratio or activity index (onsite)  & 12.07 & 1130.57 & 51.29 & 1.19  \\ \bottomrule\bottomrule
    \end{tabular}
    \begin{minipage}{13.5cm}
        \vspace{0.05in}
        \tiny NOTES: The table contains state-level descriptive statistics as of the year immediately before the first initial MW change. The sample is restricted to border counties in treated and control states (See Table~\ref{tab:states-mw-adjustments-t-and-c}).
    \end{minipage}
\end{table}

    \begin{figure}[H]
    \centering
    \includegraphics[width=1\textwidth, height=0.45\textheight]{fig_pre_evolution}
    \caption{County-level Macroeconomic Trends in Border Counties}
    \label{fig:county-level-macroeconomic-trends-in-border-counties}
    \begin{minipage}{18cm}
        \vspace{0.05in}
        {NOTES: This figure is obtained from estimating this equation $y_{c,t} = \sum_{t = 2011}^{2013} \beta_{c,t} (Treated \cdot B)_{s,t} + \lambda_{t} + \Phi_{c,p} + \zeta_{cp,t} + \epsilon_{c,t}$. Where $y_{c,t}$ is the vector of observables. Treated is the grouping variable that is unity for the treated states and zero for the control states. $B_{t}$ is a dummy variable with three levels of time, $2011$, $2012$, and $2013$. $\beta_{c,t}$ is the parameter vector of coefficients. $\lambda_{t}$ is the year fixed effects; $\Phi_{cp}$ is the border-county pair fixed effects; and $\zeta_{cp,t}$ is the border-county-pair-year fixed effects. $\epsilon_{c,t}$ is the error term. Robust standard errors are clustered at the state level. Row one shows the plots for the county level regressions and row two shows the plots for the state level regressions. \par}
    \end{minipage}
\end{figure}
    \begin{figure}[H]
    \centering
    \includegraphics[width=1\textwidth, height=0.45\textheight]{C:/Users/david/OneDrive/Documents/ULMS/PhD/Thesis/chapter3/src/climate_change/latex/fig_pre_evolution_state}
    \caption{State-level Macroeconomic Trends in Border States}
    \label{fig:state-level-macroeconomic-trends-in-border-states}
    \begin{minipage}{14cm}
        \vspace{0.05in}
        \tiny NOTES: This figure is obtained from estimating this equation $y_{s,t} = \sum_{t = 2011}^{2013} \beta (Treated \cdot B)_{s,t} + \lambda_{t} + \Phi_{sp} + \zeta_{sp,t} + \epsilon_{s,t}$. Where $y_{s,t}$ is the vector of observables. Treated is the grouping variable that is unity for the treated states and zero for the control states. $B_{t}$ is a dummy variable with three levels of time, $2011$, $2012$, and $2013$. $\beta$ is the state-level parameter vector of coefficients. $\lambda_{t}$ is the year fixed effects; $\Phi_{sp}$ is the border-state pair fixed effects; and $\zeta_{sp,t}$ is the border-state-pair-year fixed effects. $\epsilon_{s,t}$ is the error term. Robust standard errors are clustered at the state level.
    \end{minipage}
\end{figure}

    \section{Descriptive Statistics, List of Toxic Chemicals, Trends, Mechanisms, and Correlations}\label{sec:appendix-descriptive-stat-list-of-toxic-chemicals-trends-mechanisms-and-correlations}
    \begin{table}[H]
    \centering
    \caption{Summary Statistics (Onsite)}
    \label{tab:sumstat-onsite}
    \resizebox{\textwidth}{!}{
        \begin{tabular}{lrrrrr}
            \toprule \toprule
            Variable                                                & Obs     & Mean      & StdDev     & Min     & Max        \\ \midrule
            GDP per capita $(\$1000's)$                             & 1893689 & 44.99     & 42.29      & 2.90    & 365.80     \\
            industry employment (1000's)                            & 1893689 & 44.99     & 42.29      & 2.9     & 365.80     \\
            annual average establishments                           & 1893689 & 5.44      & 12.38      & 0.0     & 330.00     \\
            population (county) (1000's)                            & 1893689 & 693432.18 & 1247538.81 & 1466.00 & 5194675.00 \\
            city region average consumer price index $(\$)$         & 1893689 & 235.46    & 6.47       & 224.94  & 245.12     \\
            federal.facility                                        & 1893689 & 0.00      & 0.01       & 0.00    & 1.00       \\
            chemical ancillary use                                  & 1893689 & 0.25      & 0.43       & 0.0     & 1.00       \\
            chemical formulation component                          & 1893689 & 0.32      & 0.47       & 0.0     & 1.00       \\
            chemical manufacturing aid                              & 1893689 & 0.11      & 0.31       & 0.0     & 1.00       \\
            max number of chemicals at facility                     & 1893689 & 3.89      & 1.43       & 1.0     & 19.00      \\
            imported chemicals at facility                          & 1893689 & 0.07      & 0.25       & 0.0     & 1.00       \\
            produced chemicals at facility                          & 1893689 & 0.24      & 0.42       & 0.0     & 1.00       \\
            production ratio or activity index                      & 1893689 & 1.59      & 178.58     & 0.0     & 117229.00  \\
            total releases intensity (lbs)                          & 1893689 & 87.99     & 1065.44    & 0.00    & 122005.98  \\
            total air emissions intensity (lbs)                     & 1893689 & 60.02     & 616.84     & 0.00    & 40743.89   \\
            total fugitive air emissions intensity (lbs)            & 1893689 & 10.95     & 163.84     & 0.00    & 21484.45   \\
            total point air emissions intensity (lbs)               & 1893689 & 49.07     & 537.76     & 0.00    & 31559.41   \\
            total land releases intensity (lbs)                     & 1893689 & 7.93      & 701.61     & 0.00    & 122005.98  \\
            total underground injection intensity (lbs)             & 1893689 & 4.80      & 697.74     & 0.00    & 122005.98  \\
            total landfills intensity (lbs)                         & 1893689 & 1.43      & 53.40      & 0.00    & 6892.31    \\
            total releases to-land treatment intensity (lbs)        & 1893689 & 0.66      & 34.58      & 0.00    & 6006.01    \\
            total surface impoundment intensity (lbs)               & 1893689 & 0.03      & 2.40       & 0.00    & 929.15     \\
            total land releases intensity, others (lbs)             & 1893689 & 1.01      & 30.47      & 0.00    & 2299.05    \\
            total surface water discharge intensity (lbs)           & 1893689 & 20.04     & 475.39     & 0.00    & 41422.43   \\
            total number of receiving streams, onsite (lbs)         & 1893689 & 0.39      & 0.50       & 0.00    & 4.00       \\
            total release intensity, from catastrophic events (lbs) & 1893689 & 4.36      & 249.63     & 0.00   & 42103.29  \\
            total industry payroll $(\$1m)$                         & 1893689 & 2962.68   & 2630.57    & 127.00  & 16647.90   \\
            production workers (1000's)                             & 1893689 & 31.42     & 31.71      & 1.40    & 280.60     \\
            production hours (1m)                                   & 1893689 & 64.82     & 63.90      & 3.10    & 561.50     \\
%            production workers' wages $(\$1m)$                      & 1893689 & 1744.54   & 1583.39    & 64.90   & 10351.60    \\
            production workers' wages per hour                      & 1893689 & 26.56     & 7.35       & 12.24   & 54.35      \\
%            production workers' wages per worker                    & 1893689 & 55.28     & 17.85      & 24.47   & 114.48      \\
            cost of materials $(\$1m)$                              & 1893689 & 61328.05  & 162337.03  & 271.20  & 690771.20  \\
            industry value added (output) $(\$100m)$                & 1893689 & 177.13    & 272.06     & 3.00    & 1180.37    \\
            output per hour                                         & 1893689 & 2.62      & 2.82       & 0.44    & 34.28      \\
            output per worker                                       & 1893689 & 3.65      & 3.99       & 0.63    & 44.31      \\
            industry employment (1000's)                            & 1893689 & 44.99     & 42.29      & 2.90    & 365.80     \\ \bottomrule\bottomrule
        \end{tabular}
    }
\end{table}

%    \begin{table}[H]
    \centering
    \caption{Summary Statistics of Onsite Mechanisms}
    \label{tab:sumstat-onsite-mechanisms}
%    \scalebox{0.8}{
    \resizebox{\textwidth}{!}{
        \begin{tabular}{lrrrrr}
            \toprule\toprule
            Variable                                          & Obs     & Mean     & SD        & Min & Max      \\ \midrule
%            total waste management                            & 1893689 & 83667.25 & 894161.45 & 0   & 45000000 \\
            biological treatment                              & 1893689 & 0.06     & 0.24      & 0   & 1        \\
            physical treatment                                & 1893689 & 0.18     & 0.38      & 0   & 1        \\
            incineration or thermal treatment                 & 1893689 & 0.13     & 0.34      & 0   & 1        \\
%            industrial boiler energy recovery method          & 1893689 & 0.01     & 0.11      & 0   & 1        \\
            recycling quantity                                & 1893689 & 22386.84 & 427110.33 & 0   & 44938800 \\
            recycling to reuse in production process          & 1893689 & 0.04     & 0.19      & 0   & 1        \\
            source reduction activities                       & 1893689 & 0.25     & 0.43      & 0   & 1        \\
            chemical purity modification                      & 1893689 & 0.00     & 0.03      & 0   & 1        \\
            clean fuel substitution                           & 1893689 & 0.01     & 0.12      & 0   & 1        \\
            organic solvent substitution                      & 1893689 & 0.00     & 0.02      & 0   & 1        \\
            new technology or technique in production process & 1893689 & 0.00     & 0.06      & 0   & 1        \\
            recirculation in production process               & 1893689 & 0.00     & 0.04      & 0   & 1        \\
            recycling in production process                   & 1893689 & 0.03     & 0.16      & 0   & 1        \\
            product quality analysis                          & 1893689 & 0.00     & 0.02      & 0   & 1        \\
            operating practices training                      & 1893689 & 0.00     & 0.03      & 0   & 1        \\
            changing size of storage containers               & 1893689 & 0.00     & 0.02      & 0   & 1        \\
            improved material handling                        & 1893689 & 0.00     & 0.02      & 0   & 1        \\ \bottomrule
        \end{tabular}
    }
%    }
\end{table}
    \begin{table}[H]
    \centering
    \caption{Analyzed Chemicals}
    \label{tab:analyzed-chemicals}
%    \scalebox{0.35}{
    \resizebox{\textwidth}{!}{
        \begin{tabular}{llllllllllll}
            \toprule\toprule
            chemical name                                                              & classification & attribute             & onsite & offsite & potw & chemical name                                                                                                      & classification & attribute & onsite & offsite & potw\\
            \midrule
            1-Chloro-1,1-difluoroethane (HCFC-142b)                                    & TRI            & formulation component & yes    & NA      & NA   & Dimethylamine dicamba & TRI & others & yes & NA & NA\\
            1,2-Dibromoethane                                                          & TRI            & carcinogenic          & yes    & yes     & NA   & Dioxin and dioxin-like compounds                                                                                   & DIOXIN & carcinogenic & yes & yes & NA\\
            1,2-Dichloroethane                                                         & TRI            & carcinogenic          & yes    & yes     & NA   & Epichlorohydrin                                                                                                    & TRI            & carcinogenic          & yes & yes & NA\\
            1,2,4-Trimethylbenzene                                                     & TRI            & formulation component & yes    & yes     & yes  & Ethyl acrylate                                                                                                     & TRI            & carcinogenic & yes & yes & yes\\
            1,3-Butadiene                                                              & TRI            & carcinogenic          & yes    & NA      & NA   & Ethylbenzene                                                                                                       & TRI            & carcinogenic          & yes    & yes     & yes  \\
            1,3-Phenylenediamine                                                       & TRI            & formulation component & yes    & NA      & NA   & Ethylene                                                                                                           & TRI            & formulation component & yes & NA & NA\\
            1,4-Dioxane                                                                & TRI            & carcinogenic          & yes    & yes     & NA   & Ethylene glycol                                                                                                    & TRI            & clean air act         & yes    & yes & yes\\
            2-Ethoxyethanol                                                            & TRI            & formulation component & yes    & yes     & NA   & Ethylene oxide                                                                                                     & TRI            & carcinogenic & yes & NA & NA\\
            2-Phenylphenol                                                             & TRI            & formulation component & yes    & NA      & NA   & Fomesafen                                                                                                          & TRI            & formulation component & yes & yes & yes\\
            2,2-Bis(bromomethyl)-1,3-propanediol                                       & TRI            & carcinogenic          & yes    & NA      & NA   & Formaldehyde                                                                                                       & TRI & carcinogenic & yes & yes & yes\\
            2,2-Dichloro-1,1,1-trifluoroethane (HCFC-123)                              & TRI            & manufacturing aid     & yes    & yes     & NA   & Formic acid & TRI & formulation component & yes & yes & NA\\
            2,4-D                                                                      & TRI            & carcinogenic          & yes    & yes     & NA   & Hexachlorobenzene                                                                                                  & PBT            & carcinogenic          & yes    & yes     & yes  \\
            2,4-D 2-ethylhexyl ester                                                   & TRI            & carcinogenic          & yes    & yes     & NA   & Hydrazine                                                                                                          & TRI            & carcinogenic          & yes & NA & NA\\
            2,4-Dimethylphenol                                                         & TRI            & ancillary use         & yes    & NA      & NA   & Hydrochloric acid (acid aerosols including mists, vapors, gas, fog, and other airborne forms of any particle size) & TRI & clean air act & yes & NA & NA\\
            2,4-Dinitrotoluene                                                         & TRI            & carcinogenic          & yes    & NA      & NA   & Hydrogen cyanide                                                                                                   & TRI            & article component & yes & yes & NA\\
            2,6-Dinitrotoluene                                                         & TRI            & carcinogenic          & yes    & NA      & NA   & Hydrogen fluoride                                                                                                  & TRI            & clean air act & yes & yes & yes\\
            3-Iodo-2-propynyl butylcarbamate                                           & TRI            & formulation component & yes    & yes     & NA   & Hydroquinone & TRI & clean air act & yes & yes & yes\\
            4,4'-Isopropylidenediphenol                                                & TRI            & formulation component & yes    & yes     & NA   & Lead                                                                                                               & PBT            & carcinogenic & yes & yes & yes\\
            Acetaldehyde                                                               & TRI            & carcinogenic          & yes    & yes     & NA   & Lead compounds                                                                                                     & PBT            & clean air act         & yes    & yes & yes\\
            Acetonitrile                                                               & TRI            & clean air act         & yes    & yes     & NA   & Lithium carbonate                                                                                                  & TRI            & metal restricted & yes & yes & NA\\
            Acetophenone                                                               & TRI            & clean air act         & yes    & yes     & NA   & Maleic anhydride                                                                                                   & TRI            & clean air act         & yes & yes & yes\\
            Acrolein                                                                   & TRI            & clean air act         & yes    & NA      & NA   & Manganese                                                                                                          & TRI            & clean air act         & yes    & yes     & yes  \\
            Acrylamide                                                                 & TRI            & carcinogenic          & yes    & yes     & NA   & Manganese compounds                                                                                                & TRI            & clean air act         & yes & yes & yes\\
            Acrylic acid                                                               & TRI            & clean air act         & yes    & yes     & NA   & Mercury                                                                                                            & PBT            & clean air act         & yes    & yes     & yes  \\
            Acrylonitrile                                                              & TRI            & carcinogenic          & yes    & yes     & yes  & Mercury compounds                                                                                                  & PBT            & clean air act         & yes & yes & yes\\
            Allyl alcohol                                                              & TRI            & article component     & yes    & yes     & NA   & Methanol                                                                                                           & TRI            & clean air act         & yes    & yes & yes\\
            Aluminum (fume or dust)                                                    & TRI            & metal restricted      & yes    & yes     & NA   & Methoxone                                                                                                          & TRI            & carcinogenic & yes & NA & NA\\
            Aluminum oxide (fibrous forms)                                             & TRI            & metal restricted      & yes    & yes     & NA   & Methyl acrylate                                                                                                    & TRI & formulation component & yes & yes & NA\\
            Ammonia                                                                    & TRI            & formulation component & yes    & yes     & yes  & Methyl isobutyl ketone                                                                                             & TRI            & carcinogenic & yes & yes & yes\\
            Anthracene                                                                 & TRI            & formulation component & yes    & yes     & yes  & Methyl methacrylate                                                                                                & TRI            & clean air act & yes & yes & yes\\
            Antimony                                                                   & TRI            & clean air act         & yes    & yes     & yes  & Methyl tert-butyl ether                                                                                            & TRI            & clean air act & yes & yes & yes\\
            Antimony compounds                                                         & TRI            & clean air act         & yes    & yes     & yes  & Molybdenum trioxide                                                                                                & TRI            & metal restricted & yes & yes & NA\\
            Arsenic                                                                    & TRI            & carcinogenic          & yes    & NA      & NA   & n-Butyl alcohol                                                                                                    & TRI            & formulation component & yes & yes & yes\\
            Arsenic compounds                                                          & TRI            & clean air act         & yes    & yes     & NA   & n-Hexane                                                                                                           & TRI            & clean air act         & yes    & yes & yes\\
            Asbestos (friable)                                                         & TRI            & carcinogenic          & yes    & NA      & NA   & N-Methyl-2-pyrrolidone                                                                                             & TRI            & formulation component & yes & yes & yes\\
            Atrazine                                                                   & TRI            & formulation component & yes    & yes     & yes  & N-Methylolacrylamide                                                                                               & TRI            & others                & yes & yes & NA\\
            Barium                                                                     & TRI            & metal restricted      & yes    & NA      & NA   & N,N-Dimethylaniline                                                                                                & TRI            & clean air act         & yes    & NA & NA\\
            Barium compounds (except for barium sulfate (CAS No. 7727-43-7))           & TRI            & metal restricted & yes & yes & yes & N,N-Dimethylformamide & TRI & clean air act & yes & yes & yes\\
            Benzal chloride                                                            & TRI            & others                & yes    & yes     & NA   & Naphthalene                                                                                                        & TRI            & carcinogenic          & yes    & yes     & yes  \\
            Benzene                                                                    & TRI            & carcinogenic          & yes    & yes     & yes  & Nickel                                                                                                             & TRI            & carcinogenic          & yes    & yes     & yes  \\
            Benzo[g,h,i]perylene                                                       & PBT            & clean air act         & yes    & yes     & yes  & Nickel compounds                                                                                                   & TRI            & carcinogenic & yes & yes & yes\\
            Benzoyl peroxide                                                           & TRI            & formulation component & yes    & yes     & NA   & Nitrate compounds (water dissociable; reportable only when in aqueous solution) & TRI & formulation component & yes & yes & yes\\
            Benzyl chloride                                                            & TRI            & clean air act         & yes    & NA      & NA   & Nitric acid                                                                                                        & TRI            & formulation component & yes & yes & yes\\
            Biphenyl                                                                   & TRI            & clean air act         & yes    & NA      & NA   & Nitrobenzene                                                                                                       & TRI            & carcinogenic          & yes    & NA      & NA   \\
            Boron trichloride                                                          & TRI            & metal restricted      & yes    & NA      & NA   & o-Xylene                                                                                                           & TRI            & clean air act         & yes    & yes & NA\\
            Bromomethane                                                               & TRI            & clean air act         & yes    & NA      & NA   & Ozone                                                                                                              & TRI            & ancillary use         & yes    & NA      & NA   \\
            Butyl acrylate                                                             & TRI            & formulation component & yes    & yes     & yes  & p-Xylene                                                                                                           & TRI            & clean air act         & yes & NA & NA\\
            Cadmium                                                                    & TRI            & carcinogenic          & yes    & yes     & NA   & Pentachlorophenol                                                                                                  & TRI            & carcinogenic          & yes    & yes     & NA   \\
            Cadmium compounds                                                          & TRI            & carcinogenic          & yes    & yes     & yes  & Peracetic acid                                                                                                     & TRI            & formulation component & yes & yes & NA\\
            Carbon disulfide                                                           & TRI            & clean air act         & yes    & NA      & NA   & Phenanthrene                                                                                                       & TRI            & clean air act         & yes    & yes & yes\\
            Carbonyl sulfide                                                           & TRI            & clean air act         & yes    & NA      & NA   & Phenol                                                                                                             & TRI            & clean air act         & yes    & yes     & yes  \\
            Catechol                                                                   & TRI            & carcinogenic          & yes    & NA      & NA   & Phosphorus (yellow or white)                                                                                       & TRI            & clean air act & yes & NA & NA\\
            Certain glycol ethers                                                      & TRI            & clean air act         & yes    & yes     & yes  & Phthalic anhydride                                                                                                 & TRI            & clean air act & yes & yes & yes\\
            Chlorine                                                                   & TRI            & clean air act         & yes    & NA      & NA   & Polychlorinated biphenyls                                                                                          & PBT            & carcinogenic          & yes & NA & NA\\
            Chlorine dioxide                                                           & TRI            & article component     & yes    & NA      & NA   & Polycyclic aromatic compounds                                                                                      & PBT & carcinogenic & yes & yes & yes\\
            Chlorobenzene                                                              & TRI            & clean air act         & yes    & yes     & NA   & Propiconazole                                                                                                      & TRI            & formulation component & yes & yes & yes\\
            Chloroethane                                                               & TRI            & clean air act         & yes    & yes     & NA   & Propylene                                                                                                          & TRI            & formulation component & yes & yes & NA\\
            Chloroform                                                                 & TRI            & carcinogenic          & yes    & yes     & yes  & Propylene oxide                                                                                                    & TRI            & carcinogenic          & yes    & NA      & NA   \\
            Chloromethane                                                              & TRI            & clean air act         & yes    & NA      & NA   & Pyridine                                                                                                           & TRI            & article component     & yes    & yes & NA\\
            Chromium                                                                   & TRI            & clean air act         & yes    & yes     & yes  & sec-Butyl alcohol                                                                                                  & TRI            & formulation component & yes & yes & yes\\
            Chromium compounds (except for chromite ore mined in the Transvaal Region) & TRI            & clean air act & yes & yes & yes & Selenium compounds & TRI & clean air act & yes & yes & NA\\
            Cobalt                                                                     & TRI            & carcinogenic          & yes    & yes     & yes  & Silver                                                                                                             & TRI            & metal restricted      & yes    & yes     & yes  \\
            Cobalt compounds                                                           & TRI            & clean air act         & yes    & yes     & yes  & Silver compounds                                                                                                   & TRI            & metal restricted & yes & yes & yes\\
            Copper                                                                     & TRI            & metal restricted      & yes    & yes     & yes  & Sodium dimethyldithiocarbamate                                                                                     & TRI            & formulation component & yes & NA & NA\\
            Copper compounds                                                           & TRI            & metal restricted      & yes    & yes     & yes  & Sodium nitrite                                                                                                     & TRI            & metal restricted & yes & yes & yes\\
            Creosote                                                                   & TRI            & carcinogenic          & yes    & yes     & NA   & Styrene                                                                                                            & TRI            & carcinogenic          & yes    & yes     & yes  \\
            Cresol (mixed isomers)                                                     & TRI            & clean air act         & yes    & yes     & NA   & Sulfuric acid (acid aerosols including mists, vapors, gas, fog, and other airborne forms of any particle size) & TRI & formulation component & yes & NA & NA\\
            Cumene                                                                     & TRI            & carcinogenic          & yes    & yes     & NA   & tert-Butyl alcohol                                                                                                 & TRI            & formulation component & yes & NA & NA\\
            Cumene hydroperoxide                                                       & TRI            & manufacturing aid     & yes    & yes     & NA   & Tetrabromobisphenol A                                                                                              & PBT            & formulation component & yes & yes & yes\\
            Cyanide compounds                                                          & TRI            & clean air act         & yes    & yes     & yes  & Tetrachloroethylene                                                                                                & TRI            & carcinogenic & yes & yes & NA\\
            Cyclohexane                                                                & TRI            & formulation component & yes    & yes     & yes  & Thiabendazole                                                                                                      & TRI            & formulation component & yes & yes & yes\\
            Decabromodiphenyl oxide                                                    & TRI            & formulation component & yes    & yes     & NA   & Thiram                                                                                                             & TRI            & article component & yes & yes & yes\\
            Di(2-ethylhexyl) phthalate                                                 & TRI            & carcinogenic          & yes    & yes     & yes  & Toluene                                                                                                            & TRI            & clean air act & yes & yes & yes\\
            Diaminotoluene (mixed isomers)                                             & TRI            & carcinogenic          & yes    & NA      & NA   & Toluene-2,4-diisocyanate & TRI & carcinogenic & yes & yes & NA\\
            Dibenzofuran                                                               & TRI            & clean air act         & yes    & yes     & NA   & Toluene diisocyanate (mixed isomers)                                                                               & TRI & carcinogenic & yes & yes & NA\\
            Dibutyl phthalate                                                          & TRI            & clean air act         & yes    & NA      & NA   & Trichloroethylene                                                                                                  & TRI            & carcinogenic & yes & yes & NA\\
            Dichloromethane                                                            & TRI            & carcinogenic          & yes    & yes     & yes  & Triethylamine                                                                                                      & TRI            & clean air act         & yes & yes & NA\\
            Dicyclopentadiene                                                          & TRI            & formulation component & yes    & yes     & yes  & Trifluralin                                                                                                        & PBT            & clean air act & yes & NA & NA\\
            Diethanolamine                                                             & TRI            & clean air act         & yes    & yes     & NA   & Vanadium compounds                                                                                                 & TRI            & metal restricted & yes & yes & NA\\
            Diglycidyl resorcinol ether                                                & TRI            & carcinogenic          & yes    & NA      & NA   & Vinyl acetate                                                                                                      & TRI            & carcinogenic & yes & yes & NA\\
            Diisocyanates                                                              & TRI            & clean air act         & yes    & yes     & NA   & Vinyl chloride                                                                                                     & TRI            & carcinogenic          & yes    & yes & NA\\
            Dimethyl phthalate                                                         & TRI            & clean air act         & yes    & yes     & NA   & Xylene (mixed isomers)                                                                                             & TRI            & clean air act & yes & yes & yes\\
            Dimethylamine                                                              & TRI            & formulation component & yes    & yes     & NA   & Zinc (fume or dust)                                                                                                & TRI            & metal restricted & yes & yes & NA\\
            Dimethylamine dicamba                                                      & TRI            & others                & yes    & NA      & NA   & Zinc compounds                                                                                                     & TRI            & metal restricted      & yes & yes & yes\\ \bottomrule\bottomrule
        \end{tabular}
    }
%    }
    \begin{minipage}
        \linewidth
        \vspace{0.01in}
        \tiny NOTES: NA means absent in that sample; POTW means publicly owned treatment works.
    \end{minipage}
\end{table}

    \begin{figure}[H]
    \centering
    \includegraphics[width=\textwidth]{C:/Users/david/OneDrive/Documents/ULMS/PhD/Thesis/chapter3/src/climate_change/latex/motivation_plots}
    \caption{Trends in Releases Intensities by Treatment Status}
    \label{fig:releases-plots-treatment}
\end{figure}
    \begin{figure}[H]
    \centering
    \includegraphics[width=1\textwidth,keepaspectratio]{C:/Users/david/OneDrive/Documents/ULMS/PhD/Thesis/chapter3/src/climate_change/latex/fig_sdid_profits}
    \caption{Manufacturing Industry Profits}
    \label{fig:baseline-manufacturing-industry-profits}
    \begin{minipage}{\columnwidth}
        \vspace{0.05in}
        \tiny NOTES: The event study model of equation~\ref{eq:baseline-wages} is $R_{i,cp,t} = \sum_{{e = -3},{e \neq -1}}^{3} \beta Treated_{s,t}^e = \textbf{1}[t - G_{s,t}] + \delta X_{v,c,t-1} + \omega F_{f,t} + \lambda_{t} + \sigma_{c} + \phi_{cp} + \zeta_{cp,t} + \epsilon_{i,cp,t}$; where $R_{i,cp,t}$ is the profit and margin vector. Standard errors are clustered at the state level. de Chaisemartin and D'Haultfoeuille Decomposition: $\sum dCDH_{ATTs}^{weights(+)} = 1$ and $\sum dCDH_{ATTs}^{weights(-)} = 0$.
    \end{minipage}
\end{figure}
    \begin{figure}[H]
    \centering
    \includegraphics[width=0.85\textwidth]{C:/Users/david/OneDrive/Documents/ULMS/PhD/Thesis/chapter3/src/climate_change/latex/fig_correlation_het}
    \caption{Correlation matrix of heterogeneous dimensions}
    \label{fig:correlation-het}
\end{figure}


    \section{Distribution of Industries and Pollution Emissions Intensities}\label{sec:appendix-distribution-of-industries-and-pollution-emissions-intensities}
    \begin{figure}[H]
    \centering
    \includegraphics[width=0.85\textwidth]{C:/Users/david/OneDrive/Documents/ULMS/PhD/Thesis/chapter3/src/climate_change/latex/fig_naics_distribution}
    \caption{Distribution of Manufacturing Industries in the Sample}
    \label{fig:naics-manufacturing-industries}
\end{figure}
    \begin{figure}[H]
    \centering
    \includegraphics[width = 0.8\textwidth]{fig_releases_distribution}
    \caption{Distribution of Total Onsite Releases Intensity across Manufacturing Industries}
    \label{fig:releases-distribution}
\end{figure}
    \begin{figure}[H]
    \centering
    \includegraphics[width = 0.8\textwidth]{C:/Users/david/OneDrive/Documents/ULMS/PhD/Thesis/chapter3/src/climate_change/latex/fig_air_emissions_distribution_naics}
    \caption{Distribution of Total Onsite Air Emissions Intensity across Manufacturing Industries}
    \label{fig:air-emissions-distribution-naics}
\end{figure}
    \begin{figure}[H]
    \centering
    \includegraphics[width = 0.8\textwidth]{fig_water_distribution_naics}
    \caption{Distribution of Total Onsite Surface Water Discharge Intensity across Manufacturing Industries}
    \label{fig:water-discharge-distribution-naics}
\end{figure}
    \begin{figure}[H]
    \centering
    \includegraphics[width = 0.8\textwidth]{C:/Users/david/OneDrive/Documents/ULMS/PhD/Thesis/chapter3/src/climate_change/latex/fig_land_releases_distribution_naics}
    \caption{Distribution of Total Onsite Land Releases Intensity across Manufacturing Industries}
    \label{fig:land-releases-distribution-naics}
\end{figure}
    \begin{figure}[H]
    \centering
    \includegraphics[width = 0.8\textwidth]{fig_releases_distribution_states}
    \caption{Distribution of Total Onsite Releases Intensity between the Treated and Control States}
    \label{fig:releases-distribution}
\end{figure}
    \begin{figure}[H]
    \centering
    \includegraphics[width = 0.8\textwidth]{fig_air_emissions_distribution_state}
    \caption{Distribution of Total Air Emission Intensity between the Treated and Control States.}
    \label{fig:air-emissions-distribution}
\end{figure}
    \begin{figure}[H]
    \centering
    \includegraphics[width = 0.8\textwidth]{C:/Users/david/OneDrive/Documents/ULMS/PhD/Thesis/chapter3/src/climate_change/latex/fig_water_discharge_distribution_state}
    \caption{Distribution of Total Surface Water Discharge Intensity between the Treated and Control States}
    \label{fig:water-discharge-distribution}
\end{figure}
    \begin{figure}[H]
    \centering
    \includegraphics[width = 0.8\textwidth]{fig_land_releases_distribution_state}
    \caption{Distribution of Total Onsite Land Releases Intensity between the Treated and Control States}
    \label{fig:land-releases-distribution}
\end{figure}
    \begin{figure}[H]
    \centering
    \includegraphics[width = 0.8\textwidth]{fig_releases_distribution_carcinogenic}
    \caption{Distribution of Average Total Onsite Carcinogenic Releases Intensity between the Treated and Control States}
    \label{fig:releases-distribution-carcinogenic}
\end{figure}
    \begin{figure}[H]
    \centering
    \includegraphics[width = 0.8\textwidth]{C:/Users/david/OneDrive/Documents/ULMS/PhD/Thesis/chapter3/src/climate_change/latex/fig_releases_distribution_caa}
    \caption{Distribution of Average Total Onsite CAA Releases Intensity between the Treated and Control States}
    \label{fig:releases-distribution-caa}
\end{figure}
    \begin{figure}[H]
    \centering
    \includegraphics[width = 0.8\textwidth]{fig_releases_distribution_haps}
    \caption{Distribution of Total Onsite HAPs Releases Intensity between the Treated and Control States}
    \label{fig:releases-distribution-haps}
\end{figure}
    \begin{figure}[H]
    \centering
    \includegraphics[width = 0.8\textwidth]{fig_releases_distribution_pbts}
    \caption{Distribution of Average Total Onsite PBT Releases Intensity between the Treated and Control States}
    \label{fig:releases-distribution-pbts}
\end{figure}


    \section{Baseline Robustness Tables and Figures}\label{sec:appendix-baseline-robustness-tables-and-figures}
    % Please add the following required packages to your document preamble:
% \usepackage{booktabs}
% \usepackage{graphicx}
\begin{table}[H]
    \centering
    \caption{Treatment Selection}
    \label{tab:treatment-selection}
    \resizebox{\columnwidth}{!}{%
        \begin{tabular}{@{}llllll@{}}
            \toprule\toprule
            $Treated^{e}$                         & 1                 & 2                 & 3              & 4              & 5              \\ \midrule
            $GDP_{1}$                             & -0.000 (0.000)    & -0.000 (0.000)    & 0.000 (0.000)  & 0.000 (0.000)  & 0.000 (0.000)  \\
            $GDPPC_{1}$                           & 0.001 (0.002)     & 0.001 (0.002)     & 0.001 (0.002)  & 0.001 (0.002)  & -0.002 (0.002) \\
            $Personal\_Income_{1}$                & 0.000 (0.000)     & 0.000 (0.000)     & -0.000 (0.000) & -0.000 (0.000) & 0.000 (0.000)  \\
            $Average\_Number\_Establishments_{1}$ & -0.001 (0.001)    & -0.000 (0.001)    & -0.000 (0.000) & -0.000 (0.000) & 0.000 (0.000)  \\
            $Inflation_{1}$                       & 0.010 (0.007)     & NA                & NA             & NA             & NA             \\
            Entire Facility (0,1)                 & -0.637*** (0.149) & -0.654*** (0.144) & 0.015 (0.069)  & 0.015 (0.069)  & 0.006 (0.005)  \\
            Private Facility (0,1)                & 0.094 (0.060)     & 0.094 (0.059)     & 0.008 (0.005)  & 0.008 (0.005)  & 0.004 (0.003)  \\
            Federal Facility (0,1)                & -0.082 (0.074)    & -0.082 (0.074)    & 0.001 (0.031)  & 0.001 (0.031)  & -0.011 (0.008) \\
            controls                              & Yes               & Yes               & Yes            & Yes            & Yes            \\
            year FE                               & No                & Yes               & Yes            & Yes            & Yes            \\
            county FE                             & No                & No                & Yes            & Yes            & Yes            \\
            border-county FE                      & No                & No                & No             & Yes            & Yes            \\
            border-county LTs                     & No                & No                & No             & No             & Yes            \\ \midrule
            Observations                          & 1,893,689         & 1,893,689         & 1,893,689      & 1,893,689      & 1,893,689      \\
            $R^2$                                 & 0.082             & 0.110             & 0.616          & 0.616          & 0.952          \\ \bottomrule \bottomrule
        \end{tabular}%
    }
    \begin{minipage}{\columnwidth}
        \vspace{0.05in}
        \tiny NOTES: Robust standard errors clustered at the state level are reported in parentheses. ***, **, and * denote significance levels at the less than $1\%$, $5\%$ and $10\%$, respectively.

    \end{minipage}
\end{table}
    \begin{figure}[H]
    \centering
    \includegraphics[width=1\textwidth,keepaspectratio]{fig_sdid_industry_cost_heter}
    \caption{Heterogeneous Effects of the MW Policy on Manufacturing Industry Costs}
    \label{fig:baseline-manufacturing-industry-cost-heter}
    \begin{minipage}{\columnwidth}
        \vspace{0.05in}
        \tiny NOTES: The event study model of equation~\ref{eq:baseline-wages} is $C_{i,cp,t} = \sum_{{e = -3},{e \neq -1}}^{3} \beta (Treated^{e} \cdot D)_{i,s,t} + \psi (Treated^{e})_{s,t} + \vartheta (Treated \cdot D)_{i,s,t} + \mu (Post \cdot D)_{i,s,t} + \tau Treated_{s,t} + \rho D_{i,s,t} + \alpha Post_{t} + \delta X_{v,c,t-1} + \omega F_{f,t} + \lambda_{t} + \sigma_{c} + \phi_{cp} + \zeta_{cp,t} + \epsilon_{i,cp,t}$. Standard errors are clustered at the state level. $D_{i,s,t}$ is unity for low-skilled and zero for high-skilled workers; $Treated_{s,t}$ is unity for treated and zero for control states; and $Post_{t}$ is unity for post-treatment and zero for pre-treatment periods.
    \end{minipage}
\end{figure}

%\begin{figure}[H]
%    \centering
%    \includegraphics[width=1\textwidth,keepaspectratio]{fig_sdid_industry_cost_heter_hpli_lpli}
%    \caption{Heterogeneous Effects of the MW Policy on Manufacturing Industry Costs}
%    \label{fig:baseline-manufacturing-industry-cost-heter-hpli-lpli}
%    \begin{minipage}{\columnwidth}
%        \vspace{0.05in}
%        \tiny NOTES: The event study model of equation~\ref{eq:baseline-wages} is $C_{i,cp,t} = \sum_{{e = -3},{e \neq -1}}^{3} \beta (Treated^{e} \cdot D)_{i,s,t} + \psi (Treated^{e})_{s,t} + \vartheta (Treated \cdot D)_{i,s,t} + \mu (Post \cdot D)_{i,s,t} + \tau Treated_{s,t} + \rho D_{i,s,t} + \alpha Post_{t} + \delta X_{v,c,t-1} + \omega F_{f,t} + \lambda_{t} + \sigma_{c} + \phi_{cp} + \zeta_{cp,t} + \epsilon_{i,cp,t}$. Standard errors are clustered at the state level. $D_{i,s,t}$ is unity for low-skilled and zero for high-skilled workers; $Treated_{s,t}$ is unity for treated and zero for control states; and $Post_{t}$ is unity for post-treatment and zero for pre-treatment periods.
%    \end{minipage}
%\end{figure}
    \begin{figure}[H]
    \centering
    \includegraphics[width=1\textwidth,keepaspectratio]{fig_sdid_emp_hours_heter}
    \caption{Heterogeneous Effects of the MW Policy on Employment and Hours}
    \label{fig:baseline-manufacturing-industry-employment-heter}
    \begin{minipage}{\columnwidth}
        \vspace{0.05in}
        \tiny NOTES: The event study model of equation~\ref{eq:baseline-wages} is $E_{i,cp,t} = \sum_{{e = -3},{e \neq -1}}^{3} \beta (Treated^{e} \cdot D)_{i,s,t} + \psi (Treated^{e})_{s,t} + \vartheta (Treated \cdot D)_{i,s,t} + \mu (Post \cdot D)_{i,s,t} + \tau Treated_{s,t} + \rho D_{i,s,t} + \alpha Post_{t} + \delta X_{v,c,t-1} + \omega F_{f,t} + \lambda_{t} + \sigma_{c} + \phi_{cp} + \zeta_{cp,t} + \epsilon_{i,cp,t}$. Standard errors are clustered at the state level. $D_{i,s,t}$ is unity for low-skilled and zero for high-skilled workers; $Treated_{s,t}$ is unity for treated and zero for control states; and $Post_{t}$ is unity for post-treatment and zero for pre-treatment periods.
    \end{minipage}
\end{figure}

%\begin{figure}[H]
%    \centering
%    \includegraphics[width=1\textwidth,keepaspectratio]{fig_sdid_emp_hours_heter_hpli_lpli}
%    \caption{Heterogeneous Effects of the MW Policy on Employment and Hours}
%    \label{fig:baseline-manufacturing-industry-employment-heter-hpli-lpli}
%    \begin{minipage}{\columnwidth}
%        \vspace{0.05in}
%        \tiny NOTES: The event study model of equation~\ref{eq:baseline-wages} is $E_{i,cp,t} = \sum_{{e = -3},{e \neq -1}}^{3} \beta (Treated^{e} \cdot D)_{i,s,t} + \psi (Treated^{e})_{s,t} + \vartheta (Treated \cdot D)_{i,s,t} + \mu (Post \cdot D)_{i,s,t} + \tau Treated_{s,t} + \rho D_{i,s,t} + \alpha Post_{t} + \delta X_{v,c,t-1} + \omega F_{f,t} + \lambda_{t} + \sigma_{c} + \phi_{cp} + \zeta_{cp,t} + \epsilon_{i,cp,t}$. Standard errors are clustered at the state level. $D_{i,s,t}$ is unity for low-skilled and zero for high-skilled workers; $Treated_{s,t}$ is unity for treated and zero for control states; and $Post_{t}$ is unity for post-treatment and zero for pre-treatment periods.
%    \end{minipage}
%\end{figure}
    \begin{figure}[H]
    \centering
    \includegraphics[width=1\textwidth,keepaspectratio]{C:/Users/david/OneDrive/Documents/ULMS/PhD/Thesis/chapter3/src/climate_change/latex/fig_sdid_output_heter}
    \caption{Heterogeneous Effects of the MW Policy on Outputs}
    \label{fig:baseline-manufacturing-industry-output-heter}
    \begin{minipage}{\columnwidth}
        \vspace{0.05in}
        \tiny NOTES: The event study model of equation~\ref{eq:baseline-wages} is $Y_{i,cp,t} = \sum_{{e = -3},{e \neq -1}}^{3} \beta (Treated^{e} \cdot D)_{i,s,t} + \psi (Treated^{e})_{s,t} + \vartheta (Treated \cdot D)_{i,s,t} + \mu (Post \cdot D)_{i,s,t} + \tau Treated_{s,t} + \rho D_{i,s,t} + \alpha Post_{t} + \delta X_{v,c,t-1} + \omega F_{f,t} + \lambda_{t} + \sigma_{c} + \phi_{cp} + \zeta_{cp,t} + \epsilon_{i,cp,t}$. Standard errors are clustered at the state level. $D_{i,s,t}$ is unity for low-skilled and zero for high-skilled workers; $Treated_{s,t}$ is unity for treated and zero for control states; and $Post_{t}$ is unity for post-treatment and zero for pre-treatment periods.
    \end{minipage}
\end{figure}
    % Please add the following required packages to your document preamble:
% \usepackage{booktabs}
% \usepackage{graphicx}
\begin{table}[H]
    \centering
    \caption{Potential Cross County/State Mobility}
    \label{tab:baseline-cross-county-state-mobility}
    \resizebox{\columnwidth}{!}{%
        \begin{tabular}{@{}lllllll@{}}
            \toprule\toprule
            & \multicolumn{2}{c}{Employment (log)} & \multicolumn{2}{c}{Production Workers (log)} & \multicolumn{2}{c}{Production Hours (log)} \\
            \cmidrule(lr){2-3} \cmidrule(lr){4-5} \cmidrule(lr){6-7}
            employment \& hours          & 1         & 2         & 3         & 4         & 5         & 6         \\ \midrule
            $Treated^{e} \cdot distance$ & -0.000    & 0.000     & -0.000    & 0.000     & -0.001    & -0.000    \\
            & (0.001)   & (0.001)   & (0.002)   & (0.001)   & (0.001)   & (0.001)   \\
            $Treated^{e}$                & -0.005    & 0.005     & -0.017    & -0.004    & -0.010    & 0.001     \\
            & (0.034)   & (0.033)   & (0.036)   & (0.030)   & (0.034)   & (0.030)   \\
            $distance \cdot$ cohort 2014 & -0.001    & -0.001*** & -0.001    & -0.002*** & -0.001    & -0.002*** \\
            & (0.001)   & (0.000)   & (0.001)   & (0.000)   & (0.001)   & (0.000)   \\
            $distance \cdot$ cohort 2015 & -0.000    & 0.003     & 0.000     & 0.004     & -0.000    & 0.004     \\
            & (0.001)   & (0.002)   & (0.002)   & (0.002)   & (0.002)   & (0.002)   \\
            $distance \cdot$ cohort 2017 & -0.003**  & -0.019*** & -0.002    & -0.019*** & -0.002    & -0.019*** \\
            & (0.001)   & (0.003)   & (0.002)   & (0.004)   & (0.002)   & (0.003)   \\
            controls                     & Yes       & Yes       & Yes       & Yes       & Yes       & Yes       \\
            year FE                      & Yes       & Yes       & Yes       & Yes       & Yes       & Yes       \\
            county FE                    & Yes       & Yes       & Yes       & Yes       & Yes       & Yes       \\
            border-county FE             & No        & Yes       & No        & Yes       & No        & Yes       \\
            border-county LTs            & No        & Yes       & No        & Yes       & No        & Yes       \\ \midrule
            Observations                 & 1,893,689 & 1,893,689 & 1,893,689 & 1,893,689 & 1,893,689 & 1,893,689 \\
            $R^2$                        & 0.354     & 0.393     & 0.336     & 0.378     & 0.341     & 0.386     \\
            Baseline Mean                & 44.99     & 44.99     & 31.42     & 31.42     & 64.82     & 64.82     \\ \bottomrule \bottomrule
        \end{tabular}%
    }
    \begin{minipage}{\columnwidth}
        \vspace{0.05in}
        \tiny NOTES: These results are obtained from estimating model $E_{i,cp,t} = \beta (Treated^{e} \cdot D)_{h,s,t} + \psi (Treated^{e})_{s,t} + \vartheta (Treated \cdot D)_{h,s,t} + \mu (Post \cdot D)_{h,s,t} + \tau Treated_{s,t} + \rho D_{h,s,t} + \alpha Post_{t} + \delta X_{v,c,t-1} + \omega F_{f,t} + \lambda_{t} + \sigma_{h} + \phi_{cp} + \zeta_{cp,t} + \epsilon_{i,cp,t}$. Robust standard errors clustered at the state level are reported in parentheses. ***, **, and * denote significance levels at the less than $1\%$, $5\%$ and $10\%$, respectively.
    \end{minipage}
\end{table}
    % Please add the following required packages to your document preamble:
% \usepackage{booktabs}
% \usepackage{graphicx}
\begin{table}[H]
    \centering
    \caption{Manufacturing Industry Costs: Alternative Clustering of the SEs}
    \label{tab:baseline-cost-robustness}
    \resizebox{\columnwidth}{!}{%
        \begin{tabular}{@{}lllllllllllll@{}}
            \toprule  \toprule
            & \multicolumn{4}{c}{Hourly Wage} & \multicolumn{4}{c}{Total Payroll (log)} & \multicolumn{4}{c}{Material Cost (log)} \\
            \cmidrule(lr){2-5} \cmidrule(lr){6-9} \cmidrule(lr){10-13}
            industry costs    & 1         & 2         & 3         & 4         & 5         & 6         & 7         & 8         & 9         & 10        & 11        & 12        \\ \midrule
            $Treated^{e}$     & 0.889**   & 0.889**   & 0.889**   & 0.889**   & 0.043*    & 0.043     & 0.043     & 0.043     & 0.129*    & 0.129*    & 0.129*    & 0.129*    \\
            & (0.449)   & (0.404)   & (0.426)   & (0.415)   & (0.026)   & (0.038)   & (0.034)   & (0.037)   & (0.071)   & (0.069)   & (0.072)   & (0.072)   \\
            cohort 2014       & 1.200*    & 1.200**   & 1.200*    & 1.200*    & 0.004     & 0.004     & 0.004     & 0.004*    & 0.187*    & 0.187*    & 0.187*    & 0.187*    \\
            & (0.707)   & (0.593)   & (0.656)   & (0.629)   & (0.037)   & (0.049)   & (0.044)   & (0.049)   & (0.108)   & (0.107)   & (0.106)   & (0.106)   \\
            cohort 2015       & 0.382**   & 0.382     & 0.382     & 0.382     & 0.115***  & 0.115*    & 0.115**   & 0.115*    & 0.035     & 0.035     & 0.035     & 0.035     \\
            & (0.192)   & (0.379)   & (0.314)   & (0.357)   & (0.031)   & (0.065)   & (0.053)   & (0.058)   & (0.060)   & (0.077)   & (0.074)   & (0.074)   \\
            cohort 2017       & -0.208    & -0.208    & -0.208    & -0.208    & -0.612*** & -0.612*** & -0.612*** & -0.612*** & -0.343*** & -0.343*** & -0.343*** & -0.343*** \\
            & (0.436)   & (1.270)   & (1.000)   & (0.996)   & (0.166)   & (0.199)   & (0.199)   & (0.200)   & (0.131)   & (0.212)   & (0.181)   & (0.180)   \\
            controls          & Yes       & Yes       & Yes       & Yes       & Yes       & Yes       & Yes       & Yes       & Yes       & Yes       & Yes       & Yes       \\
            year FE           & Yes       & Yes       & Yes       & Yes       & Yes       & Yes       & Yes       & Yes       & Yes       & Yes       & Yes       & Yes       \\
            county FE         & Yes       & Yes       & Yes       & Yes       & Yes       & Yes       & Yes       & Yes       & Yes       & Yes       & Yes       & Yes       \\
            border-county FE  & Yes       & Yes       & Yes       & Yes       & Yes       & Yes       & Yes       & Yes       & Yes       & Yes       & Yes       & Yes       \\
            border-county LTs & Yes       & Yes       & Yes       & Yes       & Yes       & Yes       & Yes       & Yes       & Yes       & Yes       & Yes       & Yes       \\\midrule
            clustered at the: & county    & industry  & zipcode   & facility  & county    & industry  & zipcode   & facility  & county    & industry  & zipcode   & facility  \\
            Observations      & 1,893,689 & 1,893,689 & 1,893,689 & 1,893,689 & 1,893,689 & 1,893,689 & 1,893,689 & 1,893,689 & 1,893,689 & 1,893,689 & 1,893,689 & 1,893,689 \\
            $R^2$             & 0.624     & 0.624     & 0.624     & 0.624     & 0.440     & 0.440     & 0.440     & 0.440     & 0.619     & 0.619     & 0.619     & 0.619     \\ \bottomrule \bottomrule
        \end{tabular}%
    }
    \begin{minipage}{\columnwidth}
        \vspace{0.05in}
        \tiny NOTES: These results are obtained from estimating model~\ref{eq:baseline-emp-hours}. Robust standard errors clustered at the state level are reported in parentheses. ***, **, and * denote significance levels at the less than $1\%$, $5\%$ and $10\%$, respectively.
    \end{minipage}
\end{table}
    % Please add the following required packages to your document preamble:
% \usepackage{booktabs}
% \usepackage{graphicx}
\begin{table}[H]
    \centering
    \caption{Employment and Hours: Alternative Clustering of the SEs}
    \label{tab:baseline-employ-robustness}
    \resizebox{\columnwidth}{!}{%
        \begin{tabular}{@{}lllllllllllll@{}}
            \toprule\toprule
            & \multicolumn{4}{c}{Employment} & \multicolumn{4}{c}{Production Workers} & \multicolumn{4}{c}{Production Hours} \\
            \cmidrule(lr){2-5} \cmidrule(lr){6-9} \cmidrule(lr){10-13}
            employment/hours (log) & 1         & 2         & 3         & 4         & 5         & 6         & 7         & 8         & 9         & 10        & 11        & 12        \\ \midrule
            $Treated^{e}$          & -0.002    & -0.002    & -0.002    & -0.002    & -0.023    & -0.023    & -0.023    & -0.023    & -0.019    & -0.019    & -0.019    & -0.019    \\
            & (0.024)   & (0.037)   & (0.033)   & (0.036)   & (0.030)   & (0.039)   & (0.038)   & (0.039)   & (0.032)   & (0.039)   & (0.038)   & (0.040)   \\
            cohort 2014            & -0.058*   & -0.058    & -0.058    & -0.058    & -0.097**  & -0.097**  & -0.097**  & -0.097*   & -0.094**  & -0.094**  & -0.094*   & -0.094*   \\
            & (0.034)   & (0.043)   & (0.041)   & (0.045)   & (0.044)   & (0.046)   & (0.049)   & (0.052)   & (0.047)   & (0.047)   & (0.051)   & (0.054)   \\
            cohort 2015            & 0.095***  & 0.095     & 0.095*    & 0.095     & 0.104***  & 0.104     & 0.104*    & 0.104*    & 0.111***  & 0.111*    & 0.111*    & 0.111*    \\
            & (0.030)   & (0.068)   & (0.056)   & (0.062)   & (0.034)   & (0.068)   & (0.059)   & (0.063)   & (0.036)   & (0.066)   & (0.058)   & (0.061)   \\
            cohort 2017            & -0.552*** & -0.552*** & -0.552*** & -0.552*** & -0.568*** & -0.568**  & -0.568**  & -0.568**  & -0.556*** & -0.556**  & -0.556**  & -0.556**  \\
            & (0.179)   & (0.208)   & (0.211)   & (0.212)   & (0.208)   & (0.244)   & (0.241)   & (0.241)   & (0.191)   & (0.221)   & (0.229)   & (0.227)   \\
            controls               & Yes       & Yes       & Yes       & Yes       & Yes       & Yes       & Yes       & Yes       & Yes       & Yes       & Yes       & Yes       \\
            year FE                & Yes       & Yes       & Yes       & Yes       & Yes       & Yes       & Yes       & Yes       & Yes       & Yes       & Yes       & Yes       \\
            county FE              & Yes       & Yes       & Yes       & Yes       & Yes       & Yes       & Yes       & Yes       & Yes       & Yes       & Yes       & Yes       \\
            border-county FE       & Yes       & Yes       & Yes       & Yes       & Yes       & Yes       & Yes       & Yes       & Yes       & Yes       & Yes       & Yes       \\
            border-county LTs      & Yes       & Yes       & Yes       & Yes       & Yes       & Yes       & Yes       & Yes       & Yes       & Yes       & Yes       & Yes       \\ \midrule
            clustered at the:      & county    & industry  & zipcode   & facility  & county    & industry  & zipcode   & facility  & county    & industry  & zipcode   & facility  \\
            Observations           & 1,893,689 & 1,893,689 & 1,893,689 & 1,893,689 & 1,893,689 & 1,893,689 & 1,893,689 & 1,893,689 & 1,893,689 & 1,893,689 & 1,893,689 & 1,893,689 \\
            $R^2$                  & 0.393     & 0.393     & 0.393     & 0.393     & 0.378     & 0.378     & 0.378     & 0.378     & 0.385     & 0.385     & 0.385     & 0.385     \\ \bottomrule \bottomrule
        \end{tabular}%
    }
    \begin{minipage}{\columnwidth}
        \vspace{0.05in}
        \tiny NOTES: These results are obtained from estimating model~\ref{eq:baseline-emp-hours}. Robust standard errors clustered at the state level are reported in parentheses. ***, **, and * denote significance levels at the less than $1\%$, $5\%$ and $10\%$, respectively.
    \end{minipage}
\end{table}
    % Please add the following required packages to your document preamble:
% \usepackage{booktabs}
% \usepackage{graphicx}
\begin{table}[H]
    \centering
    \caption{Output and Labour Productivity: Alternative Clustering of the SEs}
    \label{tab:baseline-output-robustness}
    \resizebox{\columnwidth}{!}{%
        \begin{tabular}{@{}lllllllllllll@{}}
            \toprule\toprule
            Output (log) & \multicolumn{4}{c}{Output} & \multicolumn{4}{c}{Output per Hour} & \multicolumn{4}{c}{Output per Worker} \\
            \cmidrule(lr){2-5} \cmidrule(lr){6-9} \cmidrule(lr){10-13} & 1         & 2         & 3         & 4         & 5         & 6         & 7         & 8         & 9         & 10        & 11        & 12        \\ \midrule
            $Treated^{e}$                                              & 0.125***  & 0.125**   & 0.125***  & 0.125**   & 0.144***  & 0.144***  & 0.144***  & 0.144***  & 0.127***  & 0.127***  & 0.127***  & 0.127***  \\
            & (0.038)   & (0.051)   & (0.047)   & (0.049)   & (0.044)   & (0.045)   & (0.048)   & (0.047)   & (0.037)   & (0.041)   & (0.043)   & (0.043)   \\
            cohort 2014                                                & 0.122**   & 0.122*    & 0.122*    & 0.122*    & 0.216***  & 0.216***  & 0.216***  & 0.216***  & 0.180***  & 0.180***  & 0.180***  & 0.180***  \\
            & (0.057)   & (0.071)   & (0.067)   & (0.070)   & (0.065)   & (0.066)   & (0.073)   & (0.071)   & (0.055)   & (0.059)   & (0.066)   & (0.064)   \\
            cohort 2015                                                & 0.135***  & 0.135*    & 0.135**   & 0.135**   & 0.024     & 0.024     & 0.024     & 0.024     & 0.039     & 0.039     & 0.039     & 0.039     \\
            & (0.037)   & (0.073)   & (0.059)   & (0.061)   & (0.043)   & (0.046)   & (0.043)   & (0.045)   & (0.032)   & (0.042)   & (0.038)   & (0.041)   \\
            cohort 2017                                                & -0.447*** & -0.447**  & -0.447**  & -0.447**  & 0.108*    & 0.108     & 0.108     & 0.108     & 0.105**   & 0.105*    & 0.105     & 0.105     \\
            & (0.133)   & (0.212)   & (0.180)   & (0.180)   & (0.060)   & (0.069)   & (0.099)   & (0.098)   & (0.048)   & (0.062)   & (0.092)   & (0.088)   \\
            controls                                                   & Yes       & Yes       & Yes       & Yes       & Yes       & Yes       & Yes       & Yes       & Yes       & Yes       & Yes       & Yes       \\
            year FE                                                    & Yes       & Yes       & Yes       & Yes       & Yes       & Yes       & Yes       & Yes       & Yes       & Yes       & Yes       & Yes       \\
            county FE                                                  & Yes       & Yes       & Yes       & Yes       & Yes       & Yes       & Yes       & Yes       & Yes       & Yes       & Yes       & Yes       \\
            border-county FE                                           & Yes       & Yes       & Yes       & Yes       & Yes       & Yes       & Yes       & Yes       & Yes       & Yes       & Yes       & Yes       \\
            border-county LTs                                          & Yes       & Yes       & Yes       & Yes       & Yes       & Yes       & Yes       & Yes       & Yes       & Yes       & Yes       & Yes       \\ \midrule
            clustered at the:                                          & county    & industry  & zipcode   & facility  & county    & industry  & zipcode   & facility  & county    & industry  & zipcode   & facility  \\
            Observations                                               & 1,893,689 & 1,893,689 & 1,893,689 & 1,893,689 & 1,893,689 & 1,893,689 & 1,893,689 & 1,893,689 & 1,893,689 & 1,893,689 & 1,893,689 & 1,893,689 \\
            $R^2$                                                      & 0.549     & 0.549     & 0.549     & 0.549     & 0.630     & 0.630     & 0.630     & 0.630     & 0.648     & 0.648     & 0.648     & 0.648     \\ \bottomrule\bottomrule
        \end{tabular}%
    }
    \begin{minipage}{\columnwidth}
        \vspace{0.05in}
        \tiny NOTES: These results are obtained from estimating model~\ref{eq:baseline-output}. Robust standard errors clustered at the state level are reported in parentheses. ***, **, and * denote significance levels at the less than $1\%$, $5\%$ and $10\%$, respectively.
    \end{minipage}
\end{table}


    \section{Offsite and POTWs Toxic Release Intensities}\label{sec:offsite-and-potws-toxic-release-intensities}
    \begin{table}[H]
    \centering
    \caption{Summary Statistics (Offsite)}
    \label{tab:sumstat-offsite}
    \begin{tabular}{lrrrrr}
        \toprule \toprule
        Variable                                     & Obs     & Mean   & StdDev  & Min & Max       \\ \midrule
        total releases intensity                     & 1179754 & 257.85 & 2453.53 & 0   & 125639.48 \\
        total land releases intensity                & 1179754 & 196.20 & 2037.73 & 0   & 125639.48 \\
        total land releases other intensity          & 1179754 & 1.24   & 22.82   & 0   & 1637.57   \\
        total landfills intensity                    & 1179754 & 172.99 & 1958.09 & 0   & 125639.48 \\
        total surface impoundment intensity          & 1179754 & 0.63   & 58.77   & 0   & 7517.19   \\
        total underground injection intensity        & 1179754 & 18.44  & 557.48  & 0   & 59894.46  \\
        total wastewater releases intensity          & 1179754 & 6.00   & 125.39  & 0   & 15101.70  \\
        total releases (metal solidify) intensity    & 1179754 & 61.22  & 1507.92 & 0   & 84868.80  \\
        total releases (storage) intensity           & 1179754 & 0.87   & 34.06   & 0   & 6755.87   \\
        total releases (other mgt) intensity         & 1179754 & 5.18   & 99.26   & 0   & 11850.66  \\
        total releases (to-land) treatment intensity & 1179754 & 2.91   & 90.79   & 0   & 7875.12   \\
        total releases (unknown) intensity           & 1179754 & 6.30   & 59.87   & 0   & 3610.69   \\
        total releases (waste broker) intensity      & 1179754 & 8.28   & 115.33  & 0   & 6738.71   \\ \bottomrule\bottomrule
    \end{tabular}
\end{table}

    \begin{table}[H]
    \centering
    \caption{Summary Statistics (POTWs)}
    \label{tab:sumstat-potws}
    \begin{tabular}{lrrrrr}
        \toprule\toprule
        Variable                               & Obs    & Mean  & StdDev & Min & Max      \\ \midrule
        total releases intensity               & 308943 & 17.87 & 404.45 & 0   & 27648.29 \\
        underground releases intensity         & 308943 & 6.69  & 288.78 & 0   & 27648.29 \\
        underground releases intensity (other) & 308943 & 11.18 & 253.74 & 0   & 26548.32 \\ \bottomrule \bottomrule
    \end{tabular}
\end{table}

    \begin{figure}[H]
    \centering
    \includegraphics[width=1\textwidth, height=0.5\textheight,keepaspectratio]{fig_sdid_total_releases_offsite}
    \caption{Total Offsite Releases Intensity}
    \label{fig:baseline-offsite-total-releases-intensity}
    \begin{minipage}{12cm}
        \vspace{0.05in}
        NOTES: The event study model of equation~\ref{eq:baseline-offsite-total-releases-intensity} is $P_{f,c,i,cp,s,t}^{offsite} = \sum_{{e = 2011},{e \neq 2013}}^{2017} \beta Treated_{s,t}^e + \delta X_{v,c,t-1} + \omega F_{f,t} + \gamma_{f} + \phi_{cp} + \eta_{c,t} + \left[\lambda_{t} + \theta_{f,h} + \sigma_{s} + \zeta_{c} \right] + \varepsilon_{f,c,i,cp,s,t}$. Three-way clustered robust standard errors are reported in parentheses, and clustered at the toxic chemical, industry and state levels.
    \end{minipage}
\end{figure}
    \begin{figure}[H]
    \centering
    \includegraphics[width=1\textwidth, height=0.5\textheight,keepaspectratio]{fig_sdid_total_land_releases_offsite}
    \caption{Total Offsite Total Releases Intensity}
    \label{fig:baseline-offsite-land-releases-intensity}
    \begin{minipage}{12cm}
        \vspace{0.05in}
        NOTES: The event study model of equation~\ref{eq:baseline-offsite-land-releases-intensity} is $L_{f,c,i,cp,s,t}^{offsite} = \sum_{{e = 2011},{e \neq 2013}}^{2017} \beta Treated_{s,t}^e + \delta X_{v,c,t-1} + \omega F_{f,t} + \gamma_{f} + \phi_{cp} + \eta_{c,t} + \left[\lambda_{t} + \theta_{f,h} + \sigma_{s} + \zeta_{c} \right] + \varepsilon_{f,c,i,cp,s,t}$. Three-way clustered robust standard errors are reported in parentheses, and clustered at the toxic chemical, industry and state levels.
    \end{minipage}
\end{figure}
    \begin{figure}[H]
    \centering
    \includegraphics[width=1\textwidth, height=0.5\textheight,keepaspectratio]{C:/Users/david/OneDrive/Documents/ULMS/PhD/Thesis/chapter3/src/climate_change/latex/fig_sdid_total_releases_potws}
    \caption{Total POTWs Releases Intensity and Waste Management}
    \label{fig:baseline-potws-total-releases-intensity}
    \begin{minipage}{\columnwidth}
        \vspace{0.05in}
        \tiny NOTES: The event study model is $P_{f,cp,c,t}^{POTWs} = \sum_{{e = 2011},{e \neq 2013}}^{2017} \beta Treated_{s,t}^e + \delta X_{v,c,t-1} + \omega F_{f,t} + \lambda_{t} + \gamma_{f} + \phi_{cp} + \zeta_{c} + \eta_{c,t} + \theta_{cp,t} + \varepsilon_{f,cp,c,t}$. Three-way clustered robust standard errors are reported in parentheses, and clustered at the toxic chemical, industry and state levels.
    \end{minipage}
\end{figure}
%======================================================================================================================%
    \bibliographystyle{C:/Users/david/OneDrive/Documents/ULMS/PhD/Thesis/chapter3/src/climate_change/latex/Economic_Journal/ejbib}
    \bibliography{emissions}
\end{document}
%======================================================================================================================%