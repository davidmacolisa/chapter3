\documentclass{OUP-EJ}

\linespread{1}


\begin{document}

\noshorttitletrue

\title[How to prepare an article for~publication~using~\LaTeXe]{How to prepare an article for~publication~using~\LaTeXe}

\maketitle

\section{Introduction: file preparation and submission}


The \verb"OUP-EJ" \LaTeXe\ article class file is provided to help authors prepare articles for submission to \textit{The Economic Journal}.
This document gives advice on how to prepare articles for submission, and specific instructions on how to use \verb"OUP-EJ.cls".

The \verb"OUP-EJ" \LaTeXe\ class file is based on \verb+article.cls+, and all the basic and default \LaTeX\ commands and environments will remain the same.


If you are using BibTeX, you might need to use a ejbib.bst to generate the .bbl file


\section{Preparing your article}

\subsection{Sample coding for the start of an article}\label{sec2}

The code for the start of a title page of a typical paper in the \verb"OUP-EJ.cls" style might read:
\small
\begin{verbatim}
\documentclass{OUP-EJ}

\begin{document}

\title[Short title]{Title of the article}

\author[1,3,*]{P J Smith}

\author[2,4,\dagger]{T M Collins}

\author[3]{R J Jones}

\affil[1]{Mathematics Faculty, Open University, Milton Keynes MK7~6AA, UK}

\affil[2]{Department of Mathematics, Imperial College, Prince Consort Road,
London SW7~2BZ, UK}

\affil[3]{Department of Computer Science, University College London,
Gower Street, London WC1E~6BT, UK}

\affil[4]{Department of Physics, University of Bristol, Tyndalls Park Road,
 Bristol BS8 1TS,}

\corres[*]{Correspondence e-mail: xxxx@xxxx.com}

\aunotes[\dagger]{Present address: Department of Physics, University
of Bristol, Tyndalls Park Road, Bristol BS8 1TS, UK.}

\begin{abstract}
...
\end{abstract}

\keywords{magnetic moment, solar neutrinos, astrophysics}

\dedication{Dedication goes here.}

\thanks{Thanks statement.}

\maketitle
\end{verbatim}
\normalsize


\noindent The first line should be
\verb"\documentclass{OUP-EJ}"  to load the preprint class
file.  The normal text will be in the Computer Modern 10pt font.
Although it is possible to choose a font other than Computer Modern by loading external packages, this is not recommended.

The article text begins after \verb"\begin{document}".
Authors of very long articles may find it convenient to separate
their articles into a series of \LaTeX\ files, each containing one section, and each of which is called
in turn by the primary file.  The files for each section should be read in from the current directory;
please name the primary file clearly to enable us to run \LaTeX\ on this file.

Authors may use any common \LaTeX\ \verb".sty" files.
Authors may also define their own macros and definitions either in the main article \LaTeX\ file
or in a separate \verb".tex" or \verb".sty" file that is read in by the
main file, provided they do not overwrite existing definitions.
It is helpful to the production staff if complicated author-defined macros are explained in a \LaTeX\ comment.

\section{The title and abstract}

If you use \verb"OUP-EJ.cls", the code for setting the title page information is slightly different from
the normal default in \LaTeX.
%If you are using a different class file, you do not need to mimic the appearance of
%an \verb"OUP-EJ.cls" title page, but please ensure that all of the necessary information is present.

\subsection{Titles and article types}

The title is set using the command
\verb"\title[short title]{#1}", where \verb"#1" is the title of the article. The
first letter of the title should be capitalized with the rest in lower case.
The title appears in bold case. Authors are asked to include a short title (40 characters or less) through the command \verb+\title[short title]{....}+.\bigskip

\noindent Footnotes to titles may be given by using \verb"\thanks{Text of footnote.}" immediately after the abstract.
Acknowledgment of funding should be included in the ``Acknowledgments'' section rather than in a footnote.

\clearpage

\subsection{Authors' names and addresses}

For the authors' names, use the command
\medskip

\noindent
\verb"\author[<affil indicators>]{Author Name}"
\medskip

\noindent
Each author should be coded through the command \verb"\author{}"
\bigskip

\noindent The authors' affiliations follow the list of authors.
Each address is set by using\newline
\verb"\affil[<affil indicator>]{#1}"
\bigskip

\noindent Correspondence e-mail addresses are added by inserting the
command\medskip

\noindent
\verb"\corres[<Author indicator>]{#1}"\medskip

\noindent
after the address(es) where \verb"#1" is the e-mail address.
See Section~\ref{sec2} for sample coding. For more than one e-mail address, please use the same command with a different indicator, e.g, \verb+\corres[*]{...}+, \verb+\corres[**]{...}+.
Please ensure that a corresponding author indicator should be added in the author field, e.g., \verb+\author[1,*]{Author Name}+.
\bigskip

\noindent
Additional information that should appear as  appear as a footnote, such as a permanent address, etc, should be coded as \verb"\aunote[Author indicator]{Footnote details}"\newline e.g. \verb"\aunote[\dagger]{Footnote details}".

\smallskip

\noindent
Please ensure that a footnote indicator should be added to the author field\newline e.g. \verb+\author[1,\dagger]{Author Name}+


\subsection{The abstract}

The abstract follows the addresses and
should give readers concise information about the content
of the article and indicate the main results obtained and conclusions
drawn.  It should be enclosed between \verb"\begin{abstract}"
and \verb"\end{abstract}" commands.
Authors are requested to include an abstract, not exceeding 100 words.

% \subsection{Subject classification numbers}
% We no longer ask authors to supply Physics and Astronomy Classification System (PACS)
% classification numbers.  For submissions to {\it Nonlinearity}\/ we ask that you should
% supply Mathematics Subject Classification (MSC) codes.  MSC numbers are included after the abstract
% using \verb"\ams{#1}".


\subsection{Keywords}
A list of keywords (each word no more than 20 characters) indicating the contents of the article,
e.g., \verb"\keywords{kwd one, kwd two, kwd three}", after the end of the abstract.

\subsection{Dedication}
If a dedication is to be added in the submission, use the following code:\newline \verb+\dedication{Dedication goes here.}+


\subsection{Subject classification numbers}
Classification numbers are included after the abstract using \verb"\subjclass{....}".



\subsection{Making a  title page}
To print the title page information, use the code \verb"\maketitle"  before the start of the text.
If \verb"\maketitle" is not included, the text of the
article, title, authors' information, abstract and keywords would not be printed.

\section{The text}
\subsection{Sections, subsections and subsubsections}
The text of articles may be divided into sections, subsections and, where necessary,
subsubsections. To start a new section, end the previous paragraph and
then include \verb"\section" followed by the section heading within braces.
Numbering of sections is done {\it automatically} in the headings:
sections will be 	numbered 1, 2, 3, etc., subsections will be numbered
2.1, 2.2,  3.1, etc., and subsubsections will be numbered 2.3.1, 2.3.2,
etc.  Subsections and subsubsections are
similar to sections, but
the commands are \verb"\subsection" and \verb"\subsubsection", respectively.
\small\begin{verbatim}
\section{This is the section title}
\subsection{This is the subsection title}
\end{verbatim}\normalsize

\section{Tables}

Statistical tables should be clearly headed and the reader should be able to understand the
meaning of each row or column without searching in the text for explanations of symbols, etc.
Units of measurement, base-dates for index numbers, geographical areas covered and sources
should be clearly stated. Authors are fully responsible for the accuracy of the data and for
checking their proofs. Whenever the authors feel that the reader would have difficulty in testing
the derivation of their statistics, they should provide supplementary notes on the methods used.


\subsection{Table formatting}
\begin{itemize}
\item Ensure that tables are large enough to be clearly readable on page;
\item  Use consecutive Arabic numbers;
\item Use short titles in italics with initial capitals;
\item Do not use vertical lines or shading;
\item Do not use more than 10 columns per table.
\end{itemize}


\clearpage


\subsection{The basic table format}
The standard form for a table in \verb"OUP-EJ.cls" is:
\small\begin{verbatim}
\begin{table}
\centering
\caption{Table caption.\label{label}}
\begin{tabular}{@{}llll}
\hline
Head 1&Head 2&Head 3&Head 4\\
\hline
1.1&1.2&1.3&1.4\\
2.1&2.2&2.3&2.4\\
\hline
\end{tabular}
\end{indented}
\end{table}
\end{verbatim}\normalsize


\section{Diagrams and figures}
Diagrams and figures should be clearly drawn and accompanied by basic statistics that
were required for their preparation, the axes must be clearly labelled and the reader must be
able to understand the diagrams and figures without searching in the text for explanations.


\subsection{Diagram and figure formatting}

\begin{itemize}
\item Ensure that diagrams and figures are large enough to be clearly readable on page;
\item Use consecutive Arabic numbers;

\item Use short titles in italics with initial capitals;
\item Use just a few round numbers in the axes;
\item Diagrams and figures should be of high resolution;
\item Diagram and figure legends must be clearly readable in black and white.
\end{itemize}



\subsection{Inclusion of graphics files\label{figinc}}
Graphics files can
be included within figure and center environments at an
appropriate point within the text using the following code:
\small\begin{verbatim}
\includegraphics{file.eps}
\end{verbatim}\normalsize


\clearpage

\subsection{The basic figure format}
The standard form for a figure in \verb"OUP-EJ.cls" is:
\small\begin{verbatim}
\begin{figure}
\centering
\includegraphics{file.eps}
\caption{figure caption.\label{label}}
\end{figure}
\end{verbatim}\normalsize



\section{Footnotes}

\begin{itemize}
\item Footnotes should be kept to a minimum;
\item Use consecutive superscript Arabic numbers;
\item Footnotes should appear after punctuation.
\end{itemize}

\enlargethispage*{96pt}%%%%%%%%%%%%%
\section{Acronyms}

\begin{itemize}
\item First reference: Institute for Fiscal Studies (IFS)
\item Subsequent references: IFS
\end{itemize}

\section{Quotations}
\begin{itemize}
\item Short quotation: Short quotations `should be in single inverted commas with citation
and page reference'. (Brown, 1965, p.20)

\item Long quotation: Long quotations should be separated from the main text, starting on a
new line with a line space before and after. The paragraph should be indented. A
citation should be included immediately under the quotation to the right.

\hfill (Brown, 1965, p.20)
\end{itemize}


\section{Mathematics}
The mathematical derivations necessary for justifying each step of the argument should
accompany all articles with mathematical arguments.
Please do not use smaller fonts in complex expressions, except for superscripts and subscripts.
\begin{itemize}
\item Numbers less than 1, but greater than -1 must have 0 before the decimal point.

\item  All equations to which the text refers should be numbered consecutively as (1), (2), etc., on
the right-hand side of the page. Equations included within the main bulk of text should,
where possible, be kept on one line. If equations contain fractions, a slash “/” (solidus)
should be used and the numerator and denominator enclosed with parentheses.

\item  Algebra should include punctuation.
\item  Percent should appear in the text as \%.
\item  Algebra should have bold capitals for matrices, bold lower case for vectors and italic
lower case for scalars.
\item  Transposition is denoted by prime (A$^\prime$).
\item  Fractions that are too complex to be kept within the text should be presented on a
separate line.

\item  Assumptions, corollaries, definitions, lemma, propositions and theorems should
each be consecutively numbered 1,2,3 etc. Each category should follow on only from
other numbers in that category (i.e. PROPOSITION 1, THEOREM 1, PROPOSITION 2,
COROLLARY 1, THEOREM 2, PROPOSITION 3). Titles should be in small capitals, with
the text following in italics. Algebra that does not follow the journal style will be queried
during production.

Each element is set by using the below listed commands:

\begin{verbatim}
Theorem:     \begin{theorem}[Optional head] ....     \end{theorem}
Proposition: \begin{proposition}[Optional head] .... \end{proposition}
Corollary:   \begin{corollary}[Optional head] ....   \end{corollary}
Assumptions: \begin{assumption}[Optional head] ....  \end{assumption}
Definitions: \begin{definition}[Optional head] ....  \end{definition}
Remark:      \begin{remark}[Optional head] ....      \end{remark}
Proof:       \begin{proof}[Optional head] ....       \end{proof}
\end{verbatim}



\end{itemize}


\section{Appendices}
If there are two or more appendices, they should be called Appendix A, Appendix B, etc.
Numbered equations will be in the form (A.1), (A.2), etc.,
figures will appear as figure A1, figure B1, etc and tables as table A1,
table B1, etc.

The command \verb"\appendix" is used to signify the start of the
appendices. Thereafter, \verb"\section", \verb"\subsection", etc., will
give the headings appropriate for an appendix. To obtain a simple heading of
`Appendix', use the code \verb"\section*{Appendix}".



\section{Acknowledgments}
Authors wishing to acknowledge assistance or encouragement from
colleagues, special work by technical staff or financial support from
organizations should do so in an unnumbered `Acknowledgments' section
immediately following the last numbered section of the paper. In \verb"OUP-EJ.cls", the
command

\begin{verbatim}
\begin{ack}
....
\end{ack}
\end{verbatim}
sets the acknowledgments heading as an unnumbered
section.

The acknowledgments section should include the details of those for whom credit is due,
e.g., funders, co-authors, research assistants or others. The `Acknowledgments' section
should also provide the reference of the Research Ethics Board approval where applicable.



\clearpage

\section{Referencing\label{except}}

As per the journal-specific style, we need to use Harvard (alphabetical) system.




\subsection{Harvard (alphabetical) system}
In the Harvard system, the name of the author appears in the text together
with the year of publication. As appropriate, either the date or the name
and date is included within parentheses.

\subsection{BibTeX}

We have a \verb+Bibtex+ file to help with your references. Latex users can download the file from https://academic.oup.com/DocumentLibrary/EJ/ejbib.txt
(please note that the file extension should be changed to .bst for use in manuscript
preparation). Please provide any related .bib, .bbl or .bst files.

When using Bibtex, the bibliography style is set and the bibliography file is imported with the following two commands:
\begin{verbatim}
  \bibliographystyle{ejbib}
  \bibliography{bibfile}
\end{verbatim}
where \verb+bibfile+ is the name of the bibliography \verb+.bib+ file, without the extension and \verb+ejbib+ is the  \verb+.bst+ name



\subsection{Reference formatting in text}

Where there are only two authors,
both names should be given in the text; if there are more than two
authors, only the first name should appear followed by `{\it et al}'.
When two or
more references have the same author group with the same year,
they should be identified by including a, b, etc. after the date
(e.g.\ 2012a). If several references to different pages of the same article
occur, the appropriate page number may be given in the text, e.g.,\ Kitchen
(2011, p.39).

\begin{itemize}
\item Single reference: Brown (1964), (Brown, 1964) or Brown (1964a,b; 1965)
\item Multiple references in chronological order: (Green, 1963; Brown, 1971; Orange,
1982,1985)
\item Work with two authors: (Black and White, 2009)
\item Work with more than two authors: (Purple et al., 2008)
\item Chapters and pages: (Blue, 1996, ch.2); (Yellow, 2003, p.24)
\end{itemize}

\clearpage	

\noindent We can achieve this in-text citations through the below commands:
\begin{verbatim}
coding                          Output
======                          ======
\citet{key} ==>>                Jones et al. (1990)
\citet*{key} ==>>               Jones, Baker, and Smith (1990)
\citep{key} ==>>                (Jones et al., 1990)
\citep*{key} ==>>               (Jones, Baker, and Smith, 1990)
\citep[chap. 2]{key} ==>>       (Jones et al., 1990, chap. 2)
\citep[e.g.][]{key} ==>>        (e.g. Jones et al., 1990)
\citep[e.g.][p. 32]{key} ==>>   (e.g. Jones et al., 1990, p. 32)
\citeauthor{key} ==>>           Jones et al.
\citeauthor*{key} ==>>          Jones, Baker, and Smith
\citeyear{key} ==>>             1990
\end{verbatim}



The reference list at the end of an article consists of an
unnumbered `References' section containing an
alphabetical listing by authors' names. References with the same author list are ordered by date, with the oldest first.
The reference list in the
preprint style is started in \verb"OUP-EJ.cls" by including the command \verb"\begin{thebibliography}{}".
Individual references start with \verb"\bibitem[Author(year)]{label}" and the reference list is completed with \verb"\end{thebibliography}".


\section{Cross-referencing\label{xrefs}}

The facility to cross-reference items in the text is very useful when
composing articles as the precise form of the article may be uncertain at the start,
and  revisions and amendments may subsequently be made.
\LaTeX\ provides excellent facilities for doing cross-referencing,
and these can be very useful in preparing articles.

\subsection{References}
\label{refs}

When using \LaTeX , two passes (under certain circumstances, three passes)
are necessary initially to get the cross-references right,
but once they are correct a single run is usually sufficient provided an
\verb".aux" file is available and the file
is run to the end each time.  \verb"\label" may contain letters, numbers
or punctuation characters but must not contain spaces or commas. It is also
recommended that the underscore character \_{} is not used in cross-referencing. Thus labels of the form \verb"eq:partial", \verb"fig:run1", \verb"eq:dy'",
etc., may be used. When several
references occur together in the text, \verb"\cite" may be used with
multiple labels with commas but no spaces separating them.


\subsection{Equation numbers, sections, subsections, figures and
tables}
Labels for equation numbers, sections, subsections, figures and tables
are all defined with the \verb"\label{label}" command and cross-references
to them are made with the \verb"\ref{label}" command.

Any section, subsection, subsubsection, appendix or subappendix
command defines a section type label, e.g., 1, 2.2, A2, A1.2 depending
on the context. A typical article might have in the code of its introduction
`The results are discussed in section\verb"~\ref{disc}".' and
the heading for the discussion section would be:
\small\begin{verbatim}
\section{Results}\label{disc}
\end{verbatim}\normalsize
Labels to sections, etc., may occur anywhere within that section except
within another numbered environment.
Within a maths environment, labels can be used to tag equations which are
referred to within the text.

\end{document} 