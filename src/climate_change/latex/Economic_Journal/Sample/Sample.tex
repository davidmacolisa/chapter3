\documentclass[demo]{OUP-EJ}

\begin{document}


\title[Short Article title]{Fast and automatic periacetabular osteotomy fragment pose estimation using intraoperatively implanted fiducials and single-view fluoroscopy}

\author[1,3,*]{Michael K. Murray}

\author[2,**]{Paul Norbury}

\author[3,\S]{Robert Jones}

\affil[1]{School of Mathematical Sciences, University of Adelaide, Adelaide, SA 5005, Australia}
\affil[2]{School of Mathematics and Statistics, University of Melbourne, Vic 3010, Australia}

\affil[3]{Department of Mathematics, Duke University, 120 Science Drive, Durham NC 27708 USA}

\corres[*]{michael.murray@adelaide.edu.au}
\corres[**]{norbury@unimelb.edu.au}


\aunotes[\S]{Coppead Graduate School of Business, Federal University of Rio de Janeiro-RJ, Brazil.}


\subjclass[2010]{81T13, 53C07, 14D21}

\begin{abstract}
This paper provides a rigorous derivation of a counterexample of Bourgain, related to a well-known question of pointwise a.e. convergence for the solution of the linear Schr\"odinger equation, for initial data in a Sobolev space. This counterexample combines ideas from analysis and number theory, and the present paper demonstrates how to build such counterexamples from first principles, and then optimize them.
\end{abstract}

\keywords{transfusion; walking blood bank; military medicine, exercise physiology}

\dedication{Dedicated to the memory of Sir Michael Atiyah}

\thanks{MKM acknowledges support under the  Australian
Research Council's {\sl Discovery Projects} funding scheme (project number DP180100383).  PN acknowledges support under the  Australian
Research Council's {\sl Discovery Projects} funding scheme (project numbers DP170102028 and DP180103891).
}


\subjclass[2010]{81T13, 53C07, 14D21}

\maketitle


\section{Introduction}\label{isect1}
Positron emission tomography (PET) capabilities can be considerably improved with time-of-flight (TOF) measurement.
TOF enhances contrast-to-noise ratio in PET images, enables faster image reconstruction convergence, allows shorter acquisition time or lower injected dose, and enables improved limited angle tomography~\citep{Karp2008,Vandenberghe2016,Lecoq2017}.
These improvements can be achieved through a better coincidence time resolution (CTR) of the PET detectors~\citep{Conti2011}.

\subsection{Materials and methods}\label{isect2}



State-of-the-art CTR with clinical TOF-PET systems have recently reached 210~ps full-width at half maximum (FWHM) with 20~mm long detector crystals~\citep{vanSluis2019}.
It is known that depth-of-interaction (DOI) measurement in PET scanners with long crystals can alleviate the parallax effect that causes a degradation of the spatial resolution off-center~\citep{Lecomte2009,Cherry2012}.
A more recent motivation for DOI measurement is to counteract its degrading effect on CTR in TOF-PET detectors.
Since the velocity of scintillation photons in crystals is reduced by a factor corresponding to the refractive index $n$ of the material, their propagation time is therefore longer than annihilation photons propagating at the speed of light in vacuum $c$.

\subsubsection{Side/end signal readout}\label{isect3}

For a given coincidence between two TOF-PET detectors, the coincidence timing is impacted by this difference in propagation speed if the DOI in the two detectors differs~\citep{Moses1999}.


\paragraph{Experimental measurements}
\label{sec:expMeas}


With typical (TOF) PET detectors, the statistical nature of the scintillator light emission and the photosensor response results in a variance on the measured signal which usually is the main source of the timing imprecision.\footnote{
Significant improvements in reducing these degrading effects are achieved by using materials with fast light emission~\citep{Lucchini2016,Turtos2016a,Prochzkov2018}, efficient photosensors with low time jitter~\citep{Schaart2016,Acerbi2018,Nolet2018} or fast readout electronics~\citep{Cates2018,Gundacker2019}.}

\subparagraph{Head-to-head setup}
\label{sec:headtoheadsetup}

A CTR metric was recently proposed in which the timing signal contributions were separated in statistical variance-related effects and DOI-induced bias to estimate a mean square error~\citep{Toussaint2019}.
The metric was evaluated with a simulation framework revealing the substantial magnitude of the DOI bias against other variance-related effects when ultra-fast detection conditions are explored.

In this paper, an experimental validation of the analytical model proposed in \cite{Toussaint2019} is presented using a signal readout that allows the relative magnitude of the DOI-induced bias to be observable with typical scintillation detectors.
We show that the metric using a DOI bias separated from variance-related effects is effective to describe the measured CTR of coincident detectors with or without DOI information.
\begin{equation}
        \label{eq:rmsehtoh}
        \mathrm{RMSE}_{\mathrm{simul}}^{\mathrm{HtoH}}[\widehat{\theta}] = \left[\sum_{d_{ij}}\sum_{E_{ij}} \left( \mathrm{Var}(\widehat{\theta}^{\mathrm{coinc}}_{d_{ij},E_{ij}}) + (\mathrm{Bias}(\widehat{\theta}^{\mathrm{coinc}}_{d_{ij},E_{ij}}))^2 \right)\right]^{1/2}
\end{equation}
Assuming a Gaussian error of the estimator, a conversion to FWHM is then applied to obtain values as 2.355~$\times$~RMSE.
This metric is therefore extended compared to the one used in \cite{Toussaint2019} by adding the energy contribution to the CTR.

The following metric is used to predict the CTR of the head-to-head setup from the DOI-collimated setup experimental results
\begin{equation}
 \label{eq:rmsedoicoll}
        \mathrm{RMSE}_{\mathrm{\textrm{exp}}}^{\mathrm{pred}} = \left[\sum_{d_{j}} \left( \mathrm{Var}^{\mathrm{meas}}_{d_{j}}) + (\mathrm{Bias}^{\mathrm{corr}}_{d_{j}})^2 \right) \mathsf{P}(d_{j}) \right]^{1/2}
\end{equation}
where the corrected bias for all the aforementioned time offsets is $\mathrm{Bias}^{\mathrm{corr}}_{d_{j}} = \mu_{d_j}^{\mathrm{meas}} - \Delta t^{\mathrm{TOF}} - \overline{t}^{\mathrm{rel}} - t_{d_j}^{\mathrm{miss}}$ with $\mu_{d_j}^{\mathrm{meas}}$ being the measured coincidence mean time difference of the head-on crystal (naturally merging over DOI $d_i$ and energies $E_{i,j}$) relative to the collimated crystal at DOI $d_j$.
Note that $\mathsf{P}(d_{j})$ is adapted to reflect the experimental DOI irradiation width and the low-energy cut influence on DOI.


In most current scintillation detectors, the timing blur due to difference in DOI between coincident detectors remains marginal relative to other factors.
Here, in order to amplify the magnitude of DOI bias, a detection setup where the scintillation signal is read from the side near the end of the crystal (side/end readout), as shown in figure~\ref{fig:detector}, was chosen.


\begin{figure}[!t]
\centering
 \includegraphics{PMBaba7d0f1}
 \caption{Side/end readout schematic where the SiPM is coupled to one side face with a small opening towards the end of a crystal. This setup was chosen to enhance the effective propagation time difference between small and large DOI events.}
 \label{fig:detector}
\end{figure}


\begin{table}[!t]
\centering\caption{A summary of BB reconstruction errors for each surgery, specified by the cadaver specimen number and operative side.
The means and standard deviations of reconstruction errors are given for the separate ilium and fragment BB constellations and also the entire set of BBs.
For each surgery, four BBs were reconstructed for each of the ilium and fragment constellations.}
\label{tab:results_bb_recon_errors}
\begin{tabular*}{\textwidth}{@{\extracolsep\fill}l l l l}
\hline
Surgery & \multicolumn{3}{c}{Reconstruction errors (mm)} \\
\cline{2-4}
& Ilium BBs & Fragment BBs & All BBs \\
\hline
1 Left & $2.5 \pm        0.4$ & $3.2 \pm        0.2$ &  $2.9 \pm        0.5$ \\
1 Right & $2.1 \pm      0.2$ &  $2.7 \pm        0.5$ &  $2.4 \pm        0.5$ \\
2 Left & $1.6 \pm       0.3$ &  $1.4 \pm        0.2$ &  $1.5 \pm        0.3$ \\
2 Right & $1.3 \pm      0.5$ &  $1.1 \pm        0.1$ &  $1.2 \pm        0.3$ \\
3 Left & $1.3 \pm       0.3$ &  $1.1 \pm        0.4$ &  $1.2 \pm        0.3$ \\
3 Right & $1.4 \pm      0.2$ &  $1.6 \pm        0.2$ &  $1.5 \pm        0.2$ \\
\hline
\end{tabular*}
\end{table}


\begin{theorem}
Pose estimation was successfully performed on 18 total views (3 per surgery).
The distributions of fragment rotation, LCE angle, and translation errors in rotation were below $3^{\circ}$ for 12 of the 18 cases, with a mean of $2.4^{\circ}$.
When the rotation errors were decomposed about anatomical axes, only rotation about the IS axis had errors greater than $3^{\circ}$.
In terms of both mean and standard deviation, rotation measurements about the AP axis were the most accurate, followed by LR, and then IS.
\end{theorem}
\begin{theorem}[Theorem Head]
The maximum 3D LCE angle error was $1.8^{\circ}$ and the mean was $1.0^{\circ}$.
The mean translation error was 2.1 mm, and was less than 3 mm for 15 of the 18 estimates.
Mean translation errors about the anatomical axes were all within 0.2 mm of each other, and the maximum difference between standard deviations was 0.3 mm.
An entire listing of errors for each pose estimate is shown in supplementary table~S-4.
\end{theorem}

\begin{proposition}[Proposition Head]
The maximum 3D LCE angle error was $1.8^{\circ}$ and the mean was $1.0^{\circ}$.
The mean translation error was 2.1 mm, and was less than 3 mm for 15 of the 18 estimates.
Mean translation errors about the anatomical axes were all within 0.2 mm of each other, and the maximum difference between standard deviations was 0.3 mm.
An entire listing of errors for each pose estimate is shown in supplementary table~S-4.
\end{proposition}


\begin{proof}
In the third view for the left side of specimen~1, one ilium BB was outside the image bounds and not detected.
On the left side of specimen~2, one of the ilium BBs was occluded by K-wire in each view and therefore not detected.
Analysis of the postoperative CT revealed that this BB was actually dislodged by either: performance of the ilium osteotomy or insertion of the K-wire.
The missed ilium detections in views 1 and 2 on the right side of specimen~2, were occluded by screws.
Occlusion by K-wire also caused the missed ilium detection in view 2 on the right side of specimen~3.
However, according to the postoperative CT this ilium BB was also displaced from the bone.
The missed fragment BB detections were caused by K-wire occlusion.
There were no missed detections of ipsilateral BBs that were present in the scene and not occluded by screws or K-wires.
\end{proof}

\begin{remark}
In the third view for the left side of specimen~1, one ilium BB was outside the image bounds and not detected.
On the left side of specimen~2, one of the ilium BBs was occluded by K-wire in each view and therefore not detected.
Analysis of the postoperative CT revealed that this BB was actually dislodged by either: performance of the ilium osteotomy or insertion of the K-wire.
\end{remark}

\begin{ack}
This work was supported by a Discovery Grant from the Natural Sciences and Engineering Research Council of Canada (NSERC). F. Loignon-Houle held an Alexander Graham Bell Canada Graduate Scholarship from NSERC while M. Toussaint held an Accelerate Scholarship from Mitacs. The Sherbrooke Molecular Imaging Center is a member of the FRQS-funded Research Center of CHUS (CRCHUS).
\end{ack}




\appendix

\section{Basis Function Generation using Profile Interpolation}\label{app:basis_interp}

The accuracy of the profile interpolation procedure is assessed by splitting the set of 10 positron-emitting fragment profiles and deposited dose profiles corresponding to the monoenergetic irradiations of the PMMA target into a training set and testing set.



\section{Additional Information for $^{12}$C SOBPs} \label{app:weights}

The weight values used in Monte Carlo simulation to deliver the flat $^{12}$C SOBP, the 3 $^{12}$C SOBPs containing well-defined dose maxima and/or minima, and 4 $^{12}$C SOBPs with randomised weights are presented in figure  \ref{fig:prof_interp_validation}.


\begin{figure}[!b]
\centering\includegraphics{figname}
\caption{Profiles of $^{11}$C, $^{10}$C, and $^{15}$O fragment production and deposited dose profile resulting from irradiation of a homogeneous PMMA phantom by monoenergetic $^{12}$C beams with energies of 215~MeV\ u$^{-1}$, 245~MeV\ u$^{-1}$, 275~MeV\ u$^{-1}$ and 305~MeV\ u$^{-1}$ in Monte Carlo simulation, plotted against fragment and dose profiles estimated using the profile interpolation procedure.}\label{fig:prof_interp_validation}
\end{figure}



\begin{table}[t]
\centering\caption{Mean relative error between true and interpolated $^{11}$C, $^{10}$C, $^{15}$O and deposited dose profiles, calculated in the entrance, SOBP and tail regions, relative to the maximum of the true profile in the specified region.}\label{tab:prof_interp_validation}
\begin{tabular}{@{}lcccc@{}}
\hline
\multirow{2}{*}{Interpolated Profile}	&\multirow{2}{*}{Beam Energy (MeV\ u$^{-1}$)} & \multicolumn{3}{c}{Mean MRE (\%)}\\
\cline{3-5}
& & Entrance & SOBP & Tail \\
\hline
\multirow{4}{*}{$^{11}$C} & 215 & 0.49 & 5.81 & 0.41 \\
& 245 & 0.51 & 4.04 & 0.41 \\
& 275 & 0.92 & 5.07 & 0.37 \\
& 305 & 1.01 & 4.19 & 0.48 \\
\multirow{4}{*}{$^{10}$C} & 215 & 0.39 & 1.57 & 0.44 \\
& 245 & 0.59 & 2.71 & 0.65 \\
& 275 & 0.55 & 4.11 & 0.78 \\
& 305 & 0.52 & 2.69 & 0.93 \\
% \multirow{4}{*}{$^{15}$O} & 215 & 0.71 & 3.11 & 0.42 \\
% & 245 & 0.49 & 2.79 & 0.53 \\
% & 275 & 0.62 & 1.44 & 0.45 \\
% & 305 & 0.66 & 3.24 & 0.86 \\
\hline
\end{tabular}
\end{table}




\nocite{*}

\bibliographystyle{ejbib}

\bibliography{sample}



\end{document} 